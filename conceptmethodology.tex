\subsection{Concept}

To make a formal proof, developed in some system $X$, accessible to
the users of other systems, the first step is to express the theory
$D[X]$, implemented in the system $X$, in the logical framework {\textsf
  Dedukti}.  Then, we must instrument the system $X$ so that the proof
can be exported from it, as a piece of data, expressed as a proof in
$D[X]$ and included in {\textsf Logipedia}. Next, we need to analyze this
proof in order to determine which symbols, axioms and rewrite rules of
$D[X]$ it actually uses and, thus, in which alternative theories it
can be expressed.  Finally, we must align its concepts with the
definitions already present in {\textsf Logipedia} and decide where it
fits in the general structure of the encyclopedia.

Depending on the system $X$ we consider, more or less research needs
to be done, for instance to design the theory $D[X]$. Among the 15
systems we focus on, we plan to reach level 5 or 6 for 10 of them.
For the others, more research needs to be done and we plan to reach
level 3 only.

These goals require the expertise of computer scientists, logicians,
mathematicians who are experts on one theory or one system, and
specialists on logical frameworks, reverse mathematics, and concept
alignment.

\subsubsection{Description of the work package 1}

\subsubsection{Description of the work package 3}

\subsubsection{Description of the work package 4}

\subsubsection{Description of the work package 7}

\subsubsection{Description of the work package 2}

\subsubsection{Description of the work package 5 and 6}

\subsection{Methodology}

The methodology is the same for integrating any library to {\textsf Logipedia},
but due to a difference of readiness of the various systems we focus on,
our priorities are different.

\begin{enumerate}
\item We already know how to express most of the theories of {\textsf Atelier B},
{\textsf Coq}, {\textsf FoCaLiZe}, {\textsf HOL Light}, {\textsf HOL4}, {\textsf
Isabelle}, {\textsf Matita}, and {\textsf Rodin} in {\textsf Dedukti}. We propose
to instrument these systems so that they can produce {\textsf Dedukti}
proofs that we can include in {\textsf Logipedia}.

\item
For other theories, such as those of {\textsf Abella}, {\textsf Agda}, {\textsf
Lean}, {\textsf Minlog}, {\textsf Mizar}, {\textsf PVS}, {\textsf TLA+}, the work is
in progress, or not yet started.  So we must first understand how they
can be expressed in {\textsf Dedukti}.

\item Besides the standard libraries of these systems, large libraries
  have been developed: the {\textsf Isabelle Archive of formal Proofs} \cite{AFP},
  {\textsf Flyspeck}\cite{Flyspeck}, {\textsf MathComp}\cite{Mathcomp}, 
  {\textsf CompCert} \cite{Compcert}, {\textsf CakeML} \cite{CakeML}, ...  We aim to include
  some of these libraries in {\textsf Logipedia}.
  
\item
Besides proof systems, we also want to include proofs coming from
automated theorem provers, SAT solvers, SMT solvers, and model
checkers.  So we must instrument some of these systems so that they
can produce {\textsf Dedukti} proofs that we can include in {\textsf
Logipedia}.

\item
We want to develop algorithms to analyze which symbol, axiom, rewrite
rule is used in each proof, and consequently in which system each proof
can be used. We also want to develop algorithms to eliminate some of the 
symbols, axioms, and rewrite rules used in a proof in order to be able to 
use it in more systems.


\item
Each library imported in {\textsf Logipedia} will come with its own
definition of natural numbers, real numbers, etc. We want to develop
``concept alignments algorithms'' to transport theorems from one
structure to another isomorphic one.

\item 
Besides data, we propose to include in {\textsf Logipedia}, metadata and
an inner structure.
\end{enumerate}


\subsubsection{Methodology of the work package 1}

\subsubsection{Methodology of the work package 3}

\subsubsection{Methodology of the work package 4}

\subsubsection{Methodology of the work package 7}

\subsubsection{Methodology of the work package 2}

\subsubsection{Methodology of the work package 5 and 6}


\subsection{Readiness of the project}

This idea of building such a standard for proofs has already been
investigated in the past, such as in the Qed manifesto \cite{Qed94}, but
has produced limited results.

Our thesis is that, since the
Qed project, the situation has radically changed. After
thirty years of research, we have an empirical evidence that most of
the formal proofs developed in one of these systems can also be
developed in another. We understand the relationship between the
theories implemented in these systems much better. We have developed
several logical frameworks, extending predicate logic, in which these
theories can be expressed. And we have developed reverse mathematics
algorithms to analyze which axioms and rules are used in each proofs
and algorithms, such as constructivization algorithms, to translate
proofs from one theory to another.


%%% Local Variables:
%%%   mode: latex
%%%   ispell-local-dictionary: "english"
%%% TeX-master: "propB"
%%% End:
