\subsection{State of the art}

Formal proofs and systems manipulating such proofs have become central
tools both in safety and security, as shown by several major
successes: the correctness proof of the Paris metro line 14, the NASA
detect-and-avoid system for unmanned aircraft system proved correct in
PVS, the proved operating system seL4 \cite{Klein09}, and the proved C
compiler CompCert \cite{Leroy06}.  Also in the realm of pure
mathematics, formal proofs have convinced us of the correctness of
several complex proofs, such as those of the Feit-Thompson theorem
\cite{Gonthier13}, Hales' theorem (Kepler's conjecture)
\cite{Hales17}, etc.  The development of these formal proofs has led
to the construction of huge libraries, totalling millions of hours of
work, that cover a significant part of mankind's mathematical
knowledge.

Software development has always been accompanied with the definition
of standards that make systems interoperable and data sustainable. For
example, web browsers are interoperable and websites are sustainable
because they all comply with web standards such as HTML and
JavaScript. The area of formal proofs is however an exception. So,
while we had in the past an (informal) proof of Pythagoras' theorem or
Fermat's little theorem, the same proof now has different
formalizations in {\sc Coq}, {\sc Matita}, {\sc HOL Light}, etc. This
lack of standards is the major weakness of this area, as it
jeopardizes the usability and the sustainability of these
libraries. Indeed, each library is specific to a proof system, or
often even to a specific version of this system. In general, a library
developed in one system cannot be used in another, and when the system
is no longer maintained, the library may disappear. Being a major
weakness, designing a standard for formal proofs is also a major
challenge.

On more philosophical grounds, this lack of a standard jeopardizes the
universality of logical truth, just like non-Euclidean geometries did
in the 19th century, as some statements could be true in one
geometry, but false in others.  This lack of universality severely
limits the spreading of formal proofs in non-specialist
communities. For instance, teaching formal proving to undergraduate
students in a logic course is difficult, as it requires the choice of
a specific language, a specific theory and a specific system that
orients the course towards this language, theory, and system. The same
is true for the use of formal proofs in industry or by working
mathematicians. So another goal of this project is to make formal
proofs accessible to a much larger community, as standards often do.

At the beginning of the 20th century, a solution was found to the
crisis of non-Euclidean geometries: the definition of the various
geometries in predicate logic \cite{HilbertAckermann} restore the
universality of mathematical truth, allowing to determine which axiom
was used in which proof and which theorem held in which geometry.

In the same way, we defend the thesis that the interoperability of
proof systems can only be achieved if we are able to express the
theories implemented in these proof systems in a common logical
framework.

In 1928, predicate logic, the first logical framework in the history
of logic, was a huge success, since three important theories used at
that time (geometry, arithmetic and set theory) could be expressed in
it. But it also has limitations, which explains that another of the
major theories used at that time (Russell's type theory, from The
Principia Mathematica) cannot be expressed in it. Since then, several
other versions of type theory, such as Church's type theory
\cite{Church40}, Martin-L\"of's type Theory \cite{Martin-Lof84}, and
the Calculus of constructions \cite{CoquandHuet88}, have also been
defined as autonomous theories. These theories are the ones
implemented in most of the current proof systems, yet they cannot be
expressed in predicate logic.

This failure has led, in the field of proof systems, to the
abandonment predicate logic, and even the concept of logical
framework: the theories implemented in {\sc Coq}, {\sc Matita}, {\sc
  HOL Light}, etc. are often defined as autonomous systems, and not in
a logical framework.

However, a different line of research has attempted to understand the
limitations of predicate logic and to propose other logical frameworks
repairing them. The most prominent limitations of predicate logic are
the lack of function symbols binding variables, the lack of a syntax
for proof terms, the lack of a notion of computation, the lack of a
notion of cut for axiomatic theories, and the impossibility to express
constructive proofs. These limitations have led to the development of
logical frameworks such as $\lambda$-Prolog \cite{NadathurMiller88,
  MillerNadathur12}, Isabelle \cite{Paulson90}, the $\lambda
\Pi$-calculus \cite{HarperHonsellPlotkin91} (also called the
``Edinburgh logical framework''), Deduction modulo theory
\cite{DowekHardinKirchner03, DowekWerner03}, Pure Type Systems
\cite{Berardi88,Terlouw89}, and ecumenical logics
\cite{Prawitz15,Dowek15,PereiraRodriguez17}. The $\lambda
\Pi$-calculus modulo theory \cite{CousineauDowek07}, implemented in
the system {\sc Dedukti} \cite{Assaf16}, is a synthesis of these
frameworks. It not only allows the expression of geometry, arithmetic
and set theory, but also that of Russell's type theory, Church's type
theory, Martin-L\"of's type theory, and the Calculus of constructions.

In the years 2010--2015, it was shown that theories implemented in
{\sc HOL Light} \cite{Assaf12}, {\sc Matita} \cite{Assaf15}, and {\sc
  FoCaLiZe} \cite{Cauderlier16} could be expressed in {\sc Dedukti},
and that the libraries of these systems could be translated to {\sc
  Dedukti}, as well as the proofs produced with the automated theorem
proving systems {\sc iProver} \cite{Burel10} and {\sc Zenon}
\cite{CauderlierHalmagrand15}, and by the SMT solver {\sc Archsat}
\cite{Bury19}. In particular, this allowed {\sc Atelier B} proofs
produced by {\sc Zenon} to be expressed in {\sc Dedukti}.

Just like for non-Euclidean geometries, it is not sufficient to
express the theories implemented in {\sc Matita}, {\sc HOL Light},
etc.  in a common logical framework and to translate the proofs
originally defined in these systems to {\sc Dedukti}: we must also
analyze each proof to understand on which features of the theory it
relies and in which other theories it can be used, a domain
traditionally called ``reverse mathematics''
\cite{Friedman76,Simpson09,Dowek17}.

In the years 2015--2020, we started to focus on the translation of
proofs from one library to another \cite{Dowek17,Thire18}. This led us
to propose an on-line system-independent encyclopedia of formal
proofs, called {\sc Logipedia} ({\tt http://logipedia.science}), in
which each proof is labeled with the axioms and computation rules it
uses, indicating the systems in which it can be used. In particular,
we have shown that the arithmetic library of {\sc Matita} can be
translated into five other, significantly different, systems: {\sc HOL
  Light}, {\sc Isabelle}, {\sc PVS}, {\sc Coq}, and {\sc Lean},
preserving the readability of the statements of the theorems.

These successes have convinced us that now is the time to scale up and
develop this encyclopedia in the cloud, with the objective to include
a significant part of all formal proofs in four years time, and in
twenty years, all of them.

Such an infrastructure is, in many ways, new in the European Strategy
on Research Infrastructures. We can even say that the idea to
structure a research effort around an the construction and the use of
an infrastructure is relatively new in computer science and
mathematics, even if other projects, such as {\em OpenDreamKit} and
{\em Software Heritage} do exist. Our goal is therefore also to
trigger an little, but significative, evolution on the organization of
research in computer science and mathematics in Europe.


\subsection{Concept}

This project articulates six key notions: those of 
\begin{itemize}
\item logical framework,
\item theory,
\item instrumentation,
\item reverse mathematics,
\item concept alignmenent,
\item and structuring.
\end{itemize}

To make a formal proof, developed in some system $X$, accessible to
the users of other systems, the first step is to express the theory
$D[X]$, implemented in the system $X$, in the logical framework {\sc
  Dedukti}.  Then, we must instrument the system $X$ so that the proof
can be exported from it, as a piece of data, expressed as a proof in
$D[X]$ and included in {\sc Logipedia}. Next, we need to analyze this
proof in order to determine which symbols, axioms and rewrite rules of
$D[X]$ it actually uses and, thus, in which alternative theories it
can be expressed.  Finally, we must align its concepts with the
definitions already present in {\sc Logipedia} and decide where it
fits in the general structure of the encyclopedia.

Depending on the system $X$ we consider, more or less research needs
to be done, for instance to design the theory $D[X]$. Among the 15
systems we focus on, we plan to reach level 5 or 6 for 10 of them.
For the others, more research needs to be done and we plan to reach
level 3 only.

These goals require the expertise of computer scientists, logicians,
mathematicians who are experts on one theory or one system, and
specialists on logical frameworks, reverse mathematics, and concept
alignment.

\subsection{Methodology}

The methodology is the same for integrating any library to {\sc Logipedia},
but due to a difference of readiness of the various systems we focus on,
our priorities are different.

\begin{enumerate}
\item We already know how to express most of the theories of {\sc Atelier B},
{\sc Coq}, {\sc FoCaLiZe}, {\sc HOL Light}, {\sc HOL4}, {\sc
Isabelle}, {\sc Matita}, and {\sc Rodin} in {\sc Dedukti}. We propose
to instrument these systems so that they can produce {\sc Dedukti}
proofs that we can include in {\sc Logipedia}.

\item
For other theories, such as those of {\sc Abella}, {\sc Agda}, {\sc
Lean}, {\sc Minlog}, {\sc Mizar}, {\sc PVS}, {\sc TLA+}, the work is
in progress, or not yet started.  So we must first understand how they
can be expressed in {\sc Dedukti}.

\item Besides the standard libraries of these systems, large libraries
  have been developed: the {\sc Isabelle Archive of formal Proofs} \cite{AFP},
  {\sc Flyspeck}\cite{Flyspeck}, {\sc MathComp}\cite{Mathcomp}, 
  {\sc CompCert} \cite{Compcert}, {\sc CakeML} \cite{CakeML}, ...  We aim to include
  some of these libraries in {\sc Logipedia}.
  
\item
Besides proof systems, we also want to include proofs coming from
automated theorem provers, SAT solvers, SMT solvers, and model
checkers.  So we must instrument some of these systems so that they
can produce {\sc Dedukti} proofs that we can include in {\sc
Logipedia}.

\item
We want to develop algorithms to analyze which symbol, axiom, rewrite
rule is used in each proof, and consequently in which system each proof
can be used. We also want to develop algorithms to eliminate some of the 
symbols, axioms, and rewrite rules used in a proof in order to be able to 
use it in more systems.


\item
Each library imported in {\sc Logipedia} will come with its own
definition of natural numbers, real numbers, etc. We want to develop
``concept alignments algorithms'' to transport theorems from one
structure to another isomorphic one.

\item 
Besides data, we propose to include in {\sc Logipedia}, metadata and
an inner structure.
\end{enumerate}


\subsection{Readiness of the project}

This idea of building such a standard for proofs has already been
investigated in the past, such as in the Qed manifesto \cite{Qed94}, but
has produced limited results.

Our thesis is that, since the
Qed project, the situation has radically changed. After
thirty years of research, we have an empirical evidence that most of
the formal proofs developed in one of these systems can also be
developed in another. We understand the relationship between the
theories implemented in these systems much better. We have developed
several logical frameworks, extending predicate logic, in which these
theories can be expressed. And we have developed reverse mathematics
algorithms to analyze which axioms and rules are used in each proofs
and algorithms, such as constructivization algorithms, to translate
proofs from one theory to another.


%%% Local Variables:
%%%   mode: latex
%%%   ispell-local-dictionary: "english"
%%% TeX-master: "propB"
%%% End:
