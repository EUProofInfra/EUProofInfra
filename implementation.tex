\chapter{Implementation}\label{chap:implementation}

\section{Work plan --- Work packages, deliverables}

\subsection{Overall structure of the work plan}

Our work plan is divided into seven scientific work packages.
The first group of work packages is dedicated to the networking
activities that are needed to gather the proofs today located in
different libraries.
\begin{longtable}{|p{0.02\textwidth}|p{0.2\textwidth}|p{0.7\textwidth}|}
\hline
1
&
Integration
&
Instrument the systems for which we already know how to encode the
proofs in Dedukti, and make available these proofs in Logipedia.
\\
\hline
2
&
Automatic theorem proving
& 
Develop automatic theorem provers to populate,
help, and benefit from Logipedia.
\\
\hline
3
&
Large libraries
&
Export large dedicated libraries in curated form 
to Logipedia for end-user applications.
\\
\hline
\end{longtable}
The second is dedicated to making these proofs accessible, beyond
trans-national and virtual access.
\begin{longtable}{|p{0.02\textwidth}|p{0.2\textwidth}|p{0.7\textwidth}|}
\hline
4
&
Access
&
Define and build the Logipedia hardware and software infrastructure in
which the proofs will be integrated.
\\
\hline
5
&
Structure of the encyclopedia
&
Provide infrastructure for the structured ontological representation
of libraries and use it to enrich the information about formal
libraries in Logipedia.
\\
\hline
\end{longtable}
The third is dedicated to joint research activities that prepare
the future of Logipedia. 
\begin{longtable}{|p{0.02\textwidth}|p{0.2\textwidth}|p{0.7\textwidth}|}
\hline
6
&
Theories
&
Bringing proof systems implementing a theory 
that has not yet been expressed in Dedukti LIL 2 or better.
\\
\hline
7&Proof engineering &
Investigate methods for detecting concept alignments and apply
them to build a library of alignments present across the Logipedia database.
\\
\hline
\end{longtable}
Together with these seven scientific work packages, 
two more work packages are dedicated to dissemination, communication and
exploitation and to madagement.
\begin{longtable}{|p{0.02\textwidth}|p{0.2\textwidth}|p{0.7\textwidth}|}
\hline
8
&
Dissemination, communication, and exploitation
&
Expand the use of Logipedia in research, industry, education, and publishing.
\\
\hline
9
&
Management
&
Coordinate this large community, in a benevolent atmosphere, for optimal
efficiency.
\\
\hline
\end{longtable}

\subsection{Timing of the different work packages and their components}

Gant diagaram : a line for each task
and a column for each month.

Finishes with a miletones, deliverable and project meeting.

Put a lot of deliverables at Month 18 and 36 and 48

\ganttchart[draft,xscale=.45] 

\begin{tabular}{|c|c|c|c|c|c|c|c|c|c|c|c|c|c|c|c|c|c|c|c|c|c|c|c|c|c|c|c|c|c|c|c|c|c|c|c|c|c|c|c|c|c|c|c|c|c|c|c|c|}
\hline
& 1 & 2 & 3 & 4 & 5 & 6 & 7 & 8 & 9 & 10 & 11 & 12 & 13 & 14 & 15 & 16 & 17 & 18 & 19 & 20 & 21 & 22 & 23 & 24 & 25 & 26 & 27 & 28 & 29 & 30 & 31 & 32 & 33 & 34 & 35 & 36 & 37 & 38 & 39 & 40 & 41 & 42 & 43 & 44 & 45 & 46 & 47 & 48\\
\hline
\textsl{aaa} \\
\hline
\end{tabular}

\subsection{Detailed work description}

\wpfigstyle{\footnotesize}
\wpfig[pages,type,start,end]

\begin{workplan}
\newpage
\begin{workpackage}[id=instrumentation,type=RTD,
  short=Integration,% for Figure 5.
  title=Integration,
  lead=Del,
  DelRM=14,
  GotRM=4,
  TumRM=5,
  ChaRM=20,
  CleRM=0,  % TODO
  ImtRM=0,  % TODO
  TouRM=0,  % TODO
  BolRM=16, % 12 on Coq, 4 on Matita
  InrRM=0]  % TODO

%\ednote{MK: We need one coordinating site. original coordinators: Frédéric Blanqui and
%  Jesper Cockx}
%\ednote{MK: interested parties (add their sites and RM here): David Deharbe,
%Tobias Nipkow, Guillaume Genestier, Jesper Cockx, Guillaume Burel, Filip Marić, Makarius
%Wenzel, Nicolas Magaud, Gaspard Férey, Ulf Norell, Claudio Sacerdoti Coen}

\begin{wpobjectives}
  The objective of this work package is to bring systems already at least at
  LIL 1 to higher LIL levels, that is, to instrument the systems
  for which we already know how to encode the proofs in Dedukti, and
  make available these proofs in Logipedia so that
  they can be exported to other systems.
\end{wpobjectives}

\begin{wpdescription}
  Concretely, this work package includes the following tasks:
\end{wpdescription}

\begin{tasklist}
\begin{task}[id=agda,
  title=Instrument Agda,
  lead=Del,
  DelRM=14,
  GotRM=4]
%[G\"oteborg, Delft]

%\textbf{Budget requirements:} One research engineer at Chalmers, and one PhD student or postdoc at TU Delft.

%Question: does this task belong to WP1 or WP2?

Agda is a popular dependently typed programming language / proof
assistant based on Martin-L\"of’s intuitionistic type theory. Its theory
is similar to Coq and Lean, but is more focused on interactive
development and direct manipulation of proof terms (in contrast to
using a tactic language to generate the proof terms). Agda has a
sizable standard library (available at
https://github.com/agda/agda-stdlib) that consists of both utilities
for programming and mathematical proofs.


In the summer of 2019, Guillaume Genestier worked together with Jesper
Cockx on the implementation of an experimental translator from Agda to
Dedukti during a research visit at Chalmers University in Sweden. This
translator is still work in progress, but it is already able to
translate 142 modules of the Agda standard library to a form that can
be checked in Dedukti. This exploratory work uncovered several
challenges and opportunities for further work, which are outlined
below.

\begin{enumerate}
\item To support the construction of proof terms, Agda provides powerful
features such as dependent pattern and copattern matching, eta
equality for functions and record types, and definitional proof
irrelevance. The first one – dependent pattern matching – can be
translated directly to rewrite rules in Dedukti. However, the two
latter features – eta equality and irrelevance – rely on Agda’s
type-directed conversion algorithm, while Dedukti’s conversion is
untyped. Hence in order to translate Agda proofs to Dedukti these
features need to be encoded.

One particular concern with the encoding of eta-equality is that in
general it requires storing of additional type information in the
proof terms. It can hence lead to a large blow-up in the size of those
proof terms, and thus greatly increase the cost of typechecking. The
same problem also occurs in other parts of Agda; for example
constructors of parametrized datatypes do not store the values of the
parameters, but they need to be reconstructed in the translation to
Dedukti. We plan to investigate two possible approaches to this
problem: either we can try to find a better encoding which reduces the
size of the type annotation, or alternatively we can extend the
Dedukti language with type-directed conversion rules to render the
type annotations unneccessary.

\item Another unique feature of Agda is the support for first-class
universe level polymorphism. In particular, Agda has a built-in type
of levels that has complex structure of (in)equality between
levels. Compared to universe polymorphism in Coq, an additional
challenge is that levels in Agda can contain arbitrary terms as
subexpressions. Our plan is to define a sound and complete embedding
of Agda’s level type in Dedukti, based on the existing work on
encoding AC (associative-commutative) theories. This would both serve
as a stress test of how well Dedukti can handle complex equational
theories, and improve our understanding of type theories with
first-class universe level polymorphism, which would be useful for the
implementation of Agda.

\item In contrast to Coq and Lean, Agda does not have a well-defined
core language to which proofs are elaborated. Instead, definitions are
translated to an internal representation that is relatively close to
the user input. This provides a challenge when translating Agda proofs
to Dedukti: each feature in Agda’s internal syntax needs to have its
own translation. As part of this project, we will hence investigate
possible designs for a core language for Agda. Having such a core
language would have several benefits: it would deepen our
understanding of the Agda language, it would increase the
trustworthiness of Agda proofs, and it would make it much easier to
export Agda terms to other languages (such as Dedukti in the context
of this project).

\item Agda provides an experimental option for extending the language
with user-defined rewrite rules, which are very similar to the rewrite
rules provided by Dedukti. Because of this similarity, we expect it to
be straightforward to translate rewrite rules from Agda to
Dedukti. However, by comparing the two implementations we hope to gain
new insights and find opportunities for improvement on both sides. The
interest of some of these features goes beyond just the Agda
language. In particular, Lean also supports definitional proof
irrelevance, as does Coq with the recent addition of the SProp
universe. Hence we plan to collaborate with the teams working on those
languages to improve the support for these features where there is
overlap.
\end{enumerate}

%%% Local Variables:
%%% mode: latex
%%% TeX-master: "../propB"
%%% End:

\end{task}

\begin{task}[id=isabelle,
  title=Instrument Isabelle,
  lead=Tum,
  TumRM=5]
% task leader: Tobias

% participants: Makarius (Isabelle), David Matthews (Poly/ML)

% Moved to concept & methodology
% Isabelle as a logical framework \cite{paulson700} is an intermediate
% between Type-Theory provers (like Coq or Agda) and classic LCF-style
% systems (like HOL Light or HOL4). The inference kernel can already
% output proofs as $\lambda$-terms on request, but this has so far been
% only used for small examples \cite{Berghofer-Nipkow:2000:TPHOL}. The
% challenge is to make Isabelle proof terms work robustly for the basic
% libraries and reasonably big applications.  Preliminary work by Wenzel
% (2019) has demonstrated the feasibility for relatively small parts of
% Isabelle/HOL, but this requires scaling up.

\begin{enumerate}
  \item Improve the efficiency of important aspects of the
  Isabelle/HOL logic implementation, such as normalization of proofs,
  type-class reasoning, and special representation of derived rules
  and definition principles.
  \item Reduce the volume of proof terms in the Dedukti encoding.
  \item Improve memory usage of the Isabelle/ML implementation
  platform.
\end{enumerate}

\end{task}

\begin{task}[id=HOL4,
  title=Instrument HOL4,
  lead=Cha,
  ChaRM=20]
[G\"oteborg]

The HOL4 proof assistant is home to a few medium to large scale
specifications and associated proof developments that have value
outside of HOL4. These specifications include the formal semantics of
the CakeML language (and its verified compiler) and an extensive
specification of the ARM instruction set architecture (ISA) as
formalised by Anthony Fox at the University of Cambridge.

HOL4 has support for exporting proofs to the OpenTheory proof exchange
format, and there has been some work on importing OpenTheory proofs
into Dedukti. However, the current state of these techniques and their
implementations does not scale to real examples such as those
mentioned above.

This part of the project will be about re-thinking and re-designing
the tools HOL4-to-OpenTheory and OpenTheory-to-Dedukti tools such that
they scale to the point where real examples of interest, such as those
mentioned above, can be exported.

2 Person Years at Chalmers


\end{task}

\begin{task}[id=atelier-b,
  title=Instrument Atelier-B/Rodin,
  lead=Cle,
  CleRM=0, % TODO
  ImtRM=0, % TODO
  TouRM=0] % TODO
% task leader: Catherine

% other participants:

%\ednote{Southhampton, Toulouse, Clearsy writes this}

% Moved to concept & methodology
% Atelier B, Rodin and ProB are platforms or tools to develop models
% written in B method, Event-B or B system. The development process is
% based on formal proof: proof obligations are automatically  generated
% and must be proven by automatic or interactive provers. ProB is an
% animator and model checker, it helps users to gain confidence in their
% specifications. It is also a disprover aiming at  discovering
% counter-examples for proof obligations. Atelier B and Rodin use native
% B proof tool, they also enable the use of external provers such as SMT
% solvers. ProB calls SMT and SAT solvers, it also uses contraints
% solvers such as Sicstus Prolog. All of them relies on the B logics,
% mainly a first order language with set theory. Regarding B/Event-B/B
% system, there are some variants, mainly regarding the refinement
% process they all implement. Refinement means that models are developed
% by successive steps, from an abstract model to a more  concrete
% model. Refinement in B method mainly means deriving a program while
% EventB and B System refinement aim at defining a model of a system by
% introducing details.


% Moved to concept & methodology
% In the context of the BWare project, an encoding of the set theory of
% the B method has been provided as a theory modulo, i.e. a rewrite
% system rather than a set of axioms. This encoding is used by the
% automatic prover Zenon modulo which features a backend to
% Dedukti. Thus, as a first step through instrumentation of Atelier B
% and Rodin, proof obligations coming from Atelier B can be proved by
% Zenon modulo producing Dedukti proofs, hence providing a better
% confidence in the proofs produced by the native proof tools of Atelier
% B \cite{Bware}.

\begin{enumerate}

  \item Continue the encoding of the B set theory in Dedukti to be
  able to handle all kind of proof obligations and rules.

 %\item Instrument the native provers to produce proofs (in WP4 ?).

  \item Instrument and Exporting B models to Dedukti.

  \item Importing logipedia lemmas in B models

\end{enumerate}

\end{task}

\begin{task}[id=matita,
  title=Integrate the Matita translator in Matita itself,
  lead=Bol,
  BolRM=4]
%[Bologna]

%\textbf{Budget requirements:} One one PhD student or postdoc at UBo.

Matita is an interactive theorem prover developed at the University of Bologna and used for teaching logic courses and to verify software and mathematical proofs, with special attention to predicative foundations. The first generation of the system (up to version 0.5.9) was born as a by-product of the MoWGLI FET-Open Project, it was compatible with the logic of Coq and it could re-use its libraries. It was an important test-bench for the integration of Mathematical Knowledge Management techniques with Interactive Theorem Proving, featuring for example a library of theorems distributed over multiple servers, innovative indexing and search techniques and automatic translation of proofs between declarative and procedural styles. The second generation of the system (up to the current version 0.99.3) was a re-implementation from scratch that departed from the logic of Coq and that experimented with the most concise ways to implement an efficient theorem prover. Several ideas later migrated into Coq. The currently available largest library is the formal certification of a complexity-preserving and cost-model-inducing compiler from C to MCS-51 machine code, developed in the FET project CerCo (Certified Complexity).

The standard and arithmetic libraries of Matita has been the first libraries to be exported to Logipedia using Krajono, a fork of Matita. The forked system is also actually the only one able to import Logipedia proofs. The choice of Matita as a test-bench for Logipedia is easily understood considering that the implementation of the 0.99.x series was aimed at obtaining a well-documented, minimal but fast implementation of a theorem prover, two order of magnitudes smaller than Coq.

The task will achieve the following results
\begin{enumerate}
\item Merge Krajono and Matita, update the code to the latest version and transfer the maintenance effort to the Matita team.
\item Export all the remaining Matita libraries. In particular:
\begin{itemize}
 \item The libraries developed in CerCo contain several gigantic proof terms (nested proofs by cases on the 256 opcodes of the MCS-51 processor) that will stress the encoding and the tools developed around the Logipedia library.
 \item The proofs in the arithmetic libraries of Matita, now converted to HOL proofs inside Logipedia, do not exploit dependent types. Other libraries rely heavily on dependent types, triggering more interesting translations between theories encoded in Logipedia.
\end{itemize}
\item The logics of Matita and Coq remain quite similar, sharing a common core. However no complete automatic translation from Coq to Matita or vice versa is possible any more and only partial translations with high coverage are known, but not implemented, due to the intricacies of having to make the two code bases interact. We will study how to implement the partial translations directly in Logipedia, without knowledge of the internals of the two systems, and we will rely on automatically generated alignments to augment coverage of the translation.\ednote{CSC: this point probably does not belong to this WP}
\end{enumerate}

%%% Local Variables:
%%% mode: latex
%%% TeX-master: "../propB"
%%% End:

\end{task}

\begin{task}[id=coq,
  title=Instrument Coq,
  lead=Bol,
  BolRM=12,
  InrRM=0]  %TODO
% task leader: Claudio Sacerdoti Coen
% participants: Enrico Tassi, Claudio Sacerdoti Coen
%\textbf{Budget requirements:} One one PhD student or postdoc at UBo.

\ednote{Bol: 12 MM = 44,630 euros (39,372 euros salary + travels etc.); Inr: 12 MM}

% Moved to concept & methodology
% Coq is an interactive theorem prover developed at Inria since the 1984.
% It is based on Type Theory and was used to formally verify the correctness
% of both industrially relevant software such as the CompCert C compiler and
% complex mathematical proofs such as the one of the Four Color theorem and the
% one of the Odd Order theorem. In 2013 Coq received the ACM system award.

This task about instrumenting Coq includes the following steps:
\begin{enumerate}
% deliverable 1
\item Access Coq internal data structures to gather logical data, such as
statements and proof terms (for tasks in WP3)
% deliverable 1
\item Implement the translation of Coq terms to Dedukti terms
% deliverable 2
\item Access Coq internal data structures to gather extra-logical data,
such as the role played by a constant in the library like begin an implicit
cast from one algebraic structure to another (for tasks in WP3)
% deliverable 2
\item Make extra-logical data available in a structured and
  extensible format (for tasks in WP5)
\end{enumerate}

% Moved to concept & methodology
%A technological hurdle steps (a) and (c) have to overcome is that Coq is an actively
%developed system that is constantly evolving. Previous attempts at extracting
%data from Coq without a direct interaction with the developers of Coq resulted
%in prototypes like CoqinE that quickly became outdated.
To overcome the problems associated with external tools such as
CoqInE, we plan to have the required instrumentation merged in Coq
proper and reuse it for other projects that could benefit from it so
to amortize its development cost. More precisely E. Tassi is a core
Coq developer acquainted with its development process and he will take
care of the integration of the infrastructure in Coq and foster its
reuse in third party projects with similar needs such as SerAPI,
CoqHammer and Coq-Elpi.

% Moved to concept & methodology
% Step (b) is also problematic for two reasons. The first one is that the encoding
% in Dedukti requires some information, typically types of sub-expressions, that
% are not stored in Coq, it is transient. So the instrumentation for steps (a) and
% (c) needs to be complemented by providing not only access to existing data but
% also to log transient data or re-synthesize it on demand. Both approaches may
% be used, depending on the the tradeoff between computation time and space for
% storage. The second reason, which is more critical, is ...

\ednote{Should this part also be moved to the methodology section?}
Since the type theory of Coq is extremely large, with features that
have no corresponding representation in Dedukti, the type theory of
Coq needs to be translated to a core one representable in Dedukti.
This translation to a core calculus is not implemented in Coq and the
amount and complexity of code necessary for it is very significant and
indeed the CoqinE prototype only covers a small subset of
Coq. Feedback from WP2 will be of guidance\ednote{Tassi: there is no
task in WP2 about Coq's TT, maybe it is included in HoTT?} to extend
the translation currently available CoqinE to cover a larger subset of
Coq.

In step (d) we plan to take advantage of the work done by Sacerdoti Coen (UBo)
in 2019 in exporting non trivial logical and extra-logical data from Coq to
an XML format. Data in that format was then translated by Kohlhase et al
(UBo + FAU) to the MMT system, another logical framework to encode different
logics and their libraries. We plan to extend that format to include even
more extra-logical data as well as the data gathered in step (a) and (c). We shall
evaluate if all the data needed for step (b) can be saved in this format, and
give us the freedom to implement step (b) in a standalone tool making no
requests to Coq in order to re-synthesize missing data.

%%% Local Variables:
%%% mode: latex
%%% TeX-master: "../propB"
%%% End:

\end{task}
\end{tasklist}

\begin{wpdelivs}
  \begin{wpdeliv}[due=3,miles=startup,id=requirements,dissem=PU,nature=DEM,lead=Inr]
      {Requirements Analysis and Synchronization}
  \end{wpdeliv}
  \begin{wpdeliv}[due=12,miles=logipedia-v1,id=isabelle1,dissem=PU,nature=DEM,lead=Tum]
      {Robust export of proof terms for Isabelle}
  \end{wpdeliv}
  \begin{wpdeliv}[due=12,miles=logipedia-v1,id=isabelle1,dissem=PU,nature=DEM,lead=Tum]
      {Improved memory management and monitoring for Poly/ML}
  \end{wpdeliv}
  \begin{wpdeliv}[due=18,miles=agda-stdlib,id=agda,dissem=PU,nature=DEM,lead=Del]
      {Export Agda's standard library to Dedukti}
  \end{wpdeliv}
  \begin{wpdeliv}[due=8,miles=logipedia-v1,id=coq1,dissem=PU,nature=DEM,lead=Inr]
    {Export of proof terms from Coq, no meta data}
  \end{wpdeliv}
  \begin{wpdeliv}[due=24,miles=logipedia-v2,id=coq2,dissem=PU,nature=DEM,lead=Bol]
    {More scalable export of proof terms and meta data from Coq}
  \end{wpdeliv}
  \begin{wpdeliv}[due=12,miles=logipedia-v1,id=matita1,dissem=PU,nature=DEM,lead=Bol]
    {Export of proof terms and meta data from Matita}
  \end{wpdeliv}
\end{wpdelivs}
\end{workpackage}

%%% Local Variables:
%%% mode: latex
%%% TeX-master: "../propB"
%%% End:

\newpage  
\begin{workpackage}[id=atpetc,wphases=0-48,type=RTD,
  short=ATPs etc.,% for Figure 5.
  title={ATP, SAT, SMT, Model checkers},
  lead=ULi,
  ULiRM=10]
  
\ednote{MK: We need one coordinating site. original coordinators: Pascal Fontaine and Chantal Keller}
\ednote{MK: interested parties (add their sites and RM here): David Deharbe,
Cezary Kaliszyk, Pascal Fontaine, Dale Miller, Stephan Merz, Josef Urban, Martin Suda,
Guillaume Burel, Filip Marić, Chantal Keller, Julien Narboux, Thibault Gauthier}

\begin{wpobjectives}
  The objective of this work package is to \ldots

This includes notably:
  \begin{compactitem}
  \item \ldots
  \end{compactitem}
  A key aspect will be to foster \ldots
\end{wpobjectives}


\begin{wpdescription}

The importance of proofs in automated theorem provers, satisfiability
modulo theories solvers, propositional satisfiability solvers and
model checkers is increasingly recognized.  While for the
propositional case, the community agrees on a well defined proof
format, the situation is not clear for the other kind of automated
reasoners.  There is no clear format for SMT, and the TSTP format for
automated theorem provers fixes a syntactic template for proofs rather
than providing an unambiguous framework to express proofs
semantically.

Some preliminary works predating this proposal clearly establish that
Dedukti can accommodate proofs in Satisfiability Modulo Theories,
automated theorem provers, and SMT.  In this work package, we will
build on those preliminary work and provide a set of conduits from the
established formats used in automated tools. For the tools that do not
have yet an established format, we will make a selection of tools
(Zipperposition and E for automated theorem provers, CVC4 and veriT
for SMT, ??? for model checking) and provide a conduits for those
tools.  These conduits and the techniques used in the embedded
translation will be properly documented, to ease integration of
further tools of the kind.  If a standardized proof format appears for
some kind of tools, the conduits will be updated to adopt the new
standard.

In this work package, we also plan to integrate in Logipedia some
well-chosen proofs coming from automated tools.  Well-chosen proofs
will have to be representative of typical applications of the tools,
and be of reasonable size.  They will serve as examples to the
community, to illustrate the potentials of Dedukti and Logipedia.


Create the infrastructure to enable the long term goal: be able to split a large proof
obligation into smaller parts and distribute to the appropriate automatic engines, that
would all produce proofs, glued together in a single large proof for the original proof
obligation.
\ednote{Nancy, Liège}
\end{wpdescription}

\begin{tasklist}
  \begin{task}[id=tools,title=Automatic Tools Exporting Proofs]
  \end{task}

  \begin{task}[id=challenges,title=Logipedia as a Source of Challenges for Automatic Reasoners]
    --> Translation to TPTP, SMT-LIB, DIMACS
  \end{task}
  \begin{task}[id=commang,title=A language for Communication between Automatic Reasoners]
  \end{task}
\end{tasklist}

\begin{wpdelivs}
  \begin{wpdeliv}[due=3,miles=startup,id=requirements,dissem=PU,nature=DEM,lead=ISa]
      {Requirements Analysis and Synchronization}
\end{wpdeliv}
\end{wpdelivs}
\end{workpackage}

%%% Local Variables:
%%% mode: latex
%%% TeX-master: "../propB"
%%% End:

\newpage  
\begin{workpackage}[id=libraries,type=RTD,wphases=1-48,
  short={Large libraries},% for Figure 5.
  title={Large libraries},
  lead=Tum,
  StrRM=18,
  ChaRM=12,
  TumRM=27]
%TUM: 3 for AFP/Makarius (25k EUR) - the latter do not generate overheads!

\begin{wpobjectives}
The objective of this WP is to export large dedicated libraries in
curated form to Dedukti and thus to Logipedia for end-user applications.
The focus is \emph{Access} and \emph{Scalability}.
\begin{compactitem}
\item This WP is responsible for supplying the lion's share of proofs in
Logipedia.  As a result it will be a stress test for the results of \WPref{instrumentation}.

\item The target libraries are dedicated to particular application
areas. They provide a substantial coverage of that application area
and do so in a structured manner. This may require reworking the
libraries for better access.

\item The libraries are curated for end-user application. That is, they
are structured according to application specific ontologies that
support browsing and search. The structuring leverages the
infrastructures of \WPref{structuring} and will be a
stress test for the results of that WP.
\end{compactitem}
\end{wpobjectives}


\begin{wpdescription}
Translating the standard libraries of the systems is part of the \WPref{instrumentation}.
This WP focusses on advanced libraries selected according to the following criteria:
relevance, coverage and maturity.
As a result we selected the following libraries: MathComp, Coq's revised
Analysis library, the Archive of Formal Proofs, Isabelle's revised Analysis and Probability library,
GeoCoq, Flyspeck and CakeML. In the future we plan to incorporate
CompCert, seL4, and selected Mizar and PVS libraries (once Mizar and
PVS have reached LIL 2).
\end{wpdescription}


\begin{tasklist}
%\begin{task}[id=mathcomp,title=MathComp]
%\ednote{Sophia, Saclay (Gonthier), Paris}
%\end{task}

%\begin{task}[id=milc,title=Revised Coq Analysis Library]
%\ednote{Saclay (Boldo), Paris, Sophia}
%\end{task}

%\begin{task}[id=mizar,title=The Mizar library]
%\ednote{Innsbruck, Bialystok}
%\end{task}

\begin{task}[
  id=afp,
  title=Isabelle's Archive of Formal Proofs,
  lead=Tum,
  TumRM=3,
  wphases=13-18]
%\ednote{Wenzel}
Isabelle's Archive of Formal Proofs (AFP) \cite{isabelle-afp} is a
growing user-contributed online library for Isabelle. In Feb-2020, the
AFP consisted of more than 500 entries (articles of formalized
mathematics) by 340 authors, and required approx.\ 60h CPU time for
checking (using many gigabytes of memory).  The purpose of this task
is to scale up the Isabelle instrumentation for Dedukti further, to
cover major parts of this library. The ultimate aim is to export the main
substance of the AFP without promising full coverage: some entries
with prohibitive resource requirements will be omitted.
\end{task}

\begin{task}[
  id=isaAnalysisProb,
  title=The Isabelle Analysis \& Probability Theory library,
  lead=Tum,
  TumRM=24,
  wphases=1-24]
%
  This library consists of more than 200.000 lines
  of definitions and proofs, corresponding to almost 4000 printed
  pages. It is fair to say that it is the most advanced
  machine-checked library in the area of analysis and probability
  theory. Because analysis and probability theory are key to many
  applications in enginnering and science, this library will be a key
  exploitable result of the project: it is a fundamental enabling
  resource for almost any formal verification activity in these
  application areas. The purpose of this task is to structure,
  document and develop this library for optimal accessibility, ease of
  use and comprehensiveness.

For better access, the library needs to be modularized, which requires
a significant refactoring effort.  At the same time we need to add
metadata (as provided by \WPref{structuring}) to the source material
to turn this structured collection of theorems and proofs into a
curated library at the Logipedia level.

The following areas of the library need to be developed further. The
library support for integrals is extensive but suffers from the
coexistence of different kinds of integrals. This requires unification
and refactoring. Further essential material for mathematics, physics
and engineering needs to be added: Fourier
analysis and esp.\ the Fourier transform; stability theory for
differential equations and dynamical systems, in particular Lyapunov
functions; stochastic differential equations.
\end{task}

\begin{task}[
  id=geocoq,
  title=The GeoCoq library,
  lead=Str,
  StrRM=18,
  wphases=1-18]
%
The GeoCoq library consists of more than 100.000 lines of definitions and proofs. It is mostly based on synthetic approaches, where the axiom system is based on some geometric objects and axioms about them, but, following Descartes and Tarski, the analytic approach can be derived, where a field F is assumed (usually R) and the space is defined as $F^n$. Moreover, it contains a model of Tarski's axioms, based on the analytic approach, thus establishing the connection between these two approaches in the opposite direction. The main axiom system in this library is the one of Tarski, but Hilbert's axiom system and a version of Euclid's axioms sufficient to prove the propositions in Book 1 of Euclid's Elements are also defined. In the library, the focus is not only on axiom systems but also on axioms themselves. Eleven continuity axioms are available and are hierarchically organised. Finally, it contains a new refinement of Pejas’ classification of parallel postulates together with proofs of the classification of 34 versions of the parallel postulate.

One of the remaining obstacles is the frequent use of computational steps in Coq proofs. The issue is that proofs containing "proof by reflection" reach a level of complexity that makes verification by Dedukti impractical. An approach is to isolate these proofs by reflection so that they are not perceived as simple conversion steps in the type theory proofs, but marked as proofs to be treated by an automatic tool. Another challenge is that CoqInE, a tool developed to translate Coq proofs into Dedukti type-checkable terms, produces terms in a expressing of the Calculus of Inductive Constructions in Dedukti. Currently, it is not possible to export these Dedukti terms to other proof assistant. However, another tool, Universo, has been developed and paves the way for the export of these terms.
\end{task}

\begin{task}[
  id=flyspeck,
  title=The Flyspeck library,
  lead=Inr,
  wphases=1-36]
%\ednote{Saclay (Grienenberger)}
The {HOL Light} library is large and varied. One of its key libraries is the
multivariate analysis library
%\footnote{\url{https://github.com/jrh13/hol-light/tree/master/Multivariate}},
which spans the fields of metric spaces, topology, homology, linear algebra,
convexity, real and complex analysis and transcendentals, derivatives, and
integration. The {Flyspeck} project gives a formal proof of the {Kepler}
conjecture, based on an original proof of Thomas {Hales}
\cite{DBLP:journals/corr/HalesABDHHKMMNNNOPRSTTTUVZ15}, and formalized
largely in {HOL Light} \url{https://github.com/flyspeck/flyspeck}.
Some of these results are not formalized in any other system, motivating the
project of importing the {HOL Light} library and {Flyspeck} project in the
{Dedukti} system, in view of its integration into {Logipedia}.

\textbf{Challenges:}
Proofs coming from the HOL systems, including {HOL Light}, are known to be very
large, adding to the issue of the scalability of exporting software for large
libraries \cite{DBLP:conf/tphol/Wong95,DBLP:conf/cade/ObuaS06,
DBLP:conf/itp/KellerW10,DBLP:conf/cade/Kumar13}. Scalable export techniques
{HOL Light} proofs have been investigated \cite{KaliszykK13} and can provide a
solid base to this project.

The main milestones of this task are the further automation of the export from
{HOL Light} to {Dedukti}, the import of the multivariate analysis library, of
the whole {HOL Light} library, and of the {Flyspeck} in {Dedukti}.
\end{task}

\begin{task}[
  id=cakeml,
  title=The CakeML compiler library,
  lead=Cha,
  ChaRM=12,
  wphases=12-23]
%
The CakeML
compiler \cite{KumarMNO14} (verified with the HOL4 prover) is one of only two verified compilers for real
languages, the other being CompCert. Its export to Dedukti is one of
the KERs of this project.
% Because of the size and importance of this
%library, we will approach the export from two angles.
%
%The first approach utilizes OpenTheory-based technology.
HOL4 can export proofs in the OpenTheory format, which can in turn be
translated into Dedukti. Currently this link from HOL4 via OpenTheory
to Dedukti does not scale to something as sizeable as the CakeML
compiler proof. This part of this task will rework the route via
OpenTheory to scale better, possibly taking inspiration from an
OpenTheory-like approach that scaled well for the HOL light
prover~\cite{KaliszykK13}.

%The second approach establishes a connection from HOL4 via Isabelle to
%Dedukti. The basis is a promising new approach of virtualizing HOL4
%inside Isabelle \cite{ImmlerRW19}. That is, the inference kernel of
%HOL4 is replaced by that of Isabelle and the resulting system produces
%Isabelle theorems instead of HOL4 theorems. As a benchmark of the
%viability of this approach we plan to export CakeML via Isabelle to
%Dedukti.
\end{task}

%\begin{task}[id=unimath,title=The UniMath library]
%\ednote{Birmingham (Ahrens)}
%\end{task}

%\begin{task}[id=pvs,title=The NASA PVS library]
%\end{task}
%\begin{task}[id=sel4,title=The seL4 library]
%\end{task}

%\begin{task}[id=compcert,title=The CompCert library]
%\end{task}
\end{tasklist}

\begin{wpdelivs}
  \begin{wpdeliv}[due=3,id=requirements,dissem=PU,nature=DEM,lead=Inr]
      {Requirements Analysis and Synchronization}
  \end{wpdeliv}
  \begin{wpdeliv}[due=36,id=requirements,dissem=PU,nature=DEM,lead=Tum]
      {Scalable export of proof terms for major parts of Isabelle/AFP}
  \end{wpdeliv}
  \begin{wpdeliv}[due=36,id=requirements,dissem=PU,nature=DEM,lead=Tum]
      {Export of Isabelle's extended analysis and probability theory library}
  \end{wpdeliv}
  \begin{wpdeliv}[due=18,id=geocoq-import,dissem=PU,nature=OTH,lead=Str]
      {Export of most of GeoCoq library}
  \end{wpdeliv}
  \begin{wpdeliv}[due=26,id=requirements,dissem=PU,nature=DEM,lead=Cha]
      {Export of CakeML library}
  \end{wpdeliv}
\end{wpdelivs}
\end{workpackage}

%%% Local Variables:
%%% mode: latex
%%% TeX-master: "../propB"
%%% End:

\newpage  
\begin{workpackage}[id=access,wphases=0-48,type=MGT,
  short=Access,% for Figure 5.
  title={Access to the infrastructure},
  lead=Inr,
  InrRM=28,
  OcaRM=6]

\begin{wpobjectives}
  The objective of this work package is to \ldots

This includes notably:
  \begin{compactitem}
  \item \ldots
  \end{compactitem}
  A key aspect will be to foster \ldots
\end{wpobjectives}

\begin{wpdescription}

\end{wpdescription}

\begin{tasklist}

  \begin{task}[id=basic,title=Defining the architecture of the infrastructure]
    We will define the architecture of the infrastructure and install
    it on some server at Inria. The server will be duplicated in
    Münich for security. The architecture includes how the proof files
    for the different proof systems will be organized and stored, how
    they will be generated, etc.
  \end{task}

  \begin{task}[id=web,title=Giving access to the infrastructure on the world-wide web]
    We will develop a web interface to access the infrastructure,
    navigate into the available proofs and downaload them.
  \end{task}

  \begin{task}[id=opam,title=Giving access to the infrastructure in proof tools]
    Users need to have an easy access to the proofs in logipedia, to integrate/use
    them in their ongoing work; this access should be guaranteed universal, without
    lock-in, web standards-compliant, through an open source tool. While WP7
    will give a structure to the proof database, and T2 of this WP will give access
    to that structured database through web browsing, this task aims at
    providing a proof manager for users of Logipedia. This proof manager
    will enable users to automatically download and install proofs as well as their
    dependencies in order to ease the integration of proofs from logipedia in
    developments.

    opam \cite{opam} is an open-source source-based package manager, which has
    been successfully used by the OCaml community since 2012, where it manages
    2585 versioned packages for a total of 13196 combinations of package and
    version, guaranteeing its ability to connect people across large communities.
    Furthermore, opam is meant to provide management capabilities not only to
    OCaml, but to any language, which is why it is already used as a proof
    manager by the Coq community where it has been proven to be reliable and
    suited to managing formal proofs. This makes it a prime candidate to be the
    proof manager for logipedia.

    This task would thus use the opam management tool to develop a repository
    containing all the proofs in logipedia, allowing users across Europe to
    automatically and transparently download and install proofs and their
    dependencies via opam. This would primarily entail the creation of a new
    tool able to read the proof database of logipedia and create a corresponding
    opam repository, as well as the necessary work to automate this work so that
    it can run automatically on the infrastructure built in T1.

  \end{task}

  % Search task, importing some content that was previously in WP7
  \begin{task}[id=search,title=Providing search
    tools,lead=Inr,InrRM=28,FauRM=24,SacRM=6,BolRM=4]
    % task leader: Pierre Senellart, Inria
    We will provide users with search tools enabling them to perform
    queries on Logipedia in order to find specific theorems or proofs.
    First, users will be able to search libraries theorems by their
    names and other metadata (see task~\taskref{structuring}{strdofimpl}), including complex semantic
    queries expressed in the SPARQL language for semantic annotations
    produced in task~\taskref{structuring}{strrefonto}. Second, it will be possible to
    search theorems and proofs based on their structure and mathematical
    content (types, operators, used axioms and rules, etc.), using exact
    matching, regular expressions over fomulas, and deeper content
    matching, such as the one done in the
    \hyperlink{https://kwarc.info/systems/mws/}{MathWebSearch} system. Users can
    use this both to find a specific theorem that
    could be useful in their current development and to analyze the
    proofs themselves, e.g., to find all proofs using a given set of
    axioms.    
    Finally, users will be able to
    search in the full text of theorems and proofs that have been
    extracted from natural-language research articles in
    task~\taskref{structuring}{strtext}.
  \end{task}

\end{tasklist}

\begin{wpdelivs}
  \begin{wpdeliv}[due=3,miles=startup,id=requirements,dissem=PU,nature=DEM,lead=Inr]
      {Requirements Analysis and Synchronization}
  \end{wpdeliv}
  \begin{wpdeliv}[due=2,miles=???,id=acessopamtool,dissem=PU,nature=DEM,lead=Oca]
      {'Proof to opam' tool : Tool to translate the logipedia database format into an opam repository}
  \end{wpdeliv}
  \begin{wpdeliv}[due=1,miles=???,id=acessopamrepo,dissem=PU,nature=DEM,lead=Oca]
      {'Proof opam repository': opam repository populated with the generated proof packages }
  \end{wpdeliv}
  \begin{wpdeliv}[due=1,miles=???,id=accessopamconfig,dissem=PU,nature=DEM,lead=Oca]
    {'opam for logipedia': opam configuration to use it (only or also) for logipedia}
  \end{wpdeliv}
  \begin{wpdeliv}[due=1,miles=???,id=accessopam,dissem=PU,nature=DEM,lead=Oca]
    {'Provide logipedia opam' : installation of the repository in the infrastructure of WP9T1 }
  \end{wpdeliv}
\end{wpdelivs}
\end{workpackage}


%%% Local Variables:
%%% mode: latex
%%% TeX-master: "../propB"
%%% End:

\newpage
\begin{workpackage}[id=structuring,type=RTD,wphases=1-48,
  short={Structure of the encyclopedia},% for Figure 5.
  title={Structure of the encyclopedia},
  activity=tna,
  lead=Fau,
  SacRM=40,
  FauRM=22,
  BolRM=4
%  TouRM=12,
%  InrRM=14
]

%\ednote{Which sites are interested?
%David Deharbe and Etienne Prun (Clearsy); Nicola Gambino, Michael Rathjen, Claudio Sacerdoti Coen, Dale Miller, Emilio J. Gallego Arias, Michael Butler, Pierre Senellart}

% David Deharbe and Etienne Prun (Clearsy): use B Method, would like to doublecheck B proofs, integrate B with other proof assistant at high-level

\begin{wpobjectives}
Providing infrastructure for the structured ontological representation
of libraries and use it to enrich the information about formal
libraries in Logipedia.  Enabling the exchange and reuse the knowledge
between prover systems.
\end{wpobjectives}


\begin{wpdescription}
We proceed in three steps.
Firstly, Tasks~\localtaskref{strlibstructure} and~\localtaskref{strdofimpl} extend the Dedukti language with features for high-level representations that are critical for accessing parts of and searching libraries.
This includes a framework to \emph{define} typed meta-data in form of ontologies, and to \emph{enforce} them in 
the Dedukti libraries.
Secondly, Tasks~\localtaskref{strrefonto} builds ontologies serving as technical exchange format as well as domain-specific descriptions of libraries.
Thirdly, Tasks~\localtaskref{strontorepml} fills the ontology with data from both formal libraries and natural language articles and use the ontology to relate to each other.

This work package will be jointly led by Burkhart Wolff at \site{Sac} and Florian Rabe at \site{Fau}.
(Where a single leader is needed for formal purposes, the latter site will be the primary leader.)
Burkhart Wolff implemented a document ontology framework in Isabelle and developed several applications
in the field of formal software engineering.
% there should be no references here, move them to chapter 1
%\cite{brucker.ea:ontologies-certification:2019,brucker.ea:isabelle-ontologies:2018,brucker.ea:ontologies-certification:2019}
Florian Rabe has extensive experience in designing and implementing knowledge representation languages
%\cite{RK:mmt:10,rabe:recon:17}
as well as in exporting theorem prover libraries.
%\cite{KR:oafexp:20,CKMRSW:ulo:19}
\end{wpdescription}

\begin{tasklist}
\begin{task}[id=strlibstructure,title=Library Structure,shorttitle=Struct.,lead=Fau,FauRM=8, SacRM=6, wphases=1-28!.5]
This task extends the Dedukti language with primitives for representing library, document, informal annotations, and theory structure.
This includes in particular the definition of unique identifiers for all declarations, which is critical for alignments.
%We extend the Dedukti language with features for high-level representations.
%This will include
%\begin{compactitem}
%\item theories: a general term we use to unify a variety of module system constructs such as type classes or locales,
%\item derived declarations: high-level declarations such as inductive type definitions, whose semantics is given by elaboration into more primitive constructs,
%\item metadata annotations: a general framework for attaching information about semantics, document structure, and tool interaction.
%\end{compactitem}
%
%The low- and high-level representations will be tightly integrated: any declaration or object may be given alternatively through either or both of these.
%The prover exports from instrumentation will be such that they produce both representations whenever possible.
%
%Then we leverage this design in several applications including automated prover interaction and a Logipedia-wide search service.
\end{task} 

\begin{task}[id=strdofimpl,title=Ontological Framework for Meta-Data,shorttitle=F/W,lead=Sac,SacRM=24,wphases=1-24!1.0]
This tasks extends the Dedukti language with a framework for meta-data annotations.
This will cover all levels of the structure introduced in \localtaskref{strlibstructure} as well as the 
level of subexpressions of Dedukti expressions. It will also provide a mechanism to validate meta-data
according to assertions.
\end{task} 

% suggested for removal in budget arbitration meeting; now mentioned in task on reference ontology
%\begin{task}[id=strdomonto,title= Domain Ontologies for Formal Methods in SE,shorttitle= Domain Ontologies for Formal Methods in SE,lead=Tou,TouRM=12, SacRM=0]
%Y. Aitameur at \site{Tou}
%\begin{compactitem}
%\item domain ontologies as descriptive models for engineering domains 
%\item links/imports with/from standards and certification
%\item engineering models annotations
%\item strengthening engineering models by references to domain ontologies
%\item Case studies could be certification, safety, security.
%\end{compactitem}
%\end{task} 

\begin{task}[id=strrefonto,title=Reference Ontology,shorttitle=Ref. Ont.,lead=Sac,FauRM=6,SacRM=6,wphases=12-36!.5]
This tasks compiles, integrates, and curates the various ontologies used for describing libraries in the project.
These come from several sources:
\begin{compactitem}
 \item The ontology induced by the structuring features built in task \localtaskref{strlibstructure}.
 \item The ontologies built by users using the ontology framework built in task \localtaskref{strrefonto}.
 \item Manually written ontologies or imports of existing ontologies for knowledge formalised in prover libraries, such as the Upper Library Ontology
and domain-specific ontologies.
 The latter may include for example the ontologies for engineering and their relation to descriptive models and certification standards that are planned to be developed by \site{Tou}.
\end{compactitem}
\end{task}

\begin{task}[id=strontorepml,title=Ontological Representation of Formal Libraries,shorttitle=Ont. Repr.,lead=Fau,FauRM=6,BolRM=4,SacRM=5,wphases=12-48!.5]
  This task extends the exports from Isabelle and Coq developed in
  \WPref{libraries} with structural and ontological data that conforms to the language features introduced in Tasks~\localtaskref{strlibstructure} and \localtaskref{strdofimpl}.
  It will also build on the ontological export of RDF triples relative to
the Upper Library Ontology developed for Isabelle and Coq.

The task leader will collaborate with M. Wenzel for Isabelle and C. Sacerdoti Coen at \site{Bol} for Coq, with whom long-standing collaborations on these library exports exist.
%\cite{MRS:coq:19,CKMRSW:ulo:19,KRW:isabelle:19}
Wenzel's involvement will take the form of a sub-contract of
\site{Fau} corresponding to roughly 4 person-months, an arrangement
that has already been used twice in other projects.  The resources for
this subcontract are not included in the person-months listed here.
\end{task}

% Moved to WP9
%\begin{task}[id=strontosearch,title=Ontological Search,shorttitle=Ont. Search,lead=Fau,FauRM=12,SacRM=6]
%Search based on ontological data (RDF triples) using systems like SPARQL
%\ednote{possible participation of S. Dumbrava; is there a site for this?}
%\end{task} 

%\begin{task}[id=strformsearch,title=Formula-based Search,shorttitle=Form. Search,lead=Fau,BolRM=4,FauRM=12]
%Search based on formula structure using systems like MathWebSearch
%\end{task} 

% suggested for removal in budget arbitration meeting
%\begin{task}[id=strtext,title=Ontological Representation of Natural Language Articles,shorttitle=Articles,lead=Inr,FauRM=6,InrRM=14]
%    % task leader: Pierre Senellart, Inria
%This task extracts ontological information from natural language research articles and link them with the formal representations in Isabelle and Coq.
%P. Senellart at \site{Inr} and M. Kohlhase at \site{Fau} will work on the automatic extraction and annotation of natural-language theorem statements and proofs from published articles, as well as building libraries of such theorems and proofs.
%  % Search capabilities have been moved to WP9
%  %and search and querying capabilities
%\end{task} 

\end{tasklist}


\begin{wpdelivs}
  \begin{wpdeliv}[due=28,id=deliv-str-framework,dissem=PU,nature=R,lead=Sac]
        {This deliverable describes the language developed in Tasks 1 and 2.}
  \end{wpdeliv}
  \begin{wpdeliv}[due=36,id=deliv-str-ontology,dissem=PU,nature=R,lead=Sac]
        {This deliverable describes the reference ontology developed in Task 3.}
  \end{wpdeliv}
  \begin{wpdeliv}[due=48,id=deliv-str-libraries,dissem=PU,nature=R,lead=Fau]
        {This deliverable describes the representation of major formal libraries developed in Task 4.}
  \end{wpdeliv}
\end{wpdelivs}


%\begin{enumerate}
%\item concrete/surface syntaxes 
%\item Central Library Backend Systems 
%\item Cross-System Front-Ends/Portals (Logipedia, ...)
%\item Semantic Middleware-based System Interoperability
%\end{enumerate} 
%
%Since proof-objects for substantial theory developments tend to be
%very large (the representation of current POs for the Isabelle/AFP can
%easily reach several TB although using techniques for compression), A
%technical pre-requisite for interchangeability, connectivity and
%advanced search consists in a structured, typed format for meta-data
%together with a flexible mechanism of their validation. Technically,
%this kind of meta-data has the form of a function annoconst : arg1 ->
%... -> argn -> proof-term -> proof-term where annoconst is a constant
%symbol which represents an identity in the proof-term (so, any import
%function of a specific system can actually ignore it), and where the
%argi represent terms with meta-information such as, eg., “this
%proof-term represents a free data-type construction of the form ...”,
%or “this part of the proof is a derivation of a free data-type of the
%following form ...”, “this lifting over assumptions represents in
%Isabelle a Locale-instantiation”, “this part of a theory
%development is connected to ... ”, “this theorem belongs to the
%sub-class of XXX ... theorems”, etcpp. For arguments of annotations,
%validation-functions can be defined that may check that the argument
%terms satisfy a certain property wrt. to the proof-term and the
%current logical context. Dedukti will provide a framework that allows
%for each proof-system (Coq, HOL4, Isabelle...) to declare meta-data
%together with validations and thus communicate tool-specific knowledge
%to other systems. This framework can be seen as a particular form of
%an ontology definition language.
% 
%WP8: Indexing and browsing [?]  Construct tools to index and browse
%this encyclopedia, that is find the theorem one needs, either by
%looking for it with its name, with its statement, or with symbols
%occurring in it.


\end{workpackage}

%%% Local Variables:
%%% mode: latex
%%% TeX-master: "../propB"
%%% End:

\newpage
\begin{workpackage}[id=theories,wphases=0-48,type=RTD,
  short=Theories in Dedukti,% for Figure 5.
  title= Defining theories in Dedukti,
  lead=Inn,
  InnRM=10]

\ednote{MK: We need one coordinating site. original coordinators: Cezary Kaliszyk and Stephan Merz}

\ednote{MK: interested parties (add their sites and RM here): David Deharbe, Stephan Merz, Guillaume Genestier, Guillaume
Burel, Yamine Ait Ameur, Jean-Paul Bodeveix, Mamoun Filali, Arthur
Chargueraud, Gaspard Férey
}

\begin{wpobjectives}
  The objective of this work package is to \ldots

This includes notably:
  \begin{compactitem}
  \item \ldots
  \end{compactitem}
  A key aspect will be to foster \ldots
\end{wpobjectives}

\begin{wpdescription}
  For other theories, such as Abella, PVS, Mizar and TLA+, we have not yet investigated
  the possibility to express them in Dedukti.
\end{wpdescription}

\begin{tasklist}
\begin{task}[id=pvs,title=Express the theory of PVS in Dedukti and instrument the system]
\ednote{Saclay}
\end{task}

\begin{task}[id=mizar,title=Express the theory of Mizar in Dedukti and instrument the system]
\ednote{IInnsbruck, Bialystok}
\end{task}

\begin{task}[id=tla,title=Express the theory of TLA+ in Dedukti and instrument the system]
\ednote{Nancy, Liège}
\end{task}

\begin{task}[id=abella,title=Express the theory of Abella in Dedukti and instrument the system]
\ednote{Saclay}

The usual approach to capturing either Peano and Heyting arithmetics
is to use various axioms (and an axiom scheme for induction) on top of
classical and intuitionistic first-order logic.  Indeed, this is the
approach used in the Dedukti proof checker.


A different approach to encoding arithmetic has been developed over
the past 20-30 years, starting with papers by Schroeder-Heister and
Girard in the early 1990s and extended in a series of papers by
Baelde, Gacek, McDowell, M, Momigliano, Nadathur, and Tiu.  In this
new setting, first-order logic is extended by considering both
equality and the least fixed point operator as \emph{logical
  connectives}: these logical connectives are not available directly
in Dedukti.

This new foundations for arithmetic has been implemented in two
systems: the automated Bedwyr prover and the interactive Abella
prover.  While neither Bedwyr nor Abella are as popular as many of the
theorem provers that are covered by this proposal, there are two
important reasons to consider incorporating them into the Logipedia
effort.

First, the Bedwyr prover is capable of constructing proofs for the
kind of queries that are part of emph{model checkers}.  This class of
provers has not yet been incorporated into Dedukti.  The
proof-theoretic work behind model checking in Bedwyr should provide
some of the insights needed for allowing Dedukti to proof check the
results of model checkers.

Second, Bedwyr and Abella provide for direct and elegant support of
meta-level reasoning.  Given that the foundations for Bedwyr and
Abella have been given using Gentzen's sequent calculus, it was
possible to enrich their foundations to allow for the treatment of
binding structures within terms.  As a result, it is possible to
reason directly on terms representing $\lambda$-terms and
$\pi$-calculus expressions.  In particular, the Abella prover has
probably the most natural and compact formal treatment of the
$\pi$-calculus and its meta-theory when compared to all other attempts
in any other theorem provers.  More generally, the Abella prover
should be able to treat the meta-theory of programming and
specification languages as well as various logics and their
proofs. While these tasks are not the typical tasks considered by the
majority of theorem provers within the scope of this proposal,
meta-theory results do play an important role at times: in fact, the
ultimate questions as to whether or not a proof checkers (such as that
used by Dedukti) is correct or not will involve meta-theoretic
questions.

We propose to work on the general problem of exporting proofs from
Abella to Dedukti.  (Since all proofs that are constructed
automatically via Bedwyr can also be constructed manually within
Abella, we shall limit our discussion below to Abella only.)  The
proposed work will serve not only to answer the question of how to
relate these two different foundations for arithmetic but also to
allow Abella's particular style of proofs to find applications in the
wider world of formalized proofs.

The general problem described above has the following constituent parts.

(1) Proofs involving searching finite structures. Proofs built for
model checking problems over finite structures have two different
kinds of phases.  To illustrate, consider trying to find a specific
node within a binary tree.  If such a node exists, then the proof
essentially encodes the path to the node in the tree.  If, however, no
such node exists, then the proof of that negative fact is essentially
a computation that exhaustively explores the tree.  Using the Dedukti
terminology: in the first case, the proof involves several deduction
steps, while in the second case, the proof involves a pure
computation. When dealing with model checking problems such as
simulation (in concurrency theory) and winning strategies (in game
theory), proofs will involve alternating phases involving either
deduction or computation.  Since the notion of computation in
Abella-style proofs involves backtracking search, that style
computation will be quite different from Dedukti's notion of
computation as confluent rewriting.

(2) Extending model checking problems to the general case of infinite
structures and the associated inductive reasoning methods. Although
the formal basis of Abella uses least and greatest fixed-point
combinators and explicit (co-)invariants, the Abella implementation of
(co-)induction is based on cyclic reasoning using size-annotated
relations. It is known, in principle, how to convert cyclic proofs
using annotations to proofs with explicit invariants, but an invariant
extraction procedure that works in all cases is still missing. Once
such invariants are available, incorporating them into Dedukti should
be straightforward in association with part (1).

(3) Binding structures. Abella, as well as several other computational
logic systems ($\lambda$Prolog, Isabelle/Pure, Twelf, Beluga, etc)
make use of the so-called \emph{$\lambda$-tree syntax} (a form of
\emph{higher-order abstract syntax}, HOAS) approach to represent
bindings. This approach is further enriched in Abella with the
$\nabla$-quantifier that allows inductive and co-inductive properties
to be defined based on the \emph{structure} of $\lambda$-terms. We
propose to examine encodings of $lambda$-tree syntax in Dedukti. The
best approach probably involves extending the underlying theory of
Dedukti with a quantifier similar to Abella's $\nabla$-quantifier.

(4) Reflective treatment of unification. One of the features of
Abella's style of proofs is the use of left-introduction rules for
equality that exhaustively examine complete sets of unifiers for
$\lambda$-terms. This is implemented in terms of a unification engine
that is currently a trusted black box, which complicates any proposal
for exporting proofs to different implementations of unification or
equality. In Dedukti the unification procedure can be recast as a
rewrite system, but it is unclear how to derive reflective properties
based on the unifiability of terms.
\end{task}

\begin{task}[title=id=hott,title=expressing HoTT]
\ednote{Saclay, Leeds}
\end{task}
\end{tasklist}

\begin{wpdelivs}
  \begin{wpdeliv}[due=3,miles=startup,id=requirements,dissem=PU,nature=DEM,lead=INR]
      {Requirements Analysis and Synchronization}
\end{wpdeliv}
\end{wpdelivs}
\end{workpackage}


%%% Local Variables:
%%% mode: latex
%%% TeX-master: "../propB"
%%% End:

\newpage
\begin{workpackage}[id=alignment,wphases=0-48,type=RTD,
  short=Concept Alignment,% for Figure 5.
  title=Concept Alignment,
  lead=Pra,
  PraRM=10]
  
\ednote{We need one coordinating site. original coordinators: Filip Marić and Dale Miller}

\ednote{Parties initially expressing interest (add their sites and RM
  here): Florian Rabe, Cezary Kaliszyk, Dale Miller, Josef Urban,
  Yamine Ait Ameur, Jean-Paul Bodeveix, Mamoun Filali, Chantal Keller,
  Julien Narboux, Nicola Magaud, Arthur Charguéraud, François Thiré}

\begin{wpobjectives}
The various proof assistants have different treatments of fundamental
concepts used in logic and arithmetic.  This WP will develop
standards, tools, and techniques that will allow these concepts to be
aligned so that proofs in one proof assistant can be meaningfully used
in other systems.

The following are the three main points on which this workpackage will
focus. 
\begin{compactitem}
\item Alignments at the levels of logic.  The alignments between
  classical and intuitionistic proofs is the main challenge here.
  There are different kinds of embeddings of classical proofs into
  intuitionistic logic.  A secondary challenge is to align the various
  treatments of induction and co-induction in proof assistants: these
  treatments include explicit presentations of invariants,
  applications of invertible inference rules, and cyclic proof
  structures.

\item Alignments of theorem proving objects such as constants,
  theorems, and types.
  \begin{compactitem}
     \item Similar concepts in different libraries can have many
       significant differences once one examines the concepts in detail.

     \item approximate matching constant, where some properties that
          holds for c1 also holds for c2.
  \end{compactitem}

\item Alignments of proofs
  \begin{compactitem}
     \item Often it is not enough that we simply trust a proof to have been
          checked.  We occasionally need to work with proofs in order
          to extract an explanation or its constructive content.  
     \item Identify tactics and inference rules that have the same
       effect on proof state. 
     \item Recognizing proofs that have the same structure. (proof porting)
  \end{compactitem}
\end{compactitem}
\end{wpobjectives}

\begin{wpdescription}
Construct tools and proofs to analyze these proofs and align concepts, that is unify
concepts such as connectives and quantifiers, the concept of natural number, etc. and
theorems that occur in several libraries.  [Paris, Saclay, Innsbruck, Prague,
Strasbourg, Belgrade]

Discovery/finding objects in different theories that refers to the
same informal concepts.
There are logic inspired techniques (check that one interface formally
entails another interface): ATPs might be able to automatically handle
such checks. (Some articulation needed with WP4.)
There can be other, statistical or linguistic clues that might help
narrow down on discovering possibly useful related concepts.

\end{wpdescription}

Task: Should we attempt to ``make classical logic proofs constructive'' when possible?

\begin{tasklist}
\begin{task}[id=aligndef,title=Tracking classical and intuitionistic proof steps in logic and arithmetic] 
\end{task}

\begin{task}[id=aligndef,title=Alignment in particular domains]
  Arithmetic (FAU Erlangen-Nürnberg), Geometry (University of
  Strasbourg), and real analysis (Inria Saclay).
\end{task}

\begin{task}[id=aligndef,title=Definition of an Alignment Language]
(FAU Erlangen-Nürnberg)
\end{task}

\begin{task}[id=translate,title=Libraries as intermediates for translations]
(FAU Erlangen-Nürnberg)
\end{task}

\begin{task}[id=translate,title=Alignment of proof structures]
Inria Saclay
\end{task}

\begin{task}[id=aligntranslate,title=FAIR Services: Reuse across Libraries,lead=Fau,FauRM=12]
\ednote{Florian Rabe: At the workshop I was asked to lead this task, but the text in this file looks outdated.
So I'm putting some text here to store my notes. The coordinators should get back to me to discuss details.}
This task leverages the large library of alignments built in the previous tasks by building a major FAIR service focusing on \textbf{reuse}.
It uses alignment to translate formulas and theorems from one library to another.
\end{task}

\begin{task}[id=alignsearch,title=FAIR Services: Search across Libraries,lead=Fau,FauRM=12]
\ednote{Florian Rabe: At the workshop I was asked to add this task, but the text in this file looks outdated.
So I'm putting some text here to store my notes. The coordinators should get back to me to discuss details.}
This task leverages the large library of alignments built in the previous tasks by building a major FAIR service focusing on \textbf{search}.
Here users enter a search query relative to the central library built above and chooses which libraries to search in.
Using the functionality developed in \localtaskref{aligntranslate}, the query is then translated into the requested libraries and searched in each one.
Results from all libraries are aggregated and returned.
\end{task}

\end{tasklist}

\begin{wpdelivs}
  \begin{wpdeliv}[due=3,miles=startup,id=requirements,dissem=PU,nature=DEM,lead=Inr]
      {Requirements Analysis and Synchronization}
\end{wpdeliv}
\end{wpdelivs}
\end{workpackage}

%%% Local Variables:
%%% mode: latex
%%% TeX-master: "../propB"
%%% End:

\newpage
\begin{workpackage}[id=dissemination,wphases=0-48,
  short=Dissemination,% for Figure 5.
  title={Dissemination, communication, and exploitation},
  lead=ISa,
  ISaRM=10]
  
\begin{wpobjectives}
  The objective of this work package is to \ldots

This includes notably:
  \begin{compactitem}
  \item \ldots
  \end{compactitem}
  A key aspect will be to foster \ldots
\end{wpobjectives}

\begin{wpdescription}
  \ednote{Gilles will write} \ednote{MK: it is probably a good idea to copy from
    OpenDreamKit: see
    \url{https://github.com/OpenDreamKit/OpenDreamKit/blob/master/Proposal/WorkPackages/DisseminationCommunityBuilding.tex}}
\end{wpdescription}

\begin{tasklist}
  \begin{task}[id=industry,title=Club on industrials]
  \end{task}
  \begin{task}[id=teachers,title=Club on Teachers]
  \end{task}
\end{tasklist}

\begin{wpdelivs}
  \begin{wpdeliv}[due=3,miles=startup,id=requirements,dissem=PU,nature=DEM,lead=ISa]
      {Requirements Analysis and Synchronization}
\end{wpdeliv}
\end{wpdelivs}
\end{workpackage}


%%% Local Variables:
%%% mode: latex
%%% TeX-master: "../propB"
%%% End:

\newpage
\begin{workpackage}[id=management,type=MGT,
  short=Management,
  title=Management,
  lead=Inr,InrRM=36,BirRM=1,InnRM=1,SacRM=1,TumRM=1,IrtRM=1,LeeRM=1]
  
  \begin{wpobjectives}
    The scope of this work package is the overall management of the project activities led by the consortium. The management of the Logipedia project will ensure the necessary conditions to enable the project to achieve its objective(s) while meeting its cost, time and quality requirements. This includes the scientific, administrative, financial and legal management.

    Gilles Dowek, Inria senior researcher and professor at ENS Paris-Saclay, will be the coordinator and leader of this work package. Gilles is PI of many projects, including under the H2020 framework programme and international projects. He will also be supported by a deputy coordinator, an European project manager from the Innovation, Partnership and Transfer Office of Inria Saclay and a Chief engineer.

This therefore includes:
\begin{compactitem}
\item A scientific and technical coordination to create a vibrant scientific and technical environment within the project.
\item The overall management of the project and consortium according to the governance structure and procedures explained in section 3.2.
\item An efficient project management, as specified in 3.2, including:
  \begin{compactitem}
  \item Overall administrative and financial project management, including reporting to the European Commission.
  \item Quality management.
  \item Assessment and risk management, including conflict or dispute management.
  \end{compactitem}
\end{compactitem}
\end{wpobjectives}

\begin{tasklist}
  \begin{task}[id=coordination,title=Scientific and technical coordination,lead=Inr,InrRM=12,wphases=1-48]
    The scientific and technical coordination will be led by Inria senior researcher Gilles Dowek. He will be in charge of ensuring the implementation of the scientific strategy of Logipedia and thereby ensuring the growth of the Logipedia community. This task foresees a key role of scientific animation and to impulse the organisation of scientific activities, together with the WP leader in charge of dissemination. Gilles Dowek will supervise the ongoing scientific and technical coordination and help with the innovation management, together with the technical manager and the steering committee. The scientific coordinator will also chair the steering committee and the general assembly. The scientific coordinator will be the scientific point of contact within the consortium and for the consortium when the project needs to be represented.
  \end{task}

  \begin{task}[id=admin,title=Administrative and Financial Management,lead=Inr,InrRM=24,wphases=1-48]
    A European Project Manager (EPM) from the Transfer and Innovation team of Inria Saclay which has an extensive experience in handling innovation from research projects such as Logipedia. The consortium will therefore benefit from tailored and on-demand advice regarding the use and potential transfer of the research results during the course of the project.
The EPM will ensure the day-to-day management as it will be the administrative and financial point of contact for the consortium and a dedicated contact point for the European Commission. The meeting preparation and follow-up will be another task of the EPM and will include organising plenary meeting with the partner organisation, general assembly, minutes, review meeting with the consortium. Financial aspects will be a crucial task of the EPM and include: payment to partners; ensuring financial monitoring within the consortium and leading the financial reporting to the European Commission.
The EPM will also be responsible, together with the scientific coordinator, for ensuring the technical work and deliverables meet the technological objectives of the project according to the defined schedule. The EPM will work very closely with the work package leaders in order to monitor the progress of the technical work and to identify potential risks within each WP. The EPM also acts as a quality manager to ensure that the content of the deliverables meets the quality standards defined for the project. The EPM reports to the Scientific Coordinator.
The development and maintenance of collaborative tools will be ensured by Inria and monitored by the EPM. A common teleconference tool, storage space, reporting and intranet will be set up and detailed in the collaborative tools deliverable of M2.
  \end{task}

  \begin{task}[id=legal,title={Legal Management (data, ethics, GDPR)},wphases=1-48]
    The preparation of the Consortium and Grant agreement will be led by the European Project Manager at Inria Saclay. If a change arises during the course of the project, amendment will be prepared by the project management team, in close collaboration with Inria legal team. 
Data Protection, ethics and GDPR Compliance Management will also be ensured by INRIA. The main objective here is to provide guidance on data protection for the research activities of the project in the context of the European General Data Protection Regulation (GDPR). If at some point during the course of the project, the consortium or any scientist is unsure about how to handle a particular situation or requires advice on ethical issues, the partners or the individuals, supported by the EPM, will refer to the operational ethical committee of Inria (the COERLE) before proceeding.
  \end{task}

\end{tasklist}

\begin{wpdelivs}
  
  \begin{wpdeliv}[due=2,miles=???,id=collab-tools,dissem=PU,nature=DEC,lead=Inr]{Collaborative Tools} Document or Notice introducing the collaborative tools of the consortium
  \end{wpdeliv}

  \begin{wpdeliv}[due=3,miles=???,id=guide,dissem=PU,nature=R,lead=Inr]{Logipedia Partner Guide} In order to present the processes and governance within the consortium
  \end{wpdeliv}

  \begin{wpdeliv}[due=6,miles=???,id=data-plan,dissem=PU,nature=R,lead=Inr]{Data Management Plan}
  \end{wpdeliv}
  
\end{wpdelivs}

\end{workpackage}


%%% Local Variables:
%%% mode: latex
%%% TeX-master: "../propB"
%%% End:

\end{workplan}

\newpage

\subsubsection*{List of all deliverables}\label{sec:deliverables}

{\footnotesize\inputdelivs{8cm}}

%%% Local Variables: 
%%% mode: latex
%%% TeX-master: "propB"
%%% End: 


\subsection{Relation between the components}


\section{Management structure, milestones and procedures}

\begin{todo}{}\color{red}
  * Describe the organisational structure and the decision-making ( including a list of milestones (table 3.2a))

  * Explain why the organisational structure and decision-making mechanisms are appropriate to the complexity and scale of the project.

  * Describe, where relevant, how effective innovation management will be addressed in the management structure and work plan.

  Innovation management is a process which requires an understanding of both market and technical problems, with a goal of successfully implementing appropriate creative ideas. A new or improved product, service or process is its typical output. It also allows a consortium to respond to an external or internal opportunity.

  * Describe any critical risks, relating to project implementation, that the stated project's objectives may not be achieved. Detail any risk mitigation measures. Please provide a table with critical risks identified and mitigating actions (table 3.2b)

  * Give a summary of the trans-national and/or virtual access to be provided (table 3.2c).


  {\color{red} A table for risks}

    Risk name / WP / Impact if occurs / Probability / LEvels / Prevetive and
    Contengency action.

  {\bf Definition:}
  
  \underline{Milestones}: means control points in the project that help to chart progress. Milestones may correspond to the completion of a key deliverable, allowing the next phase of the work to begin. They may also be needed at intermediary points so that, if problems have arisen, corrective measures can be taken. A milestone may be a critical decision point in the project where, for example, the consortium must decide which of several technologies to adopt for further development.
\end{todo}


{\color{red} A table with a list of milestones}

{\color{red} Inria Saclay transfer, innovation, and partnetship
  department will contribute to the innovation management}
The Logipedia consortium will gather twenty nine beneficiaries and partners
from eleven European countries during four years. The project management
structure will be tailored to the specificities and needs of this
large consortium and its ongoing network development.

\subsection{Organisational structure}

\subsubsection*{The project management team}

{\bf The Coordinator}: the 
coordinator is responsible for the coordination of
scientific and technical activities in order to meet the objectives
set by the European Commission in the Grant Agreement. The 
coordinator works closely with the work package leaders
within the steering committee, in order to monitor the progress of the
scientific and technical work and identify potential risks within each
work package. The coordinator will daily
collaborate with the European project manager in charge of the
day-to-day management of Logipedia. The project will be managed by Pr
Gilles Dowek, permanent senior researcher at Inria Saclay. He will
also chair the meetings of both the general assembly and steering
committee.

{\bf The Deputy Coordinator}: The Deputy coordinator seconds and
replaces the coordinator.

{\bf The European Project Manager}: The European project manager
member of the Technology Transfer and Partnership Office of Inria
Saclay, is in charge of all administrative, financial and legal
management tasks as listed in \WPref{management}. The
European project manager is the interface between the project and the
European Commission as it represents the point of contact for the
European Commission. The European project manager has the overall
administrative and financial responsibility for the organisation and
administrative and financial monitoring of the project.

{\bf The Chief Engineer}: The chief engineer is an experienced
research engineer from Inria Saclay and is responsible for ensuring
the development and maintenance of tools at Inria Saclay and
supervising the development tasks achieved at the other
beneficiaries. The chief engineer will ensure the coherence of the
Logipedia tools development, according to the defined schedule in the
Grant Agreement.

Innovation management and intellectual property rights issues will be
handled by Inria and the European project manager, supported by the
experienced Technology Transfer and Partnerships Office of Inria
Saclay. The project management team will establish appropriate
policies and rules for the management of intellectual property rights
for the knowledge developed within the project, as well as the
identification of the opportunities for the exploitation of the
project results in innovation activities. Issues related to innovation
and/or intellectual property rights management will be tackled at
every steering committee meeting.

\subsubsection*{The operational level}

{\bf The Steering Committee}: The steering committee is composed of
the coordinator, the chief engineer, the European project
manager and the work package leaders. The steering committee is the
supervisory body for the implementation of the project. The steering
committee is responsible for monitoring the activities of the project
and the implementation of decisions taken by the general assembly. It
can formulate proposal for changes in the description of action and
the related consortium budget. Those changes will have to be agreed
by the general assembly first and then the European
commission. The steering committee is chaired by the 
coordinator.

{\bf The Work Package Leaders}: The work package leaders are
responsible for the monitoring and management of the activities and
results within their work packages. In particular, work package
leaders i) identify deviations from the project plan and report them
to the steering committee, ii) manage and supervise the preparation of
reports and their timely delivery, iii) control and monitor activities
of tasks and regularly meet once per month with task leaders, iv)
manage the information flow with other work packages via the steering
committee.

{\bf The Task Leaders}: The task leaders are responsible for
coordinating the scientific and technical work in their task and
making the day to day technical decisions that solely affect their
task. Inter-task decisions are coordinated with the work package
leaders.

{\bf The Club Leaders}: The club leaders are in charge of
disseminating of the tools developed by the Logipedia consortium in
various communities. They organize the activity of the club. They give
ongoing feedback to the consortium during the course of the project.


\subsubsection*{The strategic level}

{\bf The General Assembly}: The general assembly is composed by all
the members of the consortium, with each representative having one
vote. Every new partner will have a voting right. The general assembly
will gather at least once a year, and as many virtual meetings as
needed. The general assembly is the main governance and ultimate
decision-making body of the consortium. The general assembly must
review the project progress, decide on contingency actions in case of
deviations from the plan and take final decisions on policy and
contractual issues and conflicts as requested by the steering
committee.

{\bf The Advisory Board}: The advisory board is a consultation body to
the steering committee and general assembly. It will bring external
and non-legally binding perspective on the scientific and technical
development of the project, ecosystem building and the future of the
encyclopedia. The advisors of this board will attend the yearly
general assembly plenary meeting and will be consulted on the strategy
of the project. The advisory board should aim at representing the
stakeholders of the Logipedia ecosystem without including any
beneficiary or associate partner’s employees. It will be composed of,
among others, industrial and international academic partners
(including non-European ones) apointed by the coordinator after
consulting the steering committee. To start with, we suggest to include
\begin{compactitem}
\item June Andronick (Data61, Kensington NSW), 
\item Denis Cousineau (Mitsubishi Electric), 
\item Thomas Letan (ANSSI), 
\item Jacques Fleuriot, 
\item Natarajan Shankar (SRI),
\item Aaron Stump (Iowa), 
\item Laurent Voisin (Systerel).
\end{compactitem}

 \subsubsection*{Internal communication and collaborative ecosystem}

The communication of the consortium including their internal tools is
managed in task 10.2.  The consortium will make use of a number of
project management tools, such as a visio conferencing tool, a project
repository to have an updated account of the project’s important
documents, the progress of the work packages work and deliverables,
all the advances in the project and all the meetings minutes, mailing
lists, etc. that facilitate the smooth execution of the project. This
collaboration environment will be provided by the coordinator of the
project.

Work packages, chaired by work package leaders, will have monthly
planned visio conferences and meetings as need by the work plan;
additional technical meetings may be set up by task leaders or
individual partners. The steering committee will have monthly visio
conferences and will meet twice a year. Dedicated working groups will
be planned as needed according to the work plan.  All meetings will be
documented by minutes listing major decisions and action items.


The project management team will be in charge of all organisation
issues in the general assembly meetings, supported by the local
partner. The project will organise meetings of the general assembly at
least once a year. To equally share travel costs among partners,
physical meetings will be located by rotation at partners’
locations. Project review meetings will be done on a regular basis
according the Grant Agreement provisions.

\subsection{Decision-making Process}

Our approach for the decision-making process is to locate the decision
as close as possible to the level responsible for the execution (from
task level to general assembly level). Decisions are managed within
frequent project meetings, either on-site or via
teleconference. Decisions can be also managed by consultation. If
voting is needed, the agenda should clearly indicate this fact. Quorum
and voting rules will be defined in the Consortium
Agreement. Decisions are binding once the relevant part of the meeting
minutes has been accepted. Any changes to the project plan and scope
must be reviewed and approved by all levels of project management,
before proposing these changes to the steering committee and any
modifications will be considered rejected, after rejection on any of
these involved levels.

Another guiding principle is to avoid conflicts. Nevertheless, should
one arise, a conflict resolution will be ready to be put in place to
deal with it accordingly. The conflict resolution foresees that each
conflict will be mediated, solved or decided at the lowest level
possible. Attempts to solve issues within the consortium will be
carried out in increasing order of authority first at task level
(management of task leader), work package level (management of work
package leaders), and then following the management bodies till the
general assembly. Further rules related to conflict resolutions will
be laid out in the Consortium Agreement.

\subsection{Monitoring and reporting}

\subsubsection*{Internal reporting}

The project management team continuously monitors the project plan
with its milestones and critical paths. Each work package leader will be
responsible for the correct execution of the implementation plan for
the corresponding work package. In terms of reporting, this means the work package leaders
will be in charge of gathering the information related to their own
work packages.

Regular audio-conferences of the Steering Committee are foreseen,
which allows work package leaders to identify and raise risks and
discuss them together. This ensures that management (coordination,
European project manager) is aware of potential problems and
deviations and can initiate countermeasures long before a situation
becomes critical. This ensure to spot the blocking points in due time
and to find that the solutions will be available in time.

In case there is a deviation from the work plan, the 
coordinator will initiate corrective actions through the
task leader and the work package leader. The work package leader will
be responsible to implement these actions in dialogue with the
different partners involved in their work packages.


\subsubsection*{Reporting to the European Commission}

The Logipedia consortium will follow the mandatory reporting period
required by the European Commission. The following reporting will be
achieved: Period 1 (M01-M18), Period 2 (M19-M36) and Period 3
(M36-M48).

The project management team will provide the necessary templates in
order to achieve the reporting in due time. Work package leaders will
be asked to gather the relevant information provided by the task
leader regarding their work package and to summarise in order to be
reviewed by the steering committee. It will then be treated by the
coordinator and European project manager and sent to the
European Commission.

\subsection{Significant Risks and Associated Contingency Plans}\label{sec:risks}

\begin{todo}{from the proposal template}
  Describe any significant risks, and associated contingency plans
\end{todo}
\begin{oldpart}{need to integrate this somewhere. CL: I will check other proposals to see how they did it; the Guide does not really prescribe anything.}
\paragraph{Global Risk Management}
The crucial problem of \pn (and similar endeavors that offer a new basis for communication
and interaction) is that of community uptake: Unless we can convince scientists and
knowledge workers industry to use the new tools and interactions, we will
never be able to assemble the large repositories of flexiformal mathematical knowledge we
envision. We will consider uptake to be the main ongoing evaluation criterion for the network.
\end{oldpart}



1. Risks

- more difficult that expected (probability: low, severity medium). Mitigation: at least some systems

- too many specific proofs (probability: low (empirical evidence in informal maths), severity medium). Mitigation: do not translate these proofs and start understanding why they need strong axioms, but translate the rest (basic maths).

- too high complexity (time and memory), difficulty to scale up (probability medium, severity medium). Mitigation: lower the objectives (basic maths, use more powerful machines, wait for Moore's law to help you).

- one partner leaves (probability low) or does not deliver (probability medium). Mitigation: downsize the project.

- difficulty to find people (doctoral students, post-docs) in some countries. Mitigation use the size of the network to find more peoples in others.

- Logipedia splits into several libraries: face the risk  (we have avoided to have a classical and a constructive logipedia, a predicative one and a non-predicative one, the diversity of theories expressed in logipedia permits to make the probability very low). 

- a beautiful encyclopedia, but nobody cares (probability low). Mitigation: improve the interface, the communication, make it more completed

2.  Opportunities

- More people want to join: model checking, sat solvers… 

- math teachers want to use it for teaching


{\color{red} Draft a list of milestones}

\subsection{Milestones}\label{sec:milestones}

\begin{todo}{from the proposal template}
  Milestones are control points where decisions are needed with regard to the next stage
  of the project. For example, a milestone may occur when a major result has been
  achieved, if its successful attainment is a requirement for the next phase of
  work. Another example would be a point when the consortium must decide which of several
  technologies to adopt for further development.

  Means of verification: Show how you will confirm that the milestone has been
  attained. Refer to indicators if appropriate. For examples: a laboratory prototype
  completed and running flawlessly, software released and validated by a user group, field
  survey complete and data quality validated.
\end{todo}

\ednote{maybe automate the milestones}

\ednote{Rabe: I suggest having exactly 3 milestones, namely at months 18, 36, and 48 (corresponding to the EU's review schedule), possibly more milestones in the beginning e.g., at months 6 and 12}

\begin{milestones}
  \milestone[id=kickoff,verif=Inspection,month=1]
    {Organization setup}
    {Set up the organizational infrastructure of the project: mailing lists, web site, consortium agreement, activity tracking, \ldots}

  \milestone[id=logipedia-v1,verif=Inspection,month=12]
     {Logipedia v1}
     {Release of a first version of Logipedia with HOL Light standard library and parts of Matita standard library in 5 different systems: Coq, Matita, Lean, HOL and PVS}

  \milestone[id=coq-stdlib,verif=Inspection,month=24]
     {Coq in Logipedia}
     {Integration of most of Coq standard library in Logipedia}

  \milestone[id=isabelle-stdlib,verif=Inspection,month=24]
     {Isabelle/HOL in Logipedia}
     {Integration of most of Isabelle/HOL standard library in Logipedia}

  \milestone[id=compcert,verif=Inspection,month=36]
     {CompCert in Logipedia}
     {Integration of most of the CompCert library in Logipedia}

  \milestone[id=logipedia-v2,verif=Inspection,month=36]
     {Logipedia v2}
     {Release of a second version of Logipedia integrating important parts of the libraries of Isabelle, Coq, Matita and HOL4, and their translations in other systems}

  \milestone[id=atelierb,verif=Inspection,month=48]
     {Atelier B in Logipedia}
     {Release of a tool able to translate a complete development in Atelier B into a complete Dedukti proof}

\end{milestones}


\section{Consortium as a whole}\label{sec:consortium}


\begin{todo}{}\color{red}

  The individual members of the consortium are described in a separate
  section 4. There is no need to repeat that information here.

  Describe the consortium. How will it match the project’s objectives, and bring together the necessary expertise? How do the members complement one another (and cover the value chain, where appropriate)? 

  In what way does each of them contribute to the project? Show that each has a valid role, and adequate resources in the project to fulfil that role. 

  If applicable, describe the industrial/commercial involvement in the project to ensure exploitation of the results and explain why this is consistent with and will help to achieve the specific measures which are proposed for exploitation of the results of the project (see section 2.2). 

  Other countries and international organisations: If one or more of the participants requesting EU funding is based in a country or is an international organisation that is not automatically eligible for such funding (entities from Member States of the EU, from Associated Countries, from one of the countries in the exhaustive list included in General Annex A of the work programme, and, under specific conditions, from further countries identified in the work programme1,are automatically eligible for EU funding), explain why the participation of the entity in question is essential to carrying out the project .
\end{todo}

\begin{todo}{from the proposal template}
  Describe how the participants collectively constitute a consortium capable of achieving
  the project objectives, and how they are suited and are committed to the tasks assigned
  to them. Show the complementarity between participants. Explain how the composition of
  the consortium is well-balanced in relation to the objectives of the project.

  If appropriate describe the industrial/commercial involvement to ensure exploitation of
  the results. Show how the opportunity of involving SMEs has been addressed
\end{todo}

The project partners of the \pn project have a long history of successful collaboration;
Figure~\ref{tab:collaboration} gives an overview over joint projects (including proposals) and
joint publications (only international, peer reviewed ones).

\jointorga{Fau,Bol}% CICM
\jointorga{Inn,Bol}% CICM
\jointpub{Fau,Bol}% CICM paper
\jointpub{Tum,Bol}% CICM paper
\jointpub{Fau,TUM}%
%\jointsup{Fau,}
\jointsoft{Fau,Tum}% Isabelle Extension
\jointsoft{Fau,Bol}% Coq exporter
\jointpub{Inr,Bol}% ELPI
\jointproj{Inr,Bol}% MoWGLI


\jointpub{Pra,Stu}% CADE 2015 paper
\jointOrga{Inr,Stu}% 3rd PAAR, 5th PAAR

\coherencetable

\subsection{Subcontracting}\label{sec:subcontracting}

\begin{todo}{from the proposal template}
  If any part of the work is to be sub-contracted by the participant responsible for it,
  describe the work involved and explain why a sub-contract approach has been chosen for
  it.
\end{todo}

The tasks \taskref{instrumentation}{isabelle},
\taskref{libraries}{afp} (both handled by \site{Tum}) and part of
task~\taskref{structuring}{strontorepml} (handled by \site{Fau}) will
be carried out by subcontracting Dr.\ M.\ Wenzel.  Each subcontract
will cover roughly the equivalent of $3$ person-months.  Concretely,
it concerns the export of proof terms and other data from the Isabelle
system.  Wenzel is the main Isabelle developer and has spent the
last $10$ years building the technological prerequisites for the
required work and is the natural person to carry it out.  However, he
has left academia and started his own company that specializes on
Isabelle kernel development and routinely carries out subcontracts for
Isabelle-related research projects.  Therefore, a subcontract is the
best option as developing the necessary expertise in-house would take
an additional 6-12 person-months per task.  \site{Fau} has already worked with
Wenzel in similar subcontracts twice before (including the
OpenDreamKit EU infrastructure project), and the collaborations have
been very effective and efficient. Wenzel obtained his Ph.D. at
\site{Tum} advised by Nipkow.

The task \taskref{instrumentation}{isabelle} (handled by \site{Tum})
requires special assistance by Dr.\ David Matthews (PROLINGUA LTD,
Edinburgh): As provider of the underlying Poly/ML infrastructure,
Matthews is in a unique position to provide extra scalability of ML
heap management, and thus allow Isabelle to export more library
material.


\subsection{Other Countries}\label{sec:other-countries}
\begin{todo}{from the proposal template}
  If a one or more of the participants requesting EU funding is based outside of the EU
  Member states, Associated countries and the list of International Cooperation Partner
  Countries\footnote{See CORDIS web-site, and annex 1 of the work programme.}, explain in
  terms of the project’s objectives why such funding would be essential.
\end{todo}

\subsection{Additional Partners}\label{sec:assoc-partner}
\begin{todo}{from the proposal template}
  If there are as-yet-unidentified participants in the project, the expected competences,
  the role of the potential participants and their integration into the running project
  should be described
\end{todo}

%%% Local Variables:
%%% mode: latex
%%% TeX-master: "propB"
%%% End:


{\color{red} A table with all partners in lines and Key expertise in colum
  and explain who is good at what}


\section{Resources to be Committed}\label{sec:resources}

\begin{todo}{}\color{red}
Please make sure the information in this section matches the costs as stated in the budget table in section 3 of the administrative proposal forms, and the number of person months, shown in the detailed work package descriptions.

Please provide the following:

- a table showing number of person months required (table 3.4a)

- a table showing ‘other direct costs’ (table 3.4b) for participants where those  costs exceed 15\% of the personnel costs (according to the budget  table in section 3 of the administrative proposal forms) and participants providing trans-national access under this project and incurring travels and subsistence costs for supporting users' access.

Please note that the distribution of resources between access (TA/VA), JRA, and NA components must be duly justified.
\end{todo}


{\color{red} A table with WP in columns and parners in line and
  ressources in pm}

\subsection{Travel Costs and Consumables}\label{sec:travel-costs}

\subsection{Subcontracting Costs}\label{sec:subcontracting-costs}

As explained in Section~\ref{sec:subcontracting}, \site{Tum} asks for
EUR~50.000 and \site{Fau} for EUR~20.000 to subcontract Dr.\ M.\ Wenzel to
carry out (part of) the work in task
\taskref{instrumentation}{isabelle}, \taskref{libraries}{afp} and
\taskref{structuring}{strontorepml}. Further EUR~15.000 are required
by \site{Tum} to subcontract Dr.\ D.\ Matthews to participate in
critical parts of task \taskref{instrumentation}{isabelle}.
\ednote{Rabe@Dowek: This is part
  of the WP7 budget. But it is not included in the PMs in WP7 because
  money for subcontracts must be declared in a different column in the
  official EU tables. Instead, it is listed here.}




\subsection{Other Costs}

%%% Local Variables: 
%%% mode: LaTeX
%%% TeX-master: "propB"
%%% mode: flyspell
%%% ispell-local-dictionary: "english"
%%% End: 

% LocalWords:  pn newpage site-FAU site-efo site-baz jointpub efo baz
% LocalWords:  jointproj coherencetable assoc-partner
