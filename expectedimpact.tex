\subsection{Wider, simplified, and more efficient access}

The shift from informal, pencil and paper, proofs to formal
computerized proof has been a major improvement on the never ending
quest for logical rigor, with a strong impact both on computer
science, where safety and security can be dramatically improved with
the use of formal methods and on mathematics, where much more complex
proofs can be built.

But this major step forward also has a negative side effect: we have
evolved from a time where we had (informal) proofs of Fermat's little
theorem, to a time where we have (formal) proofs in Coq, in Matita, in
HOL Light, in PVS, etc.  of this theorem, jeopardizing the
universality of logical truth.

This loss of universality of logical truth is the main obstacle to the
access to formal proofs in the communities of computer scientists and
mathematicians, but also researchers of other disciplines, engineers,
and students. As a consequence, the use of formal proofs is restricted
to a community of specialists, that has fortunately been growing with
time, but that is still too small, compared to the needs, for formal
proofs in the digital society.  Our long-term goal is to resurrect the
universality of logical truth, in order to build a strong formal proof
community including specialists and non-specialists, making
formal proofs findable, accessible, interoperable, and reusable.

This requires to express the theories implemented in these systems in
a common logical framework, with axioms and rewrite rules, in order to
be able to say, not that a proof is expressed in the theory of one
system or of another, but which axioms and rewrite rules are used in
this proof, as we have been used to since the development of
non-Euclidean geometries. In such a shared public encyclopedia
providing virtual access, each user, in research, industry, and
education, can find the formal proofs she needs in the logic she
wants, regardless the logic and system this proof has been developed
in.  Alignment of isomorphic structures, both across libraries, and
within each library, provides a wider and simplified access.

Because it constitutes a central place, such a encyclopedia can foster
projects that would make less sense for a specific library. For instance,

\begin{itemize}
\item indexing mechanisms for mathematical formulas, and search engines
  that are specialized to query such formulas,

  \item a structure of mathematical knowledge in theories, books,
    chapters, categories of users (high school students, researchers),
    that can be inherited from the libraries imported in 
      Logipedia, from concept alignment, and from clustering
    algorithms in the dependence graph of the encyclopedia,

  \item ergonomic user interfaces, that allows a navigation in the
    structure of the encyclopedia, and that can be specialized from
    some domains (safety, geometry...) or some category of users,

  \item programmatic access through an Application Programming
    Interface, and a package distribution system, so that the
    mathematical knowledge can be used not only by humans, but also by
    software.
\end{itemize}

Because of its size, such an encyclopedia, provides a better dataset
for machine learning algorithms, and more generally statistical
analysis of the properties of mathematical developments.

\subsection{New or more advanced research infrastructure}

Proof systems are research infrastructure. Logipedia is not yet
another infrastructure of this kind, but it is a research
infrastructure of a completely new kind, that integrates proof systems
through data sharing.  Such as new research infrastructure will, of
course, impact research in many ways.

\begin{itemize}

\item Computer scientists will be able to prove safety and security
  properties of the software they develop faster and at a lower cost
  because they can access to already developed proofs, independently
  of the system they use.

\item When building new proof systems, computer scientists will not
  need to start from scratch, but will be able to start with an already
  existing data base of proofs.

\item In Automated Theorem Proving, computer scientists will be able
  to use already proved theorems as axioms, to enhance the power of
  their tools.

\item In their publications, computer scientists often provide access
  to formal proofs of the results they present (as required or
  suggested by several conferences and journals). They will be able to
  use Logipedia as a universal repository for such proofs, some of
  which would be available across systems, and these proofs being
  available for a long time.

\item Computer scientists conducting research on machine learning
  using mathematical dataset will have access to a wider database.

\item When they are unsure of the correctness of a proof,
  mathematicians will be able to formalize it faster and at a lower
  cost because they can access to already developed proofs,
  independently of the system they use.

\item Mathematicians who refer to a formal proof in one or their
  publication will be able to use Logipedia as a universal
  repository for such proofs, some of which would be available across
  systems.

\item Mathematicians interested in reverse mathematics will be able to
  analyze, in an easier way, the axioms used in each proof, when they
  have access to this proof formalized.

\item The results will be archived for a long time increasing the
  reproducibility of results, both in mathematics and computer science.

\item As mathematics and software are in any part of modern science,
  Logipedia will also have an impact on other sciences. In
  particular because proving properties of a piece of software driving
  a car or piloting an aircraft require to formalize part of the
  physical world in which this piece of software evolves.
\end{itemize}
  
\subsection{Operators develop synergies}

The formal proof community is currently a scattered community, each
sub-community being centered around its own system, its own theory,
and its own library.

To build a stronger community, where the researchers develop
strategies, it is not sufficient to talk to each other, or to organize
conferences, but these sub-communities must exchange data and work
together on a joint project to exchange those data.
Expressing these data in a common encyclopedia will
lead the developers of various systems to express the theories they
implement in a common logical framework, yielding a better understanding
to the similarities and differences between these theories.
Developing synergies will induce less work duplication and will increase
the efficiency of the community as a whole.

Working on common projects will not only increase the communication
between relatively close communities, such as the Coq and 
  Agda communities that meet every year at the TYPES conference, but
also to more distant communities, such as the TYPES community, the HOL
community (that already meet around the OpenTheory standard),
the B and TLA+ communities, and the Mizar community.

More importantly, this data exchange between researchers and
engineers, and this evolution towards a standard, will allow a better
cooperation between research and industry and suppress one of the main
obstacles to the diffusion of formal proofs in industry.

This evolution towards a standard and this resurrection of the
universality of logical truth will also suppress one of the main
obstacles to the diffusion of formal proofs in the community of
working mathematicians and in and education, just like the development
of the Html standard induced a renewal of document sharing in general
and the definition of predicate logic induced a renewal of logic in
the 1930's.

Sharing a logical framework will also allow new synergies between 
the formal proof community and the Automated Theorem Proving community.

It will also allow a better communication with machine learning community.

We have already organized two Logipedia workshops that have proven to
be very valuable to develop joint project and synergies.  This project
will permit to organize wider international events on this topic of
sharing formal proofs, in academia, industry, education, and
publishing.

\subsection{Innovation is fostered through a reinforced partnership of
research infrastructures with industry}

Formal methods are now an important part of some advanced industrial
projects. Mastering formal methods is key to give Europe a competitive
advantage in conquering the market of autonomous cars, trains, planes
or drones, or the block-chains and crypto-currencies. But this
penetration of formal methods in industry hits the same obstacle that
researchers often promote one method, theory or system, while their
industrial partners are in search of universality. We expect to make
formal proofs more accessible to industry by avoiding each project to
redevelop elementary proofs, but instead benefit of the formalization
work shared with other communities.

Several European companies are member of the project,
as contributors or as members of the future Logipedia
club of industrial users.

From the industry point of view the key exploitable results are threefold:

\begin{enumerate}
\item Cross-verification.

In the current state of affairs, when a company proves a piece of
software correct using a system $X$, its client, or the certification
authorities can check the proof developed by this company, but then
need to use the same system $X$ to check this proof, so they need to
trust the system $X$, which is a limitation. Several actors want to be
able to check the proofs using an independent proof system, and even
one they have developed themselves.

A side effect of the construction of the Logipedia platform is
to incent all the proof systems to be able to produce proofs in a
common language. Such proofs can then be all be checked by
Dedukti, and several other systems.

Moreover, the certification authorities can develop their own
proof-checker (the development of such a proof-checker takes a few
weeks) so that they do not need to trust anyone else.

\item Towards standardization.

The lack of standards is currently a major obstacle to the development
of formal methods in industry. Although we consider that starting a
standardization process is still premature, this project will allow us
to experiment with a common language that we shall improve until we
reach the point where it can be proposed as a standard.

\item Sustainability.

When an industrial project is over, it is sometimes difficult,
specially for small companies, to keep an archive of their work over
a long period of time. In formal methods, most of the projects are a
two-stages rocket. The first stage of the rocket contains basic
developments that can be shared between the company and its
competitors. The second contains developments that are specific to the
project and that often contain industrial secrets.

Sharing the first stage on a public encyclopedia, while keeping the
second secret, contributes to the sustainability of the
developments. They can, for instance, be reused decades later, and
cannot be lost by the company.

\item Interoperability.

Several companies use several proof systems. First because some proof
systems are better for some application domains and other are better
for others, and also because these companies hire researchers and
engineers that have different cultures and are more efficient using the
tools they know.

A side effect of the construction of the Logipedia platform is
that such development are made interoperable. First because a proof
developed in one system can be translated into another. Second because
proofs developed in different systems can be combined in 
Logipedia itself.

- {\color{red} Benjamin M.}:

Industrial programme verification tools usually rely only on one single
proof system and any missing feature of its library forces the
end-user to prove complicated theorems that probably exist in other
proof systems. Thanks to Logipedia it becomes possible to access all
standard libraries and proofs of all systems. It makes the overall
cost of the proof of programmes much less costly.

\end{enumerate}

The participation of industry to the project is twofold.
First, CEA LIST, Clearsy, Edukera, MED-EL, OCamlPro, Prove\&Run, and
SystemX, are full partners of the project and will contribute to its
work packages and tasks.

Second, we will build over the project duration time a {\em club of
  industrial users of Logipedia}. This club already contains
18 members.

\begin{framed}
\begin{center}
  {\bf \Large The current members of the club of industrial users}
\end{center}
\begin{itemize}
\item Alstom,
\item CEA LIST,
\item ClearSy,
\item Edukera,
\item Facebook France,
\item IBM Research,
\item MED-EL
\item Mitsubishi Electric R\&D Centre Europe,
\item Nomadic Labs,
\item OCamlPro,
\item Origin Labs,
\item Prove\&Run,
\item TrustInSoft,
\item RATP,
\item Siemens,
\item System X,
\item Systerel,
\item TrustInSoft.
\end{itemize}

ANSSI?

\end{framed}

These companies are working in the area of transportation, health
care, energy, cybersecurity, block-chain...

This club will organize two of meetings every year where the
industrial members will:
\begin{itemize}
\item learn about the advancement and new features of the infrastructure,
\item give feedback on the use of the infrastructure,
\item provide tests sets and proofs to include into Logipedia,
\item propose new features, services or research directions for the projet,
\item participate to the project self assessment.
\end{itemize}
From the point of view of its members, this club is a unique
opportunity for technology watch. Our empirical observations show that
many companies, working in safety critical areas, would like to be
more involved in the development of formal methods, but that the first
step into using such methods is often too costly, and that we must
offer them a smoother way to get into this technology. Such a club
where the industrial users will be able to develop a base culture in
formal methods, share experience with other companies, and conduct
technology watch on a regular basis is an efficient way to disseminate
formal methods in the European industry.

\begin{framed}
{\bf \Large Impact of Logipedia on the transportation industry}

{\color{red} Yamine}  
\end{framed}

\begin{framed}
{\bf \Large Impact of Logipedia on the block chain industry}

{\color{red} Raphaël}  
\end{framed}

\begin{framed}
{\bf \Large Impact of Logipedia for the energy industry}

{\color{red} François}  
\end{framed}

\subsection{Closer interaction between a larger number of researchers}

We have already discussed lengthily the impact of a joint project like
Logipedia on the community of academic and industrial researchers in
formal methods and, more generally, in logic.

We want to insist also on the fact that everywhere mathematics and
computer science are used (in physics, in some parts of biology and
social sciences, in engineering, in particular through simulation,
etc.) a quest for higher level of rigor should pave the way for a
development of formal proofs, but that this development is slowed down
by the multiplicity of theories, systems and libraries that make the
first step difficult for beginners. A common reference infrastructure
should simplify the access of non-specialists to formal methods.  For
instance, scientists willing to formalize hybrid systems should have
access to a good analysis library, whatever system they use.

Among the communities of researchers, one on which we can have a real
impact during the project is the community of working mathematicians.
Some of them have started using proof systems and we must take care that they
can have access to the best libraries of basic mathematics, whatever system
they use.


\begin{framed}
\begin{center}
  {\bf \Large The current members of the club of academic users}
\end{center}
  
Assia Mahboubi
  
Karl Palmskog <palmskog@gmail.com> 

Paul Jackson (not yet contacted)

Simon Foster <simon.foster@york.ac.uk> (Burkhart)

Achim D. Brucker <A.Brucker@exeter.ac.uk> (Burkhart)

Angeliki Koutsoukou-Argyraki 

Kevin Buzzard (not yet contacted)

Sébastien Gouëzel (not yet contacted)
\end{framed}

\subsection{Better management of the continuous flow of data.}

A shared encyclopedia allows a better sustainability of the formal
proofs developed over time. Too many formal proofs developed in the
past are not available any more.

UPLOAD feature of Logipedia

Service to the certification authorities

\subsection{Socio-economic impact}

{\color{red} Safety and security of software contribute to a more ethical
  and safer socity}

Make Europe a leader in ITP/ATP

EU Spent money on TYPES, Nicolas Tabareau's ERC. Formath, Circo... 

Now we integrate the results of all this research

{\color{red} Key exploitable results?}

\subsection{Education}

{\color{red} JN: I wonder if we should not extend the club of users in
  education to the club or teachers/researchers.}

Scientific research and education at all levels are concerned with the discovery, verification, communication, archival and usage of mathematical results.
These tasks have been supported by physical books, conferences and other means.

The avaibility of a formal online encyclopedia which propose in a single place the communication, archival and verification of mathematical  knowledge will be of prime importance for researchers and teachers.

Education to formal methods in computer science and to formal proofs
in mathematics always hits the same obstacle: the need to choose a
specific theory or system, the need to focus on foundational issues in contradiction with the claimed universality of logical truth. 

Education to formal methods and formal proofs will gain in universality once it will be demonstrated that this choice amounts to include, or not, a few axioms and rewrite
rules. 
The usage of formal proofs in the class room is also slown down by the lack of a large and well organized library of results.
We defend that this renewal of logic education at university level and before is of prime importance in our ``post-truth era''.

The project can lead to a societal breakthrough opening the use of formal proofs by a larger group of users, from experts users comming from the formal proof community  to a group of non expert users (mathematicians, education and researchers in other science using mathematical statements). 

The club of users in education gathers researcher and teachers who are
already actively using formal proof for teaching in computer science,
mathematics and logic but are not necessarily members of the cummunity
of researchers in formal theorem proving. These early adopters, will
provide continuous feedback to the project members on the usuability
and accessibility of the system from a user point of view.

Math students 

{\color{red} Club of users in education (Julien, can you explain the impact here?)}

{\color{red} Key exploitable results?}

{color{red} JN:  some potential ideas:}
\begin{enumerate}
\item Summer schools for introducing mathematical researchers to ITP
\item For highschool teachers: seminars/lectures about  the role of logic in maths teaching and the use of proof assistants in class.
\item For university teachers: creation and dissemination of teaching material for introduction to the concept of proof to fresh maths students.   
\item Development of an Edukera like, open source, point and click user interface for interactive formal theorem proving in the classroom : the "scratch" of maths  (a Isar like language for the proof structure but with a point and click user interface)
\item formalization of  highschool curriculums for different european countries
\item link between the project and the ThEdu community
\end{enumerate}

\subsection{Publishing}

\subsection{Open Science}


%%% Local Variables:
%%%   mode: latex
%%%   mode: flyspell
%%%   ispell-local-dictionary: "english"
%%% End:
