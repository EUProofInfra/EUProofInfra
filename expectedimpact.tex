The proposal is in coherence with the goals of the call INFRAIA-02-2020:
Integrating activities for starting communities (listed on page 56 of
the part 4 of Horizon 2020 Work Programme 2018-2020). It also clearly
refers to one of the mentioned cross-cutting activities: open science.

\subsection{Wider, simplified, and more efficient access}

The shift from informal, pencil and paper, proofs to formal
computerized proof has been a major improvement on the never ending
quest for logical rigor, with a strong impact both on computer
science, where safety and security can be dramatically improved with
the use of formal methods and on mathematics, where much more complex
proofs can be built.

But this major step forward also has a negative side effect: we have
evolved from a time where we had (informal) proofs of Pythagoras'
theorem or Fermat's little theorem, to a time where we have (formal)
proofs in {\sc Coq}, in {\sc Matita}, in {\sc HOL Light}, in {\sc
  PVS}, etc.  of these theorems, jeopardizing the universality of
logical truth.

This loss of universality of logical truth is the main obstacle to the
access to formal proofs in the communities of computer scientists and
mathematicians, but also researchers of other disciplines, engineers
and students.  Our long-term goal is to resurrect the universality of
mathematical truth in order to build a strong formal proof community
including specialists and non-specialists.

This requires to express the theories implemented in these systems in
a common logical framework, with axioms and reduction rules, in order
to be able to say, not that a proof is expressed in one system or in
another, but which axioms and reduction rules are used in this proof,
as we have been used to since the development of non-Euclidean
geometries. In such a shared public encyclopedia providing virtual
access, each user, in research, industry, and education, can find the
formal proofs she needs in the logic she wants, regardless the logic
and system this proof has been developed in.  Alignment of isomorphic
strucutes, both accross libraries, and inside the same library,
provides a wider and simplied access.

Because it constitutes a central place, such a encyclopedia can foster
projects that would make less sense for a specific library. For instance,

\begin{itemize}
\item indexing mechanisms for mathematical formulas, and search engines
  that are specialized to query such formulas,

  \item a structure of mathematical knowledge in theories, books,
    chapters, levels (from high school students to research), that can
    be inherited from the libraries imported in {\sc Logipedia}, from
    concept alignment, and from clustering algorithms in the dependence
    graph of the encyclopedia, 

  \item ergonomic user interfaces, that allows a navigation in the
    structure of the encyclopedia, and that can be specialized from
    some domains (safety, geometry...) or some category of users
    (high school students, researchers...)

  \item programmatic access through an Application Programming
    Interface, and a package distribution system, so that the
    mathematical knowledge can be used not only by humans, but also by
    software.
\end{itemize}

Because of its size, such an enclyclopedia, provides a better dataset
for machine learning algorithms, and more generaly statistical
analysis of the properties of mathematical developements.

{\color{red} Talk about licences here. Is cc-by ok for everyone?}


\subsection{New or more advanced research infrastructure}

Proof systems are research infrastructure. {\sc Logipedia} is not yet
another infrastructure of this kind, but it is a research
infrastructure of a completely new kind that integrates proof systems
through data sharing.  Such as new research infrastructure will, of
course, impact research in many ways.

\begin{itemize}

\item Computer scientists will be able to prove safety and security
  properties of the software they develop faster and at a lower cost
  because they can access to already developed proofs, independently
  of the system they use.

\item When building new proof systems, computer scientists will not
  need to start from scracth, but will be able to start with an already
  existing data base of proofs.

\item In Automated Theorem Proving, computer scientisits will be able
  to use already proved theorems as axioms, to enhance the power of
  their tools.

\item In their publications, computer scientists often provide access
  to formal proofs of the results they present (as required or
  suggested by several conferences and journals). They will be able to
  use Logipedia as a universal repository for such proofs, some of
  which would be available accross systems, and these proofs being
  available for a long time.

\item Computer scientists conducting research on machine learning using
mathematical dataset have access to a wider database.

\item When they are unsure of the corrrectness of a proof, mathematicians
will be able to formalize it faster and at a lower cost because they
can access to already developed proofs, independently of the system
they use (cite Vladimir).

\item Mathematicians who refer to a formal proof in one or their publication
will be able to use Logipedia as a universal repository for such
proofs, some of which would be available accross systems.

\item Mathematicians interesed in revere maths will be able to analyze, in
an easier way, the axioms used in each proof, when they have access to
this proof formalized.

\item The results will archived for a long time increasing the
  reproducibility of results, both in mathematics and computer science.

\item As mathematics and software are in any part of modern science,
  {\sc Logipedia} will also have an impact on other sciences. In
  particular because proving properties of a piece of software driving
  a car or piloting an aircraft require to formalize part of the
  physical world in which this piece of software evolves.
\end{itemize}
  
\subsection{Operators develop synergies}

The formal proof community is currently a scattered community, each
sub-community being centered around its own system, its own theory,
and its own library.

To build a stronger community, where the researchers develop strategies,
it is not sufficient to talk to each other, or to organize conferences,
but these sub-communities must exchange data and work together on project to
exchange those data.

For instance, expressing these data in a common encyclopedia will
lead the developers of various systems to express the theories they
implement in a common logical framework, yielding a better understanding
to the similarities and differences between these theories.

Developing synergies will induce less work duplication and will increase
the efficiency of the community as a whole.

Working on common projects will not only increase the communication
between relatively close communities, such as the {\sc Coq} and {\sc
  Agda} communities that meet every year at the Types conference, but
also to more distant communities, such as the Types community, the HOL
community (that already meet around the OpenThery standard), the B and
TLA+ communities, and the Mizar community.

More importantly, this data exchange between researchers and
engineers, and this evolution towards a standard, will allow a better
cooperation between research and industry and suppress one of the main
obstacles to the diffusion of formal proofs in industry.

This evolution towards a standard and this resurrection of the
universality of logical truth will also suppress one of the main
obstacles to the diffusion of formal proofs in the community of
working mathematicians and in and education, just like the development
of the html standard induced a renewal of document sharing in general
and the definition of predicate logic induced a renewal of logic in
the 1930's.

Sharing a logical framework will also allow new synergies between 
the formal proof community and the Automated Theorem Proving community.

It will also allow a better communication with machine learning community.

We have already organized two {\sc Logipedia} meetings that have
proven to be very valuable to develop joint project and synergies.
The project will permit to develop wider international events.

\subsection{Innovation is fostered through a reinforced partnership of research
infrastructures with industry}

Formal methods are now an important part of some advanced
industrial projects. Mastering formal methods is key to
give Europe a competitive advantage in conquering the market of
autonomous cars, trains, planes or drones, or the block-chains and crypto-currencies. But this penetration of
formal methods in industry hits the same obstacle that researchers
often promote one method, theory or system, while their industrial
partners are in search of universality. We expect to make formal
proofs more accessible to industry by avoiding each project to
redevelop elementary proofs, but instead benefit of the formalization
work shared with other communities.

Several European companies are member of the project,
as contributors or as members of the future {\sc Logipedia} industrial club.

From the industy point of view the key exploitable results are threefold:

\begin{enumerate}
\item Cross-verification.

In the current state of affairs, when a compagny proves a piece of
software correct using a system $X$, its client, or the certification
authorities can check the proof developed by this compagny, but then
need to use the same system $X$ to check this proof, so they need to
trust the system $X$, which is a limitation. Several actors want to be
able to check the proofs using an indepedent proof system, and even one
they have developed themselves.

A side effect of the construction of the {\sc Logipedia} platform is
to incent all the proof systems to be able to produce proofs in a
common language. Such proofs can then be all be checked by {\sc
Dedukti}, and several other systems.

Moreover, the certification authorities can develop their own
proof-checker (the development of such a proof-checker takes a few weeks) so that they do not need to trust anyone else.

\item Towards standardization.

The lack of standards is currently a major obstacle to the development
of formal methods in industry. Althoug we consider that starting a
standardization process is still premature, this project will allow us
to experiment with a common language that we shall improve until we
reach the point where it can be proposed as a standard.

\item Sustainability.

When an industrial project is over, it is sometimes difficult,
specially for small compagnies, to keep an archive of their work over
a long period of time. In formal methods, most of the projects are a
two-stages rocket. The first stage of the rocket contains basic
developements that can be shared between the compagny and its
competitors. The second contains developments that are specific to the
project and that often contain industrial secrets.

Sharing the first stage on a public encyclopedia, while keeping the
second secret, contributes to the sustainability of the
developements. They can, for instance, be reused decades later, and
cannot be lost by the compagny.

\item Interoperability.

Several compagnies use several proof systems. First because some proof
systems are better for some application domains and other are better
for others, and also because these compagnies hire researchers and
engineers that have different cultures and are more efficient using the
tools they know.

A side effect of the construction of the {\sc Logipedia} platform is
that such developement are made interoperable. First because a proof
developed in one system can be translated into another. Second because
proofs developed in different systems can be combined in {\sc
Logipedia} itself.
\end{enumerate}

The participation of industry to the project is twofold. First, the
project will develop a {\em club of industrial users of Logipedia} and
organize a couple of meetings every year where the industrial members
will:
\begin{itemize}
\item learn about the advancement and new features of the infrastructure,
\item give feedback on the use of the infrastructure,
\item provide tests sets and proofs to include into Logipedia,
\item propose new features, services or research directions for the projet,
\item participate to the project self assessment.
\end{itemize}

A number of companies already agreed to join our industrial club:
Alstom, RATP, Mitsubishi Electric R\&D Centre Europe, ClearSy,
Systerel, Nomadic Labs, OCamlPro, Origin Labs, TrustInSoft. Some are
working on railways, others on software verification, and others on
block-chain applications.




From the industrial point of view the participation to this club
is a way to do some technology watch. 

Then, some compagnies, specially some related to the {\sc B} community
and the railway industry, are part full partners of the project and
will contribute to the production of {\sc Logipedia} proofs from
{\sc B} provers. 

{\color{red} Club of industrials (Frédéric, can 
you explain the impact here?)}

{\color{red} Impact for the {\sc B} community and the railway industry
  (Yamine?)}

{\color{red} Key exploitable results?}

- Recall the importance of safety and security

- Block chain ({\color{red} Raphaël writes this}), smart contract : small program, lots of money. Touches different areas (cooperation of specialized tools)

- Medel (Cezary)



\subsection{Closer interaction between a larger number of researchers...}


Researchers from different groups use a common logical framework to
describe the theory implemented in their system.

- Accross fields (math computer science philosophy)

- Easing the devepeloment of formal proofs wolll bring engineers
(hybrid system can be formalized if we have a good analysis library
such an accessible library witll bring more researchers from this
field)

- Scinetists from other disciplines to the communities

- Connection with natural language 

- Atttracting working mathematicians

- {\color{red} Benjamin M.}:

Industrial program verification tools usually rely only on one single
proof system and any missing feature of its library forces the
end-user to prove complicated theorems that probably exist in other
proof systems. Thanks to Logipedia it becomes possible to access all
standard libraries and proofs of all systems. It makes the overall
cost of the proof of programs much less costly.


{\color{red} Key exploitable results?}

- Representation of theories implemented in various systems

- Some specific developements will be kept secret by the industrial parners
but sharing the basics  (90\%) is win-win

{\color{red} Speak here about the club of Academic users

  
Karl Palmskog <palmskog@gmail.com> 


Jacques Fleuriot (not yet contacted)

Paul Jackson (not yet contacted)

Simon Foster <simon.foster@york.ac.uk> (Burkhart)

Achim D. Brucker <A.Brucker@exeter.ac.uk> (Burkhart)
}





\subsubsection{Education}

{\color{red} JN: I wonder if we should not extend the club of teachers to the club or teachers/researchers.}

Scientific research and education at all levels are concerned with the discovery, verification, communication, archival and usage of mathematical results.
These tasks have been supported by physical books, conferences and other means.

The avaibility of a formal online encyclopedia which propose in a single place the communication, archival and verification of mathematical  knowledge will be of prime importance for researchers and teachers.

Education to formal methods in computer science and to formal proofs
in mathematics always hits the same obstacle: the need to choose a
specific theory or system, the need to focus on foundational issues in contradiction with the claimed universality of logical truth. 

Education to formal methods and formal proofs will gain in universality once it will be demonstrated that this choice amounts to include, or not, a few axioms and reduction
rules. 
The usage of formal proofs in the class room is also slown down by the lack of a large and well organized library of results.
We defend that this renewal of logic education at university level and before is of prime importance in our ``post-truth era''.

The project can lead to a societal breakthrough opening the use of formal proofs by a larger group of users, from experts users comming from the formal proof community  to a group of non expert users (mathematicians, education and researchers in other science using mathematical statements). 

The club of teachers gathers researcher and teachers who are already actively using formal proof for teaching in computer science, mathematics and logic but are not necessarily members of the cummunity of researchers in formal theorem proving. These early adopters, will provide continuous feedback to the project members on the usuability and accessibility of the system from a user point of view.

Math students 

{\color{red} Club of teachers (Julien, can you explain the impact here?)}

{\color{red} Key exploitable results?}

{color{red} JN:  some potential ideas:}
\begin{enumerate}
\item Summer schools for introducing mathematical researchers to ITP
\item For highschool teachers: seminars/lectures about  the role of logic in maths teaching and the use of proof assistants in class.
\item For university teachers: creation and dissemination of teaching material for introduction to the concept of proof to fresh maths students.   
\item Development of an Edukera like, open source, point and click user interface for interactive formal theorem proving in the classroom : the "scratch" of maths  (a Isar like language for the proof structure but with a point and click user interface)
\item formalization of  highschool curriculums for different european countries
\item link between the project and the ThEdu community
\end{enumerate}

\subsection{Better management of the continuous flow of data.}

A shared encyclopedia allows a better sustainability of the formal
proofs developed over time. Too many formal proofs developed in the
past are not available any more.

{\color{red} Key exploitable results?}

UPLOAD feature of Logipedia

Sercice to the certification authorities

\subsection{Socio-economic impact}

{\color{red} Safety and security of software contribute to a more ethical
  and safer socity}

Make Europe a leader in ITP/ATP

EU Spent money on TYPES, Nicolas Tabareau's ERC. Formath, Circo... 

Now we integrate the results of all this research

{\color{red} Key exploitable results?}


\section{Key Exploitable results}

Logipedia is a Key Exploitable Results, each domain specific library
(on real analysis, probability...) is also a Key Exploitable Result,
each specific development made available to all provers (CompCert?,
CakeML?...) is a Key Exploitable Result, as well as each proof in
Logipedia. Formalized datastructures and algorithms shared accros
provers are a Key Exploitable Result.

Indexing algorithms and seach engines are Key Exploitable Results.

User interfaces are Key Exploitable results.

API and package distribution systems are Key Exploitable Results.

Logipedia as a repository for proofs referenced in publications, 
in mathematics, in computer science and in other sciences is a
Key Exploitable Result.

- Each import feature to {\sc Logipedia} and each export
feature from {\sc Logipedia} is a 
Key Exploitable Result.

%%% Local Variables:
%%%   mode: latex
%%%   mode: flyspell
%%%   ispell-local-dictionary: "english"
%%% End:
