{\color{red} Speak about barriers to impact and obstacles}

\subsection*{Expected impacts from the call}

\begin{longtable}{|p{0.2\textwidth}|p{0.75\textwidth}|}
%%%%%%%%%%%%%%%%%%%%%%%%%%%%%%%%%%%%%%%%%%%%%%%%%%%%%%%%%%%%%%%%%%%%%%%%%%%%%%
\hline
{\bf Wider, simplified, and more efficient access}&

{\color{red} This is context, remove}

The shift from informal pencil-and-paper proofs to formal computerized
proofs has been a major improvement on the never ending quest for
logical rigor. It has made and continues to make a strong impact both
on computer science, where it allows to dramatically improve safety
and security, and on mathematics, where it can be used to build much
more complex proofs.\\
&
\hspace{0.4cm}
But this major step forward also has a negative side effect: we have
evolved from a time where we had (informal) proofs of Fermat's little
theorem, to a time where we have (formal) proofs in Coq, in Matita, in
HOL Light, in PVS, etc.  of this theorem, jeopardizing the
universality of logical truth.\\
&
\hspace*{0.4cm} This loss of universality of logical truth is the main
obstacle to the access to formal proofs in the communities of computer
scientists and mathematicians, but also researchers of other
disciplines, engineers, and students. As a consequence, the use of
formal proofs is restricted to a community of specialists, that has
fortunately been growing with time, but not fast enough
compared to the needs for formal proofs in the digital society.  Our
long-term goal is to restore the universality of logical truth, in
order to build a strong formal proof community including specialists
and non-specialists, making formal proofs findable, accessible,
interoperable, and reusable.\\ 
&
\hspace{0.4cm}
This requires to express the theories implemented in these systems in
a common logical framework, with axioms and computation rules. This allows
to say which axioms and computation rules are used in
each proof, as we have been used to, since the development of
non-Euclidean geometries.\\
&
\hspace{0.4cm}
In such a shared public encyclopedia providing virtual access, each
user, in research, industry, and education, can find the formal proofs
she needs in the logic she wants, independently of the logic and system in
which this
proof has been developed.  Alignment of isomorphic structures, both
across libraries, and
within each library, provides a wider and simplified access.\\
&
\hspace{0.4cm}
Because it constitutes a central place, such a encyclopedia can foster
projects that would make less sense for a specific library. For instance,\\
\\
&
$\blacktriangleright$
indexing mechanisms for mathematical formulas, and search engines
  that are specialized to query such formulas,
\\
&
$\blacktriangleright$
a structure of mathematical knowledge divided in theories, books,
    chapters, categories of users (high school students, researchers),
    that can be inherited from the libraries imported in
      Logipedia, from concept alignment, and from clustering
    algorithms in the dependence graph of the encyclopedia,
\\
&
$\blacktriangleright$
  ergonomic user interfaces, that allows a navigation in the
    structure of the encyclopedia, and that can be specialized from
    some domains (safety, geometry...) or some category of users,
\\
&
$\blacktriangleright$ programmatic access through an Application Programming
    Interface, and a package distribution system, so that the
    mathematical knowledge can be used not only by humans, but also by
    software.
\\
&
$\blacktriangleright$
an effort will be made to familiarize new researchers and engineers,
who want to contribute to the Logipedia:
an on line tutorial to explain how to use the infrastructure,
summer schools, etc.
\\
&
\colorbox{color2}{\bf Performance indicators:}
\begin{compactitem}
\item number of accesses on the web interface
\item number of downloads from the server
\item number of uses of the teaching interface
\item number of teachers and courses using Logipedia
\end{compactitem}

{\color{red} give targets as figures}

This must match the objectives. 

\\
%%%%%%%%%%%%%%%%%%%%%%%%%%%%%%%%%%%%%%%%%%%%%%%%%%%%%%%%%%%%%%%%%%%%%%%%%%%%%%
\hline
{\bf New and more advanced research infrastructure}
&
Logipedia is a research
infrastructure of a completely new kind, as it integrates proof systems
through data sharing. Such a new research infrastructure will, of
course, impact research in many ways:
\\
&
$\blacktriangleright$ Computer scientists
will be able to prove safety and security
  properties of the software they develop faster and at a lower cost
  because they can access to already developed proofs, independently
  of the system they use.
\\
&
$\blacktriangleright$ When building new proof systems, computer scientists will not
  need to start from scratch, but will be able to start with an already
  existing database of proofs.
\\
&
$\blacktriangleright$ In Automatic Theorem Proving, computer scientists will be able
  to use already proved theorems as axioms, to enhance the power of
  their tools.
\\
&
$\blacktriangleright$ Computer scientists conducting research on machine learning
using mathematical data sets will have access, thanks to Logipedia,
to a wider database of formal proofs.
\\
&
$\blacktriangleright$ When they are unsure of the correctness of a proof,
  mathematicians will be able to formalize it faster and at a lower
  cost because they can access to already developed proofs,
  independently of the system they use.
\\
&
$\blacktriangleright$
Mathematicians interested in reverse mathematics (that is the analysis
of the minimal means to prove a theorem) will be able to analyze, in
an easier way, the axioms used in each proof.
\\
&

$\blacktriangleright$ Scientists, who provide access
  to formal proofs of the results they present (as required or
  suggested by several conferences and journals, in computer science), 
  will be able to
  use Logipedia as a universal repository for such proofs, some of
  which would be available across systems, and these proofs being
  available for a long time.
\\
&
$\blacktriangleright$ As mathematics and software are ubiquitous in 
  modern science, Logipedia will also have an impact on other sciences. 
  In particular, because proving properties of a piece of software driving 
  a car or piloting an aircraft require to formalize part of the physical 
  world in which this piece of software evolves.
\\
& 
$\blacktriangleright$ The results will be archived for a long time, increasing the
  reproducibility of results, both in mathematics and computer science.
\\
&
$\blacktriangleright$ As mathematics and software are ubiquitous in modern science,
  Logipedia will also have an impact on other sciences. In
  particular, proving properties of a piece of software driving
  a car or piloting an aircraft require formalizing part of the
  physical world in which this piece of software evolves.

\\
&
\colorbox{color2}{\bf Performance indicators:}
\begin{compactitem}
\item number of proofs added in Logipedia
\item number of proof systems supported by Logipedia
\end{compactitem}
\\
%%%%%%%%%%%%%%%%%%%%%%%%%%%%%%%%%%%%%%%%%%%%%%%%%%%%%%%%%%%%%%%%%%%%%%%%%%%%%%
\hline
{\bf Operators develop synergies}
&
The formal proof community is currently a scattered community, each
sub-community being centered around its own system, its own theory,
and its own library or libraries.\\
&
\hspace{0.4cm}
To build a stronger community, where the researchers develop
strategies, it is not sufficient to talk to each other, or to organize
conferences, but these sub-communities must exchange data and work
together on a joint project to exchange those data.  Expressing these
data in the same encyclopedia will lead the developers of various
systems to express the theories they implement in the same logical
framework, yielding a better understanding to the similarities and
differences between these theories.  Developing synergies will induce
less work duplication and will increase the efficiency of the
community as a whole.
\\
&
\hspace{0.4cm}
Working on common projects will not only increase the communication
between relatively close communities, such as the Coq and
  Agda communities that meet every year at the TYPES conference, but
also to more distant communities, such as the TYPES community, the HOL
community (that already meet around the OpenTheory standard),
the B and TLA+ communities, and the Mizar community.\\
&
\hspace{0.4cm}
More importantly, this data exchange between researchers and engineers
will allow a better cooperation between research and industry and
suppress one of the main obstacles to the diffusion of formal proofs
in industry.\\
&
\hspace{0.4cm}
This evolution towards a standard and this restoration of the
universality of logical truth will also suppress one of the main
obstacles to the diffusion of formal proofs in the community of
working mathematicians and in education, just like the development
of the HTML standard induced a renewal of document sharing in general
and the definition of predicate logic induced a renewal of logic in
the 1930's.\\
&
\hspace{0.4cm}
Sharing a logical framework will also allow new synergies between the
formal proof community and the Automatic Theorem Proving community
that currently share the same notion of proof, but not the same tools
to manipulate them.
\\
&
\hspace{0.4cm}
It will also allow a better communication with the machine learning community.\\
&
\hspace{0.4cm}
We have already organized two Logipedia workshops that have proven to
be very valuable to develop joint project and synergies.  This project
will permit to organize wider international events on this topic of
sharing formal proofs, in academia, industry, education, and
publishing.\\
&
\colorbox{color2}{\bf Performance indicators:}
\begin{compactitem}
  \item number of proofs developed in one system and used in another
  \item number of interactive proof systems generating a checkable certificate
  \item number of automatic theorem provers generating a checkable certificate 
  \item activity of the clubs (meetings, workshops, etc.)
\end{compactitem}
\\
%%%%%%%%%%%%%%%%%%%%%%%%%%%%%%%%%%%%%%%%%%%%%%%%%%%%%%%%%%%%%%%%%%%%%%%%%%%%%%
\hline
{\bf Innovation is fostered through a reinforced partnership
with industry}
&
Formal methods are now an important part of some advanced industrial
projects. Mastering formal methods is key to give Europe a competitive
advantage in conquering the market of autonomous cars, trains, planes
or drones, or the blockchains and cryptocurrencies. However, this
penetration of formal methods in industry hits the same obstacle that
researchers often promote one method, theory or system, while their
industrial partners are in search of universality. We expect to make
formal proofs more accessible to industry by avoiding having to redevelop
elementary proofs again in each system, but instead benefit from
sharing work on formalization across communities.\\
&
\hspace{0.4cm}
Eight European enterprises are partners.
Others are members of the club of
industrial users.  From the point of view of industry, this
project has several types of key exploitable results:\\
&
$\blacktriangleright$
{\bf Cross-verification.}
In the current state of affairs, when a enterprise proves a piece of
software correct using a system $X$, its client, or the certification
authorities can check the proof developed by this enterprise, but they
can only do so by using the same system $X$. Hence a limitation of
the current state is that they need to trust the system $X$.
Several actors want to be able to check the proofs using an
independent proof system, or even a proof system they have developed
themselves.

A side effect of the construction of the Logipedia platform is
to incent all the proof systems to be able to produce proofs in a
common language. Such proofs can then all be checked by
Dedukti, and several other systems.

Moreover, the certification authorities can develop their own
proof-checker (the development of such a proof-checker takes a few
months) so that they do not need to trust anyone else.\\
&
$\blacktriangleright$
{\bf Towards standardization.}
The lack of standards is currently a major obstacle to the development
of formal methods in industry. Although we consider starting a
standardization process to be still premature, this project will allow us
to experiment with a common language that we shall improve until we
reach the point where it can be proposed as a standard.\\
&
$\blacktriangleright$
{\bf Sustainability.}
When an industrial project is over, it is sometimes difficult,
especially for small enterprises, to keep an archive of their work over
a long period of time. In formal methods, most of the projects are a
two-stages rocket. The first stage of the rocket contains basic
developments that can be shared between the enterprise and its
competitors. The second contains developments that are specific to the
project and that often contains industrial secrets.
Sharing the first stage on a public encyclopedia, while keeping the
second secret, contributes to the sustainability of the
developments. They can, for instance, be reused decades later, and
cannot be lost by the enterprise.\\
&\\
&
$\blacktriangleright$
{\bf Interoperability.}
  Some industrial program verification tools rely only on a single
  proof system, so any missing feature of its library forces the
  end-user to prove complicated theorems that probably exist in other
  proof systems. Logipedia will make it possible to access all
  standard libraries and proofs coming from all systems. It hence
  makes the overall process of proving program correctness much less costly.
  Other industrial program verification tools use several proof
  systems. The reason for this is that some proof systems are better for some
  specific application domains, and also
  because these enterprises hire researchers and engineers that have
  different cultures and are more efficient using the tools they know.
  A side effect of the construction of the Logipedia platform is that
  such cross-system developments are made interoperable: proofs
  developed in one system can be translated into another, and
  proofs developed in different systems can be combined in
  Logipedia itself.\\
&\\
&
\hspace{0.4cm}
The participation of industry to the project is twofold. First,
some enterprises that have a strong formal proof activity (CEA
LIST, Clearsy, Edukera, MED-EL, OCamlPro, Prove\&Run, IRT SystemX,
and Runtime Verification) are full partners of the project and
contribute to its work packages and tasks to build the
infrastructure.
\\
&
\hspace{0.4cm}
Second, we will build over the project duration time a {\em club of
  industrial users}. This club already contains
19 members who provided us with letters of support available on request.
\\
&
\hspace{0.4cm}
\definecolor{shadecolor}{named}{color1}
\begin{shaded}
\begin{center}
  {\bf\large Current members of the club of industrial users\footnote{Their letters of support are available on request.}}
\end{center}
% in alphabetical order
Alstom (Luis-Fernando Meija, FR),
Arm (Dominic Mulligan, UK),
CEA List (Florent Kirchner, FR),
ClearSy (David Déharbe, FR),
Edukera (Benoit Rognier, FR),
Facebook France (François Charton, FR),
IBM Research (Louis Mandel, FR),
MED-EL (Michal Zaremba, AT),
Mitsubishi Electric R\&D Centre Europe (Denis Cousineau, FR),
Nomadic Labs (Raphaël Cauderlier, FR),
OCamlPro (Guillaume Bury, FR),
Onera (Virginie Wiels, FR),
Origin Labs (Fabrice Le Fessant, FR),
Prove\&Run (Stéphane Lescuyer, FR),
RATP (Julien Ordioni, FR),
Runtime Verification (Grigore Rosu, RO),
Siemens (Danko Ilik, FR),
IRT System X (Patrice Aknin, FR),
Systerel (Laurent Voisin, FR),
% Thales (FR),
TrustedLabs (Quang-Huy Nguyen, FR),
TrustInSoft (Benjamin Monate, FR),
\end{shaded}\\
&
\hspace{0.4cm}
These enterprises are working in the area of transportation, health
care, energy, cybersecurity and blockchain.
The club will organize a meeting every year where the
industrial members will:\\
&
$\blacktriangleright$
learn about the advancement and new features of the infrastructure,\\
&
$\blacktriangleright$
give feedback on the use of the infrastructure,
\\
&
$\blacktriangleright$ provide tests sets and proofs to include into Logipedia,
\\
&
$\blacktriangleright$ propose new features, services or research directions for the project,
\\
&
$\blacktriangleright$ participate to the project self-assessment by providing
some feedback on the activities of the consortium.\\
&
\hspace{0.4cm}
From the point of view of its members, this club is a unique
opportunity for watching the current state of technology. Our empirical observations show that
many enterprises, working in safety-critical areas, would like to be
more involved in the development of formal methods, but that the first
step into using such methods is often too costly. This club will offer
them a smoother way to get into this technology. Such a club
where the industrial users will be able to develop a base culture in
formal methods, share experience with other enterprises, and conduct
technology watch on a regular basis is an efficient way to disseminate
formal methods in the European industry.\\
&
\hspace{0.4cm}
Several key results of the project can lead to an industrial
exploitation. This is the case of Logipedia and Dedukti themselves,
which cannot be marketed as products but that can be used to develop safer
software at a lower cost, to simplify simultaneous certification
processes in different countries, and to preserve formal developments
over time. More specifically, a shared library of mathematical analysis is a Key
Exploitable Result, as is a shared specification of the semantics of
programming languages, that can be used both for proving properties of
programmes and in particular of compilers.\\
&\colorbox{color2}{\bf Performance indicators:}
\begin{compactitem}
  \item growth of the club of industrial users
  \item activity of the clubs of industrial users
  \item number of theses with an academic and an industrial co-advisor
  \item number of proofs in Logipedia provided by industrial partners
\end{compactitem}
\\
&
\definecolor{shadecolor}{named}{color1}
\begin{shaded}
{\bf\large Impact on the transportation industry}

For the transportation industry, the railway industry has been
historically a big user of mechanical theorem provers to support the
design of software sub-systems. For instance, CLEARSY has been
applying proof-based development and maintenance of automatic train
control software for large European enterprises. More recently, it has
expanded its offer to formal systems and software analysis services,
supported by formal methods and mechanical proof systems.
By promoting an open framework for the combination of proof systems,
including independent proof verification, and the constitution of open
libraries of mathematical lemmas, Logipedia will produce an ecosystem
that will simplify the constitution of safety cases for product
qualification as well as reduce the cost of proof-based developments.
This should result in an increase in competitivity without compromising
safety.
\end{shaded}\\
&
\definecolor{shadecolor}{named}{color1}
\begin{shaded}
{\bf\large Impact on the health care industry}

Given strict and specific quality requirements in the healthcare
domain, medical software already requires a different approach to testing
and much stricter verification of requirements than a traditional
software. Healthcare software should provide for reliable data
exchange, save patients and professionals time and effort on routine
procedures, show the stable performance, and securely deal with the
sensitive data. Therefore, such a software should be validated from
the perspectives of interoperability, usability, performance, and
compliance with requirements, risk factors as well as industry
regulations. We consider that a traditional software testing might be
still enhanced with methods ensuring that under every condition
medical software provides desired results. Hence by adding formal
verification to medical software testing toolkit we could considerably
increase probability of delivering new functionalities without
unexpected behaviors. In particular, the Logipedia approach will allow
sharing proofs, which will facilitate adaptation in the medical domain.
\end{shaded}\\
&
\definecolor{shadecolor}{named}{color1}
\begin{shaded}
{\bf\large Impact on the energy industry}

Energy, especially nuclear energy, is a one of the key industrial
sectors where safety and security are of prime importance.  The
certification process in the energy industry is very specific
depending on the country. Indeed they do not share common
certification practices, unlike in the aeronautic industry. So a
enterprise who certified critical software used in a nuclear plant in one
country needs to redo the certification in other country using
different tools. The tools recognized by one certification authority
are different in another.  Logipedia by allowing to share proof and
models, would ease the adaptation for one certification authority to
another.
\end{shaded}
\\
&
\definecolor{shadecolor}{named}{color1}
\begin{shaded}
{\bf\large Impact on the blockchain industry}

Public blockchains are about removing the need for trusted third
parties. Formal proofs, as promoted by the Logipedia project, can be
checked independently of any central authority. Thanks to this key
aspect of formal proofs, they can be used in blockchain protocols. For
example, the Zen protocol\cite{zenprotocol_whitepaper} uses formal
proofs to guarantee resource bounds on the execution of its smart
contracts (a smart contract is a program running on a blockchain).


\hspace{0.4cm}
Blockchain protocols require standardisation of data formats but,
because the blockchain technology and its applications evolve fast,
keeping these standards as generic as possible is needed to avoid
forbidding future yet-unknown use-cases. Formal proof systems are very
generic languages but Logipedia pushes genericity even further by
promoting a proof format that is powerful enough to encode most (if
not all) other proof formats.

\hspace{0.4cm}
Last but not least, the importance of formal methods is acknowledged
by a growing proportion of the blockchain community. In the case of
the Tezos\cite{Tezos_whitepaper} blockchain for example, applicability
of formal methods has even been one of the main advertised features
since the start of the project. Parts of the Tezos node software are
certified in F* and Coq and various formal verification systems are
used or being developed to certify Tezos smart contracts.  Because of
the decentralised nature of blockchains, software is hard to update in
this industry. Moreover, software in the blockchain industry often
manages valuable digital assets, evaluating the risk of software bugs
and budgeting for formal verification is easier in this industry than
in others. Blockchain systems are however complex because they are
related to many fields of computer science (cryptography, theory of
programming languages, network security, decentralised computing, game
theory) so verifying their code usually requires a combination of
specialised tools which raise the question of interoperability between
these specialized proof systems. Answering this question of
interoperability is one of the main purposes of the Logipedia project.
\end{shaded}
\\
%%%%%%%%%%%%%%%%%%%%%%%%%%%%%%%%%%%%%%%%%%%%%%%%%%%%%%%%%%%%%%%%%%%%%%%%%%%%%%
\hline
{\bf A new generation of researchers is educated}
&
The availability of a formal online encyclopedia that offers, in a
single place, the communication, archival and verification of
mathematical knowledge will be of prime importance for teachers.\\
&
{\bf PhD Students.} The project will contribute to the formation of a new
generation of researchers, through the many theses and post-doc it shall
propose.\\
&
{\bf At the university.}
Education of formal methods in computer science and of formal proofs
in mathematics always hits the same obstacle: the need to choose a
specific theory or system. This forces the focus on foundational issues, which is in
contradiction with the claimed universality of logical truth.
By demonstrating that this choice amounts to including or excluding
certain axioms and rewrite rules, Logipedia will bring back this
universality to the education of formal methods.  Interactive
theorem proving is already used in major universities to introduce
students to the concept of proof and to teach software foundations.
But, currently it cannot be used for courses with significant
mathematical content because of the lack of a large standard libraries
of formalized results.  By collecting the results already available in
the different interactive proof systems, Logipedia will ease the
adoption to teach maths at university level.
\\
&
{\bf In secondary education.}
The usage of formal proofs in the class room is also slowed down by the
lack of a large and well organized library of results.  Logipedia,
through a specific interface, can be used to teach mathematics, for
instance geometry, in secondary school.  It can also be used to write
textbooks that promote mathematical rigor by including only
theorems that have formal proofs, even if the proofs in the textbook
are not presented formally.
This is why we have included, in the project, an enterprise that has
already experimented the development of software to teach rigorous proof
in high school.
We also defend that this renewal of logic education in secondary education
is of prime importance in our ``post-truth era''.
Because several of us are involved in the renewal of teaching
mathematics and computer science in secondary education and in the
first years of university, we felt the need to include in the project
a club of users in education.
\\
&
\colorbox{color2}{\bf Performance indicators:}
\begin{compactitem}
\item number of training sessions organized
\item growth of the club of users in education
\item number of pedagogical experiments using formal proofs
\item use of the pedagogical interface
\end{compactitem}
\\
&
\definecolor{shadecolor}{named}{color1}
\begin{shaded}
\begin{center}
{\bf\large Current members of the club of users in education}
\end{center}
% in alphabetical order
Jeremy Avigad (Carnegie Mellon University, US),
Thibaut Balabonski (Université Paris Sud, FR),
Jean-Paul Bodeveix (Université Paul Sabatier, FR),
Quentin Bramas (Université de Strasbourg, FR),
Richard Cabassut (Université de Strasbourg, FR),
James Davenport (University of Bath, UK),
Marc De Falco (Centre international de Valbonne, FR),
David Delahaye (Université de Montpellier, FR),
Viviane Durand-Guerrier (Université de Montpellier, FR),
Séverine Fratani (Université de Provence, FR),
Tetsuo Ida (University of Tsukuba, JP),
Bartzia Iro (Université de Montpellier, FR),
Mathieu Jaume (Sorbonne Université, FR),
Magdalena Kobylanski (Université Gustave Eiffel, FR),
Zoltán Kovács (Johannes Kepler University, AT),
Frédéric Le Roux (Sorbonne Université, FR),
Joao Marcos (Universidade Federal do Rio Grande do Norte, BR),
Erik Martin-Dorel (Université Paul Sabatier, FR),
Patrick Massot (Université Paris-Sud, FR),
Antoine Meyer (Université Paris Est, FR),
Simon Modeste (Université de Montpellier, FR),
Walther Neuper (Graz University of Technology, AT),
Paige Randall (North University of Birmingham, UK),
Marc Pantel (Université de Toulouse, FR),
Vincent Pavan (Aix-Marseille Université, FR),
Pedro Quaresma (University of Coimbra, PT),
Jean-Baptiste Raclet (Université Paul Sabatier, FR),
Philippe Richard (Université de Montreal, CA),
Damien Rouhling (Université de Strasbourg, FR),
Sylvain Salvati (Inria, FR),
Gert Smolka (Saarland University,DE),
Martin Strecker (Université Paul Sabatier, FR),
Pierre-Yves Strub (Ecole Polytechnique, FR),
Christine Tasson (Université Paris Diderot, FR),
Damien Thomine (Université Paris-Saclay, FR),
Daniel Violato (University of Brasilia, BR),
Théo Zimmermann (Inria FR)
\end{shaded}
\\
%%%%%%%%%%%%%%%%%%%%%%%%%%%%%%%%%%%%%%%%%%%%%%%%%%%%%%%%%%%%%%%%%%%%%%%%%%%%%%
\hline
{\bf Closer interaction between a larger number of researchers}
&
We have already discussed lengthily the impact of Logipedia on the
community of academic and industrial researchers in formal methods
and, more generally, in logic. Logipedia gathers almost all the
research groups in formal proofs in Europe, and its long-term ambition
is to gather all of them.
\\
&
\hspace{0.4cm}
But Logipedia can lead to a more important evolution, opening the use
of formal proofs by a larger group of users, from experts coming
from the formal proof community to non experts: 
mathematicians, researchers in other sciences using mathematics. 
\\
&
\hspace{0.4cm}
We want to insist on this last point: scientific research and
education at all levels are concerned with the discovery,
verification, communication, archival and usage of mathematical
results.  These tasks have been supported by physical books,
conferences and other means.  Everywhere mathematics and computer
science are used (in physics, in some parts of biology and social
sciences, in engineering, in particular through simulation, etc.) a
quest for higher level of rigor will pave the way for a development
of formal proofs. But, here also, this development is slowed down by the
multiplicity of theories, systems and libraries that make the first
step difficult for beginners. A common reference infrastructure should
simplify the access of non-specialists to formal methods.  For
instance, scientists willing to formalize hybrid systems should have
access to a good analysis library, whatever system they use.
\\
& 
\hspace{0.4cm} Among the communities of researchers, one on which we
can have a real impact during the project is the community of working
mathematicians.  Some of them have started using proof systems and we
must take care that they can have access to the best libraries of
basic mathematics, whatever system
they use.\\
&
{\color{red} A diagram with all the clubs}

\colorbox{color2}{\bf Performance indicators:}
\begin{compactitem}
\item number of Logipedia workshops organized
\item growth of the club of academic users
\item number of papers in computer science, mathematics, and other sciences
  using formal proofs in their quest for rigor
\end{compactitem}
\\
&
\definecolor{shadecolor}{named}{color1}
\begin{shaded}
\begin{center}
  {\bf\large Current members of the club of academic users}
\end{center}
% in alphabetical order
Achim D. Brucker (University of Exeter, UK),
Michael Butler (University of Southampton, UK),
Kevin Buzzard (Imperial College, UK),
Johan Commelin (University of Freiburg, DE),
David Delahaye (Université de Montpellier, FR),
Simon Foster (University of York, UK),
Georges Gonthier (Inria, FR),
Sébastien Gouëzel (Université de Nantes, FR),
Angeliki Koutsoukou-Argyraki (Cambridge University, UK),
Michael Leuschel (Universität Düsseldorf, DE),
Tadeusz Litak (FAU Erlangen-Nürnberg, DE),
Patrick Massot (Université Paris-Sud, FR),
Assia Mahboubi (Inria, FR),
Dale Miller (Inria, FR),
Paige Randall North (Ohio State University, US),
Karl Palmskog (KTH Royal Institute of Technology, NO),
Michael Shulman (University of San Diego, US),
Gert Smolka (Universität Saarland, DE),
Andrew Sogokon (University of Southampton, UK),
Martin Suda (Czech Institute of Informatics, CZ),
Josef Urban (Czech Institute of Informatics, CZ)
%Paul Jackson (contacted, awaiting response)
%Kaustuv Chaudhuri (contacted, awaiting response)
%Martín Escardó (contacted, awaiting response)
%Vincent Rahli (contacted, awaiting response)
%Thibault Gauthier (contacted, awaiting response)
\end{shaded}
\\
%%%%%%%%%%%%%%%%%%%%%%%%%%%%%%%%%%%%%%%%%%%%%%%%%%%%%%%%%%%%%%%%%%%%%%%%%%%%%%
\hline
{\bf Integrated and harmonised access to resources facilitating evidence-based policy making}
&
Certification authorities like the French National Agency for the
Security of Information Systems (ANSSI) are very interested in having
a unique language for formal proofs, rather than several ones which
require as many experts.

Logipedia will be a service to
certification authorities, especially in security, as witnessed by the
presence of a representative of ANSSI in the advisory board,
Prove\&Run among the partners, and several other enterprises focused on
security in the club of industrial users.

Formal methods can also have applications in the verification of legal
rules like, for instance, the rules for computing one's income tax
\cite{merigoux20jfla}.
\\
&
\colorbox{color2}{\bf Performance indicators:}
\begin{compactitem}
\item number of times Dedukti and Logipedia have been used for 
cross-verification, in particular by certification authorities
\end{compactitem}
\\
%%%%%%%%%%%%%%%%%%%%%%%%%%%%%%%%%%%%%%%%%%%%%%%%%%%%%%%%%%%%%%%%%%%%%%%%%%%%%%
\hline
{\bf Better management of the continuous flow of scientific data}
&
Logipedia integrates major scientific equipments and knowledge resources
(provers and proof libraries).
Organizing this continuous flow of data is at the center of the
Logipedia project. We already insisted on the aspects of interoperability,
sustainability, cross verification, and ergonomy of the interfaces. 
The management of this continuous flow of data will be monitored by the 
metric we have defined: the Logipedia Integration Level.
A data management plan will be provided for making these data findable,
accessible, interoperable and reusable (FAIR), and Logipedia will participate
to the Open Research Data Pilot H2020 project.
\\
&
\colorbox{color2}{\bf Performance indicators:}
\begin{compactitem}
\item number of proofs added in Logipedia
\end{compactitem}
\\
%%%%%%%%%%%%%%%%%%%%%%%%%%%%%%%%%%%%%%%%%%%%%%%%%%%%%%%%%%%%%%%%%%%%%%%%%%%%%%
\hline
{\bf Socio-economic impact}
&
One aspect of the socio-economic impact of formal methods in general
and integration activities such as Logipedia specifically, is that making the
digital society safer and more secure contributes to a safer and more
secure society in general.\\
&
\hspace{0.4cm} 
We are also interested in the recent trend to add a third pillar to
safety and security: ethics.  Software must, of course, be safe and
secure, but they must also verify some ethics properties, such as
the respect of privacy, or (in the case of electronic voting systems), 
secret of vote, auditability, etc.
The development of formal methods should contribute to prove 
safety and security properties, but also ethics property.
\\
&
\colorbox{color2}{\bf Performance indicators:}
\begin{compactitem}
\item social awareness of the importance of formal proofs for safety \&
security
\item social awareness of the importance of formal proofs for ethics
\end{compactitem}
\\
%%%%%%%%%%%%%%%%%%%%%%%%%%%%%%%%%%%%%%%%%%%%%%%%%%%%%%%%%%%%%%%%%%%%%%%%%%%%%%
\hline
{\bf Open data / Open science / Open innovation}
&
Logipedia is clearly part of the Open data / Open science / Open
innovation movement, as it aims at making scientific data accessible
to everyone, amateur or professional, in academia, industry, and
education.\\
&
\hspace{0.4cm}
First, the very idea of making formal proofs findable, accessible,
interoperable, and reusable, through the construction of an
encyclopedia and proof engineering algorithms is {\em per se} in the
movement of Open data, Open science, and Open innovation.\\
&
\hspace{0.4cm}
Then, the creative common licence, and other free licences, promoted by 
Logipedia also contribute to this movement.
\\
%%%%%%%%%%%%%%%%%%%%%%%%%%%%%%%%%%%%%%%%%%%%%%%%%%%%%%%%%%%%%%%%%%%%%%%%%%%%%%
\hline
{\bf Scientific publishing, proof verification, long-term proof archiving and reproducibility}
&
Published proofs often contain gaps, sometimes errors, and can be difficult
to check or complete by humans because of their length or
complexity. Frans and Kosolosky noted that, at each level of the
publication process, there is a lack of discipline to emphasize the
importance of checking and correcting proofs
\cite{frans14theoria}. Given this state of the field, they came up
with several ways in which mathematics can and should be improved.
One way is related to the reviewing process itself. Another
way is to appeal to online databases.

\hspace{0.4cm}
Logipedia will provide various services to the community of
mathematicians and to the community of scientific publishers. By
providing hyperlinks to Logipedia proof objects, authors can focus on
the presentation of their results and help reviewers and editors in
evaluating the correctness and interest of them. Readers can be
confident in the correctness of the results as all Logipedia proofs
are formally checked. They can also get a better understanding of them
by looking at the proofs through some editor or web interface. They
can also visualize the axioms and theorems they rely on.

\hspace{0.4cm}
Moreover, as Logipedia proofs are represented in a machine-independent
format, they can be rechecked and reused as long as they are available,
thus improving result verification and reproducibility in mathematical
sciences.
\\
&
\colorbox{color2}{\bf Performance indicators:}
\begin{compactitem}
\item number of scientific publications with links to Logipedia
\item number of members of the publishers club
\item number of publication platforms providing links to Logipedia
\end{compactitem}
\\
%%%%%%%%%%%%%%%%%%%%%%%%%%%%%%%%%%%%%%%%%%%%%%%%%%%%%%%%%%%%%%%%%%%%%%%%%%%%%%
\hline
{\bf Providing data to machine learning algorithms}
&
The success of machine learning techniques heavely relies on the
availability of big amounts of data. Various enterprises (Wolfram,
Facebook, etc.) are interested in applying deep learning techniques to
automated theorem proving, symbolic computation, mathematical
knowledge, and automatic language translation (between formal
languages, or between formal languages and natural languages), with
applications in software engineering, education and computer algebra
systems.

\hspace{0.4cm}
By gathering proofs coming from many different proof systems, in a
unique format, Logipedia will be an important source of training data for
machine learning algorithms, and more generally for
statistical analysis of the properties of mathematical developments.
\\
&
\colorbox{color2}{\bf Performance indicators:}
\begin{compactitem}
\item number of proofs added in Logipedia
\item number of machine learning works using Logipedia as a data set
\end{compactitem}
\\
\hline
\end{longtable}

%%% Local Variables:
%%%   mode: latex
%%%   mode: flyspell
%%%   ispell-local-dictionary: "english"
%%% End:
