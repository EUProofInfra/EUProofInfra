\subsection{Wider, simplified, and more efficient access}

The shift from informal, pencil and paper, proofs to formal
computerized proof has been a major improvement on the never ending
quest for logical rigor, with a strong impact both on computer
science, where safety and security can be dramatically improved with
the use of formal methods and on mathematics, where much more complex
proofs can be built.

But this major step forward also has a negative side effect: we have
evolved from a time where we had (informal) proofs of Fermat's little
theorem, to a time where we have (formal) proofs in Coq, in Matita, in
HOL Light, in PVS, etc.  of this theorem, jeopardizing the
universality of logical truth.

This loss of universality of logical truth is the main obstacle to the
access to formal proofs in the communities of computer scientists and
mathematicians, but also researchers of other disciplines, engineers,
and students. As a consequence, the use of formal proofs is restricted
to a community of specialists, that has fortunately been growing with
time, but that is still too small, compared to the needs, for formal
proofs in the digital society.  Our long-term goal is to resurrect the
universality of logical truth, in order to build a strong formal proof
community including specialists and non-specialists, making
formal proofs findable, accessible, interoperable, and reusable.

This requires to express the theories implemented in these systems in
a common logical framework, with axioms and rewrite rules, in order to
be able to say, not that a proof is expressed in the theory of one
system or of another, but which axioms and rewrite rules are used in
this proof, as we have been used to since the development of
non-Euclidean geometries. In such a shared public encyclopedia
providing virtual access, each user, in research, industry, and
education, can find the formal proofs she needs in the logic she
wants, regardless the logic and system this proof has been developed
in.  Alignment of isomorphic structures, both across libraries, and
within each library, provides a wider and simplified access.

Because it constitutes a central place, such a encyclopedia can foster
projects that would make less sense for a specific library. For instance,

\begin{itemize}
\item indexing mechanisms for mathematical formulas, and search engines
  that are specialized to query such formulas,

  \item a structure of mathematical knowledge in theories, books,
    chapters, categories of users (high school students, researchers),
    that can be inherited from the libraries imported in 
      Logipedia, from concept alignment, and from clustering
    algorithms in the dependence graph of the encyclopedia,

  \item ergonomic user interfaces, that allows a navigation in the
    structure of the encyclopedia, and that can be specialized from
    some domains (safety, geometry...) or some category of users,

  \item programmatic access through an Application Programming
    Interface, and a package distribution system, so that the
    mathematical knowledge can be used not only by humans, but also by
    software.
\end{itemize}

Because of its size, such an encyclopedia, provides a better dataset
for machine learning algorithms, and more generally statistical
analysis of the properties of mathematical developments.

\subsection{New or more advanced research infrastructure}

Proof systems are research infrastructure. Logipedia is not yet
another infrastructure of this kind, but it is a research
infrastructure of a completely new kind, that integrates proof systems
through data sharing.  Such as new research infrastructure will, of
course, impact research in many ways.

\begin{itemize}

\item Computer scientists will be able to prove safety and security
  properties of the software they develop faster and at a lower cost
  because they can access to already developed proofs, independently
  of the system they use.

\item When building new proof systems, computer scientists will not
  need to start from scratch, but will be able to start with an already
  existing data base of proofs.

\item In Automated Theorem Proving, computer scientists will be able
  to use already proved theorems as axioms, to enhance the power of
  their tools.

\item In their publications, computer scientists often provide access
  to formal proofs of the results they present (as required or
  suggested by several conferences and journals). They will be able to
  use Logipedia as a universal repository for such proofs, some of
  which would be available across systems, and these proofs being
  available for a long time.

\item Computer scientists conducting research on machine learning
  using mathematical dataset will have access to a wider database.

\item When they are unsure of the correctness of a proof,
  mathematicians will be able to formalize it faster and at a lower
  cost because they can access to already developed proofs,
  independently of the system they use.

\item Mathematicians who refer to a formal proof in one or their
  publication will be able to use Logipedia as a universal
  repository for such proofs, some of which would be available across
  systems.

\item Mathematicians interested in reverse mathematics will be able to
  analyze, in an easier way, the axioms used in each proof, when they
  have access to this proof formalized.

\item The results will be archived for a long time increasing the
  reproducibility of results, both in mathematics and computer science.

\item As mathematics and software are in any part of modern science,
  Logipedia will also have an impact on other sciences. In
  particular because proving properties of a piece of software driving
  a car or piloting an aircraft require to formalize part of the
  physical world in which this piece of software evolves.
\end{itemize}
  
\subsection{Operators develop synergies}

The formal proof community is currently a scattered community, each
sub-community being centered around its own system, its own theory,
and its own library.

To build a stronger community, where the researchers develop
strategies, it is not sufficient to talk to each other, or to organize
conferences, but these sub-communities must exchange data and work
together on a joint project to exchange those data.
Expressing these data in a common encyclopedia will
lead the developers of various systems to express the theories they
implement in a common logical framework, yielding a better understanding
to the similarities and differences between these theories.
Developing synergies will induce less work duplication and will increase
the efficiency of the community as a whole.

Working on common projects will not only increase the communication
between relatively close communities, such as the Coq and 
  Agda communities that meet every year at the TYPES conference, but
also to more distant communities, such as the TYPES community, the HOL
community (that already meet around the OpenTheory standard),
the B and TLA+ communities, and the Mizar community.

More importantly, this data exchange between researchers and
engineers, and this evolution towards a standard, will allow a better
cooperation between research and industry and suppress one of the main
obstacles to the diffusion of formal proofs in industry.

This evolution towards a standard and this resurrection of the
universality of logical truth will also suppress one of the main
obstacles to the diffusion of formal proofs in the community of
working mathematicians and in and education, just like the development
of the Html standard induced a renewal of document sharing in general
and the definition of predicate logic induced a renewal of logic in
the 1930's.

Sharing a logical framework will also allow new synergies between 
the formal proof community and the Automated Theorem Proving community.

It will also allow a better communication with machine learning community.

We have already organized two Logipedia workshops that have proven to
be very valuable to develop joint project and synergies.  This project
will permit to organize wider international events on this topic of
sharing formal proofs, in academia, industry, education, and
publishing.

\subsection{Innovation is fostered through a reinforced partnership of
research infrastructures with industry}

Formal methods are now an important part of some advanced industrial
projects. Mastering formal methods is key to give Europe a competitive
advantage in conquering the market of autonomous cars, trains, planes
or drones, or the block-chains and crypto-currencies. But this
penetration of formal methods in industry hits the same obstacle that
researchers often promote one method, theory or system, while their
industrial partners are in search of universality. We expect to make
formal proofs more accessible to industry by avoiding each project to
redevelop elementary proofs, but instead benefit of the formalization
work shared with other communities.

Several European companies are member of the project,
as contributors or as members of the future Logipedia
club of industrial users.

From the industry point of view the key exploitable results are threefold:

\begin{enumerate}
\item Cross-verification.

In the current state of affairs, when a company proves a piece of
software correct using a system $X$, its client, or the certification
authorities can check the proof developed by this company, but then
need to use the same system $X$ to check this proof, so they need to
trust the system $X$, which is a limitation. Several actors want to be
able to check the proofs using an independent proof system, and even
one they have developed themselves.

A side effect of the construction of the Logipedia platform is
to incent all the proof systems to be able to produce proofs in a
common language. Such proofs can then be all be checked by
Dedukti, and several other systems.

Moreover, the certification authorities can develop their own
proof-checker (the development of such a proof-checker takes a few
weeks) so that they do not need to trust anyone else.

\item Towards standardization.

The lack of standards is currently a major obstacle to the development
of formal methods in industry. Although we consider that starting a
standardization process is still premature, this project will allow us
to experiment with a common language that we shall improve until we
reach the point where it can be proposed as a standard.

\item Sustainability.

When an industrial project is over, it is sometimes difficult,
specially for small companies, to keep an archive of their work over
a long period of time. In formal methods, most of the projects are a
two-stages rocket. The first stage of the rocket contains basic
developments that can be shared between the company and its
competitors. The second contains developments that are specific to the
project and that often contain industrial secrets.

Sharing the first stage on a public encyclopedia, while keeping the
second secret, contributes to the sustainability of the
developments. They can, for instance, be reused decades later, and
cannot be lost by the company.

\item Interoperability.

  Some industrial programme verification tools rely only on one single
  proof system and any missing feature of its library forces the
  end-user to prove complicated theorems that probably exist in other
  proof systems. Logipedia will make it possible to access all
  standard libraries and proofs of all systems. It makes the overall
  proof of programmes much less costly.

  Other industrial programme verification tools use several proof
  systems. First because some proof systems are better for some
  application domains and other are better for others, and also
  because these companies hire researchers and engineers that have
  different cultures and are more efficient using the tools they know.

  A side effect of the construction of the Logipedia platform is that
  such development are made interoperable. First because a proof
  developed in one system can be translated into another. Second
  because proofs developed in different systems can be combined in
  Logipedia itself.
\end{enumerate}

The participation of industry to the project is twofold.  First, CEA
LIST, Clearsy, Edukera, MED-EL, OCamlPro, Prove\&Run, and SystemX, are
full partners of the project and will contribute to its work packages
and tasks.

Second, we will build over the project duration time a {\em club of
  industrial users of Logipedia}. This club already contains
18 members.

\begin{framed}
\begin{center}
  {\bf \Large The current members of the club of industrial users}
\end{center}
\begin{itemize}
\item Alstom
\item CEA LIST
\item ClearSy
\item Edukera
\item Facebook France
\item IBM Research
\item MED-EL
\item Mitsubishi Electric R\&D Centre Europe
\item Nomadic Labs
\item OCamlPro
\item Onera
\item Origin Labs
\item Prove\&Run
\item TrustInSoft
\item RATP
\item Siemens
\item System X
\item Systerel
\item TrustInSoft
\end{itemize}
\end{framed}

These companies are working in the area of transportation, health
care, energy, cybersecurity, block-chain, etc.

This club will organize two of meetings every year where the
industrial members will:
\begin{itemize}
\item learn about the advancement and new features of the infrastructure,
\item give feedback on the use of the infrastructure,
\item provide tests sets and proofs to include into Logipedia,
\item propose new features, services or research directions for the projet,
\item participate to the project self assessment.
\end{itemize}
From the point of view of its members, this club is a unique
opportunity for technology watch. Our empirical observations show that
many companies, working in safety critical areas, would like to be
more involved in the development of formal methods, but that the first
step into using such methods is often too costly, and that we must
offer them a smoother way to get into this technology. Such a club
where the industrial users will be able to develop a base culture in
formal methods, share experience with other companies, and conduct
technology watch on a regular basis is an efficient way to disseminate
formal methods in the European industry.

\begin{framed}
{\bf \Large Impact of Logipedia on the transportation industry}

For the transportation industry, the railway industry has been
historically a big user of mechanical theorem prover to support the
design of software sub-systems. For instance, CLEARSY has been
applying proof-based development and maintenance of automatic train
control software for large European companies. More recently, it has
expanded its offer to formal system and software analysis services,
supported by formal methods and mechanical proof tools

By promoting an open framework for the combination of proof tools,
including independent proof verification, and the constitution of open
libraries of mathematical lemmas, Logipedia will produce an ecosystem
that will simplify the constitution of safety cases for product
qualification as well as reduce the cost of proof-based developments.
This should result in an increase in competitivity without compromising
on safety.
\end{framed}

\begin{framed}
{\bf \Large Impact of Logipedia on the health care industry}

{\color{red} Cezary}  
\end{framed}

\begin{framed}
{\bf \Large Impact of Logipedia on the energy industry}

\medskip

Energy, specially nuclear energy, is a one of the key industrial sectors 
where safety and security are of prime importance.

The certification process in the energy industry is very specific
depending on the country. Indeed they don't share common certification
practice, unlike in the aeronautic industry. So a company who
certified critical software used in a nuclear plant in one country
needs to redo the certification in other country using different
tools. The tools recognized by one certification authority are
different in another.

Logipedia by allowing to share proof and models,
would ease the adaptation to another certification authority.
\end{framed}

\begin{framed}
{\bf \Large Impact of Logipedia on the block chain industry}

{\color{red} Raphaël}  
\end{framed}

\subsection{Closer interaction between a larger number of researchers}

We have already discussed lengthily the impact of Logipedia on the
community of academic and industrial researchers in formal methods
and, more generally, in logic.
The European Union has already invested a lot in logic and formal
methods, and we believe it is our duty to propose a project to
integrate the results of all this research.

Scientific research and education at all levels are concerned with the
discovery, verification, communication, archival and usage of
mathematical results.  These tasks have been supported by physical
books, conferences and other means.

The project can lead to a societal breakthrough opening the use of
formal proofs by a larger group of users, from experts users coming
from the formal proof community to a group of non expert users
(mathematicians, education and researchers in other science using
mathematical statements).

We want to insist on this last point: everywhere mathematics and
computer science are used (in physics, in some parts of biology and
social sciences, in engineering, in particular through simulation,
etc.) a quest for higher level of rigor should pave the way for a
development of formal proofs, but that this development is slowed down
by the multiplicity of theories, systems and libraries that make the
first step difficult for beginners. A common reference infrastructure
should simplify the access of non-specialists to formal methods.  For
instance, scientists willing to formalize hybrid systems should have
access to a good analysis library, whatever system they use.

Among the communities of researchers, one on which we can have a real
impact during the project is the community of working mathematicians.
Some of them have started using proof systems and we must take care that they
can have access to the best libraries of basic mathematics, whatever system
they use.


\begin{framed}
\begin{center}
  {\bf \Large The current members of the club of academic users}
\end{center}

David Delahaye


Assia Mahboubi
Gert Smolka
  
Karl Palmskog <palmskog@gmail.com> 

Paul Jackson (not yet contacted)

Simon Foster <simon.foster@york.ac.uk> (Burkhart)

Achim D. Brucker <A.Brucker@exeter.ac.uk> (Burkhart)

Angeliki Koutsoukou-Argyraki 

Kevin Buzzard (not yet contacted)

Sébastien Gouëzel (not yet contacted)
\end{framed}

\subsection{Better management of the continuous flow of data}

Organizing a continuous flow of data is at the center of a project
like Logipedia. We already instead on the aspects of interoperability,
sustainability, cross verification, and ergonomy of the interfaces.

We want also to insist on the fact that Logipedia will be a service to
certification authorities, specially in security, as witnessed by the
presence of a representative of ANSSI in the advisory board,
Prove\&Run among the partners, and several other companies focused on
security in the club of industrial users.

\subsection{Socio-economic impact}

One aspect of the socio-economic impact of formal method in general
and integration activities such as Logipedia, is that making the 
digital society safer and more secure contributes to a safer and more 
secure society in general.

We are also interested in the recent trend to add a third pillar to
safety and security: ethics. For instance, proving properties of
software including the respect of privacy or, in the case of
electronic voting systems, secret of vote, auditability, etc.  is not
directly part of our project, but the development of formal methods
contributes to promote these values.

\subsection{Education}

The availability of a formal online encyclopedia which propose in a
single place the communication, archival and verification of
mathematical knowledge will be of prime importance for teachers.


\paragraph*{At university}
Education to formal methods in computer science and to formal proofs
in mathematics always hits the same obstacle: the need to choose a
specific theory or system, the need to focus on foundational issues in
contradiction with the claimed universality of logical truth.

Education to formal methods and formal proofs will gain in
universality once it will be demonstrated that this choice amounts to
include, or not, a few axioms and rewrite rules.  

Interactive theorem proving is already used in major universities to 
introduce students to the concept of proof and to teach software foundations.

But, currently it can not used for course with significant mathematical content 
because of the lack of a large standard libraries of formalized results.
By collecting the results already available in the different interactive proof systems,
Logipedia will ease the adoption to teach maths at university level.

\paragraph*{In secondary education}
The usage of formal proofs in the class room is also slown down by the
lack of a large and well organized library of results.  Logipedia,
through a specific interface, can be used to teach mathematics, for
instance geometry, in secondary school.  It can also be used to write
textbooks promoting mathematical rigor by having in a textbook only
theorems that have formal proofs, even if the proofs in the textbook
are not presented formally. 

This is why we have included, in the project, a company that has 
already experimented the development of software to teach rigorous proof
in high school. 

We also defend that this renewal of logic education in secondary education
is of prime importance in our ``post-truth era''.

Because several of us are involved in the renewal of teaching
mathematics and computer science in secondary education and in the
first years of university, we felt the need to include in the project
a club of users of Logipedia in education.

\begin{framed}
\begin{center}
{\bf \Large The current members of the club of users in education}
\end{center}

\begin{multicols}{2}
Thibaut Balabonski,
Jean-Paul Bodeveix,
Quentin Bramas,
Richard Cabassut,
James Davenport,
David Delahaye,
Viviane Durand-Guerrier,
Séverine Fratani,
Mathieu Jaume,
Magdalena Kobylanski,
Zoltán Kovács,
Frédéric Le Roux,
Erik Martin-Dorel,
Patrick Massot,
Antoine Meyer,
Simon Modeste,
Marc Pantel,
Vincent Pavan,
Jean-Baptiste Raclet,
Philippe Richard,
Damien Rouhling,
Sylvain Salvati,
Gert Smolka,
Martin Strecker,
Christine Tasson,
Damien Thomine,
Théo Zimmermann,
Pedro Quaresma,
Marc De Falco,
Gert Smolka,
Tetsuo Ida,
Bartzia Iro,
Joao Marcos,
Pierre-Yves Strub
\end{multicols}

\end{framed}


\subsection{Publishing}

{\color{red} Frédéric}  

\subsection{Open data / Open science / Open innovation}

Logipedia is clearly part of the Open data / Open science / Open
innovation movement, as it aims at making scientific data accessible
to everyone, amateur or professional, in academia, industry, and
education.

First, the very idea of making formal proofs findable, accessible,
interoperable, and reusable, through the construction of an
encyclopedia and proof engineering algorithms is {\em per se} in the
movement of Open data, Open science, and Open innovation.

Then, the creative common licences promoted by Logipedia contribute
also to this movement.

%%% Local Variables:
%%%   mode: latex
%%%   mode: flyspell
%%%   ispell-local-dictionary: "english"
%%% End:
