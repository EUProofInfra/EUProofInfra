The proposal is in coherence with the goals of the call INFRAIA-02-2020:
Integrating activities for starting communities (listed on page 56 of
the part 4 of Horizon 2020 Work Programme 2018-2020). It also clearly
refers to one of the mentioned cross-cutting activities: open science.

\subsection{Wider, simplified, and more efficient access}

The shift from informal, pencil and paper, proofs to formal
computerized proof has been a major improvement on the never ending
quest for logical rigor, with a strong impact both on mathematics,
where much more complex proofs can be built, and computer science,
where safety and security can be dramatically improved with the use of
formal methods.  But this major step forward also has a negative side
effect: we have moved from a time where we had (informal) proofs of
Pythagoras' theorem or Fermat's little theorem, to a time where we
have (formal) proofs in {\sc Coq}, in {\sc Matita}, in {\sc HOL
  Light}, in {\sc PVS}, etc.  of these theorems, jeopardizing the
universality of mathematical truth.

This loss of universality of mathematical truth as the main obstacle
to the diffusion of the notion of formal proof in the communities of
mathematicians and computer scientists, but also engineers and
students. Our long-term goal is to resurrect the universality of
mathematical truth in order to build a strong formal proof community
including specialists and non-specialists such as working
mathematicians, engineers and students.  This requires to express the
theories implemented in these systems in a common logical framework,
with axioms and reduction rules, in order to be able to say, not that
a proof is expressed in one system or in another, but which axioms and
reduction rules are used in this proof, as we have been used to since
the development of non-Euclidean geometries.

In a shared public encyclopedia providing virtual access, each user,
in research, industry, and education, can find the formal proofs she
needs in the logic she wants, regardless the logic and system this
proof has been developed in.




{\color{red} Search engine} giving domain specific access


Central place

Alignment of siomorphic strucutes provides wider and simplied access

Large libraries allow big data, machine learning... to work better

Beautiful Presentation to the user

Programmatic access (api, opam, etc.)

{\color{red} Talk about licences here. Is cc-by ok for everyone?}

{\color{red} Key exploitable results?}

The enclyclopedai itself is a KER

The frontends are a KER

Each domain scpeific library univwrally accessible (e.g. real analysis, probability theory) is a KER


\subsection{New or more advanced research infrastructure}

{\color{red} Key exploitable results?}

Logipedia in itself from TRL 3 to TRL 4

Various clients :

- mathematicians can build formal proofs of results they are not sure about
(cite Vladimir)

- Conferences in CS where you have to poide a formal proof

- use the proofs to prove correc a piece of software

- rsearchers who want to build newe proofs to improve the datatbase (synergy)

- ressoutce for ATP

- Mathematicians interesed in revere maths can formaly analyze the axiomes used

- When building a new ATP or ITP, already a data base to satrt with
instead of starting from scratch

- Research on machine learning has a wider database

- Long term archiving


\subsection{Operators develop synergies}

Having a standard for expressing theories and proofs and resurrecting
this way the universality of mathematical truth will also make proof
systems interoperable and will allow the construction of an on-line
system-independent encyclopedia.

Instead of having a scattered community, each group developing a
library for its own logic and its own system, researchers and
engineers will be able to work together on common developments,
reusing proofs developed in other systems and in other communities.

More importantly, this will suppress one of the main obstacles to the
diffusion of formal proofs in mathematics, computer science, industry,
and education, just like the development of the html standard induced
a renewal of document sharing in general and the definition of
predicate logic induced a renewal of logic in the 1930's.

Operators can compare the teories implementes (benefit LF)

- We use to talk to each others but rarelt exchange our data

- Good communication within the TYPES community,, the HOL community,
the B + TLA+ communities, but these subcomminities do not talk to each
other much

- One to many communication instread of one to one

- *** Better communication between ITP and ATP *** TPTP

- Better communication with machin leraning





- 

- HARMONIZE

- LESS DUPLICATION

- INTEGRATE OPEN-THEORY and sclaing up

- Harmony with OpenDreamKint and Software heritage

- Logipedia should provide support for Deepspec

- Alexandria

{\color{red} Key exploitable results?}




%In terms of networking, we have already organized one logipedia
%meeting and the funding will insure we can continue to organize
%large-scale international meetings on a regular basis. The first
%logipedia event has proven to be very valuable in terms of exchange of
%best practices.

\subsection{Innovation is fostered through a reinforced partnership of research
infrastructures with industry}

Innovation Formal methods are now an important part of some advanced
industrial projects. For instance, mastering formal methods is key to
give Europe a competitive advantage in conquering the market of
autonomous cars, trains, planes, and drones. But this penetration of
formal methods in industry hits the same obstacle that researchers
often promote one method, theory or system, while their industrial
partners are in search of universality. We expect to make formal
proofs more accessible to industry by avoiding each project to
redevelop elementary proofs, but instead benefit of the formalization
work shared with other communities.

Several European and non European companies are member of the project,
as contributors or as members of the future {\sc Logipedia} user's group.

From the industy point of view the key exploitable results are threefold

\begin{enumerate}
\item Crossverification.

In the current state of affairs, when a compagny proves a piece of
software correct using a system $X$, its client, or the certification
authorities can check the proof developed by this compagny, but then
need to use the same system $X$ to check this proof, so they need to
trust the system $X$, which is a limitation. Several actors want to be
able to check the proofs using a indepedent proof system, and even one
they have developed themselves.

A side effect of the construction of the {\sc Logipedia} platform is
to incent all the proof systems to be able to produce proofs in a
common language. Such proofs can then be all be checked by {\sc
Dedukti}, and several other systems.

Moreover, the certification authorities can develop their own
proof-checker (the development of such a proof-checker takes from two
to four weeks) so that they do not need to trust anyone else.

\item Sustainability.

When an industrial project is over, it is sometimes difficult,
specially for small compagnies, to keep an archive of their work over
a long period of time. In formal methods, most of the projects are a
two-stage rocket. The first stage of the rocket contains basic
developements that can be shared between the compagny and its
competitors. The second contains developments that are specific to the
project and that often contain industrial secrets.

Sharing the first stage on a public encyclopedia (while keeping the
second secret) contributes to the sustainability of the
developements. They can, for instance, be reused decades later, and
cannot be lost by the compagny.

\item Interoperability.

Several compagnies use several proof systems. First because some proof
systems are better for some application domains and other are better
for others, and also because these compagnies hire researchers and
engineers that have different cultures and are more efficient using the
tools they know.

A side effect of the construction of the {\sc Logipedia} platform is
that such developement are made interoperable. First because a proof
developed in one system can be translated into another. Second because
proofs developed in different systems can be combined in {\sc
Logipedia} itself.
\end{enumerate}

The participation of industry to the project is twofold. First,
the project includes a {\em club of industrial users} that will
organize a couple of meeting every year. To
\begin{itemize}
\item propose new objectives for the projet,
\item propose use cases,
\item participate to the project self assessment.
\end{itemize}

From the industrial point of view the participation to this club
is a way to do some technology watch. 

Then, some compagnies, specially some related to the {\sc B} community
and the railway industry, are part full partners of the project and
will contribute to the production of {\sc Logipedia} proofs from
{\sc B} provers. 

{\color{red} Club of industrials (Frédéric, can 
you explain the impact here?)}

{\color{red} Impact for the {\sc B} community and the railway industry
  (Yamine?)}

{\color{red} Key exploitable results?}

- (introperability) Import feature is a KER

- (introperability) Export feature is a KER

- Recall the importance of safety and security

- Block chain ({\color{red} Raphaël writes this}), smart contract : small program, lots of money. Touches different areas (cooperation of specialized tools)

- Medel (Cezary)



\subsection{Closer interaction between a larger number of researchers...}


Researchers from different groups use a common logical framework to
describe the theory implemented in their system.

- Accross fields (math computer science philosophy)

- Easing the devepeloment of formal proofs wolll bring engineers
(hybrid system can be formalized if we have a good analysis library
such an accessible library witll bring more researchers from this
field)

- Scinetists from other disciplines to the communities

- Connection with natural language 

- Atttracting working mathematicians

- {\color{red} Benjamin M.}: lower the cost, using an accessible
database, hence new cooperation

{\color{red} Key exploitable results?}

- Representation of theories implemented in various systems

- Some specific developements will be kept secret by the industrial parners
but sharing the basics  (90\%) is win-win


\subsubsection{Education}

{\color{red} JN: I wonder if we should not extend the club of teachers to the club or teachers/researchers.}

Scientific research and education at all levels are concerned with the discovery, verification, communication, archival and usage of mathematical results.
These tasks have been supported by physical books, conferences and other means.

The avaibility of a formal online encyclopedia which propose in a single place the communication, archival and verification of mathematical  knowledge will be of prime importance for researchers and teachers.

Education to formal methods in computer science and to formal proofs
in mathematics always hits the same obstacle: the need to choose a
specific theory or system, the need to focus on foundational issues in contradiction with the claimed universality of logical truth. 

Education to formal methods and formal proofs will gain in universality once it will be demonstrated that this choice amounts to include, or not, a few axioms and reduction
rules. 
The usage of formal proofs in the class room is also slown down by the lack of a large and well organized library of results.
We defend that this renewal of logic education at university level and before is of prime importance in our ``post-truth era''.

The project can lead to a societal breakthrough opening the use of formal proofs by a larger group of users, from experts users comming from the formal proof community  to a group of non expert users (mathematicians, education and researchers in other science using mathematical statements). 

The club of teachers gathers researcher and teachers who are already actively using formal proof for teaching in computer science, mathematics and logic but are not necessarily members of the cummunity of researchers in formal theorem proving. These early adopters, will provide continuous feedback to the project members on the usuability and accessibility of the system from a user point of view.

Math students 

{\color{red} Club of teachers (Julien, can you explain the impact here?)}

{\color{red} Key exploitable results?}

{color{red} JN:  some potential ideas:}
\begin{enumerate}
\item Summer schools for introducing mathematical researchers to ITP
\item For highschool teachers: seminars/lectures about  the role of logic in maths teaching and the use of proof assistants in class.
\item For university teachers: creation and dissemination of teaching material for introduction to the concept of proof to fresh maths students.   
\item Development of an Edukera like, open source, point and click user interface for interactive formal theorem proving in the classroom : the "scratch" of maths  (a Isar like language for the proof structure but with a point and click user interface)
\item formalization of  highschool curriculums for different european countries
\item link between the project and the ThEdu community
\end{enumerate}

\subsection{Better management of the continuous flow of data.}

A shared encyclopedia allows a better sustainability of the formal
proofs developed over time. Too many formal proofs developed in the
past are not available any more.

{\color{red} Key exploitable results?}

UPLOAD feature of Logipedia

Sercice to the certification authorities

\subsection{Socio-economic impact}

{\color{red} Safety and security of software contribute to a more ethical
  and safer socity}

Make Europe a leader in ITP/ATP

EU Spent money on TYPES, Nicolas Tabareau's ERC. Formath, Circo... 

Now we integrate the results of all this research

{\color{red} Key exploitable results?}
