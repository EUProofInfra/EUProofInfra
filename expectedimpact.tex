The proposal is in coherence with the goals of the call INFRAIA-02-2020:
Integrating activities for starting communities (listed on page 56 of
the part 4 of Horizon 2020 Work Programme 2018-2020). It also clearly
refers to one of the mentioned cross-cutting activities: open science.

\subsection{Wider, simplified, and more efficient access}

The shift from informal, pencil and paper, proofs to formal
computerized proof has been a major improvement on the never ending
quest for logical rigour, with a strong impact both on mathematics,
where much more complex proofs can be built, and computer science,
where safety and security can be dramatically improved with the use of
formal methods.  But this major step forward has also had a negative side
effect: we have moved from a time where we had (informal) proofs of
Pythagoras' theorem or Fermat's little theorem, to a time where we
have (formal) proofs in {\sc Coq}, in {\sc Matita}, in {\sc HOL
  Light}, in {\sc PVS}, etc.\ of these theorems and it is not obvious
that the formal statements correspond to each other, jeopardizing the
universality of mathematical truth.

This loss of universality of mathematical truth is an important obstacle
to the dissemination of the notion of formal proof in the communities of
mathematicians and computer scientists, but also engineers and
students. Our long-term goal is to resurrect the universality of
mathematical truth in order to build a strong formal proof community
including specialists and non-specialists such as working
mathematicians, engineers and students.  This requires expressing the
theories implemented in these systems in a common logical framework,
with axioms and reduction rules, in order to be able to say, not that
a proof is expressed in one system or in another, but which axioms and
reduction rules are used in this proof, as we have been used to since
the development of non-Euclidean geometries.

In a shared public encyclopedia providing virtual access, each user,
in research, industry, and education, can find the formal proofs she
needs in the logic she wants, regardless of the logic and system this
proof has been developed in.

{\color{red} Talk about licences here. Is cc-by ok for everyone?}

{\color{red} Key exploitable results?}

\subsection{New or more advanced research infrastructure}

{\color{red} Key exploitable results?}

Logipedia in itself from TRL 3 to TRL 4


\subsection{Operators develop synergies}

Having a standard for expressing theories and proofs and resurrecting
in this way the universality of mathematical truth will also make proof
systems interoperable and will enable the construction of an on-line
system-independent encyclopedia.

Instead of having a scattered community, each group developing a
library for its own logic and its own system, researchers and
engineers will be able to work together on common developments,
reusing proofs developed in other systems and in other communities.

Most importantly, this will suppress one of the main obstacles to the
dissemination of formal proofs in mathematics, computer science, industry,
and education, just like the development of the html standard induced
a renewal of document sharing in general or the definition of
predicate logic induced a renewal of logic in the 1930's.


{\color{red} Key exploitable results?}

%In terms of networking, we have already organized one logipedia
%meeting and the funding will insure we can continue to organize
%large-scale international meetings on a regular basis. The first
%logipedia event has proven to be very valuable in terms of exchange of
%best practices.

\subsection{Innovation is fostered through a reinforced partnership of research
infrastructures with industry}

Formal methods are now an important part of advanced
industrial projects. For instance, mastering formal methods is key to
giving Europe a competitive advantage in conquering the market of
autonomous cars, trains, planes, and drones. But this penetration of
formal methods in industry hits the same obstacle that researchers
often promote one method, theory or system, while their industrial
partners are in search of universality. We expect to make formal
proofs more accessible to industry by avoiding that each project
redevelops elementary proofs, but instead benefits of the formalization
work shared with other communities.

Several European and non-European companies are member of the project,
as contributors or as members of the future {\sc Logipedia} user's group.

From the industry point of view the key exploitable results are threefold:

\begin{enumerate}
\item Cross-verification.

In the current state of affairs, when a company proves a piece of
software correct using a system $X$, its client, or the certification
authorities, can check the proof developed by this company, but they
need to use the same system $X$ to check this proof, so they need to
trust the system $X$, which is a limitation. Several actors want to be
able to check the proofs using an independent proof system, and even one
they have developed themselves.

A side effect of the construction of the {\sc Logipedia} platform is
to incentivize all the proof systems to be able to produce proofs in a
common language. Such proofs can all be checked by {\sc
Dedukti}, as well as several other systems.

Moreover, the certification authorities can develop their own
proof checker (the development of a proof checker requires much less effort than
that of a proof assistant)
% takes from two to four weeks)
so that they do not need to trust anyone else.

\item Sustainability.

When an industrial project has ended, it is sometimes difficult,
especially for small companies, to maintain an archive of their work over
a long period of time. In formal methods, most projects are a
two-stage rocket. The first stage of the rocket contains basic
developments that can be shared between the company and its
competitors. The second one contains developments that are specific to the
project and that often contain industrial secrets.

Sharing the first stage on a public encyclopedia (while keeping the
second one secret) contributes to the sustainability of the
developements. They can, for instance, be reused decades later, and
cannot be lost by the company.

\item Interoperability.

Different companies use different proof systems. First because some proof
systems may be specifically suited to certain application domains,
but also because these companies hire developers
that have different cultures and are more efficient using the
tools they know.

A side effect of the construction of the {\sc Logipedia} platform is
that such developments are made interoperable: a proof
developed in one system can be translated into another one, and moreover
because proofs developed in different systems can be combined in {\sc
Logipedia} itself.
\end{enumerate}

The participation of industry to the project is twofold. First,
the project includes a {\em club of industrial users} that will
organize a couple of meetings every year. To
\begin{itemize}
\item propose new objectives for the projet,
\item define use cases,
\item participate to the project self assessment.
\end{itemize}

From the industrial point of view the participation to this club
is a way to monitor new technological developments.

Second, some companies, including some related to the {\sc B} community
and the railway industry, are full partners of the project and
will contribute to the production of {\sc Logipedia} proofs from
{\sc B} provers. 

{\color{red} Club of industrials (Frédéric, can 
you explain the impact here?)}

{\color{red} Impact for the {\sc B} community and the railway industry
  (Yamine?)}

{\color{red} Key exploitable results?}

\subsection{Education}

{\color{red} JN: I wonder if we should not extend the club of teachers to the club of teachers/researchers.}

Scientific research and education at all levels are based on the discovery,
verification, communication, archival and use of mathematical results.
These tasks have been supported by physical books, conferences and other means.

The avaibility of a formal online encyclopedia that supports in a single place
the communication, archival and verification of mathematical  knowledge will be
of prime importance for researchers and teachers.

Education to formal methods in computer science and to formal proofs
in mathematics always hits the same obstacle: the need to choose a
specific theory or system, the need to focus on foundational issues in
contradiction with the claimed universality of logical truth.

Education to formal methods and formal proofs will gain in universality once it
will be demonstrated that this choice amounts to include, or not, a few axioms
and reduction rules. 
The use of formal proofs in the classroom is also slowed down by the lack of a
large and well-organized library of results.
We argue that this renewal of logic education at university level and before is
of prime importance in our ``post-truth era''.

The project can lead to a societal breakthrough, leading to the adoption of
formal proofs by a larger group of users, be they expert users comming from the
formal proof community or non-expert users (mathematicians, educators and
researchers in other sciences using mathematical statements).

The club of teachers gathers researchers and teachers who are already actively
using formal proof for teaching in computer science, mathematics and logic.


{\color{red} Club of teachers (Julien, can you explain the impact here?)}

{\color{red} Key exploitable results?}

\subsection{Closer interaction between a larger number of researchers}

Researchers from different groups use a common logical framework to
describe the theory implemented in their system.

{\color{red} Key exploitable results?}

Representation of theories implemented in various systems


\subsection{Better management of the continuous flow of data.}

A shared encyclopedia allows a better sustainability of the formal
proofs developed over time. Too many formal proofs developed in the
past are not available any more.

{\color{red} Key exploitable results?}

\subsection{Socio-economic impact}

{\color{red} Safety and security of software contribute to a more ethical
  and safer socity}

{\color{red} Key exploitable results?}
