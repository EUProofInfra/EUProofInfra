\subsection{Progress beyond current achievements}


\paragraph{Networking activities}
This project will include networking activities, to foster a culture
of cooperation between scientific communities, that are today often
too centered around one system and one theory. Instead, such a project
will incentivize them to build this encyclopedia together and to always
wonder if the proofs they develop are specific to one system or are
universal. This will also foster a culture of cooperation between
scientific communities and two communities of users: teachers and
industrial partners. This is why the project includes two ``clubs of
users'', one of teachers and one of companies using formal methods.

Besides these two clubs, the research directions will be discussed every
year by the advisory board that will gather {\sc Logipedia} contributors
and users.

The development and maintenance of {\sc Logipedia} will eventually lead
to the discussion of standards for proof languages, even if such an
effort is premature today.

We plan to organize a yearly workshop of {\sc Logipedia} developers
and users, continuing the effort started on the January 2019 meeting.

As explained before, the development of a system-independent and
theory-independent encyclopedia of proofs will trigger new
ways to teach formal proofs to new a audience.


\paragraph{Transnational access}
Our encyclopedia being on line, it will of course be accessible from
every country in Europe and beyond.


\paragraph{Joint research activities}
The project includes two types of joint research activities.  First,
as any infrastructure, it will allow joint research projects
between the users of this infrastructure that will be able to develop
new proofs together using different systems.

Second, as any infrastructure, {\sc Logipedia} raises new research
problems. Some of them have already been solved in the past and
require to be implemented jointly in a first version of the
encyclopedia. Others are newer and will trigger new cooperation
between the teams of the project.


\subsection{Innovation potential}

Formal methods are at a turning point. Several academic and
industrial successes have proved the readiness of the technology, but
this technology takes too much time to be generalized, except in the
transportation industry.

Analyzing this phenomenon, it is clear that the redundancy of the
efforts to develop proof systems and the lack of common theories,
benchmarks, and standards for these systems is a limiting factor.

This effort to integrate the scientific and technological effort
around formal proofs in Europe is a way to address this issue so that 
the economic spinoffs from the project benefits the European industry.

%%% Local Variables:
%%%   mode: latex
%%%   mode: flyspell
%%%   ispell-local-dictionary: "english"
%%% End:
