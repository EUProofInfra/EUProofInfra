\paragraph{Progress beyond current achievements}

This project will include networking activities, to foster a culture
of cooperation between scientific communities, that are today often
too centered around one system and one theory. Instead such a project
will incent them to build this encyclopedia together and to always
wonder if the proofs they develop are specific to one system or are
universal. This will also foster a culture of cooperation between
scientific communities and two communities of users: teachers and
industrial partners. This is why the project includes two ``clubs of
users'', one of teachers and one of companies using formal methods.

Our encyclopedia being on line, it will of course be accessible from
every country in Europe and beyond.

The project includes two types of joint research activities.  First,
it will as any infrastructure, it will allow joint research projects
between the users of this infrastructure that will be able to develop
new proofs together using different systems. Second, as any
infrastructure, {\sc Logipedia} raises new research problems. Some of
them have already been solved in the past and require to be
implemented jointly in a first version of the encyclopedia. Others are
newer and will trigger new cooperation between the teams of the project.

\paragraph{Innovation potential}


(e.g. ground-breaking objectives,
novel concepts and approaches, new products, services or business and
organisational models) which the proposal represents. Where relevant,
refer to products and services already available on the market. Please
refer to the results of any patent search carried out.


%%% Local Variables:
%%%   mode: latex
%%%   mode: flyspell
%%%   ispell-local-dictionary: "english"
%%% End:






