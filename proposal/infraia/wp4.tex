{\bf Coordinators:} Pascal Fontaine and Chantal Keller 

David Deharbe,
Cezary Kaliszyk,
Pascal Fontaine, Dale Miller, Stephan Merz, Josef Urban, Martin Suda,
Guillaume Burel, Filip Marić, Chantal Keller, Julien Narboux, Thibault Gauthier

[Nancy, Liège]

The importance of proofs in automated theorem provers, satisfiability
modulo theories solvers, propositional satisfiability solvers and
model checkers is increasingly recognized.  While for the
propositional case, the community agrees on a well defined proof
format, the situation is not clear for the other kind of automated
reasoners.  There is no clear format for SMT, and the TSTP format for
automated theorem provers fixes a syntactic template for proofs rather
than providing an unambiguous framework to express proofs
semantically.

Some preliminary works predating this proposal clearly establish that
Dedukti can accommodate proofs in Satisfiability Modulo Theories,
automated theorem provers, and SMT.  In this work package, we will
build on those preliminary work and provide a set of conduits from the
established formats used in automated tools. For the tools that do not
have yet an established format, we will make a selection of tools
(Zipperposition and E for automated theorem provers, CVC4 and veriT
for SMT, ??? for model checking) and provide a conduits for those
tools.  These conduits and the techniques used in the embedded
translation will be properly documented, to ease integration of
further tools of the kind.  If a standardized proof format appears for
some kind of tools, the conduits will be updated to adopt the new
standard.

In this work package, we also plan to integrate in Logipedia some
well-chosen proofs coming from automated tools.  Well-chosen proofs
will have to be representative of typical applications of the tools,
and be of reasonable size.  They will serve as examples to the
community, to illustrate the potentials of Dedukti and Logipedia.
