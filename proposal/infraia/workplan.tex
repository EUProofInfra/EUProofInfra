The project

The Logipedia kick-off meeting
http://deducteam.gforge.inria.fr/seminars/190121.html, in January
2019, brought together 38 researchers from Austria, Czech Republic,
France, Italy, the Netherlands, and Poland. Since then, colleagues
from Belgium, Germany, Serbia, Sweden, and the United Kingdom have
manifested interest. Some of these researchers are ready to contribute
to Logipedia, that currently contains a few hundred lemmas, aiming at
having in twenty years all the formal proofs then developed, in a
single encyclopedia.

Currently, we know how to express the theories of HOL Light, Matita,
and FoCaliZe in Dedukti and recheck proofs developed in these
systems. In the next five years, we plan to address the theories of
Abella, Agda, Atelier B, Coq, HOL4, Isabelle, Lean, Minlog,
Mizar, PVS, Rodin, and TLA+. Other systems, ACL2, IMPS, and Nuprl are
kept for later, except if some other groups join the project.


We propose to introduce a metric to measure the progress of the
integration of a library in Logipedia.


- Level 1: the theory implemented in the system X has been defined in
the lambda-Pi-calculus modulo theory and in Dedukti.

- Level 2: the system X has been instrumented to export of proof that
can be checked in Dedukti.

- Level 3: a significant part of the library of the system X has been exported and checked in Dedukti.

- Level 4: a tool has been defined to analyze the Dedukti X proofs,
detect those that can be expressed in a theory weaker than that of the
system X and translate those proofs in a weaker logic.

- Level 5: these proofs have been made available in Logipedia.

- Level 6: all proofs of the system X have been exported, translated
and made available in Logipedia.

The Logipedia readiness levels

The various systems addressed in the project are currently at this level 
Matita: level 5
FoCaliZe: level 3
HOL Light: level 3
Coq: level 3
Agda: 2
Atelier B: level 2
Isabelle: level 2
HOL4: level 1
Minlog: level 1
Rodin: level 1
Lean: level 0 to 1
PVS: level 0 to 1
Abella: level 0
Mizar: level 0
TLA+: level 0

The current Logipedia readiness levels of the systems addressed in the project

\subsection{Goals}

Specific, Measurable, Achievable, Relevant, Time-Bound

Matita: from level 5 to level 6
HOL Light: level 3 to level 5
FoCaliZe: from level 3 to level 5
Coq: level 3 to level 5
Agda: from level 2 to level 3
Atelier B: from level 2 to level 5
Isabelle: from level 2 to level 5
HOL4: from level 1 to level 5
Minlog: from level 1 to level 5
Rodin: from level 1 to level 5
Lean: from level 0 to 1 to level 3
PVS: from level 0 to 1 to level 2
Abella: from level 0 to level 2
Mizar: from level 0 to level 5
TLA+: from level 0 to level 2

Beyond our main focus on interactive systems, we also plan to
integrate some proofs coming from automated theorem provers, SMT
solvers, and model checkers, when these proofs have a reasonable
size. We already have experience with Zenon, iProver, and Archsat, but
we also plan to go in this direction, in cooperation with our
colleagues working on LFSC [Stump09].

Finally, we must also structure this encyclopedia: some of the
libraries we start with already have a structure (modules, qualified
names, etc.) that it is important to preserve.

But, in addition, each library contains a definition of natural
numbers, real numbers, etc. and, most importantly, logical connectors,
that must be aligned.  All these objectives contribute to building a
new formal proof community, focused on the values of knowledge
exchange and sustainability.


\subsection{Work packages and tasks}

{\color{red}

Questions :

- merge WP1 and WP4. Pros: ATP/SMT produce proofs that are not fundamentally
different from ITPs, cons: WP1 is already very big

- add a management WP

- add a dissemination WP

- Change the order : WP6 and WP7 before WP1.  
}


\subsubsection{WP 1: Instrument proof systems to produce Dedukti proofs}

{\bf coordinators}: Frédéric Blanqui and Jesper Cockx

David Deharbe, Tobias Nipkow, Guillaume Genestier, Jesper
Cockx, Guillaume Burel, Filip Marić, Makarius Wenzel, Helmut
Schwichtenberg, Nicolas Magaud, Gaspard Férey, Ulf Norell





We know how to express in Dedukti the theories implemented in Matita,
HOL Light, FoCaliZe, Coq, Agda, Lean, Minlog, Isabelle, HOL4,
Atelier B, and Rodin. The systems Matita, HOL Light, FoCaliZe, and
Coq, already have been instrumented to export proofs that can be
checked in Dedukti. Our first work package is to do the same thing for
Coq, Agda, Lean, Minlog, Isabelle, HOL4, Atelier B, and
Rodin. Three methods have to be used here: some of the systems
(Automath style), such as Coq, Agda, Lean, and Minlog already have
proof-terms that can be output, thus the main task is to translate
these proofs into the Dedukti format. Others (LCF style), such as
Isabelle and HOL4, have an inference kernel that can be
instrumented, and often already has, the main task here is to transform
the internal proof-object into an external proof-term. Others, such as
Atelier B and Rodin are slightly more difficult to address. For those,
we need to use the water ford method: extract an incomplete trace (a
sequence of lemmas) and fill the gap using automated theorem proving,
as experimented with Atelier B and Zenon.


\paragraph{Task 1.1: instrument Agda}

%[G\"oteborg, Delft]

%\textbf{Budget requirements:} One research engineer at Chalmers, and one PhD student or postdoc at TU Delft.

%Question: does this task belong to WP1 or WP2?

Agda is a popular dependently typed programming language / proof
assistant based on Martin-L\"of’s intuitionistic type theory. Its theory
is similar to Coq and Lean, but is more focused on interactive
development and direct manipulation of proof terms (in contrast to
using a tactic language to generate the proof terms). Agda has a
sizable standard library (available at
https://github.com/agda/agda-stdlib) that consists of both utilities
for programming and mathematical proofs.


In the summer of 2019, Guillaume Genestier worked together with Jesper
Cockx on the implementation of an experimental translator from Agda to
Dedukti during a research visit at Chalmers University in Sweden. This
translator is still work in progress, but it is already able to
translate 142 modules of the Agda standard library to a form that can
be checked in Dedukti. This exploratory work uncovered several
challenges and opportunities for further work, which are outlined
below.

\begin{enumerate}
\item To support the construction of proof terms, Agda provides powerful
features such as dependent pattern and copattern matching, eta
equality for functions and record types, and definitional proof
irrelevance. The first one – dependent pattern matching – can be
translated directly to rewrite rules in Dedukti. However, the two
latter features – eta equality and irrelevance – rely on Agda’s
type-directed conversion algorithm, while Dedukti’s conversion is
untyped. Hence in order to translate Agda proofs to Dedukti these
features need to be encoded.

One particular concern with the encoding of eta-equality is that in
general it requires storing of additional type information in the
proof terms. It can hence lead to a large blow-up in the size of those
proof terms, and thus greatly increase the cost of typechecking. The
same problem also occurs in other parts of Agda; for example
constructors of parametrized datatypes do not store the values of the
parameters, but they need to be reconstructed in the translation to
Dedukti. We plan to investigate two possible approaches to this
problem: either we can try to find a better encoding which reduces the
size of the type annotation, or alternatively we can extend the
Dedukti language with type-directed conversion rules to render the
type annotations unneccessary.

\item Another unique feature of Agda is the support for first-class
universe level polymorphism. In particular, Agda has a built-in type
of levels that has complex structure of (in)equality between
levels. Compared to universe polymorphism in Coq, an additional
challenge is that levels in Agda can contain arbitrary terms as
subexpressions. Our plan is to define a sound and complete embedding
of Agda’s level type in Dedukti, based on the existing work on
encoding AC (associative-commutative) theories. This would both serve
as a stress test of how well Dedukti can handle complex equational
theories, and improve our understanding of type theories with
first-class universe level polymorphism, which would be useful for the
implementation of Agda.

\item In contrast to Coq and Lean, Agda does not have a well-defined
core language to which proofs are elaborated. Instead, definitions are
translated to an internal representation that is relatively close to
the user input. This provides a challenge when translating Agda proofs
to Dedukti: each feature in Agda’s internal syntax needs to have its
own translation. As part of this project, we will hence investigate
possible designs for a core language for Agda. Having such a core
language would have several benefits: it would deepen our
understanding of the Agda language, it would increase the
trustworthiness of Agda proofs, and it would make it much easier to
export Agda terms to other languages (such as Dedukti in the context
of this project).

\item Agda provides an experimental option for extending the language
with user-defined rewrite rules, which are very similar to the rewrite
rules provided by Dedukti. Because of this similarity, we expect it to
be straightforward to translate rewrite rules from Agda to
Dedukti. However, by comparing the two implementations we hope to gain
new insights and find opportunities for improvement on both sides. The
interest of some of these features goes beyond just the Agda
language. In particular, Lean also supports definitional proof
irrelevance, as does Coq with the recent addition of the SProp
universe. Hence we plan to collaborate with the teams working on those
languages to improve the support for these features where there is
overlap.
\end{enumerate}

%%% Local Variables:
%%% mode: latex
%%% TeX-master: "../propB"
%%% End:


\paragraph{Task 1.2: instrument Lean}

[?]

\paragraph{Task 1.3: instrument Minlog}

%[LMU München]

\begin{enumerate}
\item Further develop and implement extensional realisability in Minlog: this is
  a necessary first step for bridging Minlog and Dedukti.
\item Express the core Minlog logic and proofs in Dedukti: this task is
  facilitated by the fact that both systems share proof terms and deduction
  modulo. The preliminary work on realisability is required for exporting
  programme extraction to Dedukti.
\item Properly encode coinduction and corecursion in Dedukti: these concepts are
  fundamental to Minlog, for example for representing real numbers as streams of
  digits, but they are not native to Dedukti.
\item Import a subset of Dedukti into Minlog, apply programme development by proof
  transformation, and export back. This will make Dedukti a usable tool for the
  development of proofs and programmes in constructive analysis and allow Minlog
  users to benefit from theories formalised using different proof assistants.
\end{enumerate}


%%%%%%%%% OLD TEXT %%%%
% The \href{http://minlog-system.de}{Minlog system} implements a theory of
% computable functionals (TCF) \cite{SchwichtenbergWainer12}.
% It is a form of higher order arithmetic where partial functionals are
% first-class citizens.

% The intended model of TCF is the Scott-Ershov model of partial
% continuous functionals \cite{Ershov77}. Computable functionals are defined
% by so-called computation rules, a form of (possibly non-terminating)
% defining equations understood as left-to-right conversion rules.  An
% important example is the corecursion operator, which is needed to
% define functions operating for instance on streams of signed digits (a
% convenient format to represent real numbers).  The logical framework
% allows to declare a proven equality as a rewrite rule.  Now it is
% tempting to identify two terms or formulas when they have the same
% normal form w.r.t. rewriting (including of course beta-conversion);
% this is often called deduction modulo rewriting.  However, in a setup
% like TCF where non-termination is allowed we cannot use normal forms,
% but we can consider two terms or formulas as identical when they have
% a common reduct.  This drastically simplifies proofs involving real
% number arithmetic.

% Another central feature of TCF (and hence the Minlog system) is that
% it internalizes a proof-theoretic realizability interpretation (in the
% form of Kreisel's so-called modified realizability, with realizers of
% higher type).  More precisely, for every (co)inductive predicate we
% have another one with one argument more, denoting a realizer.  It is
% important that this realizers is expressed in the term language of TCF
% (an extension of G\"odel's system $T$).  Since a realizer can be seen as
% a programme representing the computational content of a constructive
% existence proof (expressing that a certain specification has a
% solution), we now can reason about such programs in a formal way,
% inside TCF.  In fact, given a proof M in TCF of a specification
% $\forall x\exists y A(x,y)$ we can extract a term (program) $p_M$ and automatically
% generate a new proof of $\forall x A(p_M(x))$.  In other words, for
% programs generated in this way from existence proofs, formal
% verification is automatic.  Note that for a proof involving
% coinduction the extracted term contains the (non terminating)
% corecursion operator.  Of course, for efficient evaluation in a second
% step we want to translate our extracted term into an efficient
% (functional) programming language like Haskell.

% Other aspects of Minlog are more common.  It is a proof system in
% Gentzen-style natural deduction based on proof terms
% (the so-called Curry-Howard correspondence).  We distinguish between
% (co)inductive predicates with and without computational content; in
% fact, computational content only arises from (co)inductive predicates
% marked as computationally relevant (c.r.).  In particular, both
% universal und existential quantifiers do not influence computational
% content: one needs to relativize them to c.r. predicates (like
% totality) to make them computationally relevant.

% A central application area of Minlog is to formalize Bishop-style
% \cite{Bishop67} constructive analysis and extract interesting algorithms
% from proofs. An example is the Intermediate Value Theorem treated in
% \cite{LindstroemPalmgrenSegerbergStoltenberg08}.  More recently we have
% extracted algorithms operating on (both signed digit and Gray-coded)
% stream-represented real numbers from proofs which never mention
% streams.  They come in by relativizing real number quantifiers to
% appropriate coinductive predicates \cite{Berger09}.
% We will extend this
% work further into constructive analysis, e.g. Euler's existence
% proof of solutions of ordinary differential equations (ODE).



\paragraph{Task 1.4: instrument Isabelle}

% task leader: Tobias

% participants: Makarius (Isabelle), David Matthews (Poly/ML)

% Moved to concept & methodology
% Isabelle as a logical framework \cite{paulson700} is an intermediate
% between Type-Theory provers (like Coq or Agda) and classic LCF-style
% systems (like HOL Light or HOL4). The inference kernel can already
% output proofs as $\lambda$-terms on request, but this has so far been
% only used for small examples \cite{Berghofer-Nipkow:2000:TPHOL}. The
% challenge is to make Isabelle proof terms work robustly for the basic
% libraries and reasonably big applications.  Preliminary work by Wenzel
% (2019) has demonstrated the feasibility for relatively small parts of
% Isabelle/HOL, but this requires scaling up.

\begin{enumerate}
  \item Improve the efficiency of important aspects of the
  Isabelle/HOL logic implementation, such as normalization of proofs,
  type-class reasoning, and special representation of derived rules
  and definition principles.
  \item Reduce the volume of proof terms in the Dedukti encoding.
  \item Improve memory usage of the Isabelle/ML implementation
  platform.
\end{enumerate}


\paragraph{Task 1.5: instrument HOL4}

[G\"oteborg]

The HOL4 proof assistant is home to a few medium to large scale
specifications and associated proof developments that have value
outside of HOL4. These specifications include the formal semantics of
the CakeML language (and its verified compiler) and an extensive
specification of the ARM instruction set architecture (ISA) as
formalised by Anthony Fox at the University of Cambridge.

HOL4 has support for exporting proofs to the OpenTheory proof exchange
format, and there has been some work on importing OpenTheory proofs
into Dedukti. However, the current state of these techniques and their
implementations does not scale to real examples such as those
mentioned above.

This part of the project will be about re-thinking and re-designing
the tools HOL4-to-OpenTheory and OpenTheory-to-Dedukti tools such that
they scale to the point where real examples of interest, such as those
mentioned above, can be exported.

2 Person Years at Chalmers



\paragraph{Task 1.6: instrument Atelier-B}

[Southhampton, Toulouse, Clearsy].

\paragraph{Task 1.7: instrument Rodin}

[Southhampton, Toulouse, Clearsy].

\paragraph{Task 1.8: integrate the translator to Matita in Matita itself and export the full Matita library}

[Bologna]



\subsubsection{WP2: Defining theories in Dedukti}

\input{wp2}

\subsubsection{WP3 : Libraries}

Translating the standard libraries of the systems is part of the WP1. 

\paragraph{Task 3.1: MathComp}

 [Sophia, Saclay, Paris]

\paragraph{Task 3.2: the Mizar library}

[Innsbruck, Bialystok]

\paragraph{Task 3.3: Isabelle Archive of Formal Proofs}

[TU München, Saclay]

\paragraph{Task 3.4: the GeoCoq library}

[Sophia, Strasbourg, Belgrade]

\paragraph{Task 3.5: the Flyspeck library} 

[?]

\paragraph{Task 3.6: the NASA PVS library} 

[?]

\paragraph{Task 3.7: the seL4 library}

[?]

\paragraph{Task 3.8: the CompCert library}

[?]


\subsubsection{WP4: ATP, SAT, SMT, Model checkers}

{\bf Coordinators:} Pascal Fontaine and Chantal Keller 

David Deharbe,
Cezary Kaliszyk,
Pascal Fontaine, Dale Miller, Stephan Merz, Josef Urban, Martin Suda,
Guillaume Burel, Filip Marić, Chantal Keller, Julien Narboux, Thibault Gauthier

[Nancy, Liège]

The importance of proofs in automated theorem provers, satisfiability
modulo theories solvers, propositional satisfiability solvers and
model checkers is increasingly recognized.  While for the
propositional case, the community agrees on a well defined proof
format, the situation is not clear for the other kind of automated
reasoners.  There is no clear format for SMT, and the TSTP format for
automated theorem provers fixes a syntactic template for proofs rather
than providing an unambiguous framework to express proofs
semantically.

Some preliminary works predating this proposal clearly establish that
Dedukti can accommodate proofs in Satisfiability Modulo Theories,
automated theorem provers, and SMT.  In this work package, we will
build on those preliminary work and provide a set of conduits from the
established formats used in automated tools. For the tools that do not
have yet an established format, we will make a selection of tools
(Zipperposition and E for automated theorem provers, CVC4 and veriT
for SMT, ??? for model checking) and provide a conduits for those
tools.  These conduits and the techniques used in the embedded
translation will be properly documented, to ease integration of
further tools of the kind.  If a standardized proof format appears for
some kind of tools, the conduits will be updated to adopt the new
standard.

In this work package, we also plan to integrate in Logipedia some
well-chosen proofs coming from automated tools.  Well-chosen proofs
will have to be representative of typical applications of the tools,
and be of reasonable size.  They will serve as examples to the
community, to illustrate the potentials of Dedukti and Logipedia.


Create the infrastructure to enable the long term goal: be able to split a large proof obligation into smaller parts and distribute to the appropriate automatic engines, that would all produce proofs, glued together in a single large proof for the original proof obligation.

\subsection*{Automatic Tools Exporting Proofs}

\subsection*{Logipedia as a Source of Challenges for Automatic Reasoners}
--> Translation to TPTP, SMT-LIB, DIMACS

\subsection*{A language for Communication between Automatic Reasoners}



\subsubsection{WP5 : Reverse mathematics}

{\bf coordinators} : Nicola Gambino and Julien Narboux 

Nicola Gambino, Michael Rathjen, Guillaume Genestier, Julien Narboux,
François Thiré

[Saclay, Leeds]

\paragraph{Task 5.42: Ecumenical Dedukti}

[Grienenberger, Dowek]

We plan to define in {\sc Dedukti} both constructive and classical
connectives and quantifiers
following \cite{PrawitzPereira,DowekPereira,Pereira}, so that both
constructive and classical proofs can be expressed in {\sc Dedukti}.

We plan to develop constructivization algorithms to transform proofs
expressed in this theory, into its constructive fragment.

\paragraph{Task 5.43: A universal type theory}

[Grienenberger, Dowek]

A type theory that contains both a dependent and non dependent arrow
contains two fragments that correspond to Simple type theory and to
the Calculus of constructions. Such a theory can express proofs
developed in HOL Light, Isabelle, HOL4, Coq, Matita...

We plan to develop algorithms to transform proofs expressed in this
theory, into its Simple type theory fragment.


\subsubsection{WP6: Concept alignment}

Construct tools and proofs to analyze these proofs and align concepts,
that is unify concepts such as connectives and quantifiers, the
concept of natural number, etc. and theorems that occur in several
libraries.  [Paris, Saclay, Innsbruck, Prague, Strasbourg, Belgrade]


\subsubsection{WP7: Structuration of the theories}

[Erlangen, Saclay]


1. concrete/surface syntaxes 

1. Central Library Backend Systems 

2. Cross-System Front-Ends/Portals (Logipedia, ...)

3. Semantic Middleware-based System Interoperability 

Erlangen: two postdocs over the course of the project

Since proof-objects for substantial theory developments tend to be
very large (the representation of current POs for the Isabelle/AFP can
easily reach several TB although using techniques for compression), A
technical pre-requisite for interchangeability, connectivity and
advanced search consists in a structured, typed format for meta-data
together with a flexible mechanism of their validation. Technically,
this kind of meta-data has the form of a function annoconst : arg1 ->
... -> argn -> proof-term -> proof-term where annoconst is a constant
symbol which represents an identity in the proof-term (so, any import
function of a specific system can actually ignore it), and where the
argi represent terms with meta-information such as, eg., “this
proof-term represents a free data-type construction of the form ...”,
or “this part of the proof is a derivation of a free data-type of the
following form ...”, “this lifting over assumptions represents in
Isabelle/HOL a Locale-instantiation”, “this part of a theory
development is connected to ... ”, “this theorem belongs to the
sub-class of XXX ... theorems”, etcpp. For arguments of annotations,
validation-functions can be defined that may check that the argument
terms satisfy a certain property wrt. to the proof-term and the
current logical context. Dedukti will provide a framework that allows
for each proof-system (Coq, HOL4, Isabelle...) to declare meta-data
together with validations and thus communicate tool-specific knowledge
to other systems. This framework can be seen as a particular form of
an ontology definition language.
 
WP8: Indexing and browsing [?]  Construct tools to index and browse
this encyclopedia, that is find the theorem one needs, either by
looking for it with its name, with its statement, or with symbols
occurring in it.


\subsubsection{WP8: Dissemination, communication, and exploitation}

Club on industrials

Club of teachers

\subsubsection{WP9: Management}



Deliverables


Milestones

