The project

The Logipedia kick-off meeting
http://deducteam.gforge.inria.fr/seminars/190121.html, in January
2019, brought together 38 researchers from Austria, Czech Republic,
France, Italy, the Netherlands, and Poland. Since then, colleagues
from Belgium, Germany, Serbia, Sweden, and the United Kingdom have
manifested interest. Some of these researchers are ready to contribute
to Logipedia, that currently contains a few hundred lemmas, aiming at
having in twenty years all the formal proofs then developed, in a
single encyclopedia.

Currently, we know how to express the theories of HOL Light, Matita,
and FoCaliZe in Dedukti and recheck proofs developed in these
systems. In the next five years, we plan to address the theories of
Abella, Agda, Atelier B, Coq, HOL4, Isabelle, Lean, Minlog,
Mizar, PVS, Rodin, and TLA+. Other systems, ACL2, IMPS, and Nuprl are
kept for later, except if some other groups join the project.


We propose to introduce a metric to measure the progress of the
integration of a library in Logipedia.


- Level 1: the theory implemented in the system X has been defined in
the lambda-Pi-calculus modulo theory and in Dedukti.

- Level 2: the system X has been instrumented to export of proof that
can be checked in Dedukti.

- Level 3: a significant part of the library of the system X has been exported and checked in Dedukti.

- Level 4: a tool has been defined to analyze the Dedukti X proofs,
detect those that can be expressed in a theory weaker than that of the
system X and translate those proofs in a weaker logic.

- Level 5: these proofs have been made available in Logipedia.

- Level 6: all proofs of the system X have been exported, translated
and made available in Logipedia.

The Logipedia readiness levels

The various systems addressed in the project are currently at this level 
Matita: level 5
FoCaliZe: level 3
HOL Light: level 3
Coq: level 3
Agda: 2
Atelier B: level 2
Isabelle: level 2
HOL4: level 1
Minlog: level 1
Rodin: level 1
Lean: level 0 to 1
PVS: level 0 to 1
Abella: level 0
Mizar: level 0
TLA+: level 0

The current Logipedia readiness levels of the systems addressed in the project

\subsection{Goals}

Specific, Measurable, Achievable, Relevant, Time-Bound

Matita: from level 5 to level 6
HOL Light: level 3 to level 5
FoCaliZe: from level 3 to level 5
Coq: level 3 to level 5
Agda: from level 2 to level 3
Atelier B: from level 2 to level 5
Isabelle: from level 2 to level 5
HOL4: from level 1 to level 5
Minlog: from level 1 to level 5
Rodin: from level 1 to level 5
Lean: from level 0 to 1 to level 3
PVS: from level 0 to 1 to level 2
Abella: from level 0 to level 2
Mizar: from level 0 to level 5
TLA+: from level 0 to level 2

Beyond our main focus on interactive systems, we also plan to
integrate some proofs coming from automated theorem provers, SMT
solvers, and model checkers, when these proofs have a reasonable
size. We already have experience with Zenon, iProver, and Archsat, but
we also plan to go in this direction, in cooperation with our
colleagues working on LFSC [Stump09].

Finally, we must also structure this encyclopedia: some of the
libraries we start with already have a structure (modules, qualified
names, etc.) that it is important to preserve.

But, in addition, each library contains a definition of natural
numbers, real numbers, etc. and, most importantly, logical connectors,
that must be aligned.  All these objectives contribute to building a
new formal proof community, focused on the values of knowledge
exchange and sustainability.


\subsection{Work packages and tasks}

{\color{red}

Questions :

- merge WP1 and WP4. Pros: ATP/SMT produce proofs that are not fundamentally
different from ITPs, cons: WP1 is already very big

- add a management WP

- add a dissemination WP

- Change the order : WP6 and WP7 before WP1.  
}


\subsubsection{WP 1: Instrument proof systems to produce Dedukti proofs}

We know how to express in Dedukti the theories implemented in Matita,
HOL Light, FoCaliZe, Coq, Agda, Lean, Minlog, Isabelle, HOL4,
Atelier B, and Rodin. The systems Matita, HOL Light, FoCaliZe, and
Coq, already have been instrumented to export proofs that can be
checked in Dedukti. Our first work package is to do the same thing for
Coq, Agda, Lean, Minlog, Isabelle, HOL4, Atelier B, and
Rodin. Three methods have to be used here: some of the systems
(Automath style), such as Coq, Agda, Lean, and Minlog already have
proof-terms that can be output, thus the main task is to translate
this proofs into the Dedukti format. Others (LCF style), such as
Isabelle and HOL4, have an inference kernel that can be
instrumented, and often already has, the main task here is to transform
the internal proof-object into an external proof-term. Others, such as
Atelier B and Rodin are slightly more difficult to address. For those,
we need to use the water ford method: extract an incomplete trace (a
sequence of lemmas) and fill the gap using automated theorem proving,
as experimented with Atelier B and Zenon.


\paragraph{Task 1.1: instrument Agda}

% task leader: Ulf or Jesper

% other participants:

%\textbf{Budget requirements:} One research engineer at Chalmers, and one PhD student or postdoc at TU Delft.

% Moved to concept & methodology
% Agda is a popular dependently typed programming language / proof
% assistant based on Martin-L\"of’s intuitionistic type theory. Its theory
% is similar to Coq and Lean, but is more focused on interactive
% development and direct manipulation of proof terms (in contrast to
% using a tactic language to generate the proof terms). Agda has a
% sizable \href{https://github.com/agda/agda-stdlib}{standard library}
% that consists of both utilities for programming and mathematical proofs.

% Moved to concept & methodology
% In the summer of 2019, Guillaume Genestier (Inr) worked together with Jesper
% Cockx (Got) on the implementation of an experimental translator from Agda to
% Dedukti during a research visit at Chalmers University in Sweden. This
% translator is still work in progress, but it is already able to
% translate 142 modules of the Agda standard library to a form that can
% be checked in Dedukti. This exploratory work uncovered several
% challenges and opportunities for further work, which are outlined
% below.

\begin{enumerate}
  \item Agda proofs often rely on type-directed conversion rules such
  as eta-equality and definitional irrelevance, which can lead to a
  blow-up in the size of proof terms. We plan to investigate possible
  approaches to avoid this blow-up, either by finding a better
  encoding which reduces the size of the type annotation, or by
  extending the Dedukti language with type-directed conversion rules
  to render the type annotations unneccessary.

  \item Universe polymorphism in Agda relies on a built-in type of
  levels that has complex structure of (in)equalities. We plan to
  define a sound and complete embedding of Agda’s level type in
  Dedukti, based on the existing work on encoding AC
  (associative-commutative) theories. This would both serve as a
  stress test of how well Dedukti can handle complex equational
  theories, and improve our understanding of type theories with
  first-class universe level polymorphism, which would be useful for
  the implementation of Agda.

  % Removed from this proposal for now
  % \item In contrast to Coq and Lean, Agda does not have a well-defined
  % core language to which proofs are elaborated. Instead, definitions
  % are translated to an internal representation that is relatively
  % close to the user input. This provides a challenge when translating
  % Agda proofs to Dedukti: each feature in Agda’s internal syntax needs
  % to have its own translation. As part of this project, we will hence
  % investigate possible designs for a core language for Agda. Having
  % such a core language would have several benefits: it would deepen
  % our understanding of the Agda language, it would increase the
  % trustworthiness of Agda proofs, and it would make it much easier to
  % export Agda terms to other languages (such as Dedukti in the context
  % of this project).

  % \item Agda provides an experimental option for extending the
  % language with user-defined rewrite rules, which are very similar to
  % the rewrite rules provided by Dedukti. By comparing the two
  % implementations we hope to gain new insights and find opportunities
  % for improvement on both sides.
\end{enumerate}

%%% Local Variables:
%%% mode: latex
%%% TeX-master: "../propB"
%%% End:


\paragraph{Task 1.2: instrument Lean}

[?]

\paragraph{Task 1.3: instrument Minlog}

[LMU München]

The Minlog system <http://minlog-system.de> implements a theory of
computable functionals (TCF, cf. [SchwichtenbergWainer12], Ch.7).).
It is a form of higher order arithmetic where partial functionals are
first-class citizens.

The intended model of TCF is the Scott-Ershov model $C$ of partial
continuous functionals [Ershov77].  Computable functionals are defined
by so-called computation rules, a form of (possibly non-terminating)
defining equations understood as left-to-right conversion rules.  An
important example is the corecursion operator, which is needed to
define functions operating for instance on streams of signed digits (a
convenient format to represent real numbers).  The logical framework
allows to declare a proven equality as a rewrite rule.  Now it is
tempting to identify two terms or formulas when they have the same
normal form w.r.t. rewriting (including of course beta-conversion);
this is often called deduction modulo rewriting.  However, in a setup
like TCF where non-termination is allowed we cannot use normal forms,
but we can consider two terms or formulas as identical when they have
a common reduct.  This drastically simplifies proofs involving real
number arithmetic.

Another central feature of TCF (and hence the Minlog system) is that
it internalizes a proof-theoretic realizability interpretation (in the
form of Kreisel's so-called modified realizability, with realizers of
higher type).  More precisely, for every (co)inductive predicate we
have another one with one argument more, denoting a realizer.  It is
important that this realizers is expressed in the term language of TCF
(an extension of G\"odel's system $T$).  Since a realizer can be seen as
a program representing the computational content of a constructive
existence proof (expressing that a certain specification has a
solution), we now can reason about such programs in a formal way,
inside TCF.  In fact, given a proof M in TCF of a specification all x
ex y A(x,y) we can extract a term (program) t$_M$ and automatically
generate a new proof of all x A(t$_M$(x)).  In other words, for
programs generated in this way from existence proofs, formal
verification is automatic.  Note that for a proof involving
coinduction the extracted term contains the (non terminating)
corecursion operator.  Of course, for efficient evaluation in a second
step we want to translate our extracted term into an efficient
(functional) programming language like Haskell.

Other aspects of Minlog are more common.  It is a proof system in
Gentzen-style natural deduction and therefore based on proof terms
(the so-called Curry-Howard correspondence).  We distinguish between
(co)inductive predicates with and without computational content; in
fact, computational content only arises from (co)inductive predicates
marked as computationally relevant (c.r.).  In particular, both
universal und existential quantifiers do not influence computational
content: one needs to relativize them to c.r. predicates (like
totality) to make them computationally relevant.

A central application area of Minlog is to formalize Bishop-style
[Bishop67] constructive analysis and extract interesting algorithms
from proofs. An example is the Intermediate Value Theorem treated in
[LindstroemPalmgrenSegerbergStoltenberg08].  More recently we have
extracted algorithms operating on (both signed digit and Gray-coded)
stream-represented real numbers from proofs which never mention
streams.  They come in by relativizing real number quantifiers to
appropriate coinductive predicates [Berger09].

The work planned to be done by the Minlog group is to extend this kind
of work further into constructive analysis, e.g. Euler's existence
proof of solutions of ODEs.  We would like to learn from the
experience of other proof assistant developers working on related
matters, both conceptually and by sharing libraries.  It would be very
helpful if this can be done in a unified setting where different
approaches can be compared.

Here we need to go a little futher and propose to express Minlog
proofs in Dedukti.

One Postdoc and one Phd position


\paragraph{Task 1.4: instrument Isabelle}

[TU München]

\paragraph{Task 1.5: instrument HOL4}

[G\"oteborg]

\paragraph{Task 1.6: instrument Atelier-B}

[Southhampton, Toulouse, Clearsy].

\paragraph{Task 1.7: instrument Rodin}

[Southhampton, Toulouse, Clearsy].

\paragraph{Task 1.8: integrate the translator to Matita in Matita itself and export the full Matita library}

[Bologna]



\subsubsection{WP2: Defining theories in Dedukti}

For other theories, such as Abella, PVS, Mizar and TLA+, we have not yet investigated the possibility to express them in Dedukti.

Task 2.1: express the theory of PVS in Dedukti and instrument the system [Saclay].


Task 2.2: express the theory of Mizar in Dedukti and instrument the system [Innsbruck, Bialystok].

Task 2.3: express the theory of TLA+ in Dedukti and instrument the system [Nancy, Liège].

Task 2.4: express the theory of Abella in Dedukti and instrument the system [Saclay].

The usual approach to capturing either Peano and Heyting arithmetics
is to use various axioms (and an axiom scheme for induction) on top of
classical and intuitionistic first-order logic.  Indeed, this is the
approach used in the Dedukti proof checker.


A different approach to encoding arithmetic has been developed over
the past 20-30 years, starting with papers by Schroeder-Heister and
Girard in the early 1990s and extended in a series of papers by
Baelde, Gacek, McDowell, M, Momigliano, Nadathur, and Tiu.  In this
new setting, first-order logic is extended by considering both
equality and the least fixed point operator as \emph{logical
  connectives}: these logical connectives are not available directly
in Dedukti.

This new foundations for arithmetic has been implemented in two
systems: the automated Bedwyr prover and the interactive Abella
prover.  While neither Bedwyr nor Abella are as popular as many of the
theorem provers that are covered by this proposal, there are two
important reasons to consider incorporating them into the Logipedia
effort.

First, the Bedwyr prover is capable of constructing proofs for the
kind of queries that are part of emph{model checkers}.  This class of
provers has not yet been incorporated into Dedukti.  The
proof-theoretic work behind model checking in Bedwyr should provide
some of the insights needed for allowing Dedukti to proof check the
results of model checkers.

Second, Bedwyr and Abella provide for direct and elegant support of
meta-level reasoning.  Given that the foundations for Bedwyr and
Abella have been given using Gentzen's sequent calculus, it was
possible to enrich their foundations to allow for the treatment of
binding structures within terms.  As a result, it is possible to
reason directly on terms representing $\lambda$-terms and
$\pi$-calculus expressions.  In particular, the Abella prover has
probably the most natural and compact formal treatment of the
$\pi$-calculus and its meta-theory when compared to all other attempts
in any other theorem provers.  More generally, the Abella prover
should be able to treat the meta-theory of programming and
specification languages as well as various logics and their
proofs. While these tasks are not the typical tasks considered by the
majority of theorem provers within the scope of this proposal,
meta-theory results do play an important role at times: in fact, the
ultimate questions as to whether or not a proof checkers (such as that
used by Dedukti) is correct or not will involve meta-theoretic
questions.

We propose to work on the general problem of exporting proofs from
Abella to Dedukti.  (Since all proofs that are constructed
automatically via Bedwyr can also be constructed manually within
Abella, we shall limit our discussion below to Abella only.)  The
proposed work will serve not only to answer the question of how to
relate these two different foundations for arithmetic but also to
allow Abella's particular style of proofs to find applications in the
wider world of formalized proofs.

The general problem described above has the following constituent parts.

(1) Proofs involving searching finite structures. Proofs built for
model checking problems over finite structures have two different
kinds of phases.  To illustrate, consider trying to find a specific
node within a binary tree.  If such a node exists, then the proof
essentially encodes the path to the node in the tree.  If, however, no
such node exists, then the proof of that negative fact is essentially
a computation that exhaustively explores the tree.  Using the Dedukti
terminology: in the first case, the proof involves several deduction
steps, while in the second case, the proof involves a pure
computation. When dealing with model checking problems such as
simulation (in concurrency theory) and winning strategies (in game
theory), proofs will involve alternating phases involving either
deduction or computation.  Since the notion of computation in
Abella-style proofs involves backtracking search, that style
computation will be quite different from Dedukti's notion of
computation as confluent rewriting.

(2) Extending model checking problems to the general case of infinite
structures and the associated inductive reasoning methods. Although
the formal basis of Abella uses least and greatest fixed-point
combinators and explicit (co-)invariants, the Abella implementation of
(co-)induction is based on cyclic reasoning using size-annotated
relations. It is known, in principle, how to convert cyclic proofs
using annotations to proofs with explicit invariants, but an invariant
extraction procedure that works in all cases is still missing. Once
such invariants are available, incorporating them into Dedukti should
be straightforward in association with part (1).

(3) Binding structures. Abella, as well as several other computational
logic systems ($\lambda$Prolog, Isabelle/Pure, Twelf, Beluga, etc)
make use of the so-called \emph{$\lambda$-tree syntax} (a form of
\emph{higher-order abstract syntax}, HOAS) approach to represent
bindings. This approach is further enriched in Abella with the
$\nabla$-quantifier that allows inductive and co-inductive properties
to be defined based on the \emph{structure} of $\lambda$-terms. We
propose to examine encodings of $lambda$-tree syntax in Dedukti. The
best approach probably involves extending the underlying theory of
Dedukti with a quantifier similar to Abella's $\nabla$-quantifier.

(4) Reflective treatment of unification. One of the features of
Abella's style of proofs is the use of left-introduction rules for
equality that exhaustively examine complete sets of unifiers for
$\lambda$-terms. This is implemented in terms of a unification engine
that is currently a trusted black box, which complicates any proposal
for exporting proofs to different implementations of unification or
equality. In Dedukti the unification procedure can be recast as a
rewrite system, but it is unclear how to derive reflective properties
based on the unifiability of terms.

Task 2.5: expressing HoTT [Saclay, Leeds]


\subsubsection{WP3 : Libraries}

Translating the standard libraries of the systems is part of the WP1. 

Task 3.1: MathComp [Sophia, Saclay, Paris]

Task 3.2: the Mizar library [Innsbruck, Bialystok]

Task 3.3: Isabelle Archive of Formal Proofs [München, Saclay]

Task 3.4: the GeoCoq library [Sophia, Strasbourg, Belgrade]

Task 3.5: the Flyspeck library [?]

Task 3.6: the NASA PVS library [?]

Task 3.7: the seL4 library [?]

Task 3.8: the CompCert library [?]


\subsubsection{WP4: ATP, SAT, SMT, Model checkers}

{\bf Coordinators:} Pascal Fontaine and Chantal Keller 

David Deharbe,
Cezary Kaliszyk,
Pascal Fontaine, Dale Miller, Stephan Merz, Josef Urban, Martin Suda,
Guillaume Burel, Filip Marić, Chantal Keller, Julien Narboux, Thibault Gauthier

[Nancy, Liège]

The importance of proofs in automated theorem provers, satisfiability
modulo theories solvers, propositional satisfiability solvers and
model checkers is increasingly recognized.  While for the
propositional case, the community agrees on a well defined proof
format, the situation is not clear for the other kind of automated
reasoners.  There is no clear format for SMT, and the TSTP format for
automated theorem provers fixes a syntactic template for proofs rather
than providing an unambiguous framework to express proofs
semantically.

Some preliminary works predating this proposal clearly establish that
Dedukti can accommodate proofs in Satisfiability Modulo Theories,
automated theorem provers, and SMT.  In this work package, we will
build on those preliminary work and provide a set of conduits from the
established formats used in automated tools. For the tools that do not
have yet an established format, we will make a selection of tools
(Zipperposition and E for automated theorem provers, CVC4 and veriT
for SMT, ??? for model checking) and provide a conduits for those
tools.  These conduits and the techniques used in the embedded
translation will be properly documented, to ease integration of
further tools of the kind.  If a standardized proof format appears for
some kind of tools, the conduits will be updated to adopt the new
standard.

In this work package, we also plan to integrate in Logipedia some
well-chosen proofs coming from automated tools.  Well-chosen proofs
will have to be representative of typical applications of the tools,
and be of reasonable size.  They will serve as examples to the
community, to illustrate the potentials of Dedukti and Logipedia.


\subsubsection{WP5 : Reverse mathematics}

{\bf coordinators} : Nicola Gambino, Julien Narboux 

Nicola Gambino, Michael Rathjen, Guillaume Genestier, Julien Narboux,
François Thiré

[Saclay, Leeds]

With a focus on HOL Light, Isabelle, HOL4 , Coq,
Matita, Agda, Lean, in sttcoc and its stt fragment.

Reverse mathematics in Focalize, Atelier B, Rodin, PVS, Mizar, TLA+
are left for the future.



\subsubsection{WP6: Concept alignment}

{\bf Coordinators:} Dale Miller, Thibault Gauthier

Florian Rabe, Cezary Kaliszyk, Dale Miller, Josef Urban,
Filip Marić, Yamine Ait Ameur, Jean-Paul Bodeveix, Mamoun Filali,
Chanta Keller, Julien Narboux, Nicola Magaud, Arthur Charguéraud,
François Thiré

Construct tools and proofs to analyze these proofs and align concepts,
that is unify concepts such as connectives and quantifiers, the
concept of natural number, etc. and theorems that occur in several
libraries.  [Paris, Saclay, Innsbruck, Prague, Strasbourg, Belgrade]


\subsubsection{WP7: Structuration of the theories}

{\bf Coordinators}: Florian Rabe and Burkhart Wolff

David Deharbe, Nicola Gambino, Florian Rabe, Michael Rathjen,
Dale Miller



[Erlangen, Saclay]


1. concrete/surface syntaxes 

1. Central Library Backend Systems 

2. Cross-System Front-Ends/Portals (Logipedia, ...)

3. Semantic Middleware-based System Interoperability 

Erlangen: two postdocs over the course of the project

Since proof-objects for substantial theory developments tend to be
very large (the representation of current POs for the Isabelle/AFP can
easily reach several TB although using techniques for compression), A
technical pre-requisite for interchangeability, connectivity and
advanced search consists in a structured, typed format for meta-data
together with a flexible mechanism of their validation. Technically,
this kind of meta-data has the form of a function annoconst : arg1 ->
... -> argn -> proof-term -> proof-term where annoconst is a constant
symbol which represents an identity in the proof-term (so, any import
function of a specific system can actually ignore it), and where the
argi represent terms with meta-information such as, eg., “this
proof-term represents a free data-type construction of the form ...”,
or “this part of the proof is a derivation of a free data-type of the
following form ...”, “this lifting over assumptions represents in
Isabelle a Locale-instantiation”, “this part of a theory
development is connected to ... ”, “this theorem belongs to the
sub-class of XXX ... theorems”, etcpp. For arguments of annotations,
validation-functions can be defined that may check that the argument
terms satisfy a certain property wrt. to the proof-term and the
current logical context. Dedukti will provide a framework that allows
for each proof-system (Coq, HOL4, Isabelle...) to declare meta-data
together with validations and thus communicate tool-specific knowledge
to other systems. This framework can be seen as a particular form of
an ontology definition language.
 
WP8: Indexing and browsing [?]  Construct tools to index and browse
this encyclopedia, that is find the theorem one needs, either by
looking for it with its name, with its statement, or with symbols
occurring in it.





Deliverables


Milestones

