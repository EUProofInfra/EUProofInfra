The proposal meets several of the objectives of the call
INFRAIA-02-2020: Integrating activities for starting communities
(listed on page 56 of the part 4 of Horizon 2020 Work Programme
2018-2020). It also clearly refers to one of the mentioned
cross-cutting activities: open science.

\paragraph{Synergy – Open research data.}
Instead of having a scattered community,
each group developing a library for its own logic and its own system,
researchers will be able to work together on common developments,
reusing proofs developed in other systems and in other communities. In
terms of networking, we have already organized one logipedia meeting
and the funding will insure we can continue to organize large-scale
international meetings on a regular basis. The first logipedia event
has proven to be very valuable in terms of exchange of best practices.

\paragraph{Wider and more efficient access.}
In a shared public encyclopedia
providing virtual access, each user can find formal proofs in the
logic she wants, regardless the logic and system this proof has been
developed in. A data management plan will be provided according to
Inria’s compliance policy with EU regulations. We also engage
ourselves to make the evolution of the logipedia e-infrastructure
compliant with the European charter for access to research
infrastructures.


\paragraph{Education – Closer interaction between a larger number of
researchers.}
Education to formal methods in computer science and to
formal proofs in mathematics always hits the same obstacle: the need
to choose a specific theory or system, which is in contradiction with
the universality of logical truth. Education to formal methods and
formal proofs will gain in universality once it will be demonstrated
that this choice amounts to include, or not, a few axioms and
reduction rules. We believe that this renewal of logic education at
university level and before is of prime importance in our “post-truth
era”.


\paragraph{Better management of the continuous flow of data.}
A shared
encyclopedia allows a better sustainability of the formal proofs
developed over time. Too many formal proofs developed in the past are
not available any more.


\paragraph{Innovation.}
As said earlier, we expect to make formal proofs more
accessible to industry by allowing each project to benefit of the
formalization work shared with other communities.
