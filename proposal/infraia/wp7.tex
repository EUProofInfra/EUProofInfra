[Erlangen, Saclay]


1. concrete/surface syntaxes 

1. Central Library Backend Systems 

2. Cross-System Front-Ends/Portals (Logipedia, ...)

3. Semantic Middleware-based System Interoperability 

Erlangen: two postdocs over the course of the project

Since proof-objects for substantial theory developments tend to be
very large (the representation of current POs for the Isabelle/AFP can
easily reach several TB although using techniques for compression), A
technical pre-requisite for interchangeability, connectivity and
advanced search consists in a structured, typed format for meta-data
together with a flexible mechanism of their validation. Technically,
this kind of meta-data has the form of a function annoconst : arg1 ->
... -> argn -> proof-term -> proof-term where annoconst is a constant
symbol which represents an identity in the proof-term (so, any import
function of a specific system can actually ignore it), and where the
argi represent terms with meta-information such as, eg., “this
proof-term represents a free data-type construction of the form ...”,
or “this part of the proof is a derivation of a free data-type of the
following form ...”, “this lifting over assumptions represents in
Isabelle/HOL a Locale-instantiation”, “this part of a theory
development is connected to ... ”, “this theorem belongs to the
sub-class of XXX ... theorems”, etcpp. For arguments of annotations,
validation-functions can be defined that may check that the argument
terms satisfy a certain property wrt. to the proof-term and the
current logical context. Dedukti will provide a framework that allows
for each proof-system (Coq, HOL4, Isabelle...) to declare meta-data
together with validations and thus communicate tool-specific knowledge
to other systems. This framework can be seen as a particular form of
an ontology definition language.
 
WP8: Indexing and browsing [?]  Construct tools to index and browse
this encyclopedia, that is find the theorem one needs, either by
looking for it with its name, with its statement, or with symbols
occurring in it.
