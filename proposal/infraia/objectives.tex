Bugs may kill. Would you fly in an autonomous plane fully driven by a
piece of software that has not been formally verified?  Or formally
verified using a technology that Europe does not master? Today, the
trust in critical systems relies on formal methods, in particular
formal proofs, that guaranty the safety of the people using
transportation systems (autonomous car, metros, trains, planes, etc.),
health systems (robotic surgery, etc.), energy provided by nuclear
plants, etc. The crucial role of formal proof is highlighted by
several successes, like the correctness proof of the automatic Paris
metro line 14 \cite{metro14} or the detect and avoid system for
unmanned aircraft system developed by Nasa \cite{Munoz16}.

\thispagestyle{empty}

\begin{figure}
\begin{tabular}{ll}
{\sc Abella}~~~~~~~~~~~~~~~~~~~~~~~~~~~~~~&{\sc Acl2}\\
{\sc Agda}  &  {\sc HOL Light}\\
{\sc Atelier B}  &  {\sc IMPS}\\
{\sc Coq}  &  {\sc Lean}\\
{\sc FoCaliZe}  &  {\sc Nuprl}\\
{\sc HOL4}  &  {\sc PVS}\\
{\sc Isabelle}  &  {\sc TLA+}\\
{\sc Matita}\\
{\sc Minlog}\\
{\sc Mizar}\\
{\sc Rodin}\\
\end{tabular}
\caption{Some major proof systems (European ones are in the first column) \label{systems}}
\end{figure}

Unfortunately, the development of formal methods is slowed down by the
multiplicity of proof systems and the lack of a common theory used by
these systems. This restricts interoperability, sustainability,
certification, etc.  each small community being centered around one
theory and one system, and often re-doing work done elsewhere.  For
instance, the metro line 14 has been proved correct in {\sc Atelier
B}, while the Nasa detect and avoid system for unmanned aircraft
system has been proved correct in {\sc PVS}.  There are around twenty
major proof systems in the world (Figure \ref{systems}) and making
these systems interoperable would avoid work duplication, reduce the
development time, and allow independent certification.

After three decades dedicated to the developement of these systems,
allowing such a cooperation between these systems is the next step in
the developement of the formal proof technology.  To reach this goal
we propose to build an on line encyclopedia of formal proofs called
{\sc Logipedia}, that indicates which proof can be used in which
system and, when it is the case, provides a version of the proof in
this theory.

Such a project can only have a worldwide ambition. However, as more
than half of these systems are European, Europe can take the lead on
this project, so that the economic spinoffs from the project benefit
the active participants mainly based in Europe.  That is why the
consortium gathers most of the European actors active on formal proof
systems, and also proposes to develop links with non European
partners.


\paragraph{Networking activities}
This project will include networking activities, to foster a culture
of cooperation between scientific communities, that are today often
too centered around one system and one theory. Instead such a project
will incent them to build this encyclopedia together and to always
wonder if the proofs they develop are specific to one system or are
universal. This will also foster a culture of cooperation between
scientific communities and two communities of users: teachers and
industrial partners. This is why the project includes two ``clubs of
users'', one of teachers and one of companies using formal methods.

\paragraph{Transnational access or virtual access activities}
Our encyclopedia being on line, it will of course be accessible from
every country in Europe and beyond.

\paragraph{Joint research activities}
The project includes two types of Joint research activities.  First,
it will as any infrastructure, it will allow joint research projects
between the users of this infrastructure that will be able to develop
new proofs together using different systems. Second, as any
infrastructure, {\sc Logipedia} raises new research problems. Some of
them have already been solved in the past and require to be
implemented jointly in a first version of the encyclopedia. Others are
newer and will trigger new cooperation between the teams of the project.

%%% Local Variables:
%%%   mode: latex
%%%   mode: flyspell
%%%   ispell-local-dictionary: "english"
%%% End:

