Bugs may kill. Would you fly in an autonomous plane fully driven by a
piece of software that has not been formally verified?  Or formally
verified using a technology that Europe does not master? Today, the
trust in critical systems relies on formal methods, in particular
formal proofs, that guaranty the safety of the people using
transportation systems (autonomous car, metros, trains, planes, etc.),
health systems (robotic surgery, etc.), energy provided by nuclear
plants, etc. The crucial role of formal proof is highlighted by
several successes, like the correctness proof of the automatic Paris
metro line 14 \cite{metro14} or the detect and avoid system for
unmanned aircraft system developed by Nasa \cite{Munoz16}.

\thispagestyle{empty}

\begin{figure}
\begin{tabular}{ll}
{\sc Abella}~~~~~~~~~~~~~~~~~~~~~~~~~~~~~~&{\sc Acl2}\\
{\sc Agda}  &  {\sc HOL Light}\\
{\sc Atelier B}  &  {\sc IMPS}\\
{\sc Coq}  &  {\sc Lean}\\
{\sc FoCaliZe}  &  {\sc Nuprl}\\
{\sc HOL 4}  &  {\sc PVS}\\
{\sc Isabelle / HOL}  &  {\sc TLA+}\\
{\sc Matita}\\
{\sc Minlog}\\
{\sc Mizar}\\
{\sc Rodin}\\
\end{tabular}
\caption{Some major proof systems (European ones are in the first column) \label{systems}}
\end{figure}

Unfortunately, the development of formal methods is slowed down by the
multiplicity of proof systems and the lack of a common theory used by
these systems. This restricts interoperability, sustainability,
certification, etc.  each small community being centered around one
theory and one system, and often re-doing work done elsewhere.  For
instance, the metro line 14 has been proved correct in {\sc Atelier
B}, while the Nasa detect and avoid system for unmanned aircraft
system has been proved correct in {\sc PVS}.  There are around twenty
major proof systems in the world (Figure \ref{systems}) and making
these systems interoperable would avoid work duplication, reduce the
development time, and allow independent certification.

After three decades dedicated to the developement of these systems,
allowing such a cooperation between these systems is the next step in
the developement of the formal proof technology.  To reach this goal
we propose to build an on line encyclopedia of formal proofs called
{\sc Logipedia}, that indicates which proof can be used in which
system and, when it is the case, provides a version of the proof in
this theory.

Such a project can only have a worldwide ambition. However, as more
than half of these systems are European, Europe can take the lead on
this project, so that the economic spinoffs from the project benefit
the active participants mainly based in Europe.  That is why the
consortium gathers most of the European actors active on formal proof
systems, and also proposes to develop links with non European
partners.
