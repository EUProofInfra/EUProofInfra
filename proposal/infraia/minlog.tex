[München]

The Minlog system <http://minlog-system.de> implements a theory of
computable functionals (TCF, cf. [SchwichtenbergWainer12], Ch.7).).
It is a form of higher order arithmetic where partial functionals are
first-class citizens.

The intended model of TCF is the Scott-Ershov model $C$ of partial
continuous functionals [Ershov77].  Computable functionals are defined
by so-called computation rules, a form of (possibly non-terminating)
defining equations understood as left-to-right conversion rules.  An
important example is the corecursion operator, which is needed to
define functions operating for instance on streams of signed digits (a
convenient format to represent real numbers).  The logical framework
allows to declare a proven equality as a rewrite rule.  Now it is
tempting to identify two terms or formulas when they have the same
normal form w.r.t. rewriting (including of course beta-conversion);
this is often called deduction modulo rewriting.  However, in a setup
like TCF where non-termination is allowed we cannot use normal forms,
but we can consider two terms or formulas as identical when they have
a common reduct.  This drastically simplifies proofs involving real
number arithmetic.

Another central feature of TCF (and hence the Minlog system) is that
it internalizes a proof-theoretic realizability interpretation (in the
form of Kreisel's so-called modified realizability, with realizers of
higher type).  More precisely, for every (co)inductive predicate we
have another one with one argument more, denoting a realizer.  It is
important that this realizers is expressed in the term language of TCF
(an extension of G\"odel's system $T$).  Since a realizer can be seen as
a program representing the computational content of a constructive
existence proof (expressing that a certain specification has a
solution), we now can reason about such programs in a formal way,
inside TCF.  In fact, given a proof M in TCF of a specification all x
ex y A(x,y) we can extract a term (program) t$_M$ and automatically
generate a new proof of all x A(t$_M$(x)).  In other words, for
programs generated in this way from existence proofs, formal
verification is automatic.  Note that for a proof involving
coinduction the extracted term contains the (non terminating)
corecursion operator.  Of course, for efficient evaluation in a second
step we want to translate our extracted term into an efficient
(functional) programming language like Haskell.

Other aspects of Minlog are more common.  It is a proof system in
Gentzen-style natural deduction and therefore based on proof terms
(the so-called Curry-Howard correspondence).  We distinguish between
(co)inductive predicates with and without computational content; in
fact, computational content only arises from (co)inductive predicates
marked as computationally relevant (c.r.).  In particular, both
universal und existential quantifiers do not influence computational
content: one needs to relativize them to c.r. predicates (like
totality) to make them computationally relevant.

A central application area of Minlog is to formalize Bishop-style
[Bishop67] constructive analysis and extract interesting algorithms
from proofs. An example is the Intermediate Value Theorem treated in
[LindstroemPalmgrenSegerbergStoltenberg08].  More recently we have
extracted algorithms operating on (both signed digit and Gray-coded)
stream-represented real numbers from proofs which never mention
streams.  They come in by relativizing real number quantifiers to
appropriate coinductive predicates [Berger09].

The work planned to be done by the Minlog group is to extend this kind
of work further into constructive analysis, e.g. Euler's existence
proof of solutions of ODEs.  We would like to learn from the
experience of other proof assistant developers working on related
matters, both conceptually and by sharing libraries.  It would be very
helpful if this can be done in a unified setting where different
approaches can be compared.

Here we need to go a little futher and propose to express Minlog
proofs in Dedukti.

One Postdoc and one Phd position
