\paragraph{Research e-infrastructures for formal proofs}

Proofs systems and automated theorem provers are research
infrastructure, as they allow computer scientists, logicians, and
mathematicians to build and study formal proofs, just like particle
accelerators allow physicists to build and study particles. Each proof
system comes with its own library, and these libraries are also
research infrastructures.

Currently these infrastructures are small, distributed and
disconnected.  Our project aims at building a large, worldwide
infrastructure from the small ones by integrating them in an
encyclopedia.

The integration effort is substantial because the proofs in the
various libraries are expressed in different theories. But this
integration effort is doable and contributes to the challenge to bring
European researchers and engineers an effective and convenient access
to the best research infrastructures in order to conduct research for
the advancement of knowledge and technology.

\paragraph{A starting community}

This project has never been supported under FP7 or Horizon 2020 calls.
It mobilize xxx research groups in Europe, that is almost all groups
working on formal proof technology, as well as two clubs of academic
and industrial users.

\paragraph{Compliance policy with EU regulations}

A data management plan will be provided according to Inria’s
compliance policy with EU regulations. We also engage ourselves to
make the evolution of the logipedia e-infrastructure compliant with
the European charter for access to research infrastructures.

\paragraph{Networking activities, transnational access, joint research activities}

This project fosters networking activities and joint research activities that
currently do exist in Europe, but need to be strengthened, through a
joint project.

The transnational access is more or less obvious as
the encyclopedia is on-line and accessible through a web browser.

\paragraph{Towards standardization}
This project opens the door for a possible standardization of proof
languages, although such an effort would be premature today.

\paragraph{Inter-disciplinarity}
This project that includes mathematicians, logicians, and computer
scientists is clearly inter-disciplinary.

%%% Local Variables:
%%%   mode: latex
%%%   mode: flyspell
%%%   ispell-local-dictionary: "english"
%%% End:
 
