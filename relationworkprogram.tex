\subsection{Research e-infrastructures for formal proofs}

Proof systems and automated theorem provers are research
infrastructures, as they allow engineers, computer scientists, logicians, and
mathematicians to build and study formal proofs, just like particle
accelerators allow physicists to build and study particles. Each proof
system comes with its own library, and these libraries are also
part of research infrastructure.

Currently these infrastructures are small, distributed and
disconnected.  Our project aims at building a large, worldwide
infrastructure from the small ones by integrating them in an
encyclopedia of formal proofs.

The integration effort is substantial because the proofs in the
various libraries are expressed in different theories, and because
key definitions are formalized in a different way.
%
But this
integration effort is doable and contributes to the challenge to bring
European researchers and engineers an effective and convenient access
to the best research infrastructures in order to conduct research for
the advancement of knowledge and technology.

\subsection{A starting community}

This project has never been supported under FP7 or Horizon 2020 calls.
It mobilizes twenty research groups in Europe, that is almost all
groups working on formal proof technology, as well as two clubs of
academic and industrial users.

\subsection{Compliance policy with EU regulations}

A data management plan will be provided according to Inria’s
compliance policy with EU regulations. We also engage ourselves to
make the evolution of the {\sc Logipedia} e-infrastructure compliant
with the European charter for access to research infrastructures.

\subsection{Networking activities, transnational access, joint
  research activities}

This project fosters networking activities and joint research
activities. First, between the members of the consortium, as
expressing in {\sc Dedukti} the theories implemented in the various
systems raises research questions, that for most of the systems have
been solved, but for some of them, still need to be addressed.  Then
between the members of the consortium and other academic partners, in
particular those developing automated theorem proving systems as we
want to include also proofs coming from such systems (See the research
oriented work package 4). Then, between the consortium and the
industrial users of proof systems, because using such a system will
require to improve the user interface, develop a search engine,
concept alignment algorithms, etc.  Finally between the industrial
users themselves, as using proofs developed by others will foster joint
developments.

Such networking activities currently do exist in Europe, but {\sc
  Logipedia} is a unique opportunity to strengthened them, through a
joint project.

The encyclopedia in the cloud that is the main goal of this project
will be accessible online through a web browser, so transnational
access is provided directly. Its administration will be decentralized
in various places in Europe.

\subsection{Towards standardization}
This project will be a stepping stone for a possible standardization
of proof languages.

\subsection{Inter-disciplinarity}
This project includes mathematicians, logicians, and computer
scientists and therefore is clearly inter-disciplinary.

%%% Local Variables:
%%%   mode: latex
%%%   mode: flyspell
%%%   ispell-local-dictionary: "english"
%%% End:
 
