\begin{longtable}{|p{0.2\textwidth}|p{0.75\textwidth}|}
\hline
{\bf Access to the\newline best research\newline e-infrastructures}
&
Proof systems and automated theorem provers are research
infrastructures. Each proof system comes with its own library, and
these libraries are also part of the research infrastructure.  Currently
these infrastructures are small, distributed and disconnected.\\
&
\hspace{0.4cm} Logipedia aims at co-building a large, central European
infrastructure from smaller ones by an integration effort.  This
integration effort is substantial but doable. It contributes to the
challenge of bringing to European researchers and engineers effective
and convenient access to the best research infrastructures, in order
to foster further advances in knowledge and technology.
\\
&
\hspace{0.4cm} This idea to structure a networking activity around the
construction and the use of an infrastructure is relatively new in
computer science and mathematics. The novelty is that Logipedia is not
an infrastructure made of computers, but an infrastructure made of
data and of algorithms manipulating these data.
\\
\hline

%%%%%%%%%%%%%%%%%%%%%%%%%%%%%%%%%%%%%%%%%%%%%%%%%%%%%%%%%%%%%%%%%%%%%%%%%%%%%%
{\bf A starting\newline community} & The objectives of Logipedia have never
been supported under FP7 or Horizon 2020 calls.\\
& \hspace{0.4cm} It mobilizes 28 research groups (21 in academia and 7
in industry), which is almost all the groups working on formal proof
technology in Europe.  The success of the project requires such a
large network.  Indeed, according to Metcalfe's law, the effect of a
network is proportional to the square of the number of connected
users. We can postulate that the effect of an infrastructure, such as
Logipedia, is proportional to the product of the number of proofs it
contains and the number of its potential users, each of them being
proportional to the number of theories it supports. So, such a project
needs a critical mass to succeed. This is also why we have developed an
ecosystem surrounding the project, constituted of four clubs of users:
a club of industrial users, a club of academic users, a club of users
in education, and a club of users in publishing.
These clubs, already in construction, will grow during the project.  These
are the foundations of the Logipedia community on which the
infrastructure will expand.\\
& \hspace{0.4cm} These groups are located in 11 European countries.
Some of these countries have a large number of participants, some
others fewer, reflecting the diversity of maturity of the research on
formal methods in Europe. This project will contribute to develop the
formal proof
culture in the European countries where it is still inceptive.\\
& \hspace{0.4cm} We should also point out the originality of this
project within the
European strategy of Research infrastructure.\\
& $\blacktriangleright$ A novelty of this project is that it
investigates how infrastructures can be used in mathematics and in
computer science.\\
& $\blacktriangleright$ It investigates how immaterial infrastructure
can be used to structure a research community, just like material ones
do.\\
& $\blacktriangleright$ It is focused on mathematical {\em a priori}
knowledge, while most infrastructures are focused on {\em a
  posteriori} knowledge, issued from measures, observations,
experiments, and field surveys.\\
& $\blacktriangleright$ It also investigates how a common
infrastructure articulates the relation between research and industry.
\\
\hline

%%%%%%%%%%%%%%%%%%%%%%%%%%%%%%%%%%%%%%%%%%%%%%%%%%%%%%%%%%%%%%%%%%%%%%%%%%%%%%
{\bf Networking\newline activities,\newline trans-national\newline access,\newline joint research\newline activities}
&
The development of Logipedia requires the creation of a new community that
integrates the fragmented existing communities around each proof system.
Three work packages described in Part 3
are dedicated to networking activities
to bring together these communities and collect the data from each of
their respective systems: the formal proofs.

\hspace{0.4cm}
Logipedia will be publicly and freely accessible online through any
web browser and trough a package management tool, so trans-national
and virtual access are directly provided. Its administration will be
decentralized in various places in Europe with two mirror sites in
Saclay and in M\"unchen. But, beyond trans-national and virtual
access, access is at the centre of our project, with two dedicated
work packages.\\
&
\hspace{0.4cm}
Also, this project fosters new joint research activities:
between the members of the consortium, between the members of
the consortium and other academic partners (in particular those
developing automated theorem proving systems), between the
consortium and the industrial users of proof systems, between
the industrial users themselves, as using proofs developed by others
will foster joint developments.  Joint research activities
currently exist in Europe to some extent.  This project is however a
unique opportunity to strengthen the existing ones and create new
ones. Two more work packages are dedicated
to these research activities.  
\\
\hline

%%%%%%%%%%%%%%%%%%%%%%%%%%%%%%%%%%%%%%%%%%%%%%%%%%%%%%%%%%%%%%%%%%%%%%%%%%%%%%
{\bf Towards\newline standardization} & This project will be a stepping stone
for a possible standardization of proof languages, for instance with
the World Wide Web Consortium or the International Organization for
Standardization.
\\
&
\hspace{0.4cm}
Unlike other communities in computer science, the formal proof
community is somewhat sceptical with respect to standardization. And
indeed, it would not make sense to standardize a theory for proof
systems, just like it would make no sense to propose Euclidean
geometry as a standard all geometers should use.  Instead, such a
cooperative effort can lead to the standardization of a logical
framework, that is a language to express theories and proofs in any of
these theories. By promoting cooperation between the communities of
different proof systems and demonstrating that many different theories
can coexist in a single encyclopedia, we may thus change the attitude
towards standardization in the formal proof community and make a
standardization process more widely accepted.
\\
\hline

%%%%%%%%%%%%%%%%%%%%%%%%%%%%%%%%%%%%%%%%%%%%%%%%%%%%%%%%%%%%%%%%%%%%%%%%%%%%%%
{\bf Inter-disciplinarity}
&
The development of Logipedia requires mathematicians, logicians and computer
scientists. Therefore, it is clearly inter-disciplinary.
Yet, Logipedia is inter-disciplinary in an even more fundamental
way. 
\\

&
\hspace{0.4cm}
Proving properties of a piece of software driving a car or
piloting an aircraft require formalizing part of the physical world
in which this piece of software evolves. In the same way proving
properties of simulation software, requires to formalize some
properties of the simulated object.  Thus, just like mathematics and
software are involved in any part of modern science, formal proof will
eventually spread, over time, in all areas of science.\\
&
\hspace{0.4cm}
One major obstacle to the formalization of, for example, simulations
is that they typically depend on large bodies of knowledge in
mathematics and physics.  Different proof systems may be better
suited for different applications, thus there is a natural tendency
for a diversity of proof formats, just like there is a natural
tendency for diversity in the use of programming languages.  But, when
it comes to formalizing a particular physical simulation, one needs to
have all the necessary knowledge accessible from a single infrastructure.
Because interdisciplinarity is required to formalize physical world
simulations, a unifying language, such as Logipedia, for exchanging
formal proofs between proof systems is a must-have.\\
\hline
\end{longtable}
%%% Local Variables:
%%%   mode: latex
%%%   mode: flyspell
%%%   ispell-local-dictionary: "english"
%%% End:
