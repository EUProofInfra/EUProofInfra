\subsection{Research e-infrastructures for formal proofs}

Proof systems and automated theorem provers are research
infrastructures, as they allow engineers, computer scientists,
logicians, and mathematicians to build and study formal proofs, just
like particle accelerators allow physicists to build and study
particles. Each proof system comes with its own library, and these
libraries are also part of research infrastructure.

Currently these infrastructures are small, distributed and
disconnected.  Our project aims at building a large, worldwide
infrastructure from the small ones by integrating them in an
encyclopedia of formal proofs.

The integration effort is substantial because the proofs in the
various libraries are expressed in different theories, and because key
definitions are formalized in a different way.  But this integration
effort is doable and contributes to the challenge to bring European
researchers and engineers an effective and convenient access to the
best research infrastructures in order to conduct research for the
advancement of knowledge and technology.

As discussed above, the idea to structure a networking activity around
the construction and the use of an infrastructure is relatively new in
computer science and mathematics. Moreover this infrastructure is not
an infrastructure of computers, like grid 5000 is, but mostly an
infrastructure made of data and of algorithms manipulating these data.

As Logipedia is a e-infratructure, its budget structure is different
from a material infrastructure. A small part of the resources is used
for computers, and a large part of the budget is used to develop
networking activities, a trans-national, virtual and ergonomic access, 
and joint research activities.


\subsection{A starting community}

This project has never been supported under FP7 or Horizon 2020 calls.
It mobilizes more than twenty five research groups in eleven European
Countries, that is almost all groups working on formal proof
technology in Europe.

Some countries have a large number of participants, some others fewer,
reflecting the diversity of maturity of the research on formal methods
in Europe. This project will contribute to develop the formal proof
community in the European countries where it is still inceptive.

This community is also complemented with a club of industrial users, a club
of academic users and a club of teachers.u

We should also point out the originality of this project within the
European strategy of Research infrastructure. A novelty of this
project is that is investigates how infrastructures can be used in
mathematics and in computer science. It investigates how immaterial
infrastructure can be used to structure a research community, just
like material ones do. It puts mathematical a priori knowledge at the
center while most infrastructures put a posteriori knowledge, issued
from measures, observations, experiments, and field survey at the
center. It also investigate the idea to articulate the relations
between research and industry around a common infrastructure.

\subsection{Compliance policy with EU regulations}

A data management plan will be provided according to Inria’s
compliance policy with EU regulations. We also engage ourselves to
make the evolution of the {\sf Logipedia} e-infrastructure compliant
with the European charter for access to research infrastructures.

\subsection{Networking activities, transnational access, joint
  research activities}

Three work packages described below are dedicated to networking activities
to collect the data that constitute the infrastructure: the formal proofs
that constitute {\sf Logipedia}.


The encyclopedia ``in the cloud'' will be publicly and freely
accessible online through any web browser and trough a package
management tool, so transnational access is provided directly. Its
administration will be decentralized in various places in Europe with
two mirror sites in Saclay and in M\"unchen, But we need to go beyond
trans-national and virtual access.  Two work packages described below
are dedicated to improving the quality of the access to the
encyclopedia.

Finally, this project fosters new joint research activities. First,
between the members of the consortium, as expressing in {\sf Dedukti}
the theories implemented in the various systems raises research
questions, that, for most of the systems have been solved, but for
some of them, still need to be addressed.  Then, between the members
of the consortium and other academic partners, in particular those
developing automated theorem proving systems as we want to include
also proofs coming from such systems. Then, between the consortium and
the industrial users of proof systems, because using such a system
will require to improve the user interface, develop a search engine,
concept alignment algorithms, etc.  Finally between the industrial
users themselves, as using proofs developed by others will foster
joint developments. Two more work packages are dedicated to these
research activities.
Such joint research activities currently exist in Europe to some
extent. {\sf Logipedia} is a unique opportunity to strengthen the
existing ones and create new ones, through a joint project.

\subsection{Towards standardization}

This project will be a stepping stone for a possible standardization
of proof languages. Instead of choosing one system as a standard {\em
de facto}, which is always partial, such a cooperative effort will
allow to eventually develop a standard in a cooperative way, bringing
a better and more accepted standard.

\subsection{Inter-disciplinarity}
This project includes mathematicians, logicians, and computer
scientists and therefore is clearly inter-disciplinary.

That said, the project is inter-disciplinary in an even more fundamental
way. Proving properties of a piece of software driving a car or
piloting an aircraft require to formalize part of the physical world
in which this piece of software evolves. In the same way proving
properties of simulation software, requires to formalize some
properties of the simulated object.

Thus, just like mathematics and software are involved in any part of
modern science, formal proof will eventually spread, over time, in all
areas of science. The only question is: how long will it take?

One major obstacle to the formalization of simulations is that they
typically depend on large bodies of knowledge in mathematics and physics.
Different theorem provers may be better suited for different applications,
thus there is a natural tendency for a diversity of proof formats, just
like there is a natural tendency for diversity in the use of programming
languages.
Yet, when it comes to formalizing a particular phyical simulation, one needs
to have all the necessary knowledge accessible from a single theorem prover.

To summarize, formal proofs will eventually become pervasive, yet
different disciplines may be tempted to use different theorem provers.
Because inter-disciplinarity is required to formalize physical world simulations,
a unifying language such as Logipedia for exchanging formal proofs between
theorem provers appears to be a must-have.



%%% Local Variables:
%%%   mode: latex
%%%   mode: flyspell
%%%   ispell-local-dictionary: "english"
%%% End:
