% task leader: Claudio

%\textbf{Budget requirements:} One one PhD student or postdoc at UBo.

\ednote{4 MM = 13,124 euros}

Matita is an interactive theorem prover developed at the University of Bologna and used for teaching logic courses and to verify software and mathematical proofs, with special attention to predicative foundations. The first generation of the system (up to version 0.5.9) was born as a by-product of the MoWGLI FET-Open Project, it was compatible with the logic of Coq and it could re-use its libraries. It was an important test-bench for the integration of Mathematical Knowledge Management techniques with Interactive Theorem Proving, featuring for example a library of theorems distributed over multiple servers, innovative indexing and search techniques and automatic translation of proofs between declarative and procedural styles. The second generation of the system (up to the current version 0.99.3) was a re-implementation from scratch that departed from the logic of Coq and that experimented with the most concise ways to implement an efficient theorem prover. Several ideas later migrated into Coq. The currently available largest library is the formal certification of a complexity-preserving and cost-model-inducing compiler from C to MCS-51 machine code, developed in the FET project CerCo (Certified Complexity).

The standard and arithmetic libraries of Matita has been the first libraries to be exported to Logipedia using Krajono, a fork of Matita. The forked system is also actually the only one able to import Logipedia proofs. The choice of Matita as a test-bench for Logipedia is easily understood considering that the implementation of the 0.99.x series was aimed at obtaining a well-documented, minimal but fast implementation of a theorem prover, two order of magnitudes smaller than Coq.

The task will achieve the following results
\begin{enumerate}
\item Merge Krajono and Matita, update the code to the latest version and transfer the maintenance effort to the Matita team.
\item Dedukti supports a core language that has less features than the one of Matita. During the exportation the features of Matita, like inductive types, are translated to this core language. In order to be able to re-import from Dedukti using the higher order features, Krajono needs to be extended in order to record metadata that allow to invert the translation.
\item Optimize the code and the translation in order to be able to export all the libraries of Matita.
\end{enumerate}

%%% Local Variables:
%%% mode: latex
%%% TeX-master: "../propB"
%%% End:
