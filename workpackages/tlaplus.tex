%[Nancy]

\begin{enumerate}
\item Express \tlaplus set theory in Dedukti: \tlaplus is based on untyped
  Zermelo-Fraenkel set theory with choice. In this task, we will express these
  foundations in $\lambda\Pi$-calculus modulo theory in order to represent
  \tlaplus expressions, theorems, and proof rules in Dedukti.
\item Instrument TLAPS to export proofs: the \tlaplus proof system is not based
  on proof objects. We will therefore instrument some of the proof backends so
  that they can produce traces that can be checked by Dedukti, taking advantage
  of its rewriting capabilities in order to compress the size of proof terms.
\end{enumerate}

% \tlaplus~\cite{lamport:specifying} is a specification language based on
% Zermelo-Fraenkel set theory and the Temporal Logic of Actions, a dialect of
% linear-time temporal logic. It is intended for the precise description of
% discrete systems and in particular of distributed algorithms and systems.
% \tlaplus has seen substantial adoption by companies working on distributed and
% cloud systems, spurred by an influential article published by developers at
% Amazon Web Services~\cite{newcombe:amazon-cacm}. The two main software tools for
% analyzing and verifying \tlaplus specifications are the model checker
% TLC~\cite{yu:model-checking} and TLAPS, the \tlaplus Proof
% System~\cite{cousineau:tla-proofs}, which is of interest in the context of
% \textsc{Logipedia}.

% \tlaplus proofs are written in a declarative, hierarchical style that allows the
% user to break down the proof of high-level theorems into lower-level steps. The
% proof obligations corresponding to the leaves of this proof tree are discharged
% by automated prover back-ends, including SMT solvers, the Zenon tableau prover,
% an encoding of \tlaplus set theory as an object logic Isabelle/\tlaplus in the
% logical framework Isabelle (different from the object logic Isabelle/HOL
% considered in WP1), and a decision procedure for linear-time temporal logic.

% Making TLAPS interoperable with the other proof systems addressed in this
% project will provide users of \tlaplus with access to the rich libraries of
% mathematical theories that exist in more mature systems. As a first step, we
% will express the untyped set theory that underlies \tlaplus in the
% lambda-Pi-calculus and in Dedukti. The second step will be to instrument TLAPS
% in order to export proofs that can be checked in Dedukti. TLAPS already provides
% a mechanism for backends to produce Isabelle/\tlaplus proofs, which is currently
% implemented for Zenon. In preparation for checking \tlaplus proofs in Dedukti,
% we will therefore instrument the SMT backend of TLAPS, taking advantage of the
% work carried out in WP4. In turn, exporting Isabelle/\tlaplus proofs to Dedukti
% will benefit from work carried out in WP1 on the logical framework
% Isabelle/Pure.


%%% Local Variables:
%%% mode: latex
%%% TeX-master: "../propB"
%%% End:
