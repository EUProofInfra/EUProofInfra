\begin{workpackage}[id=access,type=RTD,
  short=Access,% for Figure 5
  title={Access},
  lead=Irt,IrtRM=23,OcaRM=6,EduRM=12,InrRM=18]
  % + 5K mission for Strub
  % + 15K machines

\begin{wpobjectives}
  The objectives of this work package are:
  \begin{compactenum}[(a)]
  \item Define and build the Logipedia hardware and software
    infrastructure in which the proofs developed in
    \WPref{instrumentation},
    \WPref{atpetc},
    and \WPref{libraries} will be
    integrated.
  \item Give access to the Logipedia infrastructure to everyone
    (researchers, engineers, teachers, publishers) by providing a
    publicly accessible web site for displaying the proofs added in
    Logipedia, search tools to query the database, and a tool to
    automatically download and install those proofs in one's own
    machine.
  \item Develop the tools necessary to check the correctness of proofs
    added in Logipedia and transform them from one theory to another.
  \end{compactenum}
\end{wpobjectives}

\begin{tasklist}

  %%%%%%%%%%%%%%%%%%%%%%%%%%%%%%%%%%%%%%%%%%%%%%%%%%%%%%%%%%%%%%%%%%%%%%%%%%%%
  \begin{task}[id=archi,
      title=Defining the functional and software architecture,
      lead=Irt,IrtRM=3,wphases=2-5]
    The logipedia platform will reuse efforts done on an existing
    application. However, with an ambition to be a reference platform
    accessible on internet, it is necessary to meet several
    requirements for such an application, especially scalability,
    availability and sustainability.

    To achieve this goal, we will specify the architecture of the system.
    This will be done by:
    \begin{compactitem}
    \item Collecting and formalizing users needs
    \item Defining the proof submission process
    \item Defining the reuse strategy in term of software components
    \item Defining the global software architecture
    \item Defining how proof files for the different proof systems
      will be organized and stored in logipedia database, based on
      structure defined in \taskref{structuring}{strlibstructure}.
    \end{compactitem}

    Several features will be specified in this task:
    \begin{compactitem}
    \item World-wide access through a user-friendly, ergonomic web
      browsing interface
    \item Download proof files from the web interface
    \item Interface with proof tools through the repository developed
      in task~\taskref{access}{opam}
    \item Interface with the search tools using inputs from WP7 and
      task~\taskref{access}{search}.
    \item Capability to integrate with publication systems (e.g. HAL,
      Arxiv) through permanent links
    \item Capability to easily integrate with third-party applications
      like Wikipedia or search engines
    \end{compactitem}

    Finally, we will define the tooling architecture necessary for
    continuous integration and continuous deployment (CI/CD) of
    Logipedia and new validated proof files in the database.
  \end{task}

  %%%%%%%%%%%%%%%%%%%%%%%%%%%%%%%%%%%%%%%%%%%%%%%%%%%%%%%%%%%%%%%%%%%%%%%%%%%%%%
  \begin{task}[id=infra,
      title=Defining the hardware architecture for the infrastructure,
      lead=Irt,IrtRM=1,wphases=6-7]
    Based on the work done in \taskref{access}{archi}, we will
    size the needed hardware architecture to fit the objectives of the
    project. The goal is to define a scalable infrastructure in order
    to be able to manage an increasing traffic on the website to
    several thousands downloads a day from all over the world.

    Security of the database is also very important. It must be
    possible to recover the data in a few minutes at any time to
    ensure every user will be able to continue working. Defining the
    redundancy of the infrastructure as well as the backup strategy is
    key to guarantee the security and high availability of the
    platform.

    Finally, the hosting strategy will be decided taking into account
    costs, efficiency and sustainability.  The platform can be hosted:
    \begin{compactitem}
    \item on premise on some server at Inria with redundancy in München,
    \item on public cloud services like OVH.com, Scaleway or Amazon
      Web Services,
    \item on a private cloud, which is a compromise between both
      previous options.
    \end{compactitem}
    % to be moved in section 3.3
    %SystemX operates its own private cloud based on open-source tools
    %(multiple hardware servers running OpenStack and Kubernetes) and
    %will bring its expertise, knowledge and skills to define the best
    %hosting strategy.
  \end{task}

  %%%%%%%%%%%%%%%%%%%%%%%%%%%%%%%%%%%%%%%%%%%%%%%%%%%%%%%%%%%%%%%%%%%%%%%%%%%%%%
  \begin{task}[id=web,
      title=Giving access to the infrastructure on the world-wide web,
      lead=Irt,IrtRM=18,,wphases=8-27]
    In this task, the implementation of the architecture defined in
    task~\taskref{access}{archi} and task~\taskref{access}{infra} will
    be done. The development will be done using agile methodology in
    order to get frequent feedback from end users and adjust the
    implementation to fit the needs.

    We will setup the hardware infrastructure and develop the web
    interface to access the platform, navigate into the available
    proofs and download them.  We will then deploy the software on the
    hardware infrastructure.

    % to be moved to section 3.3
    %SystemX will bring
    %its expertise on deploying a software factory and using it in
    %internal projects.  SystemX's software factory is based on
    %open-source components widely used such as gitlab or jenkins.
    All this work will be done using CI/CD tooling for:
    \begin{compactitem}
    \item automatic building/deployment of the Logipedia application
      on the hardware infrastructure,
    \item automatic integration of new validated proof files in the
      logipedia database.
    \end{compactitem}

    Finally, we will perform unitary tests and integration tests,
    especially with components developed in other tasks: opam
    repository from \taskref{access}{opam} and search tools from
    \taskref{access}{search}.
  \end{task}

  %%%%%%%%%%%%%%%%%%%%%%%%%%%%%%%%%%%%%%%%%%%%%%%%%%%%%%%%%%%%%%%%%%%%%%%%%%%%%%
  \begin{task}[id=transfer,
      title=Transfer for the sustainability of the system,
      lead=Irt,IrtRM=1,wphases=28-29]
    Once the platform will be up and running, and available to all
    users, it will be necessary to share the ownership with other
    maintainers. To this purpose, we will produce the documentation
    necessary to operate the system and organize technical transfer
    sessions in order to allow administrators to guarantee the
    sustainability of Logipedia.

    It includes administration, upgrade and backup procedures.

    Although these sessions are necessary for the launch, all the
    documentation will be accessible publicly on a website in order
    for anyone to become maintainer.
  \end{task}

  %%%%%%%%%%%%%%%%%%%%%%%%%%%%%%%%%%%%%%%%%%%%%%%%%%%%%%%%%%%%%%%%%%%%%%%%%%%%%%
  \begin{task}[id=opam,
      title=Giving access to the infrastructure in proof tools,
      lead=Oca,OcaRM=6,wphases=15-24]
    %Users need to have an easy access to the proofs in logipedia, to
    %integrate/use them in their ongoing work; this access should be
    %guaranteed universal, without lock-in, web standards-compliant,
    %through an open source tool. While \WPref{structuring} will give
    % a structure to
    %the proof database, and \taskref{access}{web} will give access to that
    %structured database through web browsing,
    This task will provide a proof manager for users of Logipedia. This proof
    manager will enable users to automatically download and install
    proofs as well as their dependencies in order to ease the
    integration of proofs from Logipedia in their own developments.

    Opam \cite{OPAM} is an open-source source-based package manager,
    which has been successfully used by the OCaml community since
    2012, where it manages 2585 versioned packages for a total of
    13196 combinations of package and version, guaranteeing its
    ability to connect people across large communities. Furthermore,
    opam is meant to provide management capabilities not only to
    OCaml, but to any language, which is why it is already used as a
    proof manager by the Coq community where it has been proven to be
    reliable and suited to managing formal proofs. This makes it a
    prime candidate to be the proof manager for Logipedia.

    This task will use the Opam management tool to develop a
    repository containing all the proofs of Logipedia, allowing users
    across Europe to automatically and transparently download and
    install proofs and their dependencies. This
    requires to develop a tool able to read the proof
    database of Logipedia and create an Opam repository,
    and integrate it in the infrastructure built in \taskref{access}{web}.
  \end{task}

  %%%%%%%%%%%%%%%%%%%%%%%%%%%%%%%%%%%%%%%%%%%%%%%%%%%%%%%%%%%%%%%%%%%%%%%%%%%%%%
  \begin{task}[id=search,
      title=Providing search tools,
      lead=Inr,InrRM=18,wphases=15-33]
    We will provide users with search tools enabling them to perform
    queries on Logipedia in order to find specific theorems or proofs.
    First, users will be able to search libraries theorems by their
    names and other metadata (see \taskref{structuring}{strdofimpl}),
    including complex semantic queries expressed in the SPARQL
    language for semantic annotations produced in
    \taskref{structuring}{strrefonto} and
    \taskref{structuring}{strontorepml}. Second, it will be possible to
    search theorems and proofs based on their structure and
    mathematical content (types, operators, used axioms and rules,
    etc.), using exact matching, regular expressions over formulas, and
    deeper content matching, such as the one done in the
    \hyperlink{https://kwarc.info/systems/mws/}{MathWebSearch}
    system. Users can use this both to find a specific theorem that
    could be useful in their current development and to analyze the
    proofs themselves, e.g., to find all proofs using a given set of
    axioms. 
  \end{task}

  %%%%%%%%%%%%%%%%%%%%%%%%%%%%%%%%%%%%%%%%%%%%%%%%%%%%%%%%%%%%%%%%%%%%%%%%%%%%%%
  \begin{task}[id=dedukti,
      title=Development of Dedukti checking and translation tools,
      lead=Inr,InrRM=24,wphases=12-36]
    We will consolidate the source code of Dedukti, the tool allowing
    to check the correctness of proofs added in Logipedia, so as to
    handle the proofs generated in \WPref{instrumentation} and the
    large libraries considered in \WPref{libraries}. We will also
    extend it so as to handle the new theories that will be developed
    in \WPref{theories}.
    All this work will be done using continuous integration and
    deployment tools.
  \end{task}
  
\end{tasklist}

%%%%%%%%%%%%%%%%%%%%%%%%%%%%%%%%%%%%%%%%%%%%%%%%%%%%%%%%%%%%%%%%%%%%%%%%%%%%%%
\begin{wpdelivs}

  % ocamlpro
  
  \begin{wpdeliv}[due=18,miles=???,id=opamtool,dissem=PU,nature=OTHER,lead=Oca]{Proof to Opam tool}
    Tool to translate the Logipedia database format into an Opam repository
  \end{wpdeliv}

  \begin{wpdeliv}[due=19,miles=???,id=opamrepo,dissem=PU,nature=OTHER,lead=Oca]{Proof Opam repository}
    Opam repository populated with the generated proof packages
  \end{wpdeliv}

  \begin{wpdeliv}[due=20,miles=???,id=opamconfig,dissem=PU,nature=OTHER,lead=Oca]{Opam for logipedia}
    Opam configuration to use it (only or also) for Logipedia
  \end{wpdeliv}

  \begin{wpdeliv}[due=21,miles=???,id=opaminstall,dissem=PU,nature=OTHER,lead=Oca]{Provide Opam for Logipedia}
    Installation of the repository in the infrastructure of task \taskref{access}{infra}
  \end{wpdeliv}

  % systemx
  
  \begin{wpdeliv}[due=5,miles=???,id=archi-spec,dissem=PU,nature=R,lead=Irt]{Software and functional architecture}
    Specification document describing the target functional and software architecture
  \end{wpdeliv}

  \begin{wpdeliv}[due=7,miles=???,id=infra-spec,dissem=PU,nature=R,lead=Irt]{Hardware architecture}
    Specification document describing the target hardware architecture for the infrastructure
  \end{wpdeliv}

  \begin{wpdeliv}[due=14,miles=???,id=website,dissem=PU,nature=DEM,lead=Irt]{Logipedia website}
    Prototype version of the Logipedia platform
  \end{wpdeliv}

  \begin{wpdeliv}[due=27,miles=???,id=website,dissem=PU,nature=OTHER,lead=Irt]{Logipedia website}
    The final Logipedia platform
  \end{wpdeliv}

  \begin{wpdeliv}[due=29,miles=???,id=transfer-doc,dissem=PU,nature=R,lead=Irt]{Transfer documentation}
    Documentation for the exploitation of the Logipedia platform
  \end{wpdeliv}

  % pierre senellart
  
  \begin{wpdeliv}[due=17,miles=???,id=search1,dissem=PU,nature=DEM,lead=Inr]{Search tools}
    Prototype version of the search tools, with basic functionalities
  \end{wpdeliv}

  \begin{wpdeliv}[due=30,miles=???,id=search2,dissem=PU,nature=OTHER,lead=Inr]{Search tools}
    Final version of the search tools, with all functionalities
  \end{wpdeliv}
    
\end{wpdelivs}

\end{workpackage}

%%% Local Variables:
%%% mode: latex
%%% TeX-master: "../propB"
%%% mode: flyspell
%%% ispell-local-dictionary: "english"
%%% End:
