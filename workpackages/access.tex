\begin{workpackage}[id=access,wphases=0-48,type=MGT,
  short=Access,% for Figure 5.
  title={Access to the infrastructure},
  lead=Inr,
  InrRM=28,
  OcaRM=6]

\begin{wpobjectives}
  The objective of this work package is to \ldots

This includes notably:
  \begin{compactitem}
  \item \ldots
  \end{compactitem}
  A key aspect will be to foster \ldots
\end{wpobjectives}

\begin{wpdescription}

\end{wpdescription}

\begin{tasklist}

  \begin{task}[id=basic,title=Defining the architecture of the infrastructure]
    We will define the architecture of the infrastructure and install
    it on some server at Inria. The server will be duplicated in
    Münich for security. The architecture includes how the proof files
    for the different proof systems will be organized and stored, how
    they will be generated, etc.
  \end{task}

  \begin{task}[id=web,title=Giving access to the infrastructure on the world-wide web]
    We will develop a web interface to access the infrastructure,
    navigate into the available proofs and downaload them.
  \end{task}

  \begin{task}[id=opam,title=Giving access to the infrastructure in proof tools]
    Users need to have an easy access to the proofs in logipedia, to integrate/use
    them in their ongoing work; this access should be guaranteed universal, without
    lock-in, web standards-compliant, through an open source tool. While WP7
    will give a structure to the proof database, and T2 of this WP will give access
    to that structured database through web browsing, this task aims at
    providing a proof manager for users of Logipedia. This proof manager
    will enable users to automatically download and install proofs as well as their
    dependencies in order to ease the integration of proofs from logipedia in
    developments.

    opam \cite{opam} is an open-source source-based package manager, which has
    been successfully used by the OCaml community since 2012, where it manages
    2585 versioned packages for a total of 13196 combinations of package and
    version, guaranteeing its ability to connect people across large communities.
    Furthermore, opam is meant to provide management capabilities not only to
    OCaml, but to any language, which is why it is already used as a proof
    manager by the Coq community where it has been proven to be reliable and
    suited to managing formal proofs. This makes it a prime candidate to be the
    proof manager for logipedia.

    This task would thus use the opam management tool to develop a repository
    containing all the proofs in logipedia, allowing users across Europe to
    automatically and transparently download and install proofs and their
    dependencies via opam. This would primarily entail the creation of a new
    tool able to read the proof database of logipedia and create a corresponding
    opam repository, as well as the necessary work to automate this work so that
    it can run automatically on the infrastructure built in T1.

  \end{task}

  % Search task, importing some content that was previously in WP7
  \begin{task}[id=search,title=Providing search
    tools,lead=Inr,InrRM=28,FauRM=24,SacRM=6,BolRM=4]
    % task leader: Pierre Senellart, Inria
    We will provide users with search tools enabling them to perform
    queries on Logipedia in order to find specific theorems or proofs.
    First, users will be able to search libraries theorems by their
    names and other metadata (see task~\taskref{structuring}{strdofimpl}), including complex semantic
    queries expressed in the SPARQL language for semantic annotations
    produced in task~\taskref{structuring}{strrefonto}. Second, it will be possible to
    search theorems and proofs based on their structure and mathematical
    content (types, operators, used axioms and rules, etc.), using exact
    matching, regular expressions over fomulas, and deeper content
    matching, such as the one done in the
    \hyperlink{https://kwarc.info/systems/mws/}{MathWebSearch} system. Users can
    use this both to find a specific theorem that
    could be useful in their current development and to analyze the
    proofs themselves, e.g., to find all proofs using a given set of
    axioms.    
    Finally, users will be able to
    search in the full text of theorems and proofs that have been
    extracted from natural-language research articles in
    task~\taskref{structuring}{strtext}.
  \end{task}

\end{tasklist}

\begin{wpdelivs}
  \begin{wpdeliv}[due=3,miles=startup,id=requirements,dissem=PU,nature=DEM,lead=Inr]
      {Requirements Analysis and Synchronization}
  \end{wpdeliv}
  \begin{wpdeliv}[due=2,miles=???,id=acessopamtool,dissem=PU,nature=DEM,lead=Oca]
      {'Proof to opam' tool : Tool to translate the logipedia database format into an opam repository}
  \end{wpdeliv}
  \begin{wpdeliv}[due=1,miles=???,id=acessopamrepo,dissem=PU,nature=DEM,lead=Oca]
      {'Proof opam repository': opam repository populated with the generated proof packages }
  \end{wpdeliv}
  \begin{wpdeliv}[due=1,miles=???,id=accessopamconfig,dissem=PU,nature=DEM,lead=Oca]
    {'opam for logipedia': opam configuration to use it (only or also) for logipedia}
  \end{wpdeliv}
  \begin{wpdeliv}[due=1,miles=???,id=accessopam,dissem=PU,nature=DEM,lead=Oca]
    {'Provide logipedia opam' : installation of the repository in the infrastructure of WP9T1 }
  \end{wpdeliv}
\end{wpdelivs}
\end{workpackage}


%%% Local Variables:
%%% mode: latex
%%% TeX-master: "../propB"
%%% End:
