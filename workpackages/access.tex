\begin{workpackage}[id=access,wphases=0-48,type=MGT,
  short=Access,% for Figure 5.
  title={Access to the infrastructure},
  lead=Inr,
  InrRM=10]
  
\begin{wpobjectives}
  The objective of this work package is to \ldots

This includes notably:
  \begin{compactitem}
  \item \ldots
  \end{compactitem}
  A key aspect will be to foster \ldots
\end{wpobjectives}

\begin{wpdescription}

\end{wpdescription}

\begin{tasklist}

  \begin{task}[id=basic,title=Defining the architecture of the infrastructure]
    We will define the architecture of the infrastructure and install
    it on some server at Inria. The server will be duplicated in
    Münich for security. The architecture includes how the proof files
    for the different proof systems will be organized and stored, how
    they will be generated, etc.
  \end{task}

  \begin{task}[id=web,title=Giving access to the infrastructure on the world-wide web]
    We will develop a web interface to access the infrastructure,
    navigate into the available proofs and downaload them.
  \end{task}

  \begin{task}[id=opam,title=Giving access to the infrastructure in proof tools]
    We will develop a tool to allow users to easily download and
    include in their proof environment proofs available in
    Logipedia. As a given proof may depend on other proofs, this tools
    will have to manage those dependencies and download the proofs it
    depends on too.
  \end{task}

  \begin{task}[id=search,title=Providing search tools]
    We will provide users with search tools enabling them to make
    request on Logipedia in order to find a specific theorem that
    could be useful in their current development, or in order to
    analyze the proofs available in Logipedia itself (e.g. all the
    proofs using some given set of axioms, etc.). In the first case,
    requests could use regular expressions on names or formulas.
  \end{task}

\end{tasklist}

\begin{wpdelivs}
  \begin{wpdeliv}[due=3,miles=startup,id=requirements,dissem=PU,nature=DEM,lead=Inr]
      {Requirements Analysis and Synchronization}
\end{wpdeliv}
\end{wpdelivs}
\end{workpackage}


%%% Local Variables:
%%% mode: latex
%%% TeX-master: "../propB"
%%% End:
