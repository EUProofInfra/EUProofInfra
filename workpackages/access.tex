\begin{workpackage}[id=access,wphases=0-48,type=MGT,
  short=Access,% for Figure 5.
  title={Access to the infrastructure},
  lead=Inr,
  InrRM=28]
  
\begin{wpobjectives}
  The objective of this work package is to \ldots

This includes notably:
  \begin{compactitem}
  \item \ldots
  \end{compactitem}
  A key aspect will be to foster \ldots
\end{wpobjectives}

\begin{wpdescription}

\end{wpdescription}

\begin{tasklist}

  \begin{task}[id=basic,title=Defining the architecture of the infrastructure]
    We will define the architecture of the infrastructure and install
    it on some server at Inria. The server will be duplicated in
    Münich for security. The architecture includes how the proof files
    for the different proof systems will be organized and stored, how
    they will be generated, etc.
  \end{task}

  \begin{task}[id=web,title=Giving access to the infrastructure on the world-wide web]
    We will develop a web interface to access the infrastructure,
    navigate into the available proofs and downaload them.
  \end{task}

  \begin{task}[id=opam,title=Giving access to the infrastructure in proof tools]
    We will develop a tool to allow users to easily download and
    include in their proof environment proofs available in
    Logipedia. As a given proof may depend on other proofs, this tools
    will have to manage those dependencies and download the proofs it
    depends on too.
  \end{task}

  % Search task, importing some content that was previously in WP7
  \begin{task}[id=search,title=Providing search
    tools,lead=Inr,InrRM=27,FauRM=24,SacRM=6,BolRM=4]
    We will provide users with search tools enabling them to perform
    queries on Logipedia in order to find specific theorems or proofs.
    First, users will be able to search libraries theorems by their
    names and other metadata (see task~\taskref{structuring}{strdofimpl}), including complex semantic
    queries expressed in the SPARQL language for semantic annotations
    produced in task~\taskref{structuring}{strrefonto}. Second, it will be possible to
    search theorems and proofs based on their structure and mathematical
    content (types, operators, used axioms and rules, etc.), using exact
    matching, regular expressions over fomulas, and deeper content
    matching, such as the one done in the
    \hyperlink{https://kwarc.info/systems/mws/}{MathWebSearch} system. Users can
    use this both to find a specific theorem that
    could be useful in their current development and to analyze the
    proofs themselves, e.g., to find all proofs using a given set of
    axioms.    
    Finally, users will be able to
    search in the full text of theorems and proofs that have been
    extracted from natural-language research articles in
    task~\taskref{structuring}{strtext}.
  \end{task}

\end{tasklist}

\begin{wpdelivs}
  \begin{wpdeliv}[due=3,miles=startup,id=requirements,dissem=PU,nature=DEM,lead=Inr]
      {Requirements Analysis and Synchronization}
\end{wpdeliv}
\end{wpdelivs}
\end{workpackage}


%%% Local Variables:
%%% mode: latex
%%% TeX-master: "../propB"
%%% End:
