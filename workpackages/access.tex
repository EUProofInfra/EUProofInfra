\begin{workpackage}[id=access,type=RTD,wphases=1-48,
  short=Access,% for Figure 5
  title={Access},
  lead=Inr,InrRM=48,OcaRM=6,EduRM=12]
  % + 5K mission for Strub
  % + 15K machines

\begin{wpobjectives}
  \begin{compactitem}
  \item Define and build the Logipedia hardware and software
    infrastructure in which the proofs developed in
    \WPref{instrumentation}, \WPref{atpetc}, \WPref{libraries} and
    \WPref{theories} will be integrated.
  \item Give access to the Logipedia infrastructure to everyone
    (researchers, engineers, teachers, publishers) by providing a
    publicly accessible web site for displaying the proofs added in
    Logipedia, search tools to query the database, and a tool to
    automatically download and install those proofs in one's own
    computer.
  \item Develop the tools necessary to check the correctness of proofs
    added in Logipedia and transform them from one theory to another
    in coordination with \WPref{alignment}.
  \end{compactitem}
\end{wpobjectives}

\begin{tasklist}

  \newcommand\hide[1]{}
  \hide{
  %%%%%%%%%%%%%%%%%%%%%%%%%%%%%%%%%%%%%%%%%%%%%%%%%%%%%%%%%%%%%%%%%%%%%%%%%%%%
  \begin{task}[id=archi,
      title=Defining the functional and software architecture,
      shorttitle=Soft. arch.,
      lead=Inr,InrRM=3,wphases=2-5]
    The logipedia platform will reuse efforts done on an existing
    application. However, with an ambition to be a reference platform
    accessible on internet, it is necessary to meet several
    requirements for such an application, especially scalability,
    availability and sustainability.

    To achieve this goal, we will specify the architecture of the system.
    This will be done by:
    \begin{compactitem}
    \item Collecting and formalizing users needs
    \item Defining the proof submission process
    \item Defining the reuse strategy in term of software components
    \item Defining the global software architecture
    \item Defining how proof files for the different proof systems
      will be organized and stored in logipedia database, based on
      structure defined in \taskref{structuring}{strlibstructure}.
    \end{compactitem}

    Several features will be specified in this task:
    \begin{compactitem}
    \item World-wide access through a user-friendly, ergonomic web
      browsing interface
    \item Download proof files from the web interface
    \item Interface with proof systems through the repository developed
      in \taskref{access}{opam}
    \item Interface with the search tools using inputs from WP7 and
      \taskref{access}{search}.
    \item Capability to integrate with publication systems (e.g. HAL,
      Arxiv) through permanent links
    \item Capability to easily integrate with third-party applications
      like Wikipedia or search engines
    \end{compactitem}

    Finally, we will define the tooling architecture necessary for
    continuous integration and continuous deployment (CI/CD) of
    Logipedia and new validated proof files in the database.
  \end{task}

  %%%%%%%%%%%%%%%%%%%%%%%%%%%%%%%%%%%%%%%%%%%%%%%%%%%%%%%%%%%%%%%%%%%%%%%%%%%%%%
  \begin{task}[id=infra,
      title=Defining the hardware architecture for the infrastructure,
      shorttitle=HW arch.,
      lead=Inr,InrRM=1,wphases=6-7]
    Based on the work done in \taskref{access}{archi}, we will
    size the needed hardware architecture to fit the objectives of the
    project. The goal is to define a scalable infrastructure in order
    to be able to manage an increasing traffic on the website to
    several thousands downloads a day from all over the world.

    Security of the database is also very important. It must be
    possible to recover the data in a few minutes at any time to
    ensure every user will be able to continue working. Defining the
    redundancy of the infrastructure as well as the backup strategy is
    key to guarantee the security and high availability of the
    platform.

    Finally, the hosting strategy will be decided taking into account
    costs, efficiency and sustainability.  The platform can be hosted:
    \begin{compactitem}
    \item on premise on some server at Inria with redundancy in München,
    \item on public cloud services like OVH.com, Scaleway or Amazon
      Web Services,
    \item on a private cloud, which is a compromise between both
      previous options.
    \end{compactitem}
    % to be moved in section 3.3
    %SystemX operates its own private cloud based on open-source tools
    %(multiple hardware servers running OpenStack and Kubernetes) and
    %will bring its expertise, knowledge and skills to define the best
    %hosting strategy.
  \end{task}

  %%%%%%%%%%%%%%%%%%%%%%%%%%%%%%%%%%%%%%%%%%%%%%%%%%%%%%%%%%%%%%%%%%%%%%%%%%%%%%
  \begin{task}[id=web,
      title=Giving access to the infrastructure on the world-wide web,
      shorttitle=Web access,
      lead=Inr,InrRM=18,,wphases=8-27]
    In this task, the implementation of the architecture defined in
    \taskref{access}{archi} and \taskref{access}{infra} will
    be done. The development will be done using agile methodology in
    order to get frequent feedback from end users and adjust the
    implementation to fit the needs.

    We will setup the hardware infrastructure and develop the web
    interface to access the platform, navigate into the available
    proofs and download them.  We will then deploy the software on the
    hardware infrastructure.

    % to be moved to section 3.3
    %SystemX will bring
    %its expertise on deploying a software factory and using it in
    %internal projects.  SystemX's software factory is based on
    %open-source components widely used such as gitlab or jenkins.
    All this work will be done using CI/CD tooling for:
    \begin{compactitem}
    \item automatic building/deployment of the Logipedia application
      on the hardware infrastructure,
    \item automatic integration of new validated proof files in the
      logipedia database.
    \end{compactitem}

    Finally, we will perform unitary tests and integration tests,
    especially with components developed in other tasks: Opam
    repository from \taskref{access}{opam} and search tools from
    \taskref{access}{search}.
  \end{task}

  %%%%%%%%%%%%%%%%%%%%%%%%%%%%%%%%%%%%%%%%%%%%%%%%%%%%%%%%%%%%%%%%%%%%%%%%%%%%%%
  \begin{task}[id=transfer,
      title=Transfer for the sustainability of the system,
      shorttitle=Transfer,
      lead=Inr,InrRM=1,wphases=28-29]
    Once the platform will be up and running, and available to all
    users, it will be necessary to share the ownership with other
    maintainers. To this purpose, we will produce the documentation
    necessary to operate the system and organize technical transfer
    sessions in order to allow administrators to guarantee the
    sustainability of Logipedia.

    It includes administration, upgrade and backup procedures.

    Although these sessions are necessary for the launch, all the
    documentation will be accessible publicly on a website in order
    for anyone to become maintainer.
  \end{task}
  }

  %%%%%%%%%%%%%%%%%%%%%%%%%%%%%%%%%%%%%%%%%%%%%%%%%%%%%%%%%%%%%%%%%%%%%%%%%%%%
  \begin{task}[id=archi,
      title=Setting up the hardware and software architecture,
      shorttitle=Arch.,
      lead=Inr,InrRM=6,wphases=1-6]
    The logipedia platform will reuse efforts done on an existing
    proof-of-concept application. However, with an ambition to be a
    reference platform accessible on the Internet, it is necessary to meet
    several requirements for such an application, especially
    scalability, availability and sustainability.

    To achieve this goal, we will specify the architecture of the system by:
    \begin{compactitem}
    \item collecting and formalizing users needs,
    \item defining the proof submission process,
    \item defining the global software architecture,
    \item defining how files will be organized and stored in the
      Logipedia database,
    \item defining the reuse strategy in term of software components,
    \item defining the tooling architecture necessary for continuous
      integration and continuous deployment (CI/CD) of Logipedia.
    \end{compactitem}

    Several features will be taken into account in this task:
    \begin{compactitem}
    \item interface with proof systems through the repository developed
      in \taskref{access}{opam},
    \item interface with the search tools using inputs from
      \WPref{structuring} and \taskref{access}{search},
    \item capability to integrate with publication databases through
      permanent links (\taskref{dissemination}{zib}),
    \item capability to integrate with third-party applications like
      Wikipedia and search engines.
    \end{compactitem}

    Finally, we will size the needed hardware architecture to fit the
    objectives of the project. The goal is to define a scalable
    infrastructure in order to be able to manage an increasing traffic
    on the website to several thousands downloads a day from all over
    the world.

    Security of the database is also important. It must be possible to
    recover the data in a few minutes at any time to ensure every user
    will be able to continue working. Defining the redundancy of the
    infrastructure as well as the backup strategy is key to guarantee
    the security and high availability of the platform.

    The hosting strategy will be decided taking into account costs,
    efficiency and sustainability. The platform can be hosted:
    \begin{compactitem}
    \item on premise on some servers at Inria with redundancy at other
      sites like München,
    \item on public cloud services like OVH.com, Scaleway or Amazon
      Web Services.
    \end{compactitem}
  \end{task}

  %%%%%%%%%%%%%%%%%%%%%%%%%%%%%%%%%%%%%%%%%%%%%%%%%%%%%%%%%%%%%%%%%%%%%%%%%%%%%%
  \begin{task}[id=web,
      title=Giving access to the infrastructure on the world-wide web,
      shorttitle=Web,
      lead=Inr,InrRM=12,wphases=7-24]   
    In this task, the implementation of the architecture defined in
    \taskref{access}{archi} will be done using agile methodology in
    order to get frequent feedback from end users and adjust the
    implementation to fit the needs.

    In addition, we will develop an ergonomic and user-friendly web
    interface to access the platform, navigate into the available
    proofs and download them. To this end, we will use appropriate
    languages and tools (Unicode symbols, CCS, etc.) to provide a nice
    rendering of logical formulas and usual mathematical symbols.
    
    We will then deploy the software on the chosen hardware
    infrastructure.

    All this work will be done using CI/CD tooling for:
    \begin{compactitem}
    \item automatic building/deployment of the Logipedia application
      on the hardware infrastructure,
    \item automatic integration of new validated proof files in the
      logipedia database.
    \end{compactitem}

    Finally, we will perform unitary tests and integration tests,
    especially with components developed in other tasks: Opam
    repository from \taskref{access}{opam} and search tools from
    \taskref{access}{search}.
  \end{task}

  %%%%%%%%%%%%%%%%%%%%%%%%%%%%%%%%%%%%%%%%%%%%%%%%%%%%%%%%%%%%%%%%%%%%%%%%%%%%%%
  \begin{task}[id=opam,
      title=Giving access to the infrastructure in proof systems,
      shorttitle=Syst.,
      lead=Oca,OcaRM=6,wphases=15-24]
    %Users need to have an easy access to the proofs in logipedia, to
    %integrate/use them in their ongoing work; this access should be
    %guaranteed universal, without lock-in, web standards-compliant,
    %through an open source tool. While \WPref{structuring} will give
    % a structure to
    %the proof database, and \taskref{access}{web} will give access to that
    %structured database through web browsing,
    This task will provide a proof manager for users of Logipedia. This proof
    manager will enable users to automatically download and install
    proofs as well as their dependencies in order to ease the
    integration of proofs from Logipedia in their own developments.

    \href{https://opam.ocaml.org/}{Opam} is an open-source source-based package manager,
    which has been successfully used by the OCaml community since
    2012, where it manages 2585 versioned packages for a total of
    13196 combinations of package and version, guaranteeing its
    ability to connect people across large communities. Furthermore,
    Opam is meant to provide management capabilities not only to
    OCaml, but to any language, which is why it is already used as a
    proof manager by the Coq community where it has been proven to be
    reliable and suited to managing formal proofs. This makes it a
    prime candidate to be the proof manager for Logipedia.

    This task will use the Opam management tool to develop a
    repository containing all the proofs of Logipedia, allowing users
    across Europe to automatically and transparently download and
    install proofs and their dependencies. This
    requires to develop a tool able to read the proof
    database of Logipedia and create an Opam repository,
    and integrate it in the infrastructure built in \taskref{access}{web}.
  \end{task}

  %%%%%%%%%%%%%%%%%%%%%%%%%%%%%%%%%%%%%%%%%%%%%%%%%%%%%%%%%%%%%%%%%%%%%%%%%%%%%%
  \begin{task}[id=search,
      title=Providing search tools,
      shorttitle=Search,
      lead=Inr,InrRM=18,wphases=15-33]
    We will provide users with search tools enabling them to perform
    queries on Logipedia in order to find specific theorems or proofs.
    First, users will be able to search libraries theorems by their
    names and other metadata (see \taskref{structuring}{strdofimpl}),
    including complex semantic queries expressed in the SPARQL
    language for semantic annotations produced in
    \taskref{structuring}{strrefonto} and
    \taskref{structuring}{strontorepml}. Second, it will be possible to
    search theorems and proofs based on their structure and
    mathematical content (types, operators, used axioms and rules,
    etc.), using exact matching, regular expressions over formulas, and
    deeper content matching, such as the one done in the
    \hyperlink{https://kwarc.info/systems/mws/}{MathWebSearch}
    system. Users can use this both to find a specific theorem that
    could be useful in their current development and to analyze the
    proofs themselves, e.g., to find all proofs using a given set of
    axioms. 
  \end{task}

  %%%%%%%%%%%%%%%%%%%%%%%%%%%%%%%%%%%%%%%%%%%%%%%%%%%%%%%%%%%%%%%%%%%%%%%%%%%%%%
  \begin{task}[id=dedukti,
      title=Development of Dedukti checking and translation tools,
      shorttitle=Dev.,
      lead=Inr,InrRM=18,wphases=13-36] We will consolidate the source
    code of Dedukti, the tool that allows to check the correctness of
    proofs added in Logipedia, so as to handle the proofs generated in
    \WPref{instrumentation}, \WPref{atpetc} and \WPref{libraries}. We
    will also extend it so as to handle the new theories that will be
    developed in \WPref{theories}.

    We will also consolidate and integrate the existing tools devoted
    to the translation of proofs in Dedukti from one theory to another
    one in coordination with \WPref{alignment}, as well as the tools
    for generating proofs in target systems from Dedukti proofs.
    
    All this work will be done using continuous integration and
    deployment tools.
  \end{task}
  
\end{tasklist}

%%%%%%%%%%%%%%%%%%%%%%%%%%%%%%%%%%%%%%%%%%%%%%%%%%%%%%%%%%%%%%%%%%%%%%%%%%%%%%
%    R:  Document, report (excluding the periodic and final reports)
%  DEM:  Demonstrator, pilot, prototype, plan designs
%  DEC:  Websites, patents filing,   press & media actions, videos, etc.
% OTHER: Software, technical diagram, etc.

\begin{wpdelivs}

    % inria
  
  \begin{wpdeliv}[due=6,id=archi,dissem=PU,nature=DEM,lead=Inr,task=infra]{Specification document describing the target hardware and software architecture}\end{wpdeliv}

  \begin{wpdeliv}[due=12,id=platform,dissem=PU,nature=DEC,lead=Inr,task=infra]{Setting up the Logipedia platform}\end{wpdeliv}

  \begin{wpdeliv}[due=24,miles=platform,id=web,dissem=PU,nature=DEC,lead=Inr,task=web]{Logipedia website}\end{wpdeliv}

  \begin{wpdeliv}[due=24,id=dedukti,dissem=PU,nature=OTHER,lead=Inr,task=dedukti]{Release of an improved version of Dedukti and its associated translation tools}\end{wpdeliv}

  % ocamlprogg
  
  \begin{wpdeliv}[due=18,miles=opam, id=opam,dissem=PU,nature=OTHER,lead=Oca,task=opam]{Opam repository for Logipedia}\end{wpdeliv}

  % pierre senellart
  
  \begin{wpdeliv}[due=17,id=search1,dissem=PU,nature=DEM,lead=Inr,task=search]{Prototype version of the search tools with basic functionalities}\end{wpdeliv}

  \begin{wpdeliv}[due=30,id=search2,dissem=PU,nature=OTHER,lead=Inr,task=search]{Final version of the search tools with all functionalities}\end{wpdeliv}
    
\end{wpdelivs}

\end{workpackage}

%%% Local Variables:
%%% mode: latex
%%% TeX-master: "../propB"
%%% mode: flyspell
%%% ispell-local-dictionary: "english"
%%% End:
