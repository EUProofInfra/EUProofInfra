\begin{workpackage}[id=theories,wphases=0-48,type=RTD,
  short=Theories in Dedukti,% for Figure 5.
  title= Defining theories in Dedukti,
  lead=Inn,
  InnRM=10]

\ednote{MK: We need one coordinating site. original coordinators: Cezary Kaliszyk and Stephan Merz}

\ednote{MK: interested parties (add their sites and RM here): David Deharbe,
  Stephan Merz, Helmut Schwichtenberg, Guillaume Genestier, Guillaume Burel,
  Yamine Ait Ameur, Jean-Paul Bodeveix, Mamoun Filali, Arthur Chargueraud,
  Gaspard Férey
}

\begin{wpobjectives}
  The objective of this work package is to lay the groundwork for making
  available in \textsc{Logipedia} proofs carried out in systems that are
  currently at LIL $0$, i.e.\ for which we do not know today how to express
  their logic and theories in \textsc{Dedukti}. We intend to bring these systems
  to LIL $2$ over the span of the project.

This includes notably:
  \begin{compactitem}
  \item expressing the logics and foundational theories that underlie these
    systems in the lambda-Pi-calculus modulo theory and in \textsc{Dedukti};
  \item instrumenting these systems so that they can export proofs that can be
    checked in \textsc{Dedukti}.
  \end{compactitem}
  A key aspect will be to foster cooperation between the participating project
  sites by focusing on similar foundational aspects such as set theory vs.\ type
  theories, predicate subtyping, dependent types, proofs by induction and
  coinduction and so on.
\end{wpobjectives}

\begin{wpdescription}
  For the logical theories of the systems considered in this work package,
  namely Abella, HoTT, Minlog, Mizar, PVS, and \tlaplus, we have not yet
  investigated the possibility to express them in Dedukti and to instrument the
  systems so that they can export proofs that Dedukti can check. The individual
  tasks listed below address the challenges related to each system.

  It is interesting to note that similar problems occur in different systems,
  which are related to concepts such as set-theoretic or type-theoretic
  foundations of the system. Different type theories exhibit common traits such
  as the presence of predicate subtyping or dependent types. Similarly,
  inductive and coinductive definitions, and correspondingly proofs by induction
  or coinduction, are ubiquitous. Taking advantage of the network established
  within this project, we will avoid ``reinventing the wheel'' by designing
  generic mechanisms that can be instantiated for the different systems, also
  taking advantage of the lessons learnt for the systems with higher LIL
  considered in WP1. These challenges represent fundamental research problems:
  for example, Dedukti does not yet support coinductive proofs.
\end{wpdescription}

\begin{tasklist}
\begin{task}[id=abella,title=Express the theory of Abella in Dedukti and
  instrument the system]
  \ednote{Saclay}
  \input{workpackages/abella}

\end{task}

\begin{task}[title=id=hott,title=expressing HoTT]
  \ednote{Saclay, Leeds}
\end{task}

\begin{task}[id=minlog,title=Express the theory of Minlog in Dedukti and
  instrument the system]
  %[LMU München]

\begin{enumerate}
\item Further develop and implement extensional realisability in Minlog: this is
  a necessary first step for bridging Minlog and Dedukti.
\item Express the core Minlog logic and proofs in Dedukti: this task is
  facilitated by the fact that both systems share proof terms and deduction
  modulo. The preliminary work on realisability is required for exporting
  programme extraction to Dedukti.
\item Properly encode coinduction and corecursion in Dedukti: these concepts are
  fundamental to Minlog, for example for representing real numbers as streams of
  digits, but they are not native to Dedukti.
\item Import a subset of Dedukti into Minlog, apply programme development by proof
  transformation, and export back. This will make Dedukti a usable tool for the
  development of proofs and programmes in constructive analysis and allow Minlog
  users to benefit from theories formalised using different proof assistants.
\end{enumerate}


%%%%%%%%% OLD TEXT %%%%
% The \href{http://minlog-system.de}{Minlog system} implements a theory of
% computable functionals (TCF) \cite{SchwichtenbergWainer12}.
% It is a form of higher order arithmetic where partial functionals are
% first-class citizens.

% The intended model of TCF is the Scott-Ershov model of partial
% continuous functionals \cite{Ershov77}. Computable functionals are defined
% by so-called computation rules, a form of (possibly non-terminating)
% defining equations understood as left-to-right conversion rules.  An
% important example is the corecursion operator, which is needed to
% define functions operating for instance on streams of signed digits (a
% convenient format to represent real numbers).  The logical framework
% allows to declare a proven equality as a rewrite rule.  Now it is
% tempting to identify two terms or formulas when they have the same
% normal form w.r.t. rewriting (including of course beta-conversion);
% this is often called deduction modulo rewriting.  However, in a setup
% like TCF where non-termination is allowed we cannot use normal forms,
% but we can consider two terms or formulas as identical when they have
% a common reduct.  This drastically simplifies proofs involving real
% number arithmetic.

% Another central feature of TCF (and hence the Minlog system) is that
% it internalizes a proof-theoretic realizability interpretation (in the
% form of Kreisel's so-called modified realizability, with realizers of
% higher type).  More precisely, for every (co)inductive predicate we
% have another one with one argument more, denoting a realizer.  It is
% important that this realizers is expressed in the term language of TCF
% (an extension of G\"odel's system $T$).  Since a realizer can be seen as
% a programme representing the computational content of a constructive
% existence proof (expressing that a certain specification has a
% solution), we now can reason about such programs in a formal way,
% inside TCF.  In fact, given a proof M in TCF of a specification
% $\forall x\exists y A(x,y)$ we can extract a term (program) $p_M$ and automatically
% generate a new proof of $\forall x A(p_M(x))$.  In other words, for
% programs generated in this way from existence proofs, formal
% verification is automatic.  Note that for a proof involving
% coinduction the extracted term contains the (non terminating)
% corecursion operator.  Of course, for efficient evaluation in a second
% step we want to translate our extracted term into an efficient
% (functional) programming language like Haskell.

% Other aspects of Minlog are more common.  It is a proof system in
% Gentzen-style natural deduction based on proof terms
% (the so-called Curry-Howard correspondence).  We distinguish between
% (co)inductive predicates with and without computational content; in
% fact, computational content only arises from (co)inductive predicates
% marked as computationally relevant (c.r.).  In particular, both
% universal und existential quantifiers do not influence computational
% content: one needs to relativize them to c.r. predicates (like
% totality) to make them computationally relevant.

% A central application area of Minlog is to formalize Bishop-style
% \cite{Bishop67} constructive analysis and extract interesting algorithms
% from proofs. An example is the Intermediate Value Theorem treated in
% \cite{LindstroemPalmgrenSegerbergStoltenberg08}.  More recently we have
% extracted algorithms operating on (both signed digit and Gray-coded)
% stream-represented real numbers from proofs which never mention
% streams.  They come in by relativizing real number quantifiers to
% appropriate coinductive predicates \cite{Berger09}.
% We will extend this
% work further into constructive analysis, e.g. Euler's existence
% proof of solutions of ordinary differential equations (ODE).


\end{task}

\begin{task}[id=mizar,title=Express the theory of Mizar in Dedukti and
  instrument the system]
  \ednote{Innsbruck, Bialystok, task leader: Artur Kornilowicz}
  %[Białystok,Innsbruck]

\begin{enumerate}
\item Express the foundations of Mizar in Dedukti: Mizar is based on
  Tarski-Grothendieck set theory. The first challenge is to represent it in
  $\lambda\Pi$-calculus modulo theory in a way that takes advantage of Dedukti's
  rewriting capabilities to automate small steps.
\item Instrument Mizar to export type disambiguation data: Mizar offers a soft
  type system, and benefitting from it will enable us to optimize the
  representation of Mizar statements in Dedukti.
\item Express the Mizar checker for equality checking and unification: the
  algorithms underlying the Mizar checker must be expressed as a mix of small
  proof steps and rewrite rules so that Dedukti's proof kernel can verify them.
  This requires exporting semantic information for proof obligations that is not
  currently available outside of the Mizar checker and will enable us to check
  the basis of the Mizar library and make it available in Logipedia.
\end{enumerate}

%%%%%%% OLD TEXT %%%%
% Mizar~\cite{bancerek:mizar2015} is one of the earliest proof assistants. 
% It was initially created as a typesetting system for mathematics with proof
% checking functionality added later. Mizar relies on a soft type system which
% is used to specify set theoretic foundations. The library developed together
% with the system--the Mizar Mathematical Library (MML)~\cite{bancerek:mml2017}--is
% until today one of the largest libraries of formal mathematics with a number
% of results not present in other systems.

% There are multiple reasons why checking Mizar proofs in Dedukti will be one of
% the very challenging tasks in this proposal. First, the Mizar language is close
% to the standard mathematical language and as such its syntax is very far from
% the languages of most other proof assistants. Second, the semantics of a Mizar
% proof step correspond to the research on the notion of ``obviousness'' of a single
% proof step to a human mathematician. This means that Mizar proofs do not
% reference proof procedures or tactics and a lot of background knowledge is used
% implicitly. Finally, the Mizar proof system has not been originally designed to save
% any proof objects.

% \begin{enumerate}
% \item Instrument the Mizar proof checker to export more detailed proof objects \ednote{Białystok, major effort}

% \item We will express the Mizar foundations in Dedukti. This means expressing the
%   Mizar soft type system, the initial set theory, as well as Mizar structures and
%   comprehensions. Recently, we have expressed major parts of Mizar in the logical
%   framework Isabelle~\cite{ckkp:isabellemizar2019}, focusing on preserving as much
%   of the syntax as possible. Here the work will focus on being able to check all
%   the proofs instead. 

% \item We will develop a soft type checking system in the style of Mizar on the level
%   of Dedukti. We expect that the whole algorithm can be expressed as a rewrite system.
%   However, as the algorithm has been so far expressed as an imperative program this will
%   require significant work.

% \item Develop Dedukti techniques corresponding to Mizar proof checking \ednote{Białystok, major effort}.
% \end{enumerate}


\end{task}

\begin{task}[id=pvs,title=Express the theory of PVS in Dedukti and instrument
  the system]
  \ednote{Saclay}
\end{task}

\begin{task}[id=tla,title=Express the theory of \tlaplus in Dedukti and
  instrument the system]
  \ednote{Nancy, Liège}
  %[Nancy]

\begin{enumerate}
\item Express \tlaplus set theory in Dedukti: \tlaplus is based on untyped
  Zermelo-Fraenkel set theory with choice. In this task, we will express these
  foundations in $\lambda\Pi$-calculus modulo theory in order to represent
  \tlaplus expressions, theorems, and proof rules in Dedukti.
\item Instrument TLAPS to export proofs: the \tlaplus proof system is not based
  on proof objects. We will therefore instrument some of the proof backends so
  that they can produce traces that can be checked by Dedukti, taking advantage
  of its rewriting capabilities in order to compress the size of proof terms.
\end{enumerate}

% \tlaplus~\cite{lamport:specifying} is a specification language based on
% Zermelo-Fraenkel set theory and the Temporal Logic of Actions, a dialect of
% linear-time temporal logic. It is intended for the precise description of
% discrete systems and in particular of distributed algorithms and systems.
% \tlaplus has seen substantial adoption by companies working on distributed and
% cloud systems, spurred by an influential article published by developers at
% Amazon Web Services~\cite{newcombe:amazon-cacm}. The two main software tools for
% analyzing and verifying \tlaplus specifications are the model checker
% TLC~\cite{yu:model-checking} and TLAPS, the \tlaplus Proof
% System~\cite{cousineau:tla-proofs}, which is of interest in the context of
% Logipedia.

% \tlaplus proofs are written in a declarative, hierarchical style that allows the
% user to break down the proof of high-level theorems into lower-level steps. The
% proof obligations corresponding to the leaves of this proof tree are discharged
% by automated prover back-ends, including SMT solvers, the Zenon tableau prover,
% an encoding of \tlaplus set theory as an object logic Isabelle/\tlaplus in the
% logical framework Isabelle (different from the object logic Isabelle/HOL
% considered in WP1), and a decision procedure for linear-time temporal logic.

% Making TLAPS interoperable with the other proof systems addressed in this
% project will provide users of \tlaplus with access to the rich libraries of
% mathematical theories that exist in more mature systems. As a first step, we
% will express the untyped set theory that underlies \tlaplus in the
% lambda-Pi-calculus and in Dedukti. The second step will be to instrument TLAPS
% in order to export proofs that can be checked in Dedukti. TLAPS already provides
% a mechanism for backends to produce Isabelle/\tlaplus proofs, which is currently
% implemented for Zenon. In preparation for checking \tlaplus proofs in Dedukti,
% we will therefore instrument the SMT backend of TLAPS, taking advantage of the
% work carried out in WP4. In turn, exporting Isabelle/\tlaplus proofs to Dedukti
% will benefit from work carried out in WP1 on the logical framework
% Isabelle/Pure.


%%% Local Variables:
%%% mode: latex
%%% TeX-master: "../propB"
%%% End:

\end{task}

\end{tasklist}

\begin{wpdelivs}
  \begin{wpdeliv}[due=3,miles=startup,id=requirements,dissem=PU,nature=DEM,lead=Inr]
      {Requirements Analysis and Synchronization}
\end{wpdeliv}
\end{wpdelivs}
\end{workpackage}


%%% Local Variables:
%%% mode: latex
%%% TeX-master: "../propB"
%%% End:
