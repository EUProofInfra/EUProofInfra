\begin{workpackage}[id=theories,wphases=0-48,type=RTD,
  short=Theories in Dedukti,% for Figure 5.
  title= Defining theories in Dedukti,
  lead=Inn,
  InnRM=10]

\ednote{MK: We need one coordinating site. original coordinators: Cezary Kaliszyk and Stephan Merz}

\ednote{MK: interested parties (add their sites and RM here): David Deharbe,
  Stephan Merz, Helmut Schwichtenberg, Guillaume Genestier, Guillaume Burel,
  Yamine Ait Ameur, Jean-Paul Bodeveix, Mamoun Filali, Arthur Chargueraud,
  Gaspard Férey
}

\begin{wpobjectives}
  The objective of this work package is to lay the groundwork for making
  available in \textsc{Logipedia} proofs carried out in systems that are
  currently at LRL $0$, i.e.\ for which we do not know today how to express
  their logic and theories in \textsc{Dedukti}. We intend to bring these systems
  to LRL $2$ over the span of the project.

This includes notably:
  \begin{compactitem}
  \item expressing the logics and foundational theories that underlie these
    systems in the lambda-Pi-calculus modulo theory and in \textsc{Dedukti};
  \item instrumenting these systems so that they can export proofs that can be
    checked in \textsc{Dedukti}.
  \end{compactitem}
  A key aspect will be to foster cooperation between the participating project
  sites by focusing on similar foundational aspects such as set theory vs.\ type
  theories, predicate subtyping, dependent types, proofs by induction and
  coinduction and so on.
\end{wpobjectives}

\begin{wpdescription}
  For other theories, such as Abella, PVS, Mizar and \tlaplus, we have not yet
  investigated the possibility to express them in Dedukti.
\end{wpdescription}

\begin{tasklist}
\begin{task}[id=pvs,title=Express the theory of PVS in Dedukti and instrument
  the system]
\ednote{Saclay}
\end{task}

\begin{task}[id=minlog,title=Express the theory of Minlog in Dedukti and
  instrument the system]
[LMU München]

The Minlog system <http://minlog-system.de> implements a theory of
computable functionals (TCF, cf. [SchwichtenbergWainer12], Ch.7).).
It is a form of higher order arithmetic where partial functionals are
first-class citizens.

The intended model of TCF is the Scott-Ershov model $C$ of partial
continuous functionals [Ershov77].  Computable functionals are defined
by so-called computation rules, a form of (possibly non-terminating)
defining equations understood as left-to-right conversion rules.  An
important example is the corecursion operator, which is needed to
define functions operating for instance on streams of signed digits (a
convenient format to represent real numbers).  The logical framework
allows to declare a proven equality as a rewrite rule.  Now it is
tempting to identify two terms or formulas when they have the same
normal form w.r.t. rewriting (including of course beta-conversion);
this is often called deduction modulo rewriting.  However, in a setup
like TCF where non-termination is allowed we cannot use normal forms,
but we can consider two terms or formulas as identical when they have
a common reduct.  This drastically simplifies proofs involving real
number arithmetic.

Another central feature of TCF (and hence the Minlog system) is that
it internalizes a proof-theoretic realizability interpretation (in the
form of Kreisel's so-called modified realizability, with realizers of
higher type).  More precisely, for every (co)inductive predicate we
have another one with one argument more, denoting a realizer.  It is
important that this realizers is expressed in the term language of TCF
(an extension of G\"odel's system $T$).  Since a realizer can be seen as
a program representing the computational content of a constructive
existence proof (expressing that a certain specification has a
solution), we now can reason about such programs in a formal way,
inside TCF.  In fact, given a proof M in TCF of a specification all x
ex y A(x,y) we can extract a term (program) t$_M$ and automatically
generate a new proof of all x A(t$_M$(x)).  In other words, for
programs generated in this way from existence proofs, formal
verification is automatic.  Note that for a proof involving
coinduction the extracted term contains the (non terminating)
corecursion operator.  Of course, for efficient evaluation in a second
step we want to translate our extracted term into an efficient
(functional) programming language like Haskell.

Other aspects of Minlog are more common.  It is a proof system in
Gentzen-style natural deduction and therefore based on proof terms
(the so-called Curry-Howard correspondence).  We distinguish between
(co)inductive predicates with and without computational content; in
fact, computational content only arises from (co)inductive predicates
marked as computationally relevant (c.r.).  In particular, both
universal und existential quantifiers do not influence computational
content: one needs to relativize them to c.r. predicates (like
totality) to make them computationally relevant.

A central application area of Minlog is to formalize Bishop-style
[Bishop67] constructive analysis and extract interesting algorithms
from proofs. An example is the Intermediate Value Theorem treated in
[LindstroemPalmgrenSegerbergStoltenberg08].  More recently we have
extracted algorithms operating on (both signed digit and Gray-coded)
stream-represented real numbers from proofs which never mention
streams.  They come in by relativizing real number quantifiers to
appropriate coinductive predicates [Berger09].

The work planned to be done by the Minlog group is to extend this kind
of work further into constructive analysis, e.g. Euler's existence
proof of solutions of ODEs.  We would like to learn from the
experience of other proof assistant developers working on related
matters, both conceptually and by sharing libraries.  It would be very
helpful if this can be done in a unified setting where different
approaches can be compared.

Here we need to go a little futher and propose to express Minlog
proofs in Dedukti.

One Postdoc and one Phd position

\end{task}

\begin{task}[id=mizar,title=Express the theory of Mizar in Dedukti and
  instrument the system]
\ednote{IInnsbruck, Bialystok}
\end{task}

\begin{task}[id=tla,title=Express the theory of \tlaplus in Dedukti and
  instrument the system]
\ednote{Nancy, Liège}
\end{task}

\begin{task}[id=abella,title=Express the theory of Abella in Dedukti and
  instrument the system]
\ednote{Saclay}

The usual approach to capturing either Peano and Heyting arithmetics
is to use various axioms (and an axiom scheme for induction) on top of
classical and intuitionistic first-order logic.  Indeed, this is the
approach used in the Dedukti proof checker.


A different approach to encoding arithmetic has been developed over
the past 20-30 years, starting with papers by Schroeder-Heister and
Girard in the early 1990s and extended in a series of papers by
Baelde, Gacek, McDowell, M, Momigliano, Nadathur, and Tiu.  In this
new setting, first-order logic is extended by considering both
equality and the least fixed point operator as \emph{logical
  connectives}: these logical connectives are not available directly
in Dedukti.

This new foundations for arithmetic has been implemented in two
systems: the automated Bedwyr prover and the interactive Abella
prover.  While neither Bedwyr nor Abella are as popular as many of the
theorem provers that are covered by this proposal, there are two
important reasons to consider incorporating them into the Logipedia
effort.

First, the Bedwyr prover is capable of constructing proofs for the
kind of queries that are part of emph{model checkers}.  This class of
provers has not yet been incorporated into Dedukti.  The
proof-theoretic work behind model checking in Bedwyr should provide
some of the insights needed for allowing Dedukti to proof check the
results of model checkers.

Second, Bedwyr and Abella provide for direct and elegant support of
meta-level reasoning.  Given that the foundations for Bedwyr and
Abella have been given using Gentzen's sequent calculus, it was
possible to enrich their foundations to allow for the treatment of
binding structures within terms.  As a result, it is possible to
reason directly on terms representing $\lambda$-terms and
$\pi$-calculus expressions.  In particular, the Abella prover has
probably the most natural and compact formal treatment of the
$\pi$-calculus and its meta-theory when compared to all other attempts
in any other theorem provers.  More generally, the Abella prover
should be able to treat the meta-theory of programming and
specification languages as well as various logics and their
proofs. While these tasks are not the typical tasks considered by the
majority of theorem provers within the scope of this proposal,
meta-theory results do play an important role at times: in fact, the
ultimate questions as to whether or not a proof checkers (such as that
used by Dedukti) is correct or not will involve meta-theoretic
questions.

We propose to work on the general problem of exporting proofs from
Abella to Dedukti.  (Since all proofs that are constructed
automatically via Bedwyr can also be constructed manually within
Abella, we shall limit our discussion below to Abella only.)  The
proposed work will serve not only to answer the question of how to
relate these two different foundations for arithmetic but also to
allow Abella's particular style of proofs to find applications in the
wider world of formalized proofs.

The general problem described above has the following constituent parts.

(1) Proofs involving searching finite structures. Proofs built for
model checking problems over finite structures have two different
kinds of phases.  To illustrate, consider trying to find a specific
node within a binary tree.  If such a node exists, then the proof
essentially encodes the path to the node in the tree.  If, however, no
such node exists, then the proof of that negative fact is essentially
a computation that exhaustively explores the tree.  Using the Dedukti
terminology: in the first case, the proof involves several deduction
steps, while in the second case, the proof involves a pure
computation. When dealing with model checking problems such as
simulation (in concurrency theory) and winning strategies (in game
theory), proofs will involve alternating phases involving either
deduction or computation.  Since the notion of computation in
Abella-style proofs involves backtracking search, that style
computation will be quite different from Dedukti's notion of
computation as confluent rewriting.

(2) Extending model checking problems to the general case of infinite
structures and the associated inductive reasoning methods. Although
the formal basis of Abella uses least and greatest fixed-point
combinators and explicit (co-)invariants, the Abella implementation of
(co-)induction is based on cyclic reasoning using size-annotated
relations. It is known, in principle, how to convert cyclic proofs
using annotations to proofs with explicit invariants, but an invariant
extraction procedure that works in all cases is still missing. Once
such invariants are available, incorporating them into Dedukti should
be straightforward in association with part (1).

(3) Binding structures. Abella, as well as several other computational
logic systems ($\lambda$Prolog, Isabelle/Pure, Twelf, Beluga, etc)
make use of the so-called \emph{$\lambda$-tree syntax} (a form of
\emph{higher-order abstract syntax}, HOAS) approach to represent
bindings. This approach is further enriched in Abella with the
$\nabla$-quantifier that allows inductive and co-inductive properties
to be defined based on the \emph{structure} of $\lambda$-terms. We
propose to examine encodings of $lambda$-tree syntax in Dedukti. The
best approach probably involves extending the underlying theory of
Dedukti with a quantifier similar to Abella's $\nabla$-quantifier.

(4) Reflective treatment of unification. One of the features of
Abella's style of proofs is the use of left-introduction rules for
equality that exhaustively examine complete sets of unifiers for
$\lambda$-terms. This is implemented in terms of a unification engine
that is currently a trusted black box, which complicates any proposal
for exporting proofs to different implementations of unification or
equality. In Dedukti the unification procedure can be recast as a
rewrite system, but it is unclear how to derive reflective properties
based on the unifiability of terms.
\end{task}

\begin{task}[title=id=hott,title=expressing HoTT]
\ednote{Saclay, Leeds}
\end{task}
\end{tasklist}

\begin{wpdelivs}
  \begin{wpdeliv}[due=3,miles=startup,id=requirements,dissem=PU,nature=DEM,lead=Inr]
      {Requirements Analysis and Synchronization}
\end{wpdeliv}
\end{wpdelivs}
\end{workpackage}


%%% Local Variables:
%%% mode: latex
%%% TeX-master: "../propB"
%%% End:
