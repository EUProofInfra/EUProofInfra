\begin{workpackage}[id=theories,type=RTD,
  short=Theories,% for Figure 5.
  title= Theories,
  lead=Inn,
  BiaRM=70,
  BirRM=3,
  IasRM=5,
  InnRM=9,
  InrRM=83,
  LeeRM=3,
  LmuRM=16,
  MedRM=4,
  ProRM=11,
  RunRM=7,
  wphases=1-48,
  ]

\begin{wpobjectives}
  This work package addresses proof systems that are currently at LIL 0, i.e.\
  whose logic and theories have not yet been expressed in \textsc{Dedukti}. Our
  objective is to bring them to LIL 2 or better over the duration of the
  project so that at least elementary proofs can be exported and checked in
  \textsc{Dedukti}.

  Achieving these goals requires (i) expressing the logics and proof systems
  that underlie these systems in Dedukti and
  (ii) instrumenting the original proof systems so that they can export proofs
  that can be checked in Dedukti. Much of the work required in this work package
  is of foundational nature.
\end{wpobjectives}
\begin{wpdescription}
  The tasks are presented according to the individual systems. Nevertheless,
  the project addresses cross-cutting concerns and foundational aspects such
  as set-theoretic vs.\ type-theoretic foundations, predicate and dependent
  types, (co-)recursive definitions and (co-)inductive proofs and so on. The
  network established in this project will avoid ``reinventing the wheel'', also
  taking advantage of the experiences of the more mature systems considered in
  \WPref{instrumentation} and of the work carried out in \WPref{atpetc}.
\end{wpdescription}

\begin{tasklist}
% \begin{task}[id=abella,title=Express the theory of Abella in Dedukti]
%   \ednote{K. Chaudhuri, Saclay}
%   \input{workpackages/abella}
% \end{task}

\begin{task}[id=hott,
  title=Express Cubical Type Theory in Dedukti,
  lead=Inr, % B. Barras
  InrRM=42, % 36 PhD + 6 in kind B. Barras
  BirRM=3,  % 3 in kind B. Ahrens
  LeeRM=3,  % 3 in kind N. Gambino
  wphases=1-48,
  ]
  \begin{enumerate}
  \item Express 2-Level Type Theory (2LTT) as an object theory in Dedukti, as
    a stepping stone towards encoding more expressive variants of homotopy type
    theory.
  \item Express a core Cubical Type Theory (CubTT) in 2LTT: here, the main challenge is
    that equality in cubical type theory is more expressive than what can be
    expressed through rewrite rules in Dedukti.
  \item Define structures such as cartesian cubical type theory on top of the
    core, and compare these structures.
  \item Import the UniMath library into Dedukti and translate it into Cubical
    Type Theory. The UniMath library extends a version of
    Martin-Löf type theory by the univalence axiom. It provides an interesting
    case study for our encoding of CubTT.
  \end{enumerate}
\end{task}

\begin{task}[id=matching,
  title=Express Matching Logic in Dedukti and instrument the K prover,
  lead=Ias,
  IasRM=5,
  RunRM=7,
  wphases=1-36,
  ]
  \begin{enumerate}
  \item Express Kore in Dedukti: Kore is the specification language used for
    describing Matching Logic theories in K and will serve as the interface
    between the K prover and Logipedia.
  \item Instrument the K prover to produce detailed proof traces expressed in
    Kore.
  \item Integrate an instrumented automated prover (cf.\ tasks
    \taskref{atpetc}{instrumenting} and \taskref{atpetc}{tracetodedukti}) into
    the K prover for obtaining proofs for the queries generated by the K prover.
  \item Combine the translations into Dedukti from Kore and from the automated
    prover into a single Dedukti proof.
  \end{enumerate}
\end{task}

\begin{task}[id=minlog,
  title=Express the theory of Minlog in Dedukti,
  lead=Lmu,
  LmuRM=16, % 10 post-doc + 6 in kind H. Schwichtenberg + K. Miyamoto
  wphases=1-48,
  ]
  \begin{enumerate}
  \item Further develop and implement extensional realizability in Minlog as
    a necessary first step for bridging Minlog and Dedukti.
  \item Express the core Minlog logic and proofs in Dedukti, benefitting from
    the work on realizability for exporting programme extraction to Dedukti.
  \item Properly encode coinduction and corecursion in Dedukti: these concepts are
    fundamental to Minlog, for example for representing real numbers as streams of
    digits, but they are not native to Dedukti.
  \item Import a subset of Dedukti into Minlog, apply program development by proof
    transformation, and export back. This will make Dedukti a usable tool for the
    development of proofs and programs in constructive analysis and allow Minlog
    users to benefit from theories formalized using other proof assistants.
  \end{enumerate}
\end{task}

\begin{task}[id=mizar,
  title=Express the theory of Mizar in Dedukti,
  lead=Bia,   % Artur Korni{\l}owicz
  BiaRM=70, % 48 post-doc + 22 in kind
  InnRM=9,   % 6 post-doc + 3 C. Kaliszyk
  wphases=1-48,
  ]
  \begin{enumerate}
  \item Express the foundations of Mizar (first-order Tarski-Grothendieck set
    theory) in Dedukti, together with Mizar's soft
    type system. Take advantage of Dedukti's rewriting capabilities to automate
    parts of type inference.
  \item Instrument Mizar to export type disambiguation data. Exporting the
    information present in the Mizar types will enable us to optimize the
    representation of Mizar statements in Dedukti.
  \item Express Mizar's equality checking and unification steps as a mix of
    small proof steps and rewrite rules so that Dedukti's proof kernel can
    verify them. Export necessary semantic information that is not currently
    available outside of the Mizar checker in order to check the basis of the
    Mizar library and make it available in Logipedia.
  \end{enumerate}
\end{task}

\begin{task}[id=pvs,
  title=Express the theory of PVS in Dedukti,
  lead=Inr,   % Gabriel Hondet
  InrRM=20,  % 20 in kind G. Hondet
  wphases=1-48,
  ]
  \begin{enumerate}
  \item Extend the existing encoding in Dedukti,
    restricted to a fragment of PVS with decidable type checking, and represent
    proofs of type checking conditions for predicate subtypes.
  \item Instrument PVS to export proof traces to Dedukti given that PVS proof
    tactics do not produce proof terms.
  \item Design and implement a PVS proof checker in Dedukti based on the
    reconstruction of proof traces exported from PVS.
  \end{enumerate}
\end{task}

\begin{task}[id=smart,
  title=Express \textsf{Smart} models and proofs in Dedukti,
  lead=Pro,   % Stephane Lescuyer
  ProRM=11,
  wphases=1-48,
  ]
  \begin{enumerate}
  \item Choose a viable translation strategy from \textsf{Smil} programs (the
    intermediate form of \textsf{Smart} models used by ProvenTools) into Dedukti
    by experimenting with direct translations or translations through a
    different tool such as Coq or Why3.
  \item Develop a prototype capable of translating the definitions and proof
    obligations corresponding to relevant examples from the \textsf{Smart}
    standard library.
  \item Integrate the prototype into ProvenTools: integrate proof rules internal
    to the prover and evaluate the scalability to cross-verifying real-world use
    cases.
  \end{enumerate}
\end{task}

\begin{task}[id=tla,
  title=Express the theory of \tlaplus in Dedukti,
  lead=Inr,   % Stephan Merz
  InrRM=21,   % 18 post-doc + 3 in-kind S. Merz
  MedRM=4,
  wphases=1-48,
  ]
  \begin{enumerate}
  \item Express the untyped \tlaplus set theory with choice in
    Dedukti.
  \item Instrument backends of the \tlaplus Proof System to export proofs,
    taking advantage of Dedukti's rewriting capabilities in order to compress
    the size of proofs.
  \item Formalize a distributed assignment of rehabilitation programs to
    patients and updates of the process in MED-EL's software.
  \end{enumerate}
\end{task}

\end{tasklist}

\begin{wpdelivs}
  \begin{wpdeliv}[due=18,id=wp2midterm,dissem=PU,nature=R,lead=Inn]
      {Report on defining theories in Dedukti}
  \end{wpdeliv}
  \begin{wpdeliv}[due=48,id=wp2cubical,dissem=PU,nature=R,lead=Inr]
      {Report on the integration of Cubical Type Theory in Dedukti}
  \end{wpdeliv}
  \begin{wpdeliv}[due=36,id=wp2matching,dissem=PU,nature=R,lead=Ias]
      {Report on the integration of Matching Logic in Dedukti}
  \end{wpdeliv}
  \begin{wpdeliv}[due=36,id=wp2minlog,dissem=PU,nature=R,lead=Lmu]
      {Report on the integration of Minlog in Dedukti}
  \end{wpdeliv}
  \begin{wpdeliv}[due=48,id=wp2mizar,dissem=PU,nature=R,lead=Bia]
      {Report on the integration of Mizar in Dedukti}
  \end{wpdeliv}
  \begin{wpdeliv}[due=36,id=wp2pvs,dissem=PU,nature=R,lead=Inr]
      {Report on the integration of PVS in Dedukti}
  \end{wpdeliv}
  \begin{wpdeliv}[due=48,id=wp2smart,dissem=PU,nature=R,lead=Pro]
      {Report on the integration of \textsf{Smart} in Dedukti}
  \end{wpdeliv}
  \begin{wpdeliv}[due=48,id=wp2tlaplus,dissem=PU,nature=R,lead=Inr]
      {Report on the integration of \tlaplus in Dedukti}
  \end{wpdeliv}
\end{wpdelivs}
\end{workpackage}


%%% Local Variables:
%%% mode: latex
%%% TeX-master: "../propB"
%%% End:
