\begin{workpackage}[id=theories,type=RTD,wphases=1-48,
  short=Theories,% for Figure 5.
  title= Theories,
  activity=jra,
  lead=Inn,
  BiaRM=70,
  BirRM=3,
  IasRM=5,
  InnRM=12,
  InrRM=83,
  LeeRM=3,
  LmuRM=16,
  MedRM=4,
  ProRM=11,
  RunRM=7,
  wphases=1-48,
  ]

\begin{wpobjectives}
Bringing proof systems that are currently at LIL 0, i.e.\ whose logic
and theories have not yet been expressed in Dedukti, to LIL 2 or so
that at least elementary proofs can be exported and checked in
Dedukti.

  Achieving these goals requires (i) expressing the logics
  that underlie these systems in Dedukti and
  (ii) instrumenting the original proof systems so that they can export proofs
  that can be checked in Dedukti. Much of the work required in this work package
  is of foundational nature.
\end{wpobjectives}
\begin{wpdescription}
  The tasks are presented according to the individual systems. Nevertheless,
  the project addresses cross-cutting concerns and foundational aspects such
  as set-theoretic vs.\ type-theoretic foundations, predicate and dependent
  types, (co-)recursive definitions and (co-)inductive proofs and so on. The
  network established in this project will avoid ``reinventing the wheel'', also
  taking advantage of the experiences of the more mature systems considered in
  \WPref{instrumentation} and of the work carried out in \WPref{atpetc}.
\end{wpdescription}

\begin{tasklist}
% \begin{task}[id=abella,title=Express the theory of Abella in Dedukti,shorttitle=Abella]
%   \ednote{K. Chaudhuri, Saclay}
%   \begin{enumerate}
\item Encoding proofs on finite structures: model checking queries supported by
  Abella's Bedwyr prover rely on the exhaustive exploration of finite structures
  and support quantifier alternation. Representing such proofs in Dedukti
  involves alternating phases of deduction and computation. A particular
  challenge is handling backtracking search, which is different from Dedukti's
  notion of computation as confluent rewriting.
\item Encoding cyclic proofs: Abella's implementation of (co-)induction is based
  on cyclic reasoning with size annotated relations. A representation in Dedukti
  requires extracting explicit invariants from cyclic proofs.
\item Equality and unification: Abella's equality proofs examine complete sets
  of unifiers for terms assumed to be equal, based on a trusted unification
  engine. The unification procedure can be recast as a rewrite system in
  Dedukti, but it remains to derive facts from the unifiability of terms.
\item Support $\lambda$-tree syntax: an encoding of Abella in Dedukti requires
  $\lambda$-terms to be considered modulo $\alpha\beta\eta$-equivalence, which
  is not natively supported. Moreover, for reasoning about the inductive
  structure of terms, Abella provides the $\nabla$-quantifier, which provides a
  challenge for representing the full logic in Dedukti.  
\end{enumerate}

%%%%% OLD TEXT %%%%%
% The usual approach to capturing either Peano and Heyting arithmetics
% is to use various axioms (and an axiom scheme for induction) on top of
% classical and intuitionistic first-order logic.  Indeed, this is the
% approach used in the Dedukti proof checker.


% A different approach to encoding arithmetic has been developed over
% the past 20--30 years, starting with papers by Schroeder-Heister and
% Girard in the early 1990s and extended in a series of papers by
% Baelde, Gacek, McDowell, M, Momigliano, Nadathur, and Tiu.  In this
% new setting, first-order logic is extended by considering both
% equality and the least fixed point operator as \emph{logical
%   connectives}: these logical connectives are not available directly
% in Dedukti.

% This new foundations for arithmetic has been implemented in two
% systems: the automated Bedwyr prover and the interactive Abella
% prover.  While neither Bedwyr nor Abella are as popular as many of the
% theorem provers that are covered by this proposal, there are two
% important reasons to consider incorporating them into the Logipedia
% effort.

% First, the Bedwyr prover is capable of constructing proofs for the
% kind of queries that are part of \emph{model checkers}.  This class of
% provers has not yet been incorporated into Dedukti.  The
% proof-theoretic work behind model checking in Bedwyr should provide
% some of the insights needed for allowing Dedukti to proof check the
% results of model checkers.

% Second, Bedwyr and Abella provide for direct and elegant support of
% meta-level reasoning.  Given that the foundations for Bedwyr and
% Abella have been given using Gentzen's sequent calculus, it was
% possible to enrich their foundations to allow for the treatment of
% binding structures within terms.  As a result, it is possible to
% reason directly on terms representing $\lambda$-terms and
% $\pi$-calculus expressions.  In particular, the Abella prover has
% probably the most natural and compact formal treatment of the
% $\pi$-calculus and its meta-theory when compared to all other attempts
% in any other theorem provers.  More generally, the Abella prover
% should be able to treat the meta-theory of programming and
% specification languages as well as various logics and their
% proofs. While these tasks are not the typical tasks considered by the
% majority of theorem provers within the scope of this proposal,
% meta-theory results do play an important role at times: in fact, the
% ultimate questions as to whether or not a proof checkers (such as that
% used by Dedukti) is correct or not will involve meta-theoretic
% questions.

% We propose to work on the general problem of exporting proofs from
% Abella to Dedukti.  (Since all proofs that are constructed
% automatically via Bedwyr can also be constructed manually within
% Abella, we shall limit our discussion below to Abella only.)  The
% proposed work will serve not only to answer the question of how to
% relate these two different foundations for arithmetic but also to
% allow Abella's particular style of proofs to find applications in the
% wider world of formalized proofs.

% The general problem described above has the following constituent parts.

% (1) Proofs involving searching finite structures. Proofs built for
% model checking problems over finite structures have two different
% kinds of phases.  To illustrate, consider trying to find a specific
% node within a binary tree.  If such a node exists, then the proof
% essentially encodes the path to the node in the tree.  If, however, no
% such node exists, then the proof of that negative fact is essentially
% a computation that exhaustively explores the tree.  Using the Dedukti
% terminology: in the first case, the proof involves several deduction
% steps, while in the second case, the proof involves a pure
% computation. When dealing with model checking problems such as
% simulation (in concurrency theory) and winning strategies (in game
% theory), proofs will involve alternating phases involving either
% deduction or computation.  Since the notion of computation in
% Abella-style proofs involves backtracking search, that style
% computation will be quite different from Dedukti's notion of
% computation as confluent rewriting.

% (2) Extending model checking problems to the general case of infinite
% structures and the associated inductive reasoning methods. Although
% the formal basis of Abella uses least and greatest fixed-point
% combinators and explicit (co-)invariants, the Abella implementation of
% (co-)induction is based on cyclic reasoning using size-annotated
% relations. It is known, in principle, how to convert cyclic proofs
% using annotations to proofs with explicit invariants, but an invariant
% extraction procedure that works in all cases is still missing. Once
% such invariants are available, incorporating them into Dedukti should
% be straightforward in association with part (1).

% (3) Binding structures. Abella, as well as several other computational
% logic systems ($\lambda$Prolog, Isabelle/Pure, Twelf, Beluga, etc)
% make use of the so-called \emph{$\lambda$-tree syntax} (a form of
% \emph{higher-order abstract syntax}, HOAS) approach to represent
% bindings. This approach is further enriched in Abella with the
% $\nabla$-quantifier that allows inductive and co-inductive properties
% to be defined based on the \emph{structure} of $\lambda$-terms. We
% propose to examine encodings of $\lambda$-tree syntax in Dedukti. The
% best approach probably involves extending the underlying theory of
% Dedukti with a quantifier similar to Abella's $\nabla$-quantifier.

% (4) Reflective treatment of unification. One of the features of
% Abella's style of proofs is the use of left-introduction rules for
% equality that exhaustively examine complete sets of unifiers for
% $\lambda$-terms. This is implemented in terms of a unification engine
% that is currently a trusted black box, which complicates any proposal
% for exporting proofs to different implementations of unification or
% equality. In Dedukti the unification procedure can be recast as a
% rewrite system, but it is unclear how to derive reflective properties
% based on the unifiability of terms.

% \end{task}

\begin{task}[id=hott,
  title=Express Cubical Type Theory in Dedukti,
  shorttitle=CuTT,
  lead=Inr, % B. Barras
  InrRM=42, % 36 PhD + 6 in kind B. Barras
  BirRM=3,  % 3 in kind B. Ahrens
  LeeRM=3,  % 3 in kind N. Gambino
  wphases=1-48,
  ]
  \vspace{-5mm}
  \begin{compactitem}
  \item Express 2-Level Type Theory (2LTT) as an object theory in Dedukti, as
    a stepping stone towards encoding more expressive variants of Homotopy Type
    Theory.
  \item Express a core Cubical Type Theory (CubTT) in 2LTT: here, the main challenge is
    that equality in cubical type theory is more expressive than what can be
    expressed through rewrite rules in Dedukti.
  \item Define structures such as cartesian cubical type theory on top of the
    core, and compare these structures.
  \item Import the UniMath library into Dedukti and translate it into Cubical
    Type Theory. The UniMath library extends a version of
    Martin-Löf type theory by the univalence axiom. It provides an interesting
    case study for our encoding of CubTT.
  \end{compactitem}
\end{task}

\begin{task}[id=matching,
  title=Express Matching Logic in Dedukti and instrument the K prover,
  shorttitle=K,
  lead=Ias,
  IasRM=5,
  RunRM=7,
  wphases=1-42,
  ]
  \vspace{-5mm}
  \begin{compactitem}
  \item Express Kore in Dedukti: Kore is the specification language used for
    describing Matching Logic theories in K and will serve as the interface
    between the K prover and Logipedia.
  \item Instrument the K prover to produce detailed proof traces expressed in
    Kore.
  \item Integrate an instrumented automated prover (cf.\ tasks
    \taskref{atpetc}{instrumenting} and \taskref{atpetc}{tracetodedukti}) into
    the K prover for obtaining proofs for the queries generated by the K prover.
  \item Combine the translations into Dedukti from Kore and from the automated
    prover into a single Dedukti proof.
  \end{compactitem}
\end{task}

\begin{task}[id=minlog,
  title=Express the theory of Minlog in Dedukti,
  shorttitle=Minlog,
  lead=Lmu,
  LmuRM=16, % 10 post-doc + 6 in kind H. Schwichtenberg + K. Miyamoto
  wphases=1-30,
  ]
  \vspace{-5mm}
  \begin{compactitem}
  \item Further develop and implement extensional realizability in Minlog as
    a necessary first step for bridging Minlog and Dedukti.
  \item Express the core Minlog logic and proofs in Dedukti, benefitting from
    the work on realizability for exporting programme extraction to Dedukti.
  \item Properly encode coinduction and corecursion in Dedukti: these concepts are
    fundamental to Minlog, for example for representing real numbers as streams of
    digits, but they are not native to Dedukti.
  \item Import a subset of Dedukti into Minlog, apply program development by proof
    transformation, and export back. This will make Dedukti a usable tool for the
    development of proofs and programs in constructive analysis and allow Minlog
    users to benefit from theories formalized using other proof assistants.
  \end{compactitem}
\end{task}

\begin{task}[id=mizar,
  title=Express the theory of Mizar in Dedukti,
  shorttitle=Mizar,
  lead=Bia,   % Artur Korni{\l}owicz
  BiaRM=70, % 48 post-doc + 22 in kind
  InnRM=12,   % 9 post-doc + 3 C. Kaliszyk
  wphases=1-48,
  ]
  \vspace{-5mm}
  \begin{compactitem}
  \item Express the foundations of Mizar (first-order Tarski-Grothendieck set
    theory) in Dedukti, together with Mizar's soft
    type system. Take advantage of Dedukti's rewriting capabilities to automate
    parts of type inference.
  \item Instrument Mizar to export type disambiguation data. Exporting the
    information present in the Mizar types will enable us to optimize the
    representation of Mizar statements in Dedukti.
  \item Express Mizar's equality checking and unification steps as a mix of
    small proof steps and rewrite rules so that Dedukti's proof kernel can
    verify them. Export necessary semantic information that is not currently
    available outside of the Mizar checker in order to check the basis of the
    Mizar library and make it available in Logipedia.
  \end{compactitem}
\end{task}

\begin{task}[id=pvs,
  title=Express the theory of PVS in Dedukti,
  shorttitle=PVS,
  lead=Inr,   % Gabriel Hondet
  InrRM=20,  % 20 in kind G. Hondet
  wphases=1-48,
  ]
  \vspace{-5mm}
  \begin{compactitem}
  \item Extend the existing encoding in Dedukti,
    restricted to a fragment of PVS with decidable type checking, and represent
    proofs of type checking conditions for predicate subtypes.
  \item Instrument PVS to export proof traces to Dedukti given that PVS proof
    tactics do not produce proof terms.
  \item Design and implement a PVS proof checker in Dedukti based on the
    reconstruction of proof traces exported from PVS.
  \end{compactitem}
\end{task}

\begin{task}[id=smart,
  title=Express Smart models and proofs in Dedukti,
  shorttitle=Smart,
  lead=Pro,   % Stephane Lescuyer
  ProRM=11,
  wphases=1-48,
  ]
  \vspace{-5mm}
  \begin{compactitem}
  \item Choose a viable translation strategy from Smil programs (the
    intermediate form of Smart models used by ProvenTools) into Dedukti
    by experimenting with direct translations or translations through a
    different tool such as Coq or Why3.
  \item Develop a prototype capable of translating the definitions and proof
    obligations corresponding to relevant examples from the Smart
    standard library.
  \item Integrate the prototype into ProvenTools: integrate proof rules internal
    to the prover and evaluate the scalability to cross-verifying real-world use
    cases.
  \end{compactitem}
\end{task}

\begin{task}[id=tla,
  title=Express the theory of \tlaplus in Dedukti,
  shorttitle=\tlaplus,
  lead=Inr,   % Stephan Merz
  InrRM=21,   % 18 post-doc + 3 in-kind S. Merz
  MedRM=4,
  wphases=1-48,
  ]
  \vspace{-5mm}
  \begin{compactitem}
  \item Express the untyped \tlaplus set theory with choice in
    Dedukti.
  \item Instrument backends of the \tlaplus Proof System to export proofs,
    taking advantage of Dedukti's rewriting capabilities in order to compress
    the size of proofs.
  \item Formalize a distributed assignment of rehabilitation programs to
    patients and updates of the process in MED-EL's software.
  \end{compactitem}
\end{task}

\end{tasklist}

\begin{wpdelivs}
  \begin{wpdeliv}[due=18,id=wp2midterm,dissem=PU,nature=R,lead=Inn]{Report on defining theories in Dedukti}\end{wpdeliv}

  \begin{wpdeliv}[due=48,id=wp2cubical,dissem=PU,nature=R,lead=Inr,task=hott]{Report on the integration of Cubical Type Theory in Dedukti}\end{wpdeliv}

  \begin{wpdeliv}[due=42,id=wp2matching,dissem=PU,nature=R,lead=Ias,task=matching]{Report on the integration of Matching Logic}\end{wpdeliv}

  \begin{wpdeliv}[due=48,id=wp2minlog,dissem=PU,nature=R,lead=Lmu,task=minlog]{Report on the integration of Minlog}\end{wpdeliv}

  \begin{wpdeliv}[due=48,id=wp2mizar,dissem=PU,nature=R,lead=Bia,task=mizar]{Report on the integration of Mizar}\end{wpdeliv}

  \begin{wpdeliv}[due=48,id=wp2pvs,dissem=PU,nature=R,lead=Inr,task=pvs]{Report on the integration of PVS}\end{wpdeliv}
  
  \begin{wpdeliv}[due=48,id=wp2smart,dissem=PU,nature=R,lead=Pro,task=smart]{Report on the integration of Smart}\end{wpdeliv}
  
  \begin{wpdeliv}[due=48,id=wp2tlaplus,dissem=PU,nature=R,lead=Inr,task=tla]{Report on the integration of \tlaplus}\end{wpdeliv}
\end{wpdelivs}
\end{workpackage}


%%% Local Variables:
%%% mode: latex
%%% TeX-master: "../propB"
%%% End:
