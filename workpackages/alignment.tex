\begin{workpackage}[id=alignment,wphases=0-48,type=RTD,
  short=Concept Alignment,% for Figure 5.
  title=Concept Alignment,
  lead=UPra,
  UPraRM=10]
  
\ednote{We need one coordinating site. original coordinators: Filip Marić and Dale Miller}

\ednote{Parties initially expressing interest (add their sites and RM
  here): Florian Rabe, Cezary Kaliszyk, Dale Miller, Josef Urban,
  Yamine Ait Ameur, Jean-Paul Bodeveix, Mamoun Filali, Chantal Keller,
  Julien Narboux, Nicola Magaud, Arthur Charguéraud, François Thiré}

\begin{wpobjectives}
The various proof assistants have different treatments of fundamental
concepts used in logic and arithmetic.  This WP will develop
standards, tools, and techniques that will allow these concepts to be
aligned so that proofs in one proof assistant can be meaningfully used
in other systems.

The following are the three main points on which this workpackage will
focus. 
\begin{compactitem}
\item Alignments at the levels of logic.  The alignments between
  classical and intuitionistic proofs is the main challenge here.
  There are different kinds of embeddings of classical proofs into
  intuitionistic logic.  A secondary challenge is to align the various
  treatments of induction and co-induction in proof assistants: these
  treatments include explicit presentations of invariants,
  applications of invertible inference rules, and cyclic proof
  structures.

\item Alignments of theorem proving objects such as constants,
  theorems, and types.
  \begin{compactitem}
     \item Similar concepts in different libraries can have many
       significant differences once one examines the concepts in detail.

     \item approximate matching constant, where some properties that
          holds for c1 also holds for c2.
  \end{compactitem}

\item Alignments of proofs
  \begin{compactitem}
     \item Often it is not enough that we simply trust a proof to have been
          checked.  We occasionally need to work with proofs in order
          to extract an explanation or its constructive content.  
     \item Identify tactics and inference rules that have the same
       effect on proof state. 
     \item Recognizing proofs that have the same structure. (proof porting)
  \end{compactitem}
\end{compactitem}
\end{wpobjectives}

\begin{wpdescription}
Construct tools and proofs to analyze these proofs and align concepts, that is unify
concepts such as connectives and quantifiers, the concept of natural number, etc. and
theorems that occur in several libraries.  [Paris, Saclay, Innsbruck, Prague,
Strasbourg, Belgrade]

Discovery/finding objects in different theories that refers to the
same informal concepts.
There are logic inspired techniques (check that one interface formally
entails another interface): ATPs might be able to automatically handle
such checks. (Some articulation needed with WP4.)
There can be other, statistical or linguistic clues that might help
narrow down on discovering possibly useful related concepts.

\end{wpdescription}

Task: Should we attempt to ``make classical logic proofs constructive'' when possible?

\begin{tasklist}
\begin{task}[id=aligndef,title=Tracking classical and intuitionistic proof steps in logic and arithmetic] 
\end{task}

\begin{task}[id=aligndef,title=Alignment in particular domains]
  Arithmetic (FAU Erlangen-Nürnberg), Geometry (University of
  Strasbourg), and real analysis (Inria Saclay).
\end{task}

\begin{task}[id=aligndef,title=Definition of an Alignment Language]
(FAU Erlangen-Nürnberg)
\end{task}

\begin{task}[id=translate,title=Alignment-Based Translation]
(FAU Erlangen-Nürnberg)
\end{task}

\begin{task}[id=translate,title=Libraries as intermediates for translations]
(FAU Erlangen-Nürnberg)
\end{task}

\begin{task}[id=translate,title=Alignment of proof structures]
Inria Saclay
\end{task}
\end{tasklist}

\begin{wpdelivs}
  \begin{wpdeliv}[due=3,miles=startup,id=requirements,dissem=PU,nature=DEM,lead=Inr]
      {Requirements Analysis and Synchronization}
\end{wpdeliv}
\end{wpdelivs}
\end{workpackage}

%%% Local Variables:
%%% mode: latex
%%% TeX-master: "../propB"
%%% End:
