\begin{workpackage}[id=alignment,type=RTD,
  short={Proof engineering},% for Figure 5.
  title={Proof engineering},
  lead=Lee,
  LeeRM=12,  % +2 unpaid
  StrRM=18,  % +10 unpaid
  BelRM=18,  %
  ImtRM=6,  % 
  InnRM=6,  %
  SacRM=6,  %
  FauRM=11, % 
  BolRM=13, %
  InrRM=6   %
  ]

  \begin{wpobjectives}
    The aim of this workpackage is to investigate methods for
    detecting concept alignments, to apply them to build a library of
    alignments present across the Logipedia database, and to build a
    set of alignment-based services.
  \end{wpobjectives}

  \begin{wpdescription}
    Within tasks \localtaskref{alignlogic} and
    \localtaskref{aligncasestudies} manual investigations will be
    applied to detect and to align basic mathematical objects (logics,
    numbers, sets, functions, relations etc.). As a case study,
    various formalizations of geometry will manually be aligned. Task
    \localtaskref{aligntheories} is devoted to automated detection of
    an ontology alignments by employing database and semantic-web
    technology as well as unsupervised machine translation algorithms.
    Tasks \localtaskref{alignsearch} and \localtaskref{alignproofs}
    are devoting to building alignment based services: search and
    proof-rewriting.
  \end{wpdescription}

\begin{tasklist}
  \begin{task}[id=alignlogic,title=Alignment of logical foundations,lead=Lee,LeeRM=12]
    Mechanized mathematical theories span a wide spectrum of different
    logical foundations (e.g., set theory, first-order logic,
    higher-order logic, or different variants of type theory) and it
    is critically important that these can be related to each other in
    ways that will ensure their interoperability. This task will be
    devoted to aligning
    \begin{itemize}
    \item logical connectives,
    \item classical and intuitionistic logic,
    \item eliminating second-order proof,
    \item predicative vs impredicative theories,
    \item and to identification of abstraction layers.
    \end{itemize}
  \end{task}
  
  \begin{task}[id=aligncasestudies,title=Case study: aligning geometry,lead=Str,StrRM=18,BelRM=18]
    Several large-scale formalizations of geometry are available and a
    very interesting case study is to align fundamental objects that
    are introduced quite differently (both synthetically or
    analyticaly). Since geometrical objects are sometimes introduced
    using real or complex numbers, within this case study alignment of
    different numbers ($\mathbb{N}$, $\mathbb{Z}$, $\mathbb{Q}$,
    $\mathbb{R}$, $\mathbb{C}$), as well as alignment of other
    fundamental objects (sets, relations, functions) will be
    investigated.
  \end{task}

  \begin{task}[id=aligntheories,title=Automated theory alignment,lead=Imt,ImtRM=6,InnRM=6,SacRM=6]
    Since manual concept alignment is very tedious and time-consuming
    task, automated methods for detecting and organizing alignments
    will be developed on top of an ontology framework that defines
    mappings between base concepts belonging to different
    theories. Unsupervised machine learning methods to directly find
    correspondences between statements and their constituent constants
    and types will be developed and applied.
  \end{task}

  \begin{task}[id=alignsearch,title=Alignment-Based Search,lead=Fau,FauRM=18]
    This task designs and implements the alignment-expression
    translation function and uses it to realize search and browsing
    service modulo alignment.
  \end{task}
  
  \begin{task}[id=alignproofs,title=Alignment-Based Proof-Rewriting,lead=Bol,BolRM=13,InrRM=6]
    This task will develop methods to automaticaly rewrite proofs and
    statements by a mix of ELPI (developed by a join Ubo-Inr team) and
    Dedukti rewrite rules. Statement rewriting will find direct
    application to alignment based search and browsing as well
    (developed in \localtaskref{alignsearch}).
  \end{task}
\end{tasklist}

\begin{wpdelivs}
  \begin{wpdeliv}[due=24,miles=startup,id=prooftheoretical,dissem=PU,nature=DEM,lead=Lee]
    {Proof-theoretical results relating different logical foundations}
  \end{wpdeliv}
  \begin{wpdeliv}[due=42,miles=startup,id=translatingstatements,dissem=PU,nature=DEM,lead=Lee]
    {Efficient algorithms for translating statements from one logical system to another}
  \end{wpdeliv}
  \begin{wpdeliv}[due=24,miles=startup,id=aligningnumbers,dissem=PU,nature=DEM,lead=Str]
    {Manually created basic alignments for some of the examples of the
      study: $\mathbb{N}$, $\mathbb{R}$, $\mathbb{C}$, $\mathbb{R}
      \rightarrow \mathbb{R}$, between Coq and Isabelle and between
      Coq and MinlogTranslator from a subsystem of Dedukti to
      Minlog.}
  \end{wpdeliv}
  \begin{wpdeliv}[due=36,miles=startup,id=aligninggeometries,dissem=PU,nature=DEM,lead=Bel]
    {A manually created alignments between Tarski's geometry defined
      in Coq, Tarski's geometry defined in Isabelle, Analytic Geometry
      defined in Coq, Analytic Geometry in Isabelle.}
  \end{wpdeliv}
  \begin{wpdeliv}[due=36,miles=???,id=alignsearch,dissem=PU,nature=DEM,lead=Fau]
    {Implementation of the alignment-based search service.}
  \end{wpdeliv}
	
  \begin{wpdeliv}[due=36,miles=???,id=alignproofrewr,dissem=PU,nature=DEM,lead=Bol]
    {Implementation of the alignment based proof-rewriting service.}
  \end{wpdeliv}

\end{wpdelivs}
\end{workpackage}

%%% Local Variables:
%%% mode: latex
%%% TeX-master: "../propB"
%%% End:
