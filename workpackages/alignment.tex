\begin{workpackage}[id=alignment,wphases=0-48,type=RTD,
  short=Concept Alignment,% for Figure 5.
  title=Concept Alignment,
  lead=UPra,
  UPraRM=10]
  
\ednote{MK: We need one coordinating site. original coordinators: Thibault Gauthier and Dale Miller}
\ednote{MK: interested parties (add their sites and RM here): 
Florian Rabe, Cezary Kaliszyk, Dale Miller, Josef Urban,
Filip Marić, Yamine Ait Ameur, Jean-Paul Bodeveix, Mamoun Filali,
Chanta Keller, Julien Narboux, Nicola Magaud, Arthur Charguéraud,
François Thiré}

\begin{wpobjectives}
  The objective of this work package is to \ldots

This includes notably:
  \begin{compactitem}
  \item \ldots
  \end{compactitem}
  A key aspect will be to foster \ldots
\end{wpobjectives}


\begin{wpdescription}
Construct tools and proofs to analyze these proofs and align concepts, that is unify
concepts such as connectives and quantifiers, the concept of natural number, etc. and
theorems that occur in several libraries.  \ednote{Paris, Saclay, Innsbruck, Prague,
Strasbourg, Belgrade}

\end{wpdescription}

\begin{tasklist}
\begin{task}[id=aligndef,title=Definition of an Alignment Language]
\end{task}

\begin{task}[id=translate,title=Alignment-Based Translation]
\end{task}
\end{tasklist}

\begin{wpdelivs}
  \begin{wpdeliv}[due=3,miles=startup,id=requirements,dissem=PU,nature=DEM,lead=ISa]
      {Requirements Analysis and Synchronization}
\end{wpdeliv}
\end{wpdelivs}
\end{workpackage}

%%% Local Variables:
%%% mode: latex
%%% TeX-master: "../propB"
%%% End:
