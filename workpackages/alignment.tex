\begin{workpackage}[id=alignment,type=RTD,wphases=1-48,
  short={Proof engineering},% for Figure 5.
  title={Proof engineering},
  lead=Lee,
  LeeRM=12,  % +2 unpaid
  StrRM=18,  % +10 unpaid
  BelRM=18,  %
  ImtRM=6,  % 
  InnRM=6,  %
  SacRM=6,  %
  FauRM=11, % 
  BolRM=13, %
  InrRM=6   %
  ]

  \begin{wpobjectives}
    The aim of this work package is to investigate proof engineering
    methods for detecting corresponding concept, to apply those methods
    to build a library of alignments present across the Logipedia
    database, and to build a set of alignment-based services.
  \end{wpobjectives}

  \begin{wpdescription}
    Within tasks \localtaskref{alignlogic} and
    \localtaskref{aligncasestudies} manual investigations will be
    applied to detect and to align basic mathematical objects (logics,
    numbers, sets, functions, relations etc.). As a case study,
    various formalizations of geometry will manually be aligned. Task
    \localtaskref{aligntheories} is devoted to automated detection of
    an ontology alignments by employing database and semantic-web
    technology as well as unsupervised machine translation algorithms.
    Tasks \localtaskref{alignsearch} and \localtaskref{alignproofs}
    are devoting to building alignment based services: search and
    proof-rewriting.
  \end{wpdescription}

\begin{tasklist}
  \begin{task}[id=alignlogic,title=Logical foundations,shorttitle=Logical foundations,lead=Lee,LeeRM=12,wphases=6-24!.67]
    Mechanized mathematical theories span a wide spectrum of different
    logical foundations (e.g., set theory, first-order logic,
    higher-order logic, or different variants of type theory) and it
    is critically important that these can be related to each other in
    ways that will ensure their interoperability. This task will be
    devoted to aligning logical connectives, classical and
    intuitionistic logic, eliminating second-order proof, predicative
    vs impredicative theories, and to identification of abstraction
    layers. The focus will be on the Coq and Adga systems.
  \end{task}
  
  \begin{task}[id=aligncasestudies,title=Case study: geometry,shorttitle=Case study: geometry,lead=Str,StrRM=18,BelRM=18,wphases=6-42!1]
    Several large-scale formalizations of geometry are available and a
    very interesting case study is to align fundamental objects that
    are introduced quite differently (both synthetically or
    analyticaly). Since geometrical objects are sometimes introduced
    using real or complex numbers, this case study will also
    investigate alignment of different numbers ($\mathbb{N}$,
    $\mathbb{Z}$, $\mathbb{Q}$, $\mathbb{R}$, $\mathbb{C}$), as well
    as alignment of other fundamental objects (sets, relations,
    functions).
  \end{task}

  \begin{task}[id=aligntheories,title=Automated proof engineering,shorttitle=Automated proof engineering,lead=Imt,ImtRM=6,InnRM=6,SacRM=6,wphases=6-24!1]
    Since manual concept alignment is very tedious and time-consuming
    task, automated methods for detecting and organizing alignments
    will be developed on top of an ontology framework that defines
    mappings between base concepts belonging to different
    theories. Unsupervised machine learning methods to directly find
    correspondences between statements and their constituent constants
    and types will be developed and applied.
  \end{task}

  \begin{task}[id=alignsearch,title=Alignment-Based Search,shorttitle=Alignment-Based Search,lead=Fau,FauRM=11,wphases=5-48!.33]
    This task designs and implements alignment-expression translation
    functions and uses them to realize search and browsing service
    modulo alignment.
  \end{task}
  
  \begin{task}[id=alignproofs,title=Proof-Rewriting,shorttitle=Proof-Rewriting,lead=Bol,BolRM=13,InrRM=6,wphases=36-48!1.6]
    This task will develop methods to automaticaly rewrite proofs and
    statements by a mix of ELPI (developed by a join Ubo-Inr team) and
    Dedukti rewrite rules. Statement rewriting will find direct
    application to alignment based search and browsing as well
    (developed in \localtaskref{alignsearch}).
  \end{task}
\end{tasklist}

\begin{wpdelivs}
  \begin{wpdeliv}[due=24,id=prooftheoretical,dissem=PU,nature=DEM,lead=Lee]
    {Algorithms for translating between fragments of the theories of
      Coq and Agda, possibly extended with classical logic}
  \end{wpdeliv}
  \begin{wpdeliv}[due=24,id=aligningnumbers,dissem=PU,nature=DEM,lead=Str]
    {Manually created basic alignments for some of the examples of the
      study: $\mathbb{N}$, $\mathbb{R}$, $\mathbb{C}$, $\mathbb{R}
      \rightarrow \mathbb{R}$, between Coq and Isabelle and between
      Coq via Dedukti.}
  \end{wpdeliv}
  \begin{wpdeliv}[due=36,id=aligninggeometries,dissem=PU,nature=DEM,lead=Bel]
    {A manually created alignments between Tarski's geometry defined
      in Coq, Tarski's geometry defined in Isabelle, Analytic Geometry
      defined in Coq, Analytic Geometry in Isabelle}
  \end{wpdeliv}
  \begin{wpdeliv}[due=48,id=automatedalignment,dissem=PU,nature=DEM,lead=Imt]
    {Implementation of a prototype automated alignment inference engine.}
  \end{wpdeliv}
  \begin{wpdeliv}[due=48,id=alignsearch,dissem=PU,nature=DEM,lead=Fau]
    {Implementation of the alignment-based search service.}
  \end{wpdeliv}
  \begin{wpdeliv}[due=48,id=alignproofrewr,dissem=PU,nature=DEM,lead=Bol]
    {Implementation of the alignment based proof-rewriting service}
  \end{wpdeliv}

\end{wpdelivs}
\end{workpackage}

%%% Local Variables:
%%% mode: latex
%%% TeX-master: "../propB"
%%% End:
