\begin{workpackage}[id=reversemath,wphases=0-48,type=RTD,
  short=Reverse Math,% for Figure 5.
  title=Reverse Math,
  lead=Lee,
  LeeRM=10]
  
\ednote{MK: We need one coordinating site. original coordinators:  Nicola Gambino and Julien Narboux}
\ednote{MK: interested parties (add their sites and RM here): Nicola Gambino, Michael
Rathjen, Guillaume Genestier, Julien Narboux, François Thiré}

\begin{wpobjectives}
  The objective of this work package is to \ldots

This includes notably:
  \begin{compactitem}
  \item \ldots
  \end{compactitem}
  A key aspect will be to foster \ldots
\end{wpobjectives}


\begin{wpdescription}
        \ednote{Saclay,Leeds}
\end{wpdescription}

\begin{tasklist}
\begin{task}[id=ecumenical,title=Ecumenical Dedukti]
\ednote{Grienenberger, Dowek}

We plan to define in {\sc Dedukti} both constructive and classical
connectives and quantifiers
following \cite{PrawitzPereira,DowekPereira,Pereira}, so that both
constructive and classical proofs can be expressed in {\sc Dedukti}.

We plan to develop constructivization algorithms to transform proofs
expressed in this theory, into its constructive fragment.
\end{task}

\begin{task}[id=unitt,title=A universal type theory]
\ednote{Grienenberger, Dowek}
A type theory that contains both a dependent and non dependent arrow
contains two fragments that correspond to Simple type theory and to
the Calculus of constructions. Such a theory can express proofs
developed in HOL Light, Isabelle, HOL4, Coq, Matita...

We plan to develop algorithms to transform proofs expressed in this
theory, into its Simple type theory fragment.
\end{task}
\end{tasklist}

\begin{wpdelivs}
  \begin{wpdeliv}[due=3,miles=startup,id=requirements,dissem=PU,nature=DEM,lead=INR]
      {Requirements Analysis and Synchronization}
\end{wpdeliv}
\end{wpdelivs}
\end{workpackage}

%%% Local Variables:
%%% mode: latex
%%% TeX-master: "../propB"
%%% End:
