\begin{workpackage}[id=reversemath,type=RTD,
  short=Reverse Math,% for Figure 5.
  title=Reverse Math,
  lead=Lee,
  LeeRM=10]
  
\ednote{MK: We need one coordinating site. original coordinators:  Nicola Gambino and Julien Narboux}
\ednote{MK: interested parties (add their sites and RM here): Nicola Gambino, Michael
Rathjen, Guillaume Genestier, Julien Narboux, François Thiré}

\begin{wpobjectives}
  The objective of this work package is to \ldots


This includes notably:
  \begin{compactitem}
  \item \ldots
  \end{compactitem}
  A key aspect will be to foster \ldots
\end{wpobjectives}


\begin{wpdescription}
        \ednote{Saclay,Leeds}
\end{wpdescription}

\begin{tasklist}
\begin{task}[id=ecumenical,title=Ecumenical Dedukti]
\ednote{Grienenberger, Dowek}

We plan to define in {\sc Dedukti} both constructive and classical
connectives and quantifiers
following \cite{PrawitzPereira,DowekPereira,Pereira}, so that both
constructive and classical proofs can be expressed in {\sc Dedukti}.

We plan to develop constructivization algorithms to transform proofs
expressed in this theory, into its constructive fragment.
\end{task}

\begin{task}[id=unitt,title=A universal type theory]
\ednote{Grienenberger, Dowek}
A type theory that contains both a dependent and non dependent arrow
contains two fragments that correspond to Simple type theory and to
the Calculus of constructions. Such a theory can express proofs
developed in HOL Light, Isabelle, HOL4, Coq, Matita...

We plan to develop algorithms to transform proofs expressed in this
theory, into its Simple type theory fragment.
\end{task}

\begin{task}[id=typetheoryreverse, title=Type theory for reverse mathematics]
\ednote{Gambino}

Standard reverse mathematics is developed in the context of sub-systems of second-order arithmetic, where the only primitive notions are natural numbers and subsets of natural numbers. As a consequence of this, every other mathematical object (real numbers, manifolds, algebraic varieties) needs to be encoded in this language.  This has both conceptual and practical issues. On the conceptual level, one can argue that carrying out analysis of logical strength of mathematical theorem should not involve such a heavy coding and there are questions of whether the coding actually plays a role in this. The practical issue, most relevant for this project, is that existing work on reverse mathematics does not lend itself to profitable interaction with the formalisation of mathematics in computer proof-checkers like the ones considered in this project, which are not based on  sub-systems of second-order arithmetic. 

In order to remedy these issues, we propose to investigate the formulation of type theories (both simple and dependent) on which one can develop reverse mathematics. We will test the formulation by carrying out some case studies, inspired by existing fully formalised mathematics. 

One promising starting point for this project is the formulation of so-called logic-enriched type theories, originally conceived by Peter Aczel and first presented in [AG08]. These help us because they separate out the underlying type theory from the treatment of logic, thereby allowing us greater flexibility in adding axioms to a type theory and avoid building-in proof-theoretic strength by treating propositions either via an impredicative type Prop (as in the Calculus of Constructions) or via inductive definition. 
\end{task}

\begin{task}[id=geometrictypetheory, title=Geometric type theory]
\ednote{Gambino}

This remains to be added. The idea would be to formulate a `Geometric Type Theory’ and develop counterparts of elimination of excluded middle and/or axiom of choice by adapting ideas in proof theory and category theory (Barr’s covering theorem). The upshot of this would be to implement algorithms that transform proofs using EM or AC into other proofs that do not. 
\end{task}

\begin{task}[id=removingimpred, title=Removing impredicativity]
\ednote{Gambino}

This remains to be developed. But it would be great if there was a way of taking a Coq library and have an algorithm that goes through the proofs and translates them, when possible, into proofs that do not use the impredicative type of propositions. 
\end{task}

\begin{task}[id=removingdeptypes, title=Removing dependant types]
\ednote{Narboux}

For translating formalization from logical framework with dependant types to the HOL family of proof assistants, it will be necessary to remove dependant types.


\end{task}

\begin{task}[id=detectingfirstorder, title= Detecting first order proofs]
\ednote{Narboux}

While formalizing mathematics in general require higher order logic, mathematical proofs are often locally first order proofs based on an adhoc axiom system. We need to detect and isolate  these first order proof, and to build automatically abstraction barriers.


\end{task}

\begin{task}[id=reversecasestudygeom, title = Case study: geometry]
\ednote{Gambino, Narboux} 

What is the weakest theory over which one can develop  Euclid’s geometry ?  Issues of classical vs constructive proofs.
How to eliminate second order proofs of first order statements.

\end{task}

\begin{task}[id=reversecasestudyreals, title = Case studies]
\ednote{Gambino, Narboux} 

What is the weakest theory over which one can develop theories of the reals.

\end{task}

\end{tasklist}



\begin{wpdelivs}
  \begin{wpdeliv}[due=3,miles=startup,id=requirements,dissem=PU,nature=DEM,lead=Inr]
      {Requirements Analysis and Synchronization}
\end{wpdeliv}
\end{wpdelivs}
\end{workpackage}

%%% Local Variables:
%%% mode: latex
%%% TeX-master: "../propB"
%%% End:
