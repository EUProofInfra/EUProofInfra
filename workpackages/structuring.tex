\begin{workpackage}[id=structuring,wphases=0-48,type=RTD,
  short=Structured Theories and Metadata,% for Figure 5.
  title=Structured Theories and Metadata,
  lead=Fau,
  BolRM=18,
  FauRM=48,
  SacRM=48]
  
\ednote{Which sites are interested (add their sites and RM here)? David Deharbe, Nicola Gambino, Florian Rabe, Michael Rathjen, Claudio Sacerdoti Coen,
Dale Miller}

\begin{wpobjectives}
Just like software, proofs usually exist at two levels: user-written source code at the high-level and a machine-compiled prover-specific low-level representation.
This yields two fundamentally different ways to integrate provers, both offering different advantages.

Firstly, theories pursues integration through standardization on universal *low-level* representations.
This is similar to the use of the x86 as the standard architecture for software and has similar advantages: in particular, a single architecture such as Dedukti can be used to run (i.e., check) proofs from any number of proof assistants.

Secondly, this work package extends the above by additionally developing universal standards for *high-level* representations.
It offers complementary advantages: proofs represented in this language are not necessarily checkable efficiently by Dedukti; but they are more easily portable across logics and provers, and they admit more useful knowledge management services such as search and browsing.
This is critical for provers: because logics are much harder to interpret than programming languages, each source language can be interpreted only by the associated prover.
Therefore, provers must export also standardized high-level representations in order to exploit the integration potential they offer.

This work package will be jointly coordinated by Florian Rabe at \site{Fau} and Burkhart Wolff at \site{Sac}.
\end{wpobjectives}


\begin{wpdescription}
We extend the Dedukti language with features for high-level representations.
This will include
\begin{compactitem}
\item theories: a general term we use to unify a variety of module system constructs such as type classes or locales,
\item derived declarations: high-level declarations such as inductive type definitions, whose semantics is given by elaboration into more primitive constructs,
\item metadata annotations: a general framework for attaching information about semantics, document structure, and tool interaction.
\end{compactitem}

The low- and high-level representations will be tightly integrated: any declaration or object may be given alternatively through either or both of these.
The prover exports from instrumentation will be such that they produce both representations whenever possible.

Then we leverage this design in several applications including automated prover interaction and a Logipedia-wide search service.
\end{wpdescription}

\begin{tasklist}
\begin{task}[id=tbd,title=TBD]
\end{task}
\end{tasklist}

\begin{wpdelivs}
  \begin{wpdeliv}[due=3,miles=startup,id=requirements,dissem=PU,nature=DEM,lead=Inr]
      {TBD}
\end{wpdeliv}
\end{wpdelivs}


%\begin{enumerate}
%\item concrete/surface syntaxes 
%\item Central Library Backend Systems 
%\item Cross-System Front-Ends/Portals (Logipedia, ...)
%\item Semantic Middleware-based System Interoperability
%\end{enumerate} 
%
%Since proof-objects for substantial theory developments tend to be
%very large (the representation of current POs for the Isabelle/AFP can
%easily reach several TB although using techniques for compression), A
%technical pre-requisite for interchangeability, connectivity and
%advanced search consists in a structured, typed format for meta-data
%together with a flexible mechanism of their validation. Technically,
%this kind of meta-data has the form of a function annoconst : arg1 ->
%... -> argn -> proof-term -> proof-term where annoconst is a constant
%symbol which represents an identity in the proof-term (so, any import
%function of a specific system can actually ignore it), and where the
%argi represent terms with meta-information such as, eg., “this
%proof-term represents a free data-type construction of the form ...”,
%or “this part of the proof is a derivation of a free data-type of the
%following form ...”, “this lifting over assumptions represents in
%Isabelle a Locale-instantiation”, “this part of a theory
%development is connected to ... ”, “this theorem belongs to the
%sub-class of XXX ... theorems”, etcpp. For arguments of annotations,
%validation-functions can be defined that may check that the argument
%terms satisfy a certain property wrt. to the proof-term and the
%current logical context. Dedukti will provide a framework that allows
%for each proof-system (Coq, HOL4, Isabelle...) to declare meta-data
%together with validations and thus communicate tool-specific knowledge
%to other systems. This framework can be seen as a particular form of
%an ontology definition language.
% 
%WP8: Indexing and browsing [?]  Construct tools to index and browse
%this encyclopedia, that is find the theorem one needs, either by
%looking for it with its name, with its statement, or with symbols
%occurring in it.


\end{workpackage}

%%% Local Variables:
%%% mode: latex
%%% TeX-master: "../propB"
%%% End:
