\begin{workpackage}[id=structuring,type=RTD,
  short={Structure of the encyclopedia},% for Figure 5.
  title={Structure of the encyclopedia},
  lead=Sac,
  SacRM=40,
  FauRM=20,
  BolRM=4
%  TouRM=12,
%  InrRM=14
]

%\ednote{Which sites are interested?
%David Deharbe and Etienne Prun (Clearsy); Nicola Gambino, Michael Rathjen, Claudio Sacerdoti Coen, Dale Miller, Emilio J. Gallego Arias, Michael Butler, Pierre Senellart}

% David Deharbe and Etienne Prun (Clearsy): use B Method, would like to doublecheck B proofs, integrate B with other proof assistant at high-level

\begin{wpobjectives}
We provide infrastructure for the structured ontological representation of libraries and use it to enrich the information about formal libraries in Logipedia.
This will enable the exchange and reuse the knowledge between prover systems.
\end{wpobjectives}


\begin{wpdescription}
We proceed in three steps.
Firstly, Tasks~\localtaskref{strlibstructure} and~\localtaskref{strdofimpl} extend the Dedukti language with features for high-level representations that are critical for accessing parts of and searching libraries.
This includes a framework to \emph{define} typed meta-data in form of ontologies, and to \emph{enforce} them in 
the Dedukti libraries.
Secondly, Tasks~\localtaskref{strrefonto} builds ontologies serving as technical exchange format as well as domain-specific descriptions of libraries.
Thirdly, Tasks~\localtaskref{strontorepml} fills the ontology with data from both formal libraries and natural language articles and use the ontology to relate to each other.

This work package will be jointly led by Burkhart Wolff at \site{Sac} and Florian Rabe at \site{Fau}.
(Where a single leader is needed for formal purposes, the former site will be the primary leader.)
Burkhart Wolff implemented a document ontology framework in Isabelle and developed several applications
in the field of formal software engineering \cite{brucker.ea:ontologies-certification:2019,brucker.ea:isabelle-ontologies:2018,brucker.ea:ontologies-certification:2019}.
Florian Rabe has extensive experience in designing and implementing knowledge representation languages \cite{RK:mmt:10,rabe:recon:17} as well as in exporting theorem prover libraries \cite{KR:oafexp:20,CKMRSW:ulo:19}.
\end{wpdescription}

\begin{tasklist}
\begin{task}[id=strlibstructure,title=Library Structure,lead=Fau,FauRM=8, SacRM=6, wphases=0-28!.5]
This task extends the Dedukti language with primitives for representing library, document, informal annotations, and theory structure.
This includes in particular the definition of unique identifiers for all declarations, which is critical for alignments.
%We extend the Dedukti language with features for high-level representations.
%This will include
%\begin{compactitem}
%\item theories: a general term we use to unify a variety of module system constructs such as type classes or locales,
%\item derived declarations: high-level declarations such as inductive type definitions, whose semantics is given by elaboration into more primitive constructs,
%\item metadata annotations: a general framework for attaching information about semantics, document structure, and tool interaction.
%\end{compactitem}
%
%The low- and high-level representations will be tightly integrated: any declaration or object may be given alternatively through either or both of these.
%The prover exports from instrumentation will be such that they produce both representations whenever possible.
%
%Then we leverage this design in several applications including automated prover interaction and a Logipedia-wide search service.
\end{task} 

\begin{task}[id=strdofimpl,title=Ontological Framework for Meta-Data,lead=Sac,SacRM=24,wphases=0-24!1.0]
This tasks extends the Dedukti language with a framework for meta-data annotations.
This will cover all levels of the structure introduced in \localtaskref{strlibstructure} as well as the 
level of subexpressions of Dedukti expressions. It will also provide a mechanism to validate meta-data
according to assertions.
\end{task} 

% suggested for removal in budget arbitration meeting; now mentioned in task on reference ontology
%\begin{task}[id=strdomonto,title= Domain Ontologies for Formal Methods in SE,lead=Tou,TouRM=12, SacRM=0]
%Y. Aitameur at \site{Tou}
%\begin{compactitem}
%\item domain ontologies as descriptive models for engineering domains 
%\item links/imports with/from standards and certification
%\item engineering models annotations
%\item strengthening engineering models by references to domain ontologies
%\item Case studies could be certification, safety, security.
%\end{compactitem}
%\end{task} 

\begin{task}[id=strrefonto,title=Reference Ontology,lead=Sac,FauRM=6,SacRM=6,wphases=12-36!.5]
This tasks compiles, integrates, and curates the various ontologies used for describing libraries in the project.
These come from several sources:
\begin{compactitem}
 \item The ontology induced by the structuring features built in task \localtaskref{strlibstructure}.
 \item The ontologies built by users using the ontology framework built in task \localtaskref{strrefonto}.
 \item Manually written ontologies or imports of existing ontologies for knowledge formalized in prover libraries, such as the Upper Library Ontology developed in \cite{CKMRSW:ulo:19} and domain-specific ontologies.
 The latter may include for example the ontologies for engineering and their relation to descriptive models and certification standards that are planned to be developed by \site{Tou}.
\end{compactitem}
\end{task}

\begin{task}[id=strontorepml,title=Ontological Representation of Formal Libraries,lead=Fau,FauRM=6,BolRM=4,SacRM=4,wphases=12-48!.5]
This task extends the exports from Isabelle and Coq developed in \WPref{libraries} with structural and ontological data that conforms to the language features introduced in Tasks~\localtaskref{strlibstructure} and \localtaskref{strdofimpl}.
It will also build on the ontological export of ULO data developed for Isabelle and Coq in \cite{CKMRSW:ulo:19}.

The task leader will collaborate with M. Wenzel for Isabelle and C. Sacerdoti Coen at \site{Bol} for Coq, with whom long-standing collaborations on these library exports exist \cite{MRS:coq:19,CKMRSW:ulo:19,KRW:isabelle:19}.
Wenzel's involvement will take the form of a sub-contract of \site{Fau} at 20,000 EUR (corresponding to roughly 4 person-months), an arrangement that has already been used twice in other projects.
The resources for this subcontract are declared in Section~\ref{sec:subcontracting-costs} and are not included in the person-months listed here.
\end{task}

% Moved to WP9
%\begin{task}[id=strontosearch,title=Ontological Search,lead=Fau,FauRM=12,SacRM=6]
%Search based on ontological data (RDF triples) using systems like SPARQL
%\ednote{possible participation of S. Dumbrava; is there a site for this?}
%\end{task} 

%\begin{task}[id=strformsearch,title=Formula-based Search,lead=Fau,BolRM=4,FauRM=12]
%Search based on formula structure using systems like MathWebSearch
%\end{task} 

% suggested for removal in budget arbitration meeting
%\begin{task}[id=strtext,title=Ontological Representation of Natural Language Articles,lead=Inr,FauRM=6,InrRM=14]
%    % task leader: Pierre Senellart, Inria
%This task extracts ontological information from natural language research articles and link them with the formal representations in Isabelle and Coq.
%P. Senellart at \site{Inr} and M. Kohlhase at \site{Fau} will work on the automatic extraction and annotation of natural-language theorem statements and proofs from published articles, as well as building libraries of such theorems and proofs.
%  % Search capabilities have been moved to WP9
%  %and search and querying capabilities
%\end{task} 

\end{tasklist}


\begin{wpdelivs}
  \begin{wpdeliv}[due=18,miles=???,id=deliv-str-framework,dissem=PU,nature=R,lead=Sac]
  	{This deliverable describes the language developed in Tasks taskref{structuring}{strlibstructure} and taskref{structuring}{strdofimpl}.}
  \end{wpdeliv}
  \begin{wpdeliv}[due=36,miles=???,id=deliv-str-ontology,dissem=PU,nature=R,lead=Sac]
  	{This deliverable describes the reference ontology developed in Tasks taskref{structuring}{strrefonto}.}
  \end{wpdeliv}
  \begin{wpdeliv}[due=48,miles=???,id=deliv-str-libraries,dissem=PU,nature=R,lead=Fau]
  	{This deliverable describes the representation of major formal libraries developed in Tasks taskref{structuring}{strontorepml}.}
  \end{wpdeliv}
\end{wpdelivs}



%\begin{enumerate}
%\item concrete/surface syntaxes 
%\item Central Library Backend Systems 
%\item Cross-System Front-Ends/Portals (Logipedia, ...)
%\item Semantic Middleware-based System Interoperability
%\end{enumerate} 
%
%Since proof-objects for substantial theory developments tend to be
%very large (the representation of current POs for the Isabelle/AFP can
%easily reach several TB although using techniques for compression), A
%technical pre-requisite for interchangeability, connectivity and
%advanced search consists in a structured, typed format for meta-data
%together with a flexible mechanism of their validation. Technically,
%this kind of meta-data has the form of a function annoconst : arg1 ->
%... -> argn -> proof-term -> proof-term where annoconst is a constant
%symbol which represents an identity in the proof-term (so, any import
%function of a specific system can actually ignore it), and where the
%argi represent terms with meta-information such as, eg., “this
%proof-term represents a free data-type construction of the form ...”,
%or “this part of the proof is a derivation of a free data-type of the
%following form ...”, “this lifting over assumptions represents in
%Isabelle a Locale-instantiation”, “this part of a theory
%development is connected to ... ”, “this theorem belongs to the
%sub-class of XXX ... theorems”, etcpp. For arguments of annotations,
%validation-functions can be defined that may check that the argument
%terms satisfy a certain property wrt. to the proof-term and the
%current logical context. Dedukti will provide a framework that allows
%for each proof-system (Coq, HOL4, Isabelle...) to declare meta-data
%together with validations and thus communicate tool-specific knowledge
%to other systems. This framework can be seen as a particular form of
%an ontology definition language.
% 
%WP8: Indexing and browsing [?]  Construct tools to index and browse
%this encyclopedia, that is find the theorem one needs, either by
%looking for it with its name, with its statement, or with symbols
%occurring in it.


\end{workpackage}

%%% Local Variables:
%%% mode: latex
%%% TeX-master: "../propB"
%%% End:
