\begin{workpackage}[id=structuring,wphases=0-48,type=RTD,
  short=Structured Theories and Metadata,% for Figure 5.
  title=Structured Theories and Metadata,
  lead=Fau,
  BolRM=6,
  FauRM=36,
  SacRM=36,
  TouRM=12,
  InrRM=12]
  
\ednote{Which sites are interested?
David Deharbe and Etienne Prun (Clearsy); Nicola Gambino, Michael Rathjen, Claudio Sacerdoti Coen, Dale Miller, Emilio J. Gallego Arias, Michael Butler, Pierre Senellart}

% David Deharbe and Etienne Prun (Clearsy): use B Method, would like to doublecheck B proofs, integrate B with other proof assistant at high-level


\begin{wpobjectives}
FAIR library management, specifically Access, Search

In this work-package, we provide infra-structure for the exchange of the information between prover systems as
well for technical and domain-specific search functionality providing advanced access to the 
Logipedia knowledge base.
\end{wpobjectives}


\begin{wpdescription}
We proceed in three steps.
Firstly, Tasks~\localtaskref{strlibstructure} and~\localtaskref{strdofimpl} extend the Dedukti language with features for high-level representations that are critical for accessing parts of and searching libraries.
This includes a framework to \emph{define} typed meta-data in form of ontologies, and to \emph{enforce} them in 
the Dedukti libraries.
Secondly, Tasks~\localtaskref{strrefonto}, \localtaskref{strontorepml}, and~\localtaskref{strsafeonto}\ednote{remove if canceled} built ontologies serving as technical exchange format as well as domain-specific descriptions of libraries.
Thirdly, Task~\localtaskref{strtext} builds the first block towards
    connecting Logipedia with theorems found in natural-language
    research articles, by automatically extracting information from these
    natural-language statements.

This work package will be jointly coordinated by Florian Rabe at \site{Fau} and Burkhart Wolff at \site{Sac}.
Burkhart Wolff implemented a document ontology framework in Isabelle and developed several applications
in the field of formal software engineering.
\cite{brucker.ea:ontologies-certification:2019,brucker.ea:isabelle-ontologies:2018,brucker.ea:ontologies-certification:2019} 
\end{wpdescription}

\begin{tasklist}
\begin{task}[id=strlibstructure,title=Library Structure,lead=Fau,FauRM=12]
add library, document, theory structure to Dedukti; includes unique identifiers and informal text

%We extend the Dedukti language with features for high-level representations.
%This will include
%\begin{compactitem}
%\item theories: a general term we use to unify a variety of module system constructs such as type classes or locales,
%\item derived declarations: high-level declarations such as inductive type definitions, whose semantics is given by elaboration into more primitive constructs,
%\item metadata annotations: a general framework for attaching information about semantics, document structure, and tool interaction.
%\end{compactitem}
%
%The low- and high-level representations will be tightly integrated: any declaration or object may be given alternatively through either or both of these.
%The prover exports from instrumentation will be such that they produce both representations whenever possible.
%
%Then we leverage this design in several applications including automated prover interaction and a Logipedia-wide search service.
\end{task} 

\begin{task}[id=strdofimpl,title=Ontological Framework for Meta-Data,lead=Sac,SacRM=18]
metadata annotations in Dedukti at all levels
\end{task} 

\begin{task}[id=strrefonto,title=Reference Ontology,lead=Fau,FauRM=6,SacRM=6]
This tasks compiles, integrates, and curates the various ontologies used for describing libraries in the project.
These come from several sources:
\begin{compactitem}
 \item The ontology induced by the structuring features built in task \localtaskref{strlibstructure}.
 \item The ontologies built by users using the ontology framework built in task \localtaskref{strrefonto}.
 \item Manually written ontologies or imports of existing ontologies for specific domains whose knowledge is formalized in prover libraries, such as software engineering or mathematics.
\end{compactitem}
\end{task} 

\begin{task}[id=strdomonto,title= Domain Ontologies for Formal Methods in SE,lead=Tou,TouRM=12, SacRM=6]
Y. Aitameur at \site{Tou}
\begin{compactitem}
\item domain ontologies as descriptive models for engineering domains 
\item links/imports with/from standards and certification
\item engineering models annotations
\item strengthening engineering models by references to domain ontologies
\item Case studies could be certification, safety, security.
\end{compactitem}
\end{task} 

\begin{task}[id=strontorepml,title=Ontological Representation of Formal Libraries,lead=Fau,FauRM=12,BolRM=6,SacRM=6]
export ontological data from Isabelle and Coq as prototyped in \cite{ulo}

This will be closely tied to the work by M. Wenzel as sub-contractor of \site{Fau} (6 PM) for Isabelle, and C. Sacerdoti Coen at \site{Bol} for Coq in \WPref{libraries}.
\end{task} 

% Moved to WP9
%\begin{task}[id=strontosearch,title=Ontological Search,lead=Fau,FauRM=12,SacRM=6]
%Search based on ontological data (RDF triples) using systems like SPARQL
%\ednote{possible participation of S. Dumbrava; is there a site for this?}
%\end{task} 

%\begin{task}[id=strformsearch,title=Formula-based Search,lead=Fau,BolRM=4,FauRM=12]
%Search based on formula structure using systems like MathWebSearch
%\end{task} 

\begin{task}[id=strtext,title=Ontological Representation of Natural Language Articles,lead=Inr,FauRM=6,InrRM=12]
    % task leader: Pierre Senellart, Inria
P. Senellart at \site{Inr} and M. Kohlhase at \site{Fau} will work on the automatic extraction and annotation of natural-language theorem statements and proofs from published articles, as well as building libraries of such theorems and proofs
  
  % Search capabilities have been moved to WP9
  %and search and querying capabilities
\end{task} 

%\begin{task}[id=strpresentation,title=Customizable Presentation]
%\ednote{This task is for the British site Exeter (AD Brucker) if it is added. Otherwise cancelled.}
%\end{task} 
%
%\begin{task}[id=strsafeonto,title=Safety Ontologies]
%\ednote{This task is for the British site York (S Foster) it is added. Otherwise cancelled.}
%\end{task} 

\end{tasklist}

\begin{wpdelivs}
  \begin{wpdeliv}[due=3,miles=startup,id=requirements,dissem=PU,nature=DEM,lead=Inr]
      {TBD}
\end{wpdeliv}
\end{wpdelivs}


%\begin{enumerate}
%\item concrete/surface syntaxes 
%\item Central Library Backend Systems 
%\item Cross-System Front-Ends/Portals (Logipedia, ...)
%\item Semantic Middleware-based System Interoperability
%\end{enumerate} 
%
%Since proof-objects for substantial theory developments tend to be
%very large (the representation of current POs for the Isabelle/AFP can
%easily reach several TB although using techniques for compression), A
%technical pre-requisite for interchangeability, connectivity and
%advanced search consists in a structured, typed format for meta-data
%together with a flexible mechanism of their validation. Technically,
%this kind of meta-data has the form of a function annoconst : arg1 ->
%... -> argn -> proof-term -> proof-term where annoconst is a constant
%symbol which represents an identity in the proof-term (so, any import
%function of a specific system can actually ignore it), and where the
%argi represent terms with meta-information such as, eg., “this
%proof-term represents a free data-type construction of the form ...”,
%or “this part of the proof is a derivation of a free data-type of the
%following form ...”, “this lifting over assumptions represents in
%Isabelle a Locale-instantiation”, “this part of a theory
%development is connected to ... ”, “this theorem belongs to the
%sub-class of XXX ... theorems”, etcpp. For arguments of annotations,
%validation-functions can be defined that may check that the argument
%terms satisfy a certain property wrt. to the proof-term and the
%current logical context. Dedukti will provide a framework that allows
%for each proof-system (Coq, HOL4, Isabelle...) to declare meta-data
%together with validations and thus communicate tool-specific knowledge
%to other systems. This framework can be seen as a particular form of
%an ontology definition language.
% 
%WP8: Indexing and browsing [?]  Construct tools to index and browse
%this encyclopedia, that is find the theorem one needs, either by
%looking for it with its name, with its statement, or with symbols
%occurring in it.


\end{workpackage}

%%% Local Variables:
%%% mode: latex
%%% TeX-master: "../propB"
%%% End:
