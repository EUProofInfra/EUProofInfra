% task leader: Ulf or Jesper

% other participants:

% \textbf{Budget requirements:} One research engineer at Chalmers, and one PhD student or postdoc at TU Delft.

\begin{enumerate}
  \item Implement a basic version of Agda2Dedukti that can handle core
  features of Agda such as inductive datatypes and dependent pattern
  matching.
  \item Extend Agda2Dedukti with $\eta$-equality for record types.
  \item Extend Agda2Dedukti with first-class universe level polymorphism.
\end{enumerate}

% Moved to concept & methodology
% Agda is a popular dependently typed programming language / proof
% assistant based on Martin-L\"of’s intuitionistic type theory. Its theory
% is similar to Coq and Lean, but is more focused on interactive
% development and direct manipulation of proof terms (in contrast to
% using a tactic language to generate the proof terms). Agda has a
% sizable \href{https://github.com/agda/agda-stdlib}{standard library}
% that consists of both utilities for programming and mathematical proofs.

% Moved to concept & methodology
% In the summer of 2019, Guillaume Genestier (Inr) worked together with Jesper
% Cockx (Got) on the implementation of an experimental translator from Agda to
% Dedukti during a research visit at Chalmers University in Sweden. This
% translator is still work in progress, but it is already able to
% translate 142 modules of the Agda standard library to a form that can
% be checked in Dedukti. This exploratory work uncovered several
% challenges and opportunities for further work, which are outlined
% below.

% Moved to methodology
% \begin{enumerate}
%   \item Agda proofs often rely on type-directed conversion rules such
%   as eta-equality and definitional irrelevance, which can lead to a
%   blow-up in the size of proof terms. We plan to investigate possible
%   approaches to avoid this blow-up, either by finding a better
%   encoding which reduces the size of the type annotation, or by
%   extending the Dedukti language with type-directed conversion rules
%   to render the type annotations unneccessary.

%   \item Universe polymorphism in Agda relies on a built-in type of
%   levels that has complex structure of (in)equalities. We plan to
%   define a sound and complete embedding of Agda’s level type in
%   Dedukti, based on the existing work on encoding AC
%   (associative-commutative) theories. This would both serve as a
%   stress test of how well Dedukti can handle complex equational
%   theories, and improve our understanding of type theories with
%   first-class universe level polymorphism, which would be useful for
%   the implementation of Agda.

%   % Removed from this proposal for now
%   % \item In contrast to Coq and Lean, Agda does not have a well-defined
%   % core language to which proofs are elaborated. Instead, definitions
%   % are translated to an internal representation that is relatively
%   % close to the user input. This provides a challenge when translating
%   % Agda proofs to Dedukti: each feature in Agda’s internal syntax needs
%   % to have its own translation. As part of this project, we will hence
%   % investigate possible designs for a core language for Agda. Having
%   % such a core language would have several benefits: it would deepen
%   % our understanding of the Agda language, it would increase the
%   % trustworthiness of Agda proofs, and it would make it much easier to
%   % export Agda terms to other languages (such as Dedukti in the context
%   % of this project).

%   % \item Agda provides an experimental option for extending the
%   % language with user-defined rewrite rules, which are very similar to
%   % the rewrite rules provided by Dedukti. By comparing the two
%   % implementations we hope to gain new insights and find opportunities
%   % for improvement on both sides.
% \end{enumerate}

%%% Local Variables:
%%% mode: latex
%%% TeX-master: "../propB"
%%% End:
