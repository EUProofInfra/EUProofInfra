\begin{workpackage}[id=atpetc,type=RTD,
  short={Automatic theorem provers},% for Figure 5.
  title={Automatic theorem provers},
  lead=Sac,
  LieRM=48,
  InnRM=6,
  BelRM=12,
  SacRM=9,%9
  ImtRM=9,%9
  OcaRM=14,
  CeaRM=18,
%  CleRM=24,
  StuRM=0]

% \ednote{MK: interested parties (add their sites and RM here): Guillaume Burel, Raphaël Cauderlier, David Deharbe, Pascal Fontaine, Emilio J. Gallego Arias, Olivier Hermant, Cezary Kaliszyk, Chantal Keller, Filip Marić, Stephan Merz, Dale Miller, Julien Narboux, Martin Suda, Josef Urban}

\begin{wpobjectives}
  The objective of this work package is to use and develop automatic
  theorem provers to populate, help, and benefit from Logipedia. The
  focus is networking and research.
  \begin{compactitem}
  \item Automatic theorem provers will be instrumented in a two-step
    process to produce Dedukti statements.
  \item Dedukti statements will be made understandable and usable by
    automatic theorem provers.
  \item Automatic theorem provers will increase Logipedia readiness by
    automatically putting all the pieces together.
  \end{compactitem}
  The final objective is to develop the infrastructure to safely use automation
  all along the way.
\end{wpobjectives}


\begin{wpdescription}

  The aim of this work package is to connect automated tools to the Logipedia
  infrastructure.

% https://annuel2.framapad.org/p/9ef1-logipedia?lang=fr
% https://lite.framacalc.org/9f0a-logipediawp4

The importance of proofs in automated theorem provers, satisfiability
modulo theories solvers, propositional satisfiability solvers and
model checkers is increasingly recognized.  While for the
propositional case, the community agrees on a well defined proof
format, the situation is not clear for the other kinds of automated
reasoners.  There is no clear format for SMT, and the TSTP format for
automated theorem provers fixes a syntactic template for proofs rather
than providing an unambiguous framework to express proofs
semantically.

Some preliminary works predating this proposal clearly establish that
Dedukti can accommodate proofs in Satisfiability Modulo Theories,
automated theorem provers, and SMT. In this work package, we will build
on those preliminary work and provide a set of conduits from the
established formats used in automated tools. For the tools that do not
have yet an established format, we concentrate on some provers: the
theorem prover for predicate logic E, the SMT solvers veriT and Alt-Ergo, and a
coherent logic reasoner, and provide conduits for those tools. These
conduits and the techniques used in the embedded translation will be
properly documented, to ease integration of further tools of the kind.
If a standardized proof format appears for some kind of tools, the
conduits will be updated to adopt the new standard.

In this work package, we also plan to integrate in Logipedia some well-chosen
proofs coming from automated tools.  Well-chosen proofs will have to be
representative of typical applications of the tools, and be of reasonable size.
They will serve as examples to the community, to illustrate the potentials of
Dedukti and Logipedia.

Import from Dedukti to automatic provers will also be developed, as well as
internal automation. Thus automation techniques can be put in practice to
produce new results for Logipedia and out of Logipedia.

This workpackage has the end-to-end objective to build a powerful automated
infrastructure: one will be able to split a large proof obligation into smaller
parts and distribute to the appropriate automatic engines, that would all
produce proofs, glued together in a single large proof for the original proof
obligation. A large-scale application about the semi-automatic verification of C
code will validate this infrastructure.
\end{wpdescription}


\begin{tasklist}
  \begin{task}[id=instrumenting,
      title=Instrumenting ATPs to produce traces,
      lead=Lie,
      LieRM=28,
      ImtRM=1,
      OcaRM=12,
      BelRM=6,
      wphases=0-48!1.0
    ]
  % WP Leader supervision: Pascal
  % Participants
  % Stephan Schulz
  % Martin Suda
  % Guillaume Bury (OCamlPro)
  % Guillaume Burel
  % Julien Narboux
  % Pedro Quaresma
  % Predrag Janicic
  % Resources: One PhD student shared bw Schulz and Fontaine
  % Albin Coquereau (OCamlPro)
  % Sylvain Conchon (OCamlPro)
  % François Bobot (Why3)

    Automated theorem provers should produce detailed proof traces, with
    sufficient information to be understood in the Logipedia context.  We will
    consider here the theorem prover for predicate logic E, the SMT solvers
    veriT and Alt-Ergo, and a
    coherent logic reasoner, and instrument them to produce detailed proof
    traces.  A first prototype will be made available early in the project, to
    enable work on the next task.  The final tools fully connectable to the
    Logipedia infrastructure will be delivered at the end of the project.

    %% \begin{compactitem}
    %% \item Goal: TODO
    %% \item Milestones: TODO
    %% \item Resources: 15 p.m engineer (Oca) + 36 p.m phd (Lie+Stu) + 3 p.m permanent (Bel) + 10 p.m engineer (Cle)
    %% \end{compactitem}

% Considered provers:
% \begin{itemize}
% \item Cubicle (OcamlPro, Guillaume Bury, Sylvain Conchon)
% \item provers for geometry (Julien Narboux and Pedro Quaresma)
% \item coherent logic theorem prover (Predrag Janicic)
% \item SMT (alt-ergo, veriT: Pascal Fontaine)
% \item FOL ATPs (Pascal Fontaine, Stephan Schulz)
% \item \dots
% \end{itemize}

% traces from SAT: understood

% traces from SMT: still some gaps, about arithmetic, preprocessing.  Engineering issue too because of the amount of code that is involved

% traces from geometry solvers: seems to be a good candidate, because the proof process is suitable to produce traces

% traces from FOL: there is a format, and mainstream solvers do follow the standard format, but proofs are coarse grained

% traces from HOL provers: strongly relates to FOL

\end{task}


  \begin{task}[id=tracetodedukti,
      title=Translate ATP traces into Dedukti,
      lead=Imt,
      LieRM=12,
      ImtRM=8,
      SacRM=2,
      wphases=6-48!.5
    ]
  % Pascal Fontaine
  % Chantal Keller
  % Martin Suda
  % Guillaume Bury (OCamlPro)
  % Guillaume Burel
  % Julien Narboux
  % Pedro Quaresma
  % Predrag Janicic
  % Resources: One PhD student shared bw Schulz and Fontaine
  % Albin Coquereau (OCamlPro)
  % Sylvain Conchon (OCamlPro)
    % François Bobot (Why3)

    This task focuses on the design and implementation of the tool Ekstrakto
    to translate into Dedukti the proof traces from the automatic theorem
    provers from task~\taskref{atpetc}{instrumenting}.  This task depends on
    task~\taskref{atpetc}{instrumenting}, but can be partly executed in
    parallel.

%% \begin{compactitem}
%% \item Goal: TODO
%% \item Milestones: TODO
%% \item Depends on T1 for some provers
%% \item Resources: 15 p.m engineer (Oca) + 36 p.m phd (Imt) + 2 p.m + 10 p.m engineer (Cle)
%%   permanent (Sac)
%% \end{compactitem}

% As pointed out in Task~\localtaskref{instrumenting}, it is easier to
% instrument provers to make them output traces instead of directly
% provide Dedukti proofs. The goal of this task is to reconstruct the
% proof traces in order to build Dedukti proofs from them. The proposed
% process is the following: each step of the trace is transformed into
% an independent subproblem; each of these subproblems is given to a
% prover that can output Dedukti proofs; proofs of the subproblems are
% then combined to produce a global proof of the original problem. Since
% subproblems correspond to atomic steps of the proof trace, they are
% relatively simple, so that we can hope that the prover producing
% Dedukti proofs will not struggle to find a proof. This process is
% quite similar to what is done by the hammer tools of interactive
% theorem provers (Sledgehammer in Isabelle/HOL, HOLyHammer for HOL4, etc.)
% which reconstruct proofs from traces produced by automated theorem
% provers.

% This scheme has already been prototyped in a tool called
% \href{https://github.com/Deducteam/ekstrakto}{Ekstrakto}. Ekstrakto takes a TSTP
% file, as can be produced by e.g.\ the provers E and Zipperposition, and it uses
% Zenon Modulo and ArchSAT to prove the subproblems. Ekstrakto was designed to be
% agnostic w.r.t.\ the prover producing the trace; in particular it does not
% depend on the specific set of inference rules of the prover. It was also
% designed to be agnostic w.r.t.\ the prover used to prove the subproblems; it is
% only required that the prover can output a Dedukti proof in the correct encoding
% of first-order logic.

% Although Ekstrakto has already shown that it is a
% valuable approach, it is work in progress. In particular, the
% following issues still need to be addressed:

% \begin{compactenum}
%   % extension to other proof trace formats
% \item Up to now, Ekstrakto can only take traces in TSTP format as input. We plan
%   to make it understand traces in other formats, notably traces from SMT
%   solvers, as well as all formats developed in
%   Task~\localtaskref{instrumenting}.

%   Deliverables :
%   \begin{itemize}
%   \item T+6 a version of Ekstrakto that input traces from SMT solvers
%   \end{itemize}

%   % unprovable steps

% \item  Some steps in the proof traces are not provable: their conclusion is
%   not a logical consequence of their premises. However, they preserve
%   provability: the original problem has a proof if and only if the
%   problem with the conclusion of the step also has a proof. This is the
%   case for instance of the Skolemization step in first-order automated
%   theorem provers, of the introduction of new definitions, as well as
%   the RAT property in traces produced by SAT solvers. The approach of
%   Ekstrakto cannot be used here, because the subproblem corresponding to
%   the step cannot be proved. However, since provability is preserved, it
%   should be possible to transform a
%   proof using the conclusion of the step into a proof using its
%   premises. Such a transformation depends on the nature of the step that
%   has been used. We plan to include in Ekstrakto a way to handle
%   Skolemization and definition introduction, which are the two step
%   families that are missing to be able to manage all traces from the
%   major first-order theorem provers.

%   Deliverables :
%   \begin{itemize}
%   \item T+12 a version of Ekstrakto that handles definition-introduction steps
%   \item T+24 a version of Ekstrakto that handles Skolemization steps
%   \end{itemize}


%   % specialization for theories

% \item  Dedukti-producing provers used by Ekstrakto, namely Zenon Modulo and
%   ArchSAT, are meant for pure first-order logic. However, we would like
%   to deal with proof traces that use some specialized theory,
%   e.g.\ arithmetic or bit-vectors, as could be output by SMT
%   solvers. Although such theories could be presented as a set of axioms
%   in first-order logic, it is almost certain that neither Zenon Modulo
%   nor ArchSAT could be able to find non-trivial proofs using these
%   axioms. Here, the idea would be to develop small provers dedicated to
%   a particular theory, and outputting Dedukti proofs. Such provers would
%   be called when a step in the trace relies on said theory. These
%   provers need not be very optimized, since trace steps are relatively
%   small; this should help producing Dedukti traces. A way to achieve
%   this could be to extend Zenon Modulo: indeed, Zenon modulo can find
%   proofs modulo arithmetic, but it is not able to produce a Dedukti
%   proof yet.

%   Deliverables :
%   \begin{itemize}
%   \item T+36 a prover modulo arithmetic outputting Dedukti proofs
%   \item T+48 a prover modulo another theory outputting Dedukti proofs
%   \end{itemize}

% \end{compactenum}
%   Depends on T1.

%   One PhD student
  
\end{task}


  \begin{task}[id=deduktitoatp,
      title=Translate Dedukti statements into ATPs inputs,
      lead=Oca,
      LieRM=8,
      SacRM=3,
      BelRM=3,
      OcaRM=2,
      wphases=0-48!.5
    ]
    % Task leader: Guillaume Bury (WP leader contact: Chantal Keller)
    % Guillaume Bury
    % Chantal Keller
    % Pascal Fontaine (SMT-LIB)
    % Josef Urban + Martin Suda (TPTP)
    % François Bobot (Why3)

    To be fully integrated into Logipedia, automatic tools need to understand
    Dedukti statements.  This task is about creating the software infrastructure
    to translate Logipedia statements into the standardized input format for
    some automatic tools, namely TPTP and SMT-LIB.  This task can be conducted
    in parallel of tasks~\taskref{atpetc}{instrumenting}
    and~\taskref{atpetc}{tracetodedukti}.


    %% \begin{compactitem}
    %% \item Goal: TODO
    %% \item Milestones: TODO
    %% \item Resources: 3 p.m engineer (Oca) + 3 p.m permanent (Bel) + 3 p.m permanent (Sac)+ 4 p.m engineer (Cle)
    %% \end{compactitem}
  
% \begin{itemize}
% \item encodings into FOL
% \item links to WP6: reverse mathematics
% \item benchmarks
% \end{itemize}

  \end{task}


  %% Removed task due to lack of resources
  %% \begin{task}[id=library,
  %%     title=Logipedia as a source of knowledge for ATP]
  %%   Task leader: Martin Suda (WP leader contact: Pascal Fontaine)
  %%   % Josef Urban
  %%   % Pascal Fontaine
  %%   % Stephan Schulz

  %%   \begin{compactitem}
  %%   \item Goal: TODO
  %%   \item Milestones: TODO
  %%   \item Depends on T3
  %%   \item Resources: 15 p.m post-doc (Pra) + 3 p.m permanent (Bel)
  %%   \end{compactitem}
    
  %%   % Depends on T3.

  %%   % \begin{itemize}
  %%   % \item A library of known facts for ATPs
  %%   % \item Lemma selection is crucial
  %%   % \end{itemize}

  %%   % Proposition: Add in ACSL (C specification language) an "import from
  %%   % Logipedia, ..." which allows a user to get the ressources to model the
  %%   % behavior of its code. The modelisation is at the end used by ATP through
  %%   % Frama-C-WP.

  %% \end{task}



  \begin{task}[id=readiness,
      title=ATPs to increase Logipedia readiness,
      lead=Inn,
      SacRM=4,
      InnRM=6,
      BelRM=3,
      wphases=0-48!.5
    ]
    %  Task leader: Cezary Kaliszyk

    Internal automatic strategies will be defined for Dedukti, based on
    automatic theorem proving and machine learning. In combination with
    external ATPs through the results of
    tasks~\taskref{atpetc}{instrumenting},
    \taskref{atpetc}{tracetodedukti} and \taskref{atpetc}{deduktitoatp},
    they will be used to increase Logipedia readiness, e.g.\ by helping
    proof checking, by automatically combining proofs, or by discharging
    alignment concept obligations (in a joint effort with
    \taskref{alignment}{aligntheories} of \WPref{alignment}).


  % Guillaume Burel
  % People from other packages?  

% \begin{itemize}
% \item ATPs can be used to fill the holes in proofs (e.g.\ PVS)
% \item ATPs can be used to fill the gaps between systems and alignments
% \end{itemize}

\end{task}


\begin{task}[id=cooperation,
    title=Logipedia and ATP end-to-end,
    lead=Cea,
    CeaRM=18,
    wphases=12-48!.5
    ]
%  Task leader: François Bobot (Contact: Chantal Keller)

    Everything will be put together in the verification {\em a
      posteriori} of the Frama-C-WP tool for formal verification of C
    code and its back-end Why3. In this platform, the correctness of C
    programs relies on the automatic generation of proof obligations
    from annotated programs, and their automatic proofs using a
    combinations of ATPs and user-defined rewrite rules. In this task:
    \begin{itemize}
    \item the generation of proof obligations and the rewrite engine
      will be instrumented to produce Dedukti proofs, relying on (parts
      of) Ektrakto;
    \item ATPs will produce Dedukti proofs following
      tasks~\taskref{atpetc}{instrumenting} and
      \taskref{atpetc}{tracetodedukti};
    \item users will have access to the Logipedia library following
      task~\taskref{atpetc}{deduktitoatp}.
    \end{itemize}
    In the end, Dedukti proofs will be assembled together in a coherent whole.
    This task serves as an end-to-end demonstration of the usefulness of the
    infrastructure.

  % François Bobot (Why3)
  % Chantal Keller

%% \begin{compactitem}
%% \item Goal: TODO
%% \item Milestones: TODO
%% \item Depends on T1,2,3,4
%% \item Resources: 30 p.m permanent (Cea)
%% \end{compactitem}

% \begin{itemize}
% \item Why3: gather the proof from the called provers, add traces for the
%   Why3 transformations and send all this information to Ekstrakto
% \item External preprocessing, add traces to the rewrite engine of
%   Frama-C-WP, which is used before sending goals to solvers, and give
%   them to Ekstrakto
% \item Add in ACSL (C specification language) an "import from Logipedia,
%   ..." which allows a user to get the ressources to model the behavior
%   of its code. The modelisation is at the end used by ATP through
%   Frama-C-WP. (Move to T4?)
% \end{itemize}

\end{task}
\end{tasklist}


% \begin{tasklist}
%   \begin{task}[id=tools,title=Automatic Tools Exporting Proofs]
%     %% Guillaume Burel: expertise
%     %% Raphael Cauderlier: SAT/FOL proof checking in Dedukti
%     %% David Deharbe: pr in Atelier B
%     %% Pascal Fontaine: SMT
%     %% Emilio J. Gallego Arias
%     %% Thibault Gauthier
%     %% Olivier Hermant
%     %% Cezary Kaliszyk
%     %% Chantal Keller: expertise in translating proofs from SAT/SMT into other formalisms
%     %% Filip Marić
%     %% Stephan Merz
%     %% Dale Miller
%     %% Julien Narboux
%     %% Martin Suda: FOL ATP
%     %% Josef Urban: FOL ATP
%   \end{task}

%   \begin{task}[id=challenges,title=Logipedia as a Source of Challenges for Automatic Reasoners]
%     %% --> Translation to TPTP, SMT-LIB, DIMACS
%     %% Guillaume Burel: expertise / SAT
%     %% Pascal Fontaine: SMT-LIB
%     %% Cezary Kaliszyk: TPTP
%     %% Julien Narboux: Geometric benchs
%     %% Josef Urban: TPTP
%   \end{task}
%   \begin{task}[id=commang,title=Cooperation of Reasoners via Dedukti/Logipedia]

%     %% Once ITP understand Dedukti and ATP output Dedukti, then the first step of this is trivial
%     %% But many other things can be done within this task: choosing which solver to use, cutting proof obligations into pieces, etc...

%     %% Guillaume Burel: expertise
%     %% David Deharbe: Atelier B
%     %% Pascal Fontaine: SMT ++ ???
%     %% Chantal Keller: Knowledge in combining various ATPs
%     %% Martin Suda: cooperation between FOL ?
%     %% Josef Urban: cooperation between FOL ?
%   \end{task}
%   \begin{task}[id=database,title=A Database of Theorems for Automatic Solvers]

%     %% Depends on Task 2.  Then selecting Theorems a la Urban.

%     %% Guillaume Burel
%     %% David Deharbe
%     %% Pascal Fontaine
%     %% Emilio J. Gallego Arias
%     %% Thibault Gauthier
%     %% Olivier Hermant
%     %% Cezary Kaliszyk
%     %% Chantal Keller
%     %% Filip Marić
%     %% Stephan Merz
%     %% Dale Miller
%     %% Julien Narboux
%     %% Martin Suda
%     %% Josef Urban
%   \end{task}


% \end{tasklist}

\begin{wpdelivs}
  \begin{wpdeliv}[due=6,miles=??,id=ATPtrace1,dissem=PU,nature=DEM,lead=Lie]
    {Prototypes of Trace producing Automatic Theorem Prover and Satisfiability Modulo Theories solver}
  \end{wpdeliv}

  \begin{wpdeliv}[due=18,miles=??,id=Trace2Dedukti1,dissem=PU,nature=DEM,lead=Imt]
    {Prototype of a translator for some traces to Dedukti}
  \end{wpdeliv}

  \begin{wpdeliv}[due=24,miles=??,id=Dedukti2ATP1,dissem=PU,nature=DEM,lead=Sac]
    {Prototype of a translator from Dedukti statements into TPTP and
      SMT-LIB, application to simple libraries}
  \end{wpdeliv}

  \begin{wpdeliv}[due=36,miles=??,id=ReadinessCode,dissem=PU,nature=DEM,lead=Sac]
    {Tools based on ATPs to increase Logipedia Readiness}
  \end{wpdeliv}

  \begin{wpdeliv}[due=48,miles=??,id=ATPtrace2,dissem=PU,nature=DEM,lead=Lie]
    {Trace producing Automatic Theorem Prover, Satisfiability Modulo
      Theories solvers, and Coherent Logic Solver}
  \end{wpdeliv}

  \begin{wpdeliv}[due=48,miles=??,id=Trace2Dedukti2,dissem=PU,nature=DEM,lead=Imt]
    {Translator for most traces to Dedukti}
  \end{wpdeliv}

  \begin{wpdeliv}[due=48,miles=??,id=Dedukti2ATP2,dissem=PU,nature=DEM,lead=Sac]
    {Translator from expressive Dedukti statements into TPTP and
      SMT-LIB, application to large libraries}
  \end{wpdeliv}

  \begin{wpdeliv}[due=48,miles=??,id=ReadinessReport,dissem=PU,nature=R,lead=Inn]
    {Report on the Proof Automation for Logipedia}
  \end{wpdeliv}

  \begin{wpdeliv}[due=48,miles=??,id=Why3,dissem=PU,nature=DEM,lead=Cea]
    {A posteriori verification of non trivial C programs in the
      Frama-C-WP platform}
  \end{wpdeliv}

\end{wpdelivs}
\end{workpackage}

%%% Local Variables:
%%% mode: latex
%%% TeX-master: "../propB"
%%% End:
