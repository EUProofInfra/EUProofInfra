\begin{workpackage}[id=atpetc,wphases=0-48,type=RTD,
  short=ATPs etc.,% for Figure 5.
  title={ATP, SAT, SMT, Model checkers},
  lead=Lie,
  LieRM=10]

\ednote{MK: interested parties (add their sites and RM here): Guillaume Burel, Raphaël Cauderlier, David Deharbe, Pascal Fontaine, Emilio J. Gallego Arias, Olivier Hermant, Cezary Kaliszyk, Chantal Keller, Filip Marić, Stephan Merz, Dale Miller, Julien Narboux, Martin Suda, Josef Urban}

\begin{wpobjectives}
  The objective of this work package is to \ldots

This includes notably:
  \begin{compactitem}
  \item \ldots
  \end{compactitem}
  A key aspect will be to foster \ldots
\end{wpobjectives}


\begin{wpdescription}

% https://annuel2.framapad.org/p/9ef1-logipedia?lang=fr

The importance of proofs in automated theorem provers, satisfiability
modulo theories solvers, propositional satisfiability solvers and
model checkers is increasingly recognized.  While for the
propositional case, the community agrees on a well defined proof
format, the situation is not clear for the other kinds of automated
reasoners.  There is no clear format for SMT, and the TSTP format for
automated theorem provers fixes a syntactic template for proofs rather
than providing an unambiguous framework to express proofs
semantically.

Some preliminary works predating this proposal clearly establish that
Dedukti can accommodate proofs in Satisfiability Modulo Theories,
automated theorem provers, and SMT.  In this work package, we will
build on those preliminary work and provide a set of conduits from the
established formats used in automated tools. For the tools that do not
have yet an established format, we will make a selection of tools
(Zipperposition and E for automated theorem provers, CVC4 and veriT
for SMT, ??? for model checking) and provide a conduits for those
tools.  These conduits and the techniques used in the embedded
translation will be properly documented, to ease integration of
further tools of the kind.  If a standardized proof format appears for
some kind of tools, the conduits will be updated to adopt the new
standard.

In this work package, we also plan to integrate in Logipedia some
well-chosen proofs coming from automated tools.  Well-chosen proofs
will have to be representative of typical applications of the tools,
and be of reasonable size.  They will serve as examples to the
community, to illustrate the potentials of Dedukti and Logipedia.


Create the infrastructure to enable the long term goal: be able to split a large proof
obligation into smaller parts and distribute to the appropriate automatic engines, that
would all produce proofs, glued together in a single large proof for the original proof
obligation.
\ednote{Nancy, Liège}
\end{wpdescription}


\begin{tasklist}
\begin{task}[id=instrumenting, title=Instrumenting ATPs to produce
  traces]
  Task leader: Pascal Fontaine
  %Pascal
  % Stephan Schulz, Martin Suda

  One PhD student shared bw Schulz and Fontaine
  
Considered provers:
\begin{itemize}
\item Cubicle (OcamlPro)
\item provers for geometry (Julien Narboux and Pedro Quaresma)
\item coherent logic theorem prover (Predrag Janicic)
\item SMT (Pascal Fontaine)
\item FOL ATPs (Pascal Fontaine and Stephan Schulz)
\item \dots
\end{itemize}

\end{task}


\begin{task}[id=tracetodedukti, title=Translate ATP traces into Dedukti]
  Task leader: Guillaume Burel
  %Pascal
  
  Depends on T1.

  One PhD student
  
\end{task}


\begin{task}[id=deduktitoatp, title=Translate Dedukti statements into ATPs inputs]
Task leader: ? (Contact: Chantal Keller)

\begin{itemize}
\item encodings into FOL
\item links to WP6: reverse mathematics
\item benchmarks
\end{itemize}

\end{task}


\begin{task}[id=library, title=Logipedia as a source of knowledge for ATP]
  Task leader: %Joseph Urban
  % Pascal

Depends on T3.

\begin{itemize}
\item A library of known facts for ATPs
\item Lemma selection is crucial
\end{itemize}

Proposition: Add in ACSL (C specification language) an "import from
Logipedia, ..." which allows a user to get the ressources to model the
behavior of its code. The modelisation is at the end used by ATP through
Frama-C-WP.

\end{task}


\begin{task}[id=reconstruction, title=ATPs for Logipedia]
  Task leader: Cezary Kaliszyk
  % Chantal

  Six month PhD student

\begin{itemize}
\item native provers for Logipedia
\item proof reconstruction
\end{itemize}

\end{task}


\begin{task}[id=readiness, title=Using ATPs to increase Logipedia readiness]
  Task leader: ?
  (Contact: Chantal Keller)

\begin{itemize}
\item ATPs can be used to fill the holes in proofs (e.g. PVS)
\item ATPs can be used to fill the gaps between systems and alignments
\end{itemize}

\end{task}


\begin{task}[id=cooperation, title=Make ATPs cooperate]
  Task leader: François Bobot
  % Chantal

  One PhD student joint with Chantal/François

\begin{itemize}
\item Why3: gather the proof from the called provers, add traces for the
  Why3 transformations and send all this information to Ekstrakto
\item External preprocessing, add traces to the rewrite engine of
  Frama-C-WP, which is used before sending goals to solvers, and give
  them to Ekstrakto
\item Add in ACSL (C specification language) an "import from Logipedia,
  ..." which allows a user to get the ressources to model the behavior
  of its code. The modelisation is at the end used by ATP through
  Frama-C-WP. (Move to T4?)
\end{itemize}

\end{task}
\end{tasklist}


% \begin{tasklist}
%   \begin{task}[id=tools,title=Automatic Tools Exporting Proofs]
%     %% Guillaume Burel: expertise
%     %% Raphael Cauderlier: SAT/FOL proof checking in Dedukti
%     %% David Deharbe: pr in Atelier B
%     %% Pascal Fontaine: SMT
%     %% Emilio J. Gallego Arias
%     %% Thibault Gauthier
%     %% Olivier Hermant
%     %% Cezary Kaliszyk
%     %% Chantal Keller: expertise in translating proofs from SAT/SMT into other formalisms
%     %% Filip Marić
%     %% Stephan Merz
%     %% Dale Miller
%     %% Julien Narboux
%     %% Martin Suda: FOL ATP
%     %% Josef Urban: FOL ATP
%   \end{task}

%   \begin{task}[id=challenges,title=Logipedia as a Source of Challenges for Automatic Reasoners]
%     %% --> Translation to TPTP, SMT-LIB, DIMACS
%     %% Guillaume Burel: expertise / SAT
%     %% Pascal Fontaine: SMT-LIB
%     %% Cezary Kaliszyk: TPTP
%     %% Julien Narboux: Geometric benchs
%     %% Josef Urban: TPTP
%   \end{task}
%   \begin{task}[id=commang,title=Cooperation of Reasoners via Dedukti/Logipedia]

%     %% Once ITP understand Dedukti and ATP output Dedukti, then the first step of this is trivial
%     %% But many other things can be done within this task: choosing which solver to use, cutting proof obligations into pieces, etc...

%     %% Guillaume Burel: expertise
%     %% David Deharbe: Atelier B
%     %% Pascal Fontaine: SMT ++ ???
%     %% Chantal Keller: Knowledge in combining various ATPs
%     %% Martin Suda: cooperation between FOL ?
%     %% Josef Urban: cooperation between FOL ?
%   \end{task}
%   \begin{task}[id=database,title=A Database of Theorems for Automatic Solvers]

%     %% Depends on Task 2.  Then selecting Theorems a la Urban.

%     %% Guillaume Burel
%     %% David Deharbe
%     %% Pascal Fontaine
%     %% Emilio J. Gallego Arias
%     %% Thibault Gauthier
%     %% Olivier Hermant
%     %% Cezary Kaliszyk
%     %% Chantal Keller
%     %% Filip Marić
%     %% Stephan Merz
%     %% Dale Miller
%     %% Julien Narboux
%     %% Martin Suda
%     %% Josef Urban
%   \end{task}


% \end{tasklist}

\begin{wpdelivs}
  \begin{wpdeliv}[due=3,miles=startup,id=requirements,dissem=PU,nature=DEM,lead=Inr]
      {Requirements Analysis and Synchronization}
\end{wpdeliv}
\end{wpdelivs}
\end{workpackage}

%%% Local Variables:
%%% mode: latex
%%% TeX-master: "../propB"
%%% End:
