\begin{workpackage}[id=instrumentation,type=RTD,wphases=1-48,
  short=Integration,% for Figure 5.
  title=Integration,
  lead=Del,
  DelRM=14,
  GotRM=4,
  TumRM=5,
  ChaRM=16,
  CleRM=0,  % TODO
  ImtRM=0,  % TODO
  TouRM=0,  % TODO
  BolRM=16, % 12 on Coq, 4 on Matita
  InrRM=0]  % TODO

\begin{wpobjectives}
  The objective of this work package is to bring systems already at least at
  LIL 1 to higher LIL levels, that is, to instrument the systems
  for which we already know how to encode the proofs in Dedukti, and
  make available these proofs in Logipedia so that
  they can be exported to other systems.
\end{wpobjectives}

\begin{wpdescription}
  The concrete work that has to be performed to integrate a particular
  system into Logipedia depends on how much of the theory of that
  system has already been encoded in Dedukti, and how mature the
  existing (prototype) tools for exporting proofs are already. Thus
  the tasks in this work package are divided among the different
  interactive theorem provers we consider: Agda, Isabelle, HOL4,
  Atelier-B and Rodin, and Matita.
\end{wpdescription}

\begin{tasklist}
\begin{task}[id=agda,
  title=Instrument Agda,
  lead=Del,
  DelRM=14,
  GotRM=4,
  wphases=1-18]
% % task leader: Ulf or Jesper

% other participants:

%\textbf{Budget requirements:} One research engineer at Chalmers, and one PhD student or postdoc at TU Delft.

% Moved to concept & methodology
% Agda is a popular dependently typed programming language / proof
% assistant based on Martin-L\"of’s intuitionistic type theory. Its theory
% is similar to Coq and Lean, but is more focused on interactive
% development and direct manipulation of proof terms (in contrast to
% using a tactic language to generate the proof terms). Agda has a
% sizable \href{https://github.com/agda/agda-stdlib}{standard library}
% that consists of both utilities for programming and mathematical proofs.

% Moved to concept & methodology
% In the summer of 2019, Guillaume Genestier (Inr) worked together with Jesper
% Cockx (Got) on the implementation of an experimental translator from Agda to
% Dedukti during a research visit at Chalmers University in Sweden. This
% translator is still work in progress, but it is already able to
% translate 142 modules of the Agda standard library to a form that can
% be checked in Dedukti. This exploratory work uncovered several
% challenges and opportunities for further work, which are outlined
% below.

\begin{enumerate}
  \item Agda proofs often rely on type-directed conversion rules such
  as eta-equality and definitional irrelevance, which can lead to a
  blow-up in the size of proof terms. We plan to investigate possible
  approaches to avoid this blow-up, either by finding a better
  encoding which reduces the size of the type annotation, or by
  extending the Dedukti language with type-directed conversion rules
  to render the type annotations unneccessary.

  \item Universe polymorphism in Agda relies on a built-in type of
  levels that has complex structure of (in)equalities. We plan to
  define a sound and complete embedding of Agda’s level type in
  Dedukti, based on the existing work on encoding AC
  (associative-commutative) theories. This would both serve as a
  stress test of how well Dedukti can handle complex equational
  theories, and improve our understanding of type theories with
  first-class universe level polymorphism, which would be useful for
  the implementation of Agda.

  % Removed from this proposal for now
  % \item In contrast to Coq and Lean, Agda does not have a well-defined
  % core language to which proofs are elaborated. Instead, definitions
  % are translated to an internal representation that is relatively
  % close to the user input. This provides a challenge when translating
  % Agda proofs to Dedukti: each feature in Agda’s internal syntax needs
  % to have its own translation. As part of this project, we will hence
  % investigate possible designs for a core language for Agda. Having
  % such a core language would have several benefits: it would deepen
  % our understanding of the Agda language, it would increase the
  % trustworthiness of Agda proofs, and it would make it much easier to
  % export Agda terms to other languages (such as Dedukti in the context
  % of this project).

  % \item Agda provides an experimental option for extending the
  % language with user-defined rewrite rules, which are very similar to
  % the rewrite rules provided by Dedukti. By comparing the two
  % implementations we hope to gain new insights and find opportunities
  % for improvement on both sides.
\end{enumerate}

%%% Local Variables:
%%% mode: latex
%%% TeX-master: "../propB"
%%% End:

\begin{compactitem}
\item Continue the development of the prototype Agda2Dedukti tool to
handle core features of Agda such as inductive datatypes and dependent
pattern matching in a robust manner.
\item Extend Agda2Dedukti with $\eta$-equality for record types and definitional irrelevance.
\item Extend Agda2Dedukti with first-class universe level polymorphism.
\end{compactitem}
\end{task}

\begin{task}[id=isabelle,
  title=Instrument Isabelle,
  lead=Tum,
  TumRM=5,
  wphases=1-12]
%[TU München]

Isabelle as a logical framework \cite{paulson700} is an intermediate
between Type-Theory provers (like Coq or Agda) and classic LCF-style
systems (like HOL Light or HOL4). The inference kernel can already
output proofs as $\lambda$-terms on request, but this has so far been
only used for small examples \cite{Berghofer-Nipkow:2000:TPHOL}. The
challenge is to make Isabelle proof terms work routinely for
reasonably big entries from The Archive of Formal Proofs
\cite{isabelle-afp}. Preliminary work by Wenzel (2019) has
demonstrated the feasibility for relatively small parts of
Isabelle/HOL: some orders of magnitude in scalability are still
missing.

This work package will revisit important aspects of the Isabelle/HOL
logic implementation on top of the Isabelle/Pure framework, such as
normalization of proofs, efficient type-class reasoning, special
representation of derived rules and definition principles (datatypes,
recursion, induction). The volume of proof term output may be reduced
further, by taking more structure of the target language (Dedukti)
into account and omitting certain low-level reasoning of HOL (e.g.\
for inductive types). [Subcontracted to Makarius Wenzel, Augsburg,
Germany.]

The underlying Isabelle/ML implementation platform (on top of Poly/ML)
will be revisited as well, to improve monitoring of memory usage, and
to double the standard heap size from 16\,GB to 32\,GB (without
suffering from the full overhead of the 64\,bit
addressing). [Subcontracted to David Matthews, Edinburgh, UK.]

\begin{compactitem}
\item Improve the efficiency of important aspects of the
  Isabelle/HOL logic implementation, such as normalization of proofs,
  type-class reasoning, and special representation of derived rules
  and definition principles.
\item Reduce the volume of proof terms in the Dedukti encoding, by
taking more structure of the target language (Dedukti) into account
and omitting certain low-level reasoning of HOL (e.g.\ for inductive
types).
\item Improve memory usage of the Isabelle/ML implementation platform.
\end{compactitem}
\end{task}

\begin{task}[id=HOL4,
  title=Instrument HOL4,
  lead=Cha,
  ChaRM=16.5,
  wphases=1-14]
%% task leader: Magnus

% 2 Person Years at Chalmers

% Moved to concept & methodology
% The HOL4 proof assistant is home to a few medium to large scale
% specifications and associated proof developments that have value
% outside of HOL4. These specifications include the formal semantics of
% the CakeML language (and its verified compiler) and an extensive
% specification of the ARM instruction set architecture (ISA) as
% formalised by Anthony Fox at the University of Cambridge.

% Moved to concept & methodology
% HOL4 has support for exporting proofs to the OpenTheory proof exchange
% format, and there has been some work on importing OpenTheory proofs
% into Dedukti. However, the current state of these techniques and their
% implementations does not scale to real examples such as those
% mentioned above.

This part of the project will be about re-thinking and re-designing
the tools HOL4-to-OpenTheory and OpenTheory-to-Dedukti tools such that
they scale to the point where real examples of interest, such as those
mentioned above, can be exported.



\begin{compactitem}
\item Develop the current proof-of-concept prototype
    HOL4-to-OpenTheory and OpenTheory-to-Dedukti tools to the point
    where they scale (both in terms of time and space) to cope with
    real examples of interest.
\end{compactitem}
\end{task}

\begin{task}[id=atelier-b,
  title=Instrument Atelier-B/Rodin,
  lead=Imt,
  CleRM=0, % TODO
  ImtRM=0, % TODO
  TouRM=0] % TODO
%% task leader: Catherine

% other participants: 

\ednote{Southhampton, Toulouse, Clearsy writes this}

% Moved to concept & methodology
% Atelier B, Rodin and ProB are platforms or tools to develop models
% written in B method, Event-B or B system. The development process is
% based on formal proof: proof obligations are automatically  generated
% and must be proven by automatic or interactive provers. ProB is an
% animator and model checker, it helps users to gain confidence in their
% specifications. It is also a disprover aiming at  discovering
% counter-examples for proof obligations. Atelier B and Rodin use native
% B proof tool, they also enable the use of external provers such as SMT
% solvers. ProB calls SMT and SAT solvers, it also uses contraints
% solvers such as Sicstus Prolog. All of them relies on the B logics,
% mainly a first order language with set theory. Regarding B/Event-B/B
% system, there are some variants, mainly regarding the refinement
% process they all implement. Refinement means that models are developed
% by successive steps, from an abstract model to a more  concrete
% model. Refinement in B method mainly means deriving a program while
% EventB and B System refinement aim at defining a model of a system by
% introducing details.


% Moved to concept & methodology
% In the context of the BWare project, an encoding of the set theory of
% the B method has been provided as a theory modulo, i.e. a rewrite
% system rather than a set of axioms. This encoding is used by the
% automatic prover Zenon modulo which features a backend to
% Dedukti. Thus, as a first step through instrumentation of Atelier B
% and Rodin, proof obligations coming from Atelier B can be proved by
% Zenon modulo producing Dedukti proofs, hence providing a better
% confidence in the proofs produced by the native proof tools of Atelier
% B \cite{Bware}.

This task has the following objectives:
\begin{enumerate}

  \item Continue the encoding of the B set theory in Dedukti to be
  able to handle all kind of proof obligations.

  \item Instrument the native provers to produce proofs (see also WP4).

  \item Exporting B models to Dedukti.
\end{enumerate}




\begin{compactitem}
\item Continue the encoding of the B set theory in Dedukti to be
  able to handle all kind of proof obligations in Zenon Modulo.
\item Instrument and export B/Event B models to Dedukti.
\item Develop a tool for importing lemmas coming from Logipedia
into B models.
\end{compactitem}
\end{task}

\begin{task}[id=matita,
  title=Integrate the Matita translator in Matita itself,
  lead=Bol,
  BolRM=4,
  wphases=1-12]
  % %[Bologna]

%\textbf{Budget requirements:} One one PhD student or postdoc at UBo.

Matita is an interactive theorem prover developed at the University of Bologna and used for teaching logic courses and to verify software and mathematical proofs, with special attention to predicative foundations. The first generation of the system (up to version 0.5.9) was born as a by-product of the MoWGLI FET-Open Project, it was compatible with the logic of Coq and it could re-use its libraries. It was an important test-bench for the integration of Mathematical Knowledge Management techniques with Interactive Theorem Proving, featuring for example a library of theorems distributed over multiple servers, innovative indexing and search techniques and automatic translation of proofs between declarative and procedural styles. The second generation of the system (up to the current version 0.99.3) was a re-implementation from scratch that departed from the logic of Coq and that experimented with the most concise ways to implement an efficient theorem prover. Several ideas later migrated into Coq. The currently available largest library is the formal certification of a complexity-preserving and cost-model-inducing compiler from C to MCS-51 machine code, developed in the FET project CerCo (Certified Complexity).

The standard and arithmetic libraries of Matita has been the first libraries to be exported to Logipedia using Krajono, a fork of Matita. The forked system is also actually the only one able to import Logipedia proofs. The choice of Matita as a test-bench for Logipedia is easily understood considering that the implementation of the 0.99.x series was aimed at obtaining a well-documented, minimal but fast implementation of a theorem prover, two order of magnitudes smaller than Coq.

The task will achieve the following results
\begin{enumerate}
\item Merge Krajono and Matita, update the code to the latest version and transfer the mainteinance effort from the Logipedia to the Matita team.
\item Export all the remaining Matita libraries. In particular
\begin{itemize}
 \item The libraries developed in CerCo contain several gigantic proof terms (nested proofs by cases on the 256 opcodes of the MCS-51 processor) that will stress the encoding and the tools developed around the Logipedia library.
 \item The proofs in the arithmetic libraries of Matita, now converted to HOL proofs inside Logipedia, do not exploit dependent types. Other libraries rely heavily on dependent types, triggering more interesting translations between theories encoded in Logipedia.
\end{itemize}
\item The logics of Matita and Coq remain quite similar, sharing a common core. However no complete automatic translation from Coq to Matita or vice versa is possible any more and only partial translations with high coverage are known, but not implemented, due to the intricacies of having to make the two code bases interact. We will study how to implement the partial translations directly in Logipedia, without knowledge of the internals of the two systems, and we will rely on automatically generated alignments to augment coverage of the translation.\ednote{CSC: this point probably does not belong to this WP}
\end{enumerate}

%%% Local Variables:
%%% mode: latex
%%% TeX-master: "../propB"
%%% End:

\begin{compactitem}
\item Merge Krajono and Matita, update the code to the latest
  version and transfer the maintenance effort to the Matita team.
\item Extend Krajono in order to record metadata that allow to
  invert the translation.
\item Optimize the code and the translation in order to be able to
  export all the libraries of Matita.
\end{compactitem}
\end{task}

\begin{task}[id=coq,
  title=Instrument Coq,
  lead=Bol,
  BolRM=12,
  InrRM=6,
  wphases=1-8]
  % % task leader: Claudio Sacerdoti Coen
% participants: Enrico Tassi, Claudio Sacerdoti Coen
%\textbf{Budget requirements:} One one PhD student or postdoc at UBo.

\ednote{Bol: 12 MM = 44,630 euros (39,372 euros salary + travels etc.); Inr: 12 MM}

% Moved to concept & methodology
% Coq is an interactive theorem prover developed at Inria since the 1984.
% It is based on Type Theory and was used to formally verify the correctness
% of both industrially relevant software such as the CompCert C compiler and
% complex mathematical proofs such as the one of the Four Color theorem and the
% one of the Odd Order theorem. In 2013 Coq received the ACM system award.

\begin{enumerate}
% deliverable 1
\item Access Coq internal data structures to gather logical data, such as
statements and proof terms (for tasks in WP3)
% deliverable 1
\item Implement the translation of Coq terms to Dedukti terms
% deliverable 2
\item Access Coq internal data structures to gather extra-logical data,
such as the role played by a constant in the library like begin an implicit
cast from one algebraic structure to another (for tasks in WP3)
% deliverable 2
\item Make extra-logical data available in a structured and
  extensible format (for tasks in WP5)
\end{enumerate}

% Moved to concept & methodology
%A technological hurdle steps (a) and (c) have to overcome is that Coq is an actively
%developed system that is constantly evolving. Previous attempts at extracting
%data from Coq without a direct interaction with the developers of Coq resulted
%in prototypes like CoqinE that quickly became outdated.
%To overcome the problems associated with external tools such as
%CoqInE, we plan to have the required instrumentation merged in Coq
%proper and reuse it for other projects that could benefit from it so
%to amortize its development cost. More precisely E. Tassi is a core
%Coq developer acquainted with its development process and he will take
%care of the integration of the infrastructure in Coq and foster its
%reuse in third party projects with similar needs such as SerAPI,
%CoqHammer and Coq-Elpi.

% Moved to concept & methodology
% Step (b) is also problematic for two reasons. The first one is that the encoding
% in Dedukti requires some information, typically types of sub-expressions, that
% are not stored in Coq, it is transient. So the instrumentation for steps (a) and
% (c) needs to be complemented by providing not only access to existing data but
% also to log transient data or re-synthesize it on demand. Both approaches may
% be used, depending on the the tradeoff between computation time and space for
% storage. The second reason, which is more critical, is ...

%Since the type theory of Coq is extremely large, with features that
%have no corresponding representation in Dedukti, the type theory of
%Coq needs to be translated to a core one representable in Dedukti.
%This translation to a core calculus is not implemented in Coq and the
%amount and complexity of code necessary for it is very significant and
%indeed the CoqinE prototype only covers a small subset of
%Coq. Feedback from WP2 will be of guidance\ednote{Tassi: there is no
%task in WP2 about Coq's TT, maybe it is included in HoTT?} to extend
%the translation currently available CoqinE to cover a larger subset of
%Coq.

% Moved to methodology
% In step (d) we plan to take advantage of the work done by Sacerdoti Coen (UBo)
% in 2019 in exporting non trivial logical and extra-logical data from Coq to
% an XML format. Data in that format was then translated by Kohlhase et al
% (UBo + FAU) to the MMT system, another logical framework to encode different
% logics and their libraries. We plan to extend that format to include even
% more extra-logical data as well as the data gathered in step (a) and (c). We shall
% evaluate if all the data needed for step (b) can be saved in this format, and
% give us the freedom to implement step (b) in a standalone tool making no
% requests to Coq in order to re-synthesize missing data.

%%% Local Variables:
%%% mode: latex
%%% TeX-master: "../propB"
%%% End:

\begin{compactitem}
\item Access Coq internal data structures to gather logical data, such as
statements and proof terms (for tasks in WP3).
\item Implement the required instrumentation to translate of Coq terms
to Dedukti terms as a part of Coq proper.
\item Access Coq internal data structures to gather extra-logical data,
such as the role played by a constant in the library like begin an implicit
cast from one algebraic structure to another (for tasks in WP3).
\item Make extra-logical data available in a structured and
  extensible format (for tasks in WP5).
\end{compactitem}
\end{task}
\end{tasklist}

\begin{wpdelivs}

  \begin{wpdeliv}[due=3,id=requirements,dissem=PU,nature=DEM,lead=Inr]{Requirements Analysis and Synchronization}
  \end{wpdeliv}

  \begin{wpdeliv}[due=12,miles=isabelle,id=isabelle,dissem=PU,nature=DEM,lead=Tum]{Robust export of proof terms for Isabelle}
  \end{wpdeliv}

  \begin{wpdeliv}[due=12,id=isabelleImproved,dissem=PU,nature=DEM,lead=Tum]{Improved memory management and monitoring for Poly/ML}
  \end{wpdeliv}

  \begin{wpdeliv}[due=12,miles=hol4,id=hol4,dissem=PU,nature=DEM,lead=Cha]{Export of proof terms for HOL4}
  \end{wpdeliv}

  \begin{wpdeliv}[due=18,id=agda,dissem=PU,nature=DEM,lead=Del]{Export Agda's standard library to Dedukti}
  \end{wpdeliv}

  \begin{wpdeliv}[due=8,miles=coq,id=coq1,dissem=PU,nature=DEM,lead=Inr]{Export of proof terms from Coq, no meta data}
  \end{wpdeliv}

  \begin{wpdeliv}[due=24,id=coq2,dissem=PU,nature=DEM,lead=Bol]{More scalable export of proof terms and meta data from Coq}
  \end{wpdeliv}

  \begin{wpdeliv}[due=12,id=matita1,dissem=PU,nature=DEM,lead=Bol]{Export of proof terms and meta data from Matita}
  \end{wpdeliv}

  \begin{wpdeliv}[due=24,id=atelier-b,dissem=PU,nature=DEM,lead=Cle]{a prototype of a tool exporting proof terms from Atelier B}
  \end{wpdeliv}

  \begin{wpdeliv}[due=48,id=atelier-b,dissem=PU,nature=DEM,lead=Cle]{Harness in Atelier B to export proof terms and import lemmas from Logipedia}
  \end{wpdeliv}

  \begin{wpdeliv}[due=48,id=atelier-b,dissem=PU,nature=DEM,lead=Tou]{Export/Import of B/Event B models from/in Rodin}
  \end{wpdeliv}

\end{wpdelivs}

\end{workpackage}

%%% Local Variables:
%%% mode: latex
%%% TeX-master: "../propB"
%%% End:
