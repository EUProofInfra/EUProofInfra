\begin{workpackage}[id=instrumentation,wphases=0-48,type=RTD,
  short=Instrument Provers,% for Figure 5.
  title=Instrument proof systems to produce Dedukti proof,
  lead=Inr,
  InrRM=10,
  DelRM=48,
  GURM=48]
  
\ednote{MK: We need one coordinating site. original coordinators: Frédéric Blanqui and
  Jesper Cockx}
\ednote{MK: interested parties (add their sites and RM here): David Deharbe,
Tobias Nipkow, Guillaume Genestier, Jesper Cockx, Guillaume Burel, Filip Marić, Makarius
Wenzel, Helmut Schwichtenberg, Nicolas Magaud, Gaspard Férey, Ulf Norell}

\begin{wpobjectives}
  The objective of this work package is to \ldots

This includes notably:
  \begin{compactitem}
  \item \ldots
  \end{compactitem}
  A key aspect will be to foster \ldots
\end{wpobjectives}


\begin{wpdescription}
We know how to express in Dedukti the theories implemented in Matita,
HOL Light, FoCaliZe, Coq, Agda, Lean, Minlog, Isabelle, HOL4,
Atelier B, and Rodin. The systems Matita, HOL Light, FoCaliZe, and
Coq, already have been instrumented to export proofs that can be
checked in Dedukti. Our first work package is to do the same thing for
Agda, Minlog, Isabelle, HOL4, and Atelier B.
Three methods have to be used here: some of the systems
(Automath style), such as Coq, Agda, Lean, and Minlog already have
proof-terms that can be output, thus the main task is to translate
these proofs into the Dedukti format. Others (LCF style), such as
Isabelle and HOL4, have an inference kernel that can be
instrumented, the main task here is to transform
the internal proof-object into an external proof-term. Others, such as
Atelier B and Rodin are slightly more difficult to address. For those,
we need to use the water ford method: extract an incomplete trace (a
sequence of lemmas) and fill the gap using automated theorem proving,
as experimented with Atelier B and Zenon.
\end{wpdescription}

\begin{tasklist}
\begin{task}[id=agda,title=instrument Agda]
%[G\"oteborg, Delft]

%\textbf{Budget requirements:} One research engineer at Chalmers, and one PhD student or postdoc at TU Delft.

%Question: does this task belong to WP1 or WP2?

Agda is a popular dependently typed programming language / proof
assistant based on Martin-L\"of’s intuitionistic type theory. Its theory
is similar to Coq and Lean, but is more focused on interactive
development and direct manipulation of proof terms (in contrast to
using a tactic language to generate the proof terms). Agda has a
sizable standard library (available at
https://github.com/agda/agda-stdlib) that consists of both utilities
for programming and mathematical proofs.


In the summer of 2019, Guillaume Genestier worked together with Jesper
Cockx on the implementation of an experimental translator from Agda to
Dedukti during a research visit at Chalmers University in Sweden. This
translator is still work in progress, but it is already able to
translate 142 modules of the Agda standard library to a form that can
be checked in Dedukti. This exploratory work uncovered several
challenges and opportunities for further work, which are outlined
below.

\begin{enumerate}
\item To support the construction of proof terms, Agda provides powerful
features such as dependent pattern and copattern matching, eta
equality for functions and record types, and definitional proof
irrelevance. The first one – dependent pattern matching – can be
translated directly to rewrite rules in Dedukti. However, the two
latter features – eta equality and irrelevance – rely on Agda’s
type-directed conversion algorithm, while Dedukti’s conversion is
untyped. Hence in order to translate Agda proofs to Dedukti these
features need to be encoded.

One particular concern with the encoding of eta-equality is that in
general it requires storing of additional type information in the
proof terms. It can hence lead to a large blow-up in the size of those
proof terms, and thus greatly increase the cost of typechecking. The
same problem also occurs in other parts of Agda; for example
constructors of parametrized datatypes do not store the values of the
parameters, but they need to be reconstructed in the translation to
Dedukti. We plan to investigate two possible approaches to this
problem: either we can try to find a better encoding which reduces the
size of the type annotation, or alternatively we can extend the
Dedukti language with type-directed conversion rules to render the
type annotations unneccessary.

\item Another unique feature of Agda is the support for first-class
universe level polymorphism. In particular, Agda has a built-in type
of levels that has complex structure of (in)equality between
levels. Compared to universe polymorphism in Coq, an additional
challenge is that levels in Agda can contain arbitrary terms as
subexpressions. Our plan is to define a sound and complete embedding
of Agda’s level type in Dedukti, based on the existing work on
encoding AC (associative-commutative) theories. This would both serve
as a stress test of how well Dedukti can handle complex equational
theories, and improve our understanding of type theories with
first-class universe level polymorphism, which would be useful for the
implementation of Agda.

\item In contrast to Coq and Lean, Agda does not have a well-defined
core language to which proofs are elaborated. Instead, definitions are
translated to an internal representation that is relatively close to
the user input. This provides a challenge when translating Agda proofs
to Dedukti: each feature in Agda’s internal syntax needs to have its
own translation. As part of this project, we will hence investigate
possible designs for a core language for Agda. Having such a core
language would have several benefits: it would deepen our
understanding of the Agda language, it would increase the
trustworthiness of Agda proofs, and it would make it much easier to
export Agda terms to other languages (such as Dedukti in the context
of this project).

\item Agda provides an experimental option for extending the language
with user-defined rewrite rules, which are very similar to the rewrite
rules provided by Dedukti. Because of this similarity, we expect it to
be straightforward to translate rewrite rules from Agda to
Dedukti. However, by comparing the two implementations we hope to gain
new insights and find opportunities for improvement on both sides. The
interest of some of these features goes beyond just the Agda
language. In particular, Lean also supports definitional proof
irrelevance, as does Coq with the recent addition of the SProp
universe. Hence we plan to collaborate with the teams working on those
languages to improve the support for these features where there is
overlap.
\end{enumerate}

%%% Local Variables:
%%% mode: latex
%%% TeX-master: "../propB"
%%% End:

\end{task}

\begin{task}[id=minlog,title=Instrument Minlog]
%[LMU München]

\begin{enumerate}
\item Further develop and implement extensional realisability in Minlog: this is
  a necessary first step for bridging Minlog and Dedukti.
\item Express the core Minlog logic and proofs in Dedukti: this task is
  facilitated by the fact that both systems share proof terms and deduction
  modulo. The preliminary work on realisability is required for exporting
  programme extraction to Dedukti.
\item Properly encode coinduction and corecursion in Dedukti: these concepts are
  fundamental to Minlog, for example for representing real numbers as streams of
  digits, but they are not native to Dedukti.
\item Import a subset of Dedukti into Minlog, apply programme development by proof
  transformation, and export back. This will make Dedukti a usable tool for the
  development of proofs and programmes in constructive analysis and allow Minlog
  users to benefit from theories formalised using different proof assistants.
\end{enumerate}


%%%%%%%%% OLD TEXT %%%%
% The \href{http://minlog-system.de}{Minlog system} implements a theory of
% computable functionals (TCF) \cite{SchwichtenbergWainer12}.
% It is a form of higher order arithmetic where partial functionals are
% first-class citizens.

% The intended model of TCF is the Scott-Ershov model of partial
% continuous functionals \cite{Ershov77}. Computable functionals are defined
% by so-called computation rules, a form of (possibly non-terminating)
% defining equations understood as left-to-right conversion rules.  An
% important example is the corecursion operator, which is needed to
% define functions operating for instance on streams of signed digits (a
% convenient format to represent real numbers).  The logical framework
% allows to declare a proven equality as a rewrite rule.  Now it is
% tempting to identify two terms or formulas when they have the same
% normal form w.r.t. rewriting (including of course beta-conversion);
% this is often called deduction modulo rewriting.  However, in a setup
% like TCF where non-termination is allowed we cannot use normal forms,
% but we can consider two terms or formulas as identical when they have
% a common reduct.  This drastically simplifies proofs involving real
% number arithmetic.

% Another central feature of TCF (and hence the Minlog system) is that
% it internalizes a proof-theoretic realizability interpretation (in the
% form of Kreisel's so-called modified realizability, with realizers of
% higher type).  More precisely, for every (co)inductive predicate we
% have another one with one argument more, denoting a realizer.  It is
% important that this realizers is expressed in the term language of TCF
% (an extension of G\"odel's system $T$).  Since a realizer can be seen as
% a programme representing the computational content of a constructive
% existence proof (expressing that a certain specification has a
% solution), we now can reason about such programs in a formal way,
% inside TCF.  In fact, given a proof M in TCF of a specification
% $\forall x\exists y A(x,y)$ we can extract a term (program) $p_M$ and automatically
% generate a new proof of $\forall x A(p_M(x))$.  In other words, for
% programs generated in this way from existence proofs, formal
% verification is automatic.  Note that for a proof involving
% coinduction the extracted term contains the (non terminating)
% corecursion operator.  Of course, for efficient evaluation in a second
% step we want to translate our extracted term into an efficient
% (functional) programming language like Haskell.

% Other aspects of Minlog are more common.  It is a proof system in
% Gentzen-style natural deduction based on proof terms
% (the so-called Curry-Howard correspondence).  We distinguish between
% (co)inductive predicates with and without computational content; in
% fact, computational content only arises from (co)inductive predicates
% marked as computationally relevant (c.r.).  In particular, both
% universal und existential quantifiers do not influence computational
% content: one needs to relativize them to c.r. predicates (like
% totality) to make them computationally relevant.

% A central application area of Minlog is to formalize Bishop-style
% \cite{Bishop67} constructive analysis and extract interesting algorithms
% from proofs. An example is the Intermediate Value Theorem treated in
% \cite{LindstroemPalmgrenSegerbergStoltenberg08}.  More recently we have
% extracted algorithms operating on (both signed digit and Gray-coded)
% stream-represented real numbers from proofs which never mention
% streams.  They come in by relativizing real number quantifiers to
% appropriate coinductive predicates \cite{Berger09}.
% We will extend this
% work further into constructive analysis, e.g. Euler's existence
% proof of solutions of ordinary differential equations (ODE).


\end{task}

\begin{task}[id=isabelle,title=Instrument Isabelle]
% task leader: Tobias

% participants: Makarius (Isabelle), David Matthews (Poly/ML)

% Moved to concept & methodology
% Isabelle as a logical framework \cite{paulson700} is an intermediate
% between Type-Theory provers (like Coq or Agda) and classic LCF-style
% systems (like HOL Light or HOL4). The inference kernel can already
% output proofs as $\lambda$-terms on request, but this has so far been
% only used for small examples \cite{Berghofer-Nipkow:2000:TPHOL}. The
% challenge is to make Isabelle proof terms work robustly for the basic
% libraries and reasonably big applications.  Preliminary work by Wenzel
% (2019) has demonstrated the feasibility for relatively small parts of
% Isabelle/HOL, but this requires scaling up.

\begin{enumerate}
  \item Improve the efficiency of important aspects of the
  Isabelle/HOL logic implementation, such as normalization of proofs,
  type-class reasoning, and special representation of derived rules
  and definition principles.
  \item Reduce the volume of proof terms in the Dedukti encoding.
  \item Improve memory usage of the Isabelle/ML implementation
  platform.
\end{enumerate}

\end{task}

\begin{task}[id=HOL4,title=Instrument HOL4]
[G\"oteborg]

The HOL4 proof assistant is home to a few medium to large scale
specifications and associated proof developments that have value
outside of HOL4. These specifications include the formal semantics of
the CakeML language (and its verified compiler) and an extensive
specification of the ARM instruction set architecture (ISA) as
formalised by Anthony Fox at the University of Cambridge.

HOL4 has support for exporting proofs to the OpenTheory proof exchange
format, and there has been some work on importing OpenTheory proofs
into Dedukti. However, the current state of these techniques and their
implementations does not scale to real examples such as those
mentioned above.

This part of the project will be about re-thinking and re-designing
the tools HOL4-to-OpenTheory and OpenTheory-to-Dedukti tools such that
they scale to the point where real examples of interest, such as those
mentioned above, can be exported.

2 Person Years at Chalmers


 \end{task}

\begin{task}[id=atelier-b,title=Instrument Atelier-B]
\ednote{Southhampton, Toulouse, Clearsy writes this}
\end{task}

\begin{task}[id=matita,title=Integrate the Matita translator in Matita itself]
\ednote{Bologna}
\end{task}
\end{tasklist}

\begin{wpdelivs}
  \begin{wpdeliv}[due=3,miles=startup,id=requirements,dissem=PU,nature=DEM,lead=Inr]
      {Requirements Analysis and Synchronization}
\end{wpdeliv}
\end{wpdelivs}
\end{workpackage}

%%% Local Variables:
%%% mode: latex
%%% TeX-master: "../propB"
%%% End:
