% task leader: Catherine

% other participants: 

\ednote{Southhampton, Toulouse, Clearsy writes this}

Atelier B, Rodin and ProB are platforms or tools to develop models
written in B method, Event-B or B system. The development process is
based on formal proof: proof obligations are automatically  generated
and must be proven by automatic or interactive provers. ProB is an
animator and model checker, it helps users to gain confidence in their
specifications. It is also a disprover aiming at  discovering
counter-examples for proof obligations. Atelier B and Rodin use native
B proof tool, they also enable the use of external provers such as SMT
solvers. ProB calls SMT and SAT solvers, it also uses contraints
solvers such as Sicstus Prolog. All of them relies on the B logics,
mainly a first order language with set theory. Regarding B/Event-B/B
system, there are some variants, mainly regarding the refinement
process they all implement. Refinement means that models are developed
by successive steps, from an abstract model to a more  concrete
model. Refinement in B method mainly means deriving a program while
EventB and B System refinement aim at defining a model of a system by
introducing details.


In the context of the BWare project, an encoding of the set theory of
the B method has been provided as a theory modulo, i.e. a rewrite
system rather than a set of axioms. This encoding is used by the
automatic prover Zenon modulo which features a backend to
Dedukti. Thus, as a first step through instrumentation of Atelier B
and Rodin, proof obligations coming from Atelier B can be proved by
Zenon modulo producing Dedukti proofs, hence providing a better
confidence in the proofs produced by the native proof tools of Atelier
B \cite{Bware}.


This work package has different objectives:
\begin{itemize}
\item continue the encoding
of the B set theory in Dedukti to be able to handle all kind of proof
obligations,
\item instrument the native provers to produce proofs (see
WP4) and
\item export B models.
\end{itemize}

The last objective, the exportation of  B models,
is a major task, involving research activities. A model in B method,
EventB or B System encodes a state machine constrained by
invariant properties. Verification of the model correctness implies to
verify some proof obligations produced by a weakest precondition
calculus. For example, spanning trees algorithms, distributed
algorithms, access control policies have been formalized
respectivement in EventB and B method. Besides its proof obligations,
the entire model  is worth
being exported, reused or translated 


