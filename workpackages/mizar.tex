%[Białystok,Innsbruck]

\begin{enumerate}
\item Express the foundations of Mizar in Dedukti: Mizar is based on
  Tarski-Grothendieck set theory encoded in first-order logic. The first challenge
  is to represent this foundations together with Mizar's soft type system
  in $\lambda\Pi$-calculus modulo theory in a way that takes advantage of Dedukti's
  rewriting capabilities to automate parts of type inference.
\item Instrument Mizar to export type disambiguation data: Mizar offers a soft
  type system. Exporting and using the information present in the Mizar types
  will enable us to optimize the representation of Mizar statements in Dedukti.
\item Express the Mizar checker equality checking and unification steps: the
  algorithms underlying the Mizar checker must be expressed as a mix of small
  proof steps and rewrite rules so that Dedukti's proof kernel can verify them.
  This requires exporting semantic information for proof obligations that is not
  currently available outside of the Mizar checker and will enable us to check
  the basis of the Mizar library and make it available in Logipedia.
\end{enumerate}

%%%%%%% OLD TEXT %%%%
% Mizar~\cite{bancerek:mizar2015} is one of the earliest proof assistants. 
% It was initially created as a typesetting system for mathematics with proof
% checking functionality added later. Mizar relies on a soft type system which
% is used to specify set theoretic foundations. The library developed together
% with the system--the Mizar Mathematical Library (MML)~\cite{bancerek:mml2017}--is
% until today one of the largest libraries of formal mathematics with a number
% of results not present in other systems.

% There are multiple reasons why checking Mizar proofs in Dedukti will be one of
% the very challenging tasks in this proposal. First, the Mizar language is close
% to the standard mathematical language and as such its syntax is very far from
% the languages of most other proof assistants. Second, the semantics of a Mizar
% proof step correspond to the research on the notion of ``obviousness'' of a single
% proof step to a human mathematician. This means that Mizar proofs do not
% reference proof procedures or tactics and a lot of background knowledge is used
% implicitly. Finally, the Mizar proof system has not been originally designed to save
% any proof objects.

% \begin{enumerate}
% \item Instrument the Mizar proof checker to export more detailed proof objects \ednote{Białystok, major effort}

% \item We will express the Mizar foundations in Dedukti. This means expressing the
%   Mizar soft type system, the initial set theory, as well as Mizar structures and
%   comprehensions. Recently, we have expressed major parts of Mizar in the logical
%   framework Isabelle~\cite{ckkp:isabellemizar2019}, focusing on preserving as much
%   of the syntax as possible. Here the work will focus on being able to check all
%   the proofs instead. 

% \item We will develop a soft type checking system in the style of Mizar on the level
%   of Dedukti. We expect that the whole algorithm can be expressed as a rewrite system.
%   However, as the algorithm has been so far expressed as an imperative program this will
%   require significant work.

% \item Develop Dedukti techniques corresponding to Mizar proof checking \ednote{Białystok, major effort}.
% \end{enumerate}

