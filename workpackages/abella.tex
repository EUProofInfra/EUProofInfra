\begin{enumerate}
\item Encoding proofs on finite structures: model checking queries supported by
  Abella's Bedwyr prover rely on the exhaustive exploration of finite structures
  and support quantifier alternation. Representing such proofs in Dedukti
  involves alternating phases of deduction and computation. A particular
  challenge is handling backtracking search, which is different from Dedukti's
  notion of computation as confluent rewriting.
\item Encoding cyclic proofs: Abella's implementation of (co-)induction is based
  on cyclic reasoning with size annotated relations. A representation in Dedukti
  requires extracting explicit invariants from cyclic proofs.
\item Equality and unification: Abella's equality proofs examine complete sets
  of unifiers for terms assumed to be equal, based on a trusted unification
  engine. The unification procedure can be recast as a rewrite system in
  Dedukti, but it remains to derive facts from the unifiability of terms.
\item Support $\lambda$-tree syntax: an encoding of Abella in Dedukti requires
  $\lambda$-terms to be considered modulo $\alpha\beta\eta$-equivalence, which
  is not natively supported. Moreover, for reasoning about the inductive
  structure of terms, Abella provides the $\nabla$-quantifier, which provides a
  challenge for representing the full logic in Dedukti.  
\end{enumerate}

%%%%% OLD TEXT %%%%%
% The usual approach to capturing either Peano and Heyting arithmetics
% is to use various axioms (and an axiom scheme for induction) on top of
% classical and intuitionistic first-order logic.  Indeed, this is the
% approach used in the Dedukti proof checker.


% A different approach to encoding arithmetic has been developed over
% the past 20--30 years, starting with papers by Schroeder-Heister and
% Girard in the early 1990s and extended in a series of papers by
% Baelde, Gacek, McDowell, M, Momigliano, Nadathur, and Tiu.  In this
% new setting, first-order logic is extended by considering both
% equality and the least fixed point operator as \emph{logical
%   connectives}: these logical connectives are not available directly
% in Dedukti.

% This new foundations for arithmetic has been implemented in two
% systems: the automated Bedwyr prover and the interactive Abella
% prover.  While neither Bedwyr nor Abella are as popular as many of the
% theorem provers that are covered by this proposal, there are two
% important reasons to consider incorporating them into the Logipedia
% effort.

% First, the Bedwyr prover is capable of constructing proofs for the
% kind of queries that are part of \emph{model checkers}.  This class of
% provers has not yet been incorporated into Dedukti.  The
% proof-theoretic work behind model checking in Bedwyr should provide
% some of the insights needed for allowing Dedukti to proof check the
% results of model checkers.

% Second, Bedwyr and Abella provide for direct and elegant support of
% meta-level reasoning.  Given that the foundations for Bedwyr and
% Abella have been given using Gentzen's sequent calculus, it was
% possible to enrich their foundations to allow for the treatment of
% binding structures within terms.  As a result, it is possible to
% reason directly on terms representing $\lambda$-terms and
% $\pi$-calculus expressions.  In particular, the Abella prover has
% probably the most natural and compact formal treatment of the
% $\pi$-calculus and its meta-theory when compared to all other attempts
% in any other theorem provers.  More generally, the Abella prover
% should be able to treat the meta-theory of programming and
% specification languages as well as various logics and their
% proofs. While these tasks are not the typical tasks considered by the
% majority of theorem provers within the scope of this proposal,
% meta-theory results do play an important role at times: in fact, the
% ultimate questions as to whether or not a proof checkers (such as that
% used by Dedukti) is correct or not will involve meta-theoretic
% questions.

% We propose to work on the general problem of exporting proofs from
% Abella to Dedukti.  (Since all proofs that are constructed
% automatically via Bedwyr can also be constructed manually within
% Abella, we shall limit our discussion below to Abella only.)  The
% proposed work will serve not only to answer the question of how to
% relate these two different foundations for arithmetic but also to
% allow Abella's particular style of proofs to find applications in the
% wider world of formalized proofs.

% The general problem described above has the following constituent parts.

% (1) Proofs involving searching finite structures. Proofs built for
% model checking problems over finite structures have two different
% kinds of phases.  To illustrate, consider trying to find a specific
% node within a binary tree.  If such a node exists, then the proof
% essentially encodes the path to the node in the tree.  If, however, no
% such node exists, then the proof of that negative fact is essentially
% a computation that exhaustively explores the tree.  Using the Dedukti
% terminology: in the first case, the proof involves several deduction
% steps, while in the second case, the proof involves a pure
% computation. When dealing with model checking problems such as
% simulation (in concurrency theory) and winning strategies (in game
% theory), proofs will involve alternating phases involving either
% deduction or computation.  Since the notion of computation in
% Abella-style proofs involves backtracking search, that style
% computation will be quite different from Dedukti's notion of
% computation as confluent rewriting.

% (2) Extending model checking problems to the general case of infinite
% structures and the associated inductive reasoning methods. Although
% the formal basis of Abella uses least and greatest fixed-point
% combinators and explicit (co-)invariants, the Abella implementation of
% (co-)induction is based on cyclic reasoning using size-annotated
% relations. It is known, in principle, how to convert cyclic proofs
% using annotations to proofs with explicit invariants, but an invariant
% extraction procedure that works in all cases is still missing. Once
% such invariants are available, incorporating them into Dedukti should
% be straightforward in association with part (1).

% (3) Binding structures. Abella, as well as several other computational
% logic systems ($\lambda$Prolog, Isabelle/Pure, Twelf, Beluga, etc)
% make use of the so-called \emph{$\lambda$-tree syntax} (a form of
% \emph{higher-order abstract syntax}, HOAS) approach to represent
% bindings. This approach is further enriched in Abella with the
% $\nabla$-quantifier that allows inductive and co-inductive properties
% to be defined based on the \emph{structure} of $\lambda$-terms. We
% propose to examine encodings of $\lambda$-tree syntax in Dedukti. The
% best approach probably involves extending the underlying theory of
% Dedukti with a quantifier similar to Abella's $\nabla$-quantifier.

% (4) Reflective treatment of unification. One of the features of
% Abella's style of proofs is the use of left-introduction rules for
% equality that exhaustively examine complete sets of unifiers for
% $\lambda$-terms. This is implemented in terms of a unification engine
% that is currently a trusted black box, which complicates any proposal
% for exporting proofs to different implementations of unification or
% equality. In Dedukti the unification procedure can be recast as a
% rewrite system, but it is unclear how to derive reflective properties
% based on the unifiability of terms.
