\begin{workpackage}[id=dissemination,wphases=0-48,type=MGT,
  short=Dissemination,% for Figure 5.
  title={Dissemination, communication, and exploitation},
  lead=Inr,
  BolRM=3,
  InrRM=10]
  
\begin{wpobjectives}
  The objective of this work package is to \ldots

This includes notably:
  \begin{compactitem}
  \item \ldots
  \end{compactitem}
  A key aspect will be to foster \ldots
\end{wpobjectives}

\begin{wpdescription}
  \ednote{MK: I am not sure that this is type  MGT, please check}
  \ednote{Gilles will write} \ednote{MK: it is probably a good idea to copy from
    OpenDreamKit: see
    \url{https://github.com/OpenDreamKit/OpenDreamKit/blob/master/Proposal/WorkPackages/DisseminationCommunityBuilding.tex}}
\end{wpdescription}

\begin{tasklist}
  \begin{task}[id=com,title=Communication]
    We will setup a small group in charge of setting up communication
    tools targeting the different users of Logipedia: researchers,
    engineers, certification authorities, universities, teachers,
    publishers, etc. (web site, videos, mailing, etc.)
  \end{task}

  \begin{task}[id=schools,title=Training Logipedia users]
    We will organize schools targeted to the different Logipedia users:
    \begin{itemize}
    \item Master and PhD students: training on formal proofs, proof
      translation, and automated theorem proving. This is an
      opportunity to improve the gender balance in our community by
      attracting female students.
    \item Engineers and certifiers: training on formal proofs tools.
    \item Teachers: training on the use of proof assistants and formal
      proofs in education.
    \end{itemize}
  \end{task}

  \begin{task}[id=research,title=Expanding the use of Logipedia in research]
    We will create and animate of club of researchers using Logipedia
    in their work. We will invite researchers on formal methods,
    theoretical computer science or mathematics to join this
    club. This club will be an opportunity for them to do:
    \begin{itemize}
    \item Provide some feedback on the use of Logipedia in their work.
    \item Express their needs wrt Logipedia.
    \item Provide use cases.
    \end{itemize}
  \end{task}

  \begin{task}[id=industry,title=Expanding the use of Logipedia in the industry]
    We will create and animate a club of industrial users. We will
    invite companies working on or using formal methods to join this
    club. This club will be an opportunity for them to do:
    \begin{itemize}
    \item Technology watch.
    \item Support the development of Logipedia.
    \item Express their needs wrt proof standards and proof tools.
    \item Provide use cases.
    \end{itemize}
    A meeting will be organized every year to present the advancement
    of Logipedia and discuss the use of Logipedia in the industry.
  \end{task}

  \begin{task}[id=certif,title=Expanding the use of Logipedia within certification authorities]
    We will propose to certification authorities to use the tools
    developed for Logipedia for checking some formal proofs submitted
    to them. Preliminary contact with ANSSI. Extension of other
    European certification authorities?
  \end{task}

  \begin{task}[id=education,title=Expanding the use of Logipedia in education]
    We will create and animate a club of users of formal proofs in
    education. This club will be an opportunity to:
    \begin{itemize}
    \item Express the needs of students and teachers wrt proof
      presentations and tools in Logipedia.
    \item Exchange their experience in teaching formal proofs or using
      theorem provers in class.
    \item Provide use cases.
    \end{itemize}
  \end{task}

  \begin{task}[id=publishers,title=Expanding the use of Logipedia in publishing]
    We will create and animate a club of publishers. We will invite
    people and organization working in the publication of research
    works to join this club: publication archives (arXiv, CCSD-HAL,
    Lipics, ACM, zbMath, etc.), conference steering committees (CPP,
    ITP, POPL, CICM, etc.), journal editorial boards (JFR, JFM,
    etc.). It will be an opportunity to:
    \begin{itemize}
    \item Discuss with them how to use Logipedia to store and check
      proofs presented or used in scientific publications.
    \item Express their needs wrt Logipedia.
    \end{itemize}
  \end{task}
\end{tasklist}

\begin{wpdelivs}
  \begin{wpdeliv}[due=3,miles=startup,id=requirements,dissem=PU,nature=DEM,lead=Inr]
      {Requirements Analysis and Synchronization}
\end{wpdeliv}
\end{wpdelivs}
\end{workpackage}


%%% Local Variables:
%%% mode: latex
%%% TeX-master: "../propB"
%%% End:
