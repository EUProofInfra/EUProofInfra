\begin{workpackage}[id=dissemination,type=MGT,
  short={Dissemination, communication and exploitation},
  title={Dissemination, communication and exploitation},
  lead=Inr]

\begin{wpobjectives}
  The objective of this work package is to \ldots

  Scientific research and education at all levels are concerned with
  the discovery, verification, communi- cation, archival and usage of
  mathematical results. These tasks have been supported by physical
  books, conferences and other means.  The avaibility of a formal
  online encyclopedia which propose in a single place the
  communication, archival and verification of mathematical knowledge
  will be of prime importance for researchers, industrials, teachers
  and editors.

  This includes notably:
  \begin{compactitem}
  \item \ldots
  \end{compactitem}
  A key aspect will be to foster \ldots
\end{wpobjectives}

\begin{wpdescription}
  \ednote{MK: I am not sure that this is type  MGT, please check}
  \ednote{Gilles will write} \ednote{MK: it is probably a good idea to copy from
    OpenDreamKit: see
    \url{https://github.com/OpenDreamKit/OpenDreamKit/blob/master/Proposal/WorkPackages/DisseminationCommunityBuilding.tex}}
  \ednote{StS: From my limited experience, reviewers like concrete,
    peer-reviewed publications as part of the output. Should we have
    an ongoing task \emph{Publishing Project Results} or comparable?}
\end{wpdescription}

\begin{tasklist}

  %%%%%%%%%%%%%%%%%%%%%%%%%%%%%%%%%%%%%%%%%%%%%%%%%%%%%%%%%%%%%%%%%%%%%%%%%%%%
  \begin{task}[id=com,
      title=Communication,
      lead=Inr,InrRM=4]
    We will setup a small group in charge of setting up communication
    tools targeting the different users of Logipedia: researchers,
    engineers, certification authorities, universities, teachers,
    publishers, etc. (web site, videos, mailing, etc.)
  \end{task}

  %%%%%%%%%%%%%%%%%%%%%%%%%%%%%%%%%%%%%%%%%%%%%%%%%%%%%%%%%%%%%%%%%%%%%%%%%%%%
  \begin{task}[id=training,
      title=Training Logipedia developers and users,
      lead=Inr,InrRM=2,IrtRM=2]
%     The development and growth of Logipedia depends on two separate factors:
%     \begin{compactenum}
%      \item The development and maintenance of computer-checked proofs in the computer proof assistants supported by Dedukti as part of the Logipedia project.
%      \item The development and maintenance of import and export mechanisms for libraries of computer-checked proofs.
%     \end{compactenum}
%     For each of the two items,
    We are planning to provide suitable training
    for people at various stages of their career:

%     We will organize schools targeted to the different Logipedia users:
    \begin{itemize}
    \item [\textbf{Master and PhD students}]
      We will support the planning and running of schools
      on various computer proof assistants.
      We are also planning to run a school dedicated to the topic of Logipedia, on the translation of proofs between different systems.
%       training on formal proofs, proof
%       translation, and automated theorem proving.
      Both activities form an
      opportunity to improve the gender balance in our community by
      attracting female students.
    \item [\textbf{Engineers and certifiers}] training on formal proofs tools.
    \item [\textbf{Teachers}]
       We will set up a forum for exchange of tipps and tricks for the use of computer proof assistants and formal proofs in teaching such classes.
       Particular emphasis will be on the sharing of teaching materials, which we will encourage to make available under free licenses.
    \end{itemize}

%     \textbf{Deliverables:}
%     \begin{itemize}
%      \item UniMath school 2022
%      \item Logipedia school 2023
%      \item An electronic forum for exchanging on the use of computer proof assistants in teaching and learning.
%     \end{itemize}

  \end{task}

  %%%%%%%%%%%%%%%%%%%%%%%%%%%%%%%%%%%%%%%%%%%%%%%%%%%%%%%%%%%%%%%%%%%%%%%%%%%%
  \begin{task}[id=researchers-club,
      title=Expanding the use of Logipedia in research,
      lead=Bir,BirRM=2]
%     The purpose of this task is to ensure a tight interaction, and to establish a feedback loop, between, on the one hand, the developers and, on the other hand, the intended users of Logipedia.
%     Specifically, we aim to
%     \begin{compactenum}
%      \item Provide support to researchers who want to use Logipedia in their work.
%      \item Establish a forum for researchers to describe their needs, to give feedback on Logipedia, and to request features.
%      \item Build up a community of researchers, with regular exchange of ideas and use cases.
%     \end{compactenum}
     To foster interaction between developers and users of Logipedia, we plan the following activities:
    \begin{compactenum}
     \item [\textbf{Logipedia helpdesk}]
    we will establish a ``Logipedia helpdesk'', as a unique, and easy-to-reach point of contact for any researcher seeking help on the use of Logipedia.
     The helpdesk will forward any incoming request for help to a suitable expert among the Logipedia developers.
     \item [\textbf{User day at Logipedia meeting}]
     We plan to have a ``user day'' attached to our yearly Logipedia meetings.
     During the user day, we plan to have both talks on uses of Logipedia, as well as a round table discussion featuring both developers and users, for discussing present problems and future challenges.
     \item [\textbf{Electronic seminar series}]
     We will run an electronic seminar series, thus providing an ecological and economical way of presenting one's work and giving feedback.
    \end{compactenum}
%
%     \textbf{Deliverables:}
%     \begin{itemize}
%      \item Helpdesk
%      \item User days
%      \item Electronic seminar series
%     \end{itemize}

%     We will create and animate of club of researchers using Logipedia
%     in their work. We will invite researchers on formal methods,
%     theoretical computer science or mathematics to join this
%     club. This club will be an opportunity for them to do:
%     \begin{itemize}
%     \item Provide some feedback on the use of Logipedia in their work.
%     \item Express their needs wrt Logipedia.
%     \item Provide use cases.
%     \end{itemize}
  \end{task}

  %%%%%%%%%%%%%%%%%%%%%%%%%%%%%%%%%%%%%%%%%%%%%%%%%%%%%%%%%%%%%%%%%%%%%%%%%%%%
  % \begin{task}[id=industrial-club,
  %     title=Expanding the use of Logipedia in the industry,
  %     lead=Irt,IrtRM=2]
  %   We will create and animate a club of industrial users. We will
  %   invite companies working on or using formal methods to join this
  %   club. This club will be an opportunity for them to do:
  %   \begin{itemize}
  %   \item Technology watch.
  %   \item Support the development of Logipedia.
  %   \item Express their needs wrt proof standards and proof tools.
  %   \item Provide use cases.
  %   \end{itemize}
  %   A meeting will be organized every year to present the advancement
  %   of Logipedia and discuss the use of Logipedia in the industry.
  % \end{task}


  \begin{task}[id=industrial-club,
    title=Expanding the use of Logipedia in the industry,
    lead=Irt,IrtRM=2]
    % The use of formal proof tools by companies is still limited, in particular by SMEs/SMIs. The activities of the industrial club must allow the results of other WPs to be reused, in a form adapted to the variety of the involved industrial partners and their level of prior knowledge of formal proof tools. Logipedia deliverables must be shared with the industrial club in a form that facilitates operational deployment, drawing inspiration from practical cases. It will also be wise to deploy training engineering proposals dedicated to various industrial audiences by favoring a "learn by doing" approach.
    The use of formal proof tools by companies is still limited, in particular by SMEs/SMIs. The club will be an opportunity to:
    \begin{itemize}
    \item Develop professional training and support offers for employees who respond to the evolution of the productive tool.
    \item Make significant changes to existing training offers.
    \item Propose meetings, create actions and shared services between large company(ies) and PMEs/SMIs.
    \item Develop pedagogic innovations in the delivery of training, in support of companies
    \item Share industrial experiences during dedicated meetings
    \end{itemize}
  \end{task}



  %%%%%%%%%%%%%%%%%%%%%%%%%%%%%%%%%%%%%%%%%%%%%%%%%%%%%%%%%%%%%%%%%%%%%%%%%%%%
  \begin{task}[id=certif-club,
      title=Expanding the use of Logipedia within certification authorities,
      lead=Imt,ImtRM=2]
    We will propose to certification authorities to use the tools
    developed for Logipedia for checking some formal proofs submitted
    to them. Preliminary contact with ANSSI. Extension of other
    European certification authorities?
  \end{task}

  %%%%%%%%%%%%%%%%%%%%%%%%%%%%%%%%%%%%%%%%%%%%%%%%%%%%%%%%%%%%%%%%%%%%%%%%%%%%
  \begin{task}[id=teachers-club,
      title=Expanding the use of Logipedia in education,
      lead=Str,StrRM=2] The club of users in education gathers
    teachers who are already actively using formal proof for teaching
    in computer science, mathematics and logic but are not necessarily
    members of the cummunity of researchers in formal theorem
    proving. These early adopters, will provide continuous feedback to
    the project members on the usuability and accessibility of the
    system from this particular point of view.  More than 30 persons
    have already accepted to take part in this club.

%    Along the standard intuitive descriptions of theorem statements
%    and proofs, having a formal description is crucial in the
%    education. Indeed, students are often faced to the difficulty of
%    understanding mathematical concepts and proofs based on pieces of
%    informations found in heteregoneous sources (books, lecture notes,
%    \ldots) with varying definitions, notations. The informal proofs
%    are often given omitting details or relying on implicit knowledge
%    that students may not already be familiar with.

%    Moreover, devlopment of mathematics in education, is often based
%    not on the systematic development of mathematics in the style of
%    Bourbaki, but on so-called deductive islands: sets of local
%    assumptions used in a curriculum, a lecture or an exercise.  On
%    top of the coherent foundations provided by the base libraries
%    incorporated in Logipedia, we will need to build coherent
%    collection of mathematical results suitable for being used in a
%    given classroom.

    The role of the club of users of proof assistants in education will be to:
    \begin{compactitem}
    \item Express the needs of students and teachers wrt proof
      presentations and tools in Logipedia.
    \item Exchange their experience in teaching formal proofs or using
      theorem provers in class by participating in the ThEdu community.
    \item Provide use cases.
    \item Contribute to the dissemination of information about Logipedia.
    \end{compactitem}

  \end{task}

  %%%%%%%%%%%%%%%%%%%%%%%%%%%%%%%%%%%%%%%%%%%%%%%%%%%%%%%%%%%%%%%%%%%%%%%%%%%%
  \begin{task}[id=publishers-club,
      title=Expanding the use of Logipedia in publishing,
      lead=Zib,ZibRM=2]
    We will create and animate a club of publishers. We will invite
    people and organization working in the publication of research
    works to join this club: publication archives (arXiv, CCSD-HAL,
    Lipics, ACM, zbMath, etc.), conference steering committees (CPP,
    ITP, POPL, CICM, etc.), journal editorial boards (JFR, JFM,
    etc.). It will be an opportunity to:
    \begin{compactitem}
    \item Discuss with them how to use Logipedia to store and check
      proofs presented or used in scientific publications.
    \item Express their needs wrt Logipedia.
    \end{compactitem}
  \end{task}

  %%%%%%%%%%%%%%%%%%%%%%%%%%%%%%%%%%%%%%%%%%%%%%%%%%%%%%%%%%%%%%%%%%%%%%%%%%%%
  \begin{task}[id=zib,
      title=Linking scientific publications to Logipedia,
      lead=Zib,ZibRM=12]
    We will build a service for Logipedia which provides an overview
    about the development in logic and an intuitive access to the
    following classes of objects - axioms - theories/calculi -
    software (theorem provers) - applications based on logic tools
    (e.g. the metro in Paris, other SAT problems)

    The objects and the relations between the objects will be
    realized via a graph. We will do this in the following way:
    \begin{compactitem}
    \item Defining the objects of each class (we will start from the
      theorem provers, apply the publication-based approach for the
      software and search for references to axioms, theories, and
      applications).
    \item Definition of a special ontology to describe linking and
      dependencies between the objects (metadata).
    \item Add these metadata to all objects
    \item Graph visualization of the network (lattice) and development
      of a search engine.
    \end{compactitem}
    This gives the user a comprehensive entry point into an emerging
    topic.
  \end{task}

  %%%%%%%%%%%%%%%%%%%%%%%%%%%%%%%%%%%%%%%%%%%%%%%%%%%%%%%%%%%%%%%%%%%%%%%%%%%%%%
  \begin{task}[id=edukera,
      title=Web interface for doing proofs at school,
      lead=Edu,EduRM=12]
    We build an web application for the education community on top of Logipedia. It will
    enable students to solve exercices that requires a mathematical proof with a simple
    user interface.
    The development tasks are listed below:
    \begin{compactitem}
    \item Graphical web component to display a Logipedia proof (could be used in Logipedia website)
    \item Point-and-click interaction engine on top of Logipedia
    \item Interactive proof interface application (Web interface, connection to LMS, ...)
    \end{compactitem}
  \end{task}

\end{tasklist}

%%%%%%%%%%%%%%%%%%%%%%%%%%%%%%%%%%%%%%%%%%%%%%%%%%%%%%%%%%%%%%%%%%%%%%%%%%%%
\begin{wpdelivs}

  % deliverables for T1 communication

  \begin{wpdeliv}[due=1,miles=startup,id=requirements,dissem=PU,nature=DEC,lead=Inr]{Logipedia website}Initial version of the Logipedia website
  \end{wpdeliv}

  % deliverables for T2 training

  \begin{wpdeliv}[due=18,miles=???,id=continuoused,dissem=PU,nature=R,lead=Str]{
 Report on the organization of continuous education sessions for teachers.}
  \end{wpdeliv}

  \begin{wpdeliv}[due=48,miles=???,id=school-researchers,dissem=PU,nature=R,lead=Bir]{Report on the school on the use of Logipedia targeting researchers and teachers}
  \end{wpdeliv}

%   \begin{wpdeliv}[due=12,miles=???,id=school-first-phd,dissem=PU,nature=other,lead=Bir]{First school on the foundations and development of Logipedia for our PhD students and postdocs}
%   \end{wpdeliv}

  \begin{wpdeliv}[due=24,miles=???,id=school-second-phd,dissem=PU,nature=R,lead=Bir]{Report on the two schools on the foundations and development of Logipedia for our PhD students and postdocs}
  \end{wpdeliv}

%   \begin{wpdeliv}[due=12,miles=???,id=school-first-certif,dissem=PU,nature=other,lead=Irt]{First school targeting engineers and certification authorities}
%   \end{wpdeliv}

  \begin{wpdeliv}[due=24,miles=???,id=school-second-certif,dissem=PU,nature=R,lead=Irt]{Report on the two schools targeting engineers and certification authorities}
  \end{wpdeliv}

  % edukera

  \begin{wpdeliv}[due=36,miles=???,id=edu-display,dissem=PU,nature=D,lead=Edu]{Web display a Logipedia Proof}
    Graphical web component to display a Logipedia proof
  \end{wpdeliv}

  \begin{wpdeliv}[due=36,miles=???,id=edu-app,dissem=PU,nature=OTHER,lead=Edu]{Web application for education based on Logipedia}
    Web interactive proof application for education
  \end{wpdeliv}

\end{wpdelivs}

\end{workpackage}


%%% Local Variables:
%%% mode: latex
%%% TeX-master: "../propB"
%%% mode: flyspell
%%% ispell-local-dictionary: "english"
%%% End:
