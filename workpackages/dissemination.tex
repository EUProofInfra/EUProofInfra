\begin{workpackage}[id=dissemination,wphases=0-48,type=MGT,
  short=Dissemination,% for Figure 5.
  title={Dissemination, communication and exploitation},
  lead=Inr]

\begin{wpobjectives}
  The objective of this work package is to \ldots

  Scientific research and education at all levels are concerned with
  the discovery, verification, communi- cation, archival and usage of
  mathematical results. These tasks have been supported by physical
  books, conferences and other means.  The avaibility of a formal
  online encyclopedia which propose in a single place the
  communication, archival and verification of mathematical knowledge
  will be of prime importance for researchers, industrials, teachers
  and editors.

  This includes notably:
  \begin{compactitem}
  \item \ldots
  \end{compactitem}
  A key aspect will be to foster \ldots
\end{wpobjectives}

\begin{wpdescription}
  \ednote{MK: I am not sure that this is type  MGT, please check}
  \ednote{Gilles will write} \ednote{MK: it is probably a good idea to copy from
    OpenDreamKit: see
    \url{https://github.com/OpenDreamKit/OpenDreamKit/blob/master/Proposal/WorkPackages/DisseminationCommunityBuilding.tex}}
  \ednote{StS: From my limited experience, reviewers like concrete,
    peer-reviewed publications as part of the output. Should we have
    an ongoing task \emph{Publishing Project Results} or comparable?}
\end{wpdescription}

\begin{tasklist}

  %%%%%%%%%%%%%%%%%%%%%%%%%%%%%%%%%%%%%%%%%%%%%%%%%%%%%%%%%%%%%%%%%%%%%%%%%%%%
  \begin{task}[id=com,
      title=Communication,
      lead=Inr,InrRM=4]
    We will setup a small group in charge of setting up communication
    tools targeting the different users of Logipedia: researchers,
    engineers, certification authorities, universities, teachers,
    publishers, etc. (web site, videos, mailing, etc.)
  \end{task}

  %%%%%%%%%%%%%%%%%%%%%%%%%%%%%%%%%%%%%%%%%%%%%%%%%%%%%%%%%%%%%%%%%%%%%%%%%%%%
  \begin{task}[id=training,
      title=Training Logipedia developers and users,
      lead=Inr,InrRM=2,IrtRM=2]
    The development and growth of Logipedia depends on two separate factors:
    \begin{compactenum}
     \item The development and maintenance of computer-checked proofs in the computer proof assistants supported by Dedukti as part of the Logipedia project.
     \item The development and maintenance of import and export mechanisms for libraries of computer-checked proofs.
    \end{compactenum}
    For each of the two items, we are planning to provide suitable training
    for people at various stages of their career:
    
%     We will organize schools targeted to the different Logipedia users:
    \begin{itemize}
    \item [\textbf{Master and PhD students:}] 
      We will support the planning and running of schools 
      on various computer proof assistants.
      One such school is Ahrens' School on Univalent Mathematics,
      which has taken place in 2017, 2019, and 2020.
      It trains participants in the use and development of the UniMath library of computer-checked mathematics, the import of which into Dedukti is 
      part of \taskref{theories}{hott}.
      We are going to provide financial and structural assistance in the running of this school in the future edition in 2022.
      We are also planning to run a school dedicated to the topic of Logipedia, on the translation of proofs between different systems.
      This school is planned for 2023.
%       training on formal proofs, proof
%       translation, and automated theorem proving. 
      Both activities form an
      opportunity to improve the gender balance in our community by
      attracting female students.
    \item [\textbf{Engineers and certifiers:}] training on formal proofs tools.
    \item [\textbf{Teachers:}] 
       Computer proof assistants can be used to teach a variety of 
       subjects, from undergraduate mathematics to logic and functional programming.
       We will set up a forum for exchange of tipps and tricks for the use of computer proof assistants and formal proofs in teaching such classes.
       Particular emphasis will be on the sharing of teaching materials, which we will encourage to make available under free licenses.
       In this way, best practices in teaching can be spread efficiently for the benefit of teachers and learners.
    \end{itemize}
    
    \textbf{Deliverables:}
    \begin{itemize}
     \item UniMath school 2022
     \item Logipedia school 2023
     \item An electronic forum for exchanging on the use of computer proof assistants in teaching and learning.
    \end{itemize}
   
  \end{task}

  %%%%%%%%%%%%%%%%%%%%%%%%%%%%%%%%%%%%%%%%%%%%%%%%%%%%%%%%%%%%%%%%%%%%%%%%%%%%
  \begin{task}[id=researchers-club,
      title=Expanding the use of Logipedia in research,
      lead=Bir,BirRM=2]
    The purpose of this task is to ensure a tight interaction, and to establish a feedback loop, between, on the one hand, the developers and, on the other hand, the intended users of Logipedia.
    Specifically, we aim to 
    \begin{compactenum}
     \item Provide support to researchers who want to use Logipedia in their work.
     \item Establish a forum for researchers to describe their needs, to give feedback on Logipedia, and to request features.
     \item Build up a community of researchers, with regular exchange of ideas and use cases.
    \end{compactenum}
    To implement these goals, we plan the following activities:
    \begin{compactenum}
     \item [\textbf{Logipedia helpdesk}] 
     Logipedia is a huge project involving many researchers on different sites. We anticipate that it will be difficult for someone seeking help with an aspect of Logipedia to find the right person to get in touch with.
     To solve this problem, we will establish a ``Logipedia helpdesk'', as a unique, and easy-to-reach point of contact for any researcher seeking help.
     The purpose of the helpdesk is not necessarily to provide answers, but to forward the request for help to a suitable expert.
     
     \item [\textbf{User day at Logipedia meeting}]
     It is important for the Logipedia project to understand the needs of the users of Logipedia in order to best serve the community.
     To gather feedback on the current status of Logipedia, and to discuss plans for future development, we plan to have a ``user day'' attached to our yearly Logipedia meetings. \ednote{Do we have such a meeting planned?}
     During the user day, we plan to have both talks on uses of Logipedia, as well as a round table discussion featuring both developers and users.
     During this discussion, present problems and future challenges will be discussed, and the audience will be encouraged to ask questions and provide feedback.
     \item [\textbf{Electronic seminar series}]
     To supplement the user day, and to provide a forum for discussion that does not rely on travel, we will run an electronic seminar series.
     Such seminar series are now well-established (see, e.g., the HoTTEST seminar series on Homotopy Type Theory, that has run for several years) and easy to set up. They allow everyone with an internet connection to participate, thus providing an ecological and economical way of presenting one's work and giving feedback. We plan one seminar every two to four weeks.
    \end{compactenum}

    \textbf{Deliverables:}
    \begin{itemize}
     \item Helpdesk
     \item User days
     \item Electronic seminar series
    \end{itemize}

%     We will create and animate of club of researchers using Logipedia
%     in their work. We will invite researchers on formal methods,
%     theoretical computer science or mathematics to join this
%     club. This club will be an opportunity for them to do:
%     \begin{itemize}
%     \item Provide some feedback on the use of Logipedia in their work.
%     \item Express their needs wrt Logipedia.
%     \item Provide use cases.
%     \end{itemize}
  \end{task}

  %%%%%%%%%%%%%%%%%%%%%%%%%%%%%%%%%%%%%%%%%%%%%%%%%%%%%%%%%%%%%%%%%%%%%%%%%%%%
  \begin{task}[id=industrial-club,
      title=Expanding the use of Logipedia in the industry,
      lead=Irt,IrtRM=2]
    We will create and animate a club of industrial users. We will
    invite companies working on or using formal methods to join this
    club. This club will be an opportunity for them to do:
    \begin{itemize}
    \item Technology watch.
    \item Support the development of Logipedia.
    \item Express their needs wrt proof standards and proof tools.
    \item Provide use cases.
    \end{itemize}
    A meeting will be organized every year to present the advancement
    of Logipedia and discuss the use of Logipedia in the industry.
  \end{task}

  %%%%%%%%%%%%%%%%%%%%%%%%%%%%%%%%%%%%%%%%%%%%%%%%%%%%%%%%%%%%%%%%%%%%%%%%%%%%
  \begin{task}[id=certif-club,
      title=Expanding the use of Logipedia within certification authorities,
      lead=Irt,IrtRM=2]
    We will propose to certification authorities to use the tools
    developed for Logipedia for checking some formal proofs submitted
    to them. Preliminary contact with ANSSI. Extension of other
    European certification authorities?
  \end{task}

  %%%%%%%%%%%%%%%%%%%%%%%%%%%%%%%%%%%%%%%%%%%%%%%%%%%%%%%%%%%%%%%%%%%%%%%%%%%%
  \begin{task}[id=teachers-club,
      title=Expanding the use of Logipedia in education,
      lead=Str,StrRM=2] The club of users in education gathers
    teachers who are already actively using formal proof for teaching
    in computer science, mathematics and logic but are not necessarily
    members of the cummunity of researchers in formal theorem
    proving. These early adopters, will provide continuous feedback to
    the project members on the usuability and accessibility of the
    system from this particular point of view.  More than 15 persons
    have already accepted to take part in this club.

    Along the standard intuitive descriptions of theorem statements
    and proofs, having a formal description is crucial in the
    education. Indeed, students are often faced to the difficulty of
    understanding mathematical concepts and proofs based on pieces of
    informations found in heteregoneous sources (books, lecture notes,
    \ldots) with varying definitions, notations. The informal proofs
    are often given omitting details or relying on implicit knowledge
    that students may not already be familiar with.

    Moreover, devlopment of mathematics in education, is often based
    not on the systematic development of mathematics in the style of
    Bourbaki, but on so-called deductive islands: sets of local
    assumptions used in a curriculum, a lecture or an exercise.  On
    top of the coherent foundations provided by the base libraries
    incorporated in Logipedia, we will need to build coherent
    collection of mathematical results suitable for being used in a
    given classroom.

    The role of the club of users of proof assistants in education will be to:
    \begin{compactitem}
    \item Express the needs of students and teachers wrt proof
      presentations and tools in Logipedia.
    \item Exchange their experience in teaching formal proofs or using
      theorem provers in class by participating in the ThEdu community.
    \item Provide use cases.
    \item Contribute to the dissemination of information about Logipedia.
    \end{compactitem}

  \end{task}

  %%%%%%%%%%%%%%%%%%%%%%%%%%%%%%%%%%%%%%%%%%%%%%%%%%%%%%%%%%%%%%%%%%%%%%%%%%%%
  \begin{task}[id=publishers-club,
      title=Expanding the use of Logipedia in publishing,
      lead=Zib,ZibRM=12]
    We will create and animate a club of publishers. We will invite
    people and organization working in the publication of research
    works to join this club: publication archives (arXiv, CCSD-HAL,
    Lipics, ACM, zbMath, etc.), conference steering committees (CPP,
    ITP, POPL, CICM, etc.), journal editorial boards (JFR, JFM,
    etc.). It will be an opportunity to:
    \begin{compactitem}
    \item Discuss with them how to use Logipedia to store and check
      proofs presented or used in scientific publications.
    \item Express their needs wrt Logipedia.
    \end{compactitem}
  \end{task}

  %%%%%%%%%%%%%%%%%%%%%%%%%%%%%%%%%%%%%%%%%%%%%%%%%%%%%%%%%%%%%%%%%%%%%%%%%%%%
  \begin{task}[id=zib,
      title=Expanding the use of Logipedia in publishing,
      lead=Zib,ZibRM=12]
    We will build a service for Logipedia which provides an overview
    about the development in logic and an intuitive access to the
    following classes of objects - axioms - theories/calculi -
    software (theorem provers) - applications based on logic tools
    (e.g. the metro in Paris, other SAT problems)

    The objects and the relations between the objects could be
    realized via a graph.  We could do this in the following way:
    \begin{compactitem}
    \item Defining the objects of each class (we could start from the
      theorem provers, apply the publication-based approach for the
      software and search for references to axioms, theories, and
      applications)
    \item Definition of a special ontology to describe linking and
      dependencies between the objects (metadata)
    \item add these metadata to all objects
    \item graph visualization of the network (lattice) and development
      of a search engine
    \end{compactitem}    
    This gives the user a comprehensive entry point into an emerging
    topic.
  \end{task}

  %%%%%%%%%%%%%%%%%%%%%%%%%%%%%%%%%%%%%%%%%%%%%%%%%%%%%%%%%%%%%%%%%%%%%%%%%%%%%%
  \begin{task}[id=edukera,
      title=Web interface for doing proofs at school,
      lead=Edu,EduRM=12]
      Current formal proof systems require to learn a specific formal
      language and a command-line interface to compile and execute the
      language.  This level of technicity is not compliant with mass
      adoption by teachers or students outside computer science. For
      educational purposes, it is therefore mandatory to develop an
      intuitive and easy-to-handle user-interface for the logipedia
      formal proof system. This interface, based on a WYSIWYG design
      (What You See Is What You Get), will provide two main features:
      \begin{compactitem}
      \item the structured display of the proof in a high (latex-like) quality
      \item the possibility to build the proof with simple
        point-and-click interactions
      \end{compactitem}

      The digital nature of the formal proof enables specific view features to
      understand its structure: eagle-eye view, folding/unfolding of scopes,
      highlighting the use of variables (on hover), showing/hiding context,
      and so on.

      Interactions are the point-and-click commands to develop the structure
      of the proof. They consist in applying theorems (or axioms or lemmas)
      to statements and rewriting rules to a selected element of a statement;
      this is done in deductive or abductive mode (resp. forward or backward).

      Theorems and rewrite rules should be presented and searched using
      technology developed in \WPref{dissemination}.

      The research of proof must be automated at some point in order to ease
      and speed up the process and to comply with the level of required detail
      in education; the output of \WPref{atpetc} will be used for this task.

      The development tasks are listed below:
      \begin{compactitem}
      \item graphical web component to display a Logipedia proof
      \item point-and-click interaction engine on top of Logipedia
      \item interactive proof interface application
      \end{compactitem}

      It will be possible to use the graphical proof component throughout
      the Logipedia website.
  \end{task}

\end{tasklist}

%%%%%%%%%%%%%%%%%%%%%%%%%%%%%%%%%%%%%%%%%%%%%%%%%%%%%%%%%%%%%%%%%%%%%%%%%%%%
\begin{wpdelivs}

  % deliverables for T1 communication
  
  \begin{wpdeliv}[due=1,miles=startup,id=requirements,dissem=PU,nature=DEC,lead=Inr]{Logipedia website}Initial version of the Logipedia website
  \end{wpdeliv}

  % deliverables for T2 training
  
  \begin{wpdeliv}[due=18,miles=???,id=summerschool,dissem=PU,nature=other,lead=Sac]{At university level: Organization of a summer school for introducing teacher/researchers to interactive theorem proving.}
  \end{wpdeliv}

  \begin{wpdeliv}[due=18,miles=???,id=continuoused,dissem=PU,nature=other,lead=Str]{
 For highschool teachers: Organization of seminars/continuous education sessions about the role of logic in maths teaching and the use of proof assistants in class.}
  \end{wpdeliv}

  \begin{wpdeliv}[due=18,miles=???,id=course-proof,dissem=PU,nature=other,lead=Str]{ For university teachers: creation and dissemination of teaching material for introduction to the concept of proof to fresh maths students.}
  \end{wpdeliv}

  \begin{wpdeliv}[due=18,miles=???,id=geom-curriculum,dissem=PU,nature=other,lead=Str]{ intregation in Logipedia of the formalization of highschool curriculum for geometry.}
  \end{wpdeliv}

\end{wpdelivs}

\end{workpackage}


%%% Local Variables:
%%% mode: latex
%%% TeX-master: "../propB"
%%% mode: flyspell
%%% ispell-local-dictionary: "english"
%%% End:
