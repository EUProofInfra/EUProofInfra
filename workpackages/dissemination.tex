\begin{workpackage}[id=dissemination,type=MGT,wphases=1-48,
  short={Dissemination},
  title={Dissemination, communication and exploitation},
  lead=Inr,InrRM=12,BirRM=4,InrRM=4,ImtRM=2,StrRM=2,ZibRM=14,EduRM=12]

{\color{red} too many deliverables in this WP: draft at 24 and final t 48 instead of two reports}


  \begin{wpobjectives}
    Scientific research and education at all levels are concerned with
    the discovery, verification, communication, archival and usage of
    mathematical results. These tasks have been supported by physical
    books, conferences and other means. The avaibility of a formal
    online encyclopedia which offers in a single place the
    communication, archival and verification of mathematical knowledge
    will be of prime importance for researchers, industrials, teachers
    and editors.

    This work package aims at communicating and disseminating the
    results of the project, and at further developing the communities
    of Logipedia users at the European scale. This includes:
    \begin{compactitem}
    \item providing a website and documentation about the
      infrastructure and the activities of the consortium;
    \item ensuring awareness of the results by publishing them in
      journals and presenting them in international conferences;
    \item promoting the use of Logipedia and spreading our expertise
      through training workshops;
    \item developing and structuring the various communities of users
      of Logipedia (researchers, teachers, engineers, certifiers and
      publishers) through clubs with a general meeting once a year;
    \item discussing with publishers the integration of links to
      Logipedia on their platforms;
    \item providing a teaching interface for using Logipedia at
      schools and universities.
    \end{compactitem}
  \end{wpobjectives}

%  \begin{wpdescription}
%  \ednote{MK: I am not sure that this is type  MGT, please check}
%  \ednote{StS: From my limited experience, reviewers like concrete, peer-reviewed publications as part of the output. Should we have an ongoing task \emph{Publishing Project Results} or comparable?}
%  \end{wpdescription}

\begin{tasklist}

  %%%%%%%%%%%%%%%%%%%%%%%%%%%%%%%%%%%%%%%%%%%%%%%%%%%%%%%%%%%%%%%%%%%%%%%%%%%%
  \begin{task}[id=com,
      title=Communication,
      lead=Inr,InrRM=12,wphases=1-48!.25]

    {\color{red} be more specific and precise}

    {\color{red} do not use ``We will support'' say you are goind to do it}

    This task comprises all forms of direct dissemination and public
    communication activities such as press releases, creation of the
    project website including visitor analysis and monitoring
    tools, scientific and technical publications, outreach activities
    (seminars, keynote talks, media interviews, press releases),
    promotion through social media (e.g.  Twitter, Facebook,
    LinkedIn), creation of advertisement

{\color{red} do not use the word advertisement}

    materials such as flyers,
    posters, and electronic feeds as well as their distribution. We
    will use standard community building technology such as mailing
    lists, Wikis and Forums, to ensure dissemination and engagement of
    the community to support this. We will also generate press
    releases at appropriate moments.
  \end{task}

  %%%%%%%%%%%%%%%%%%%%%%%%%%%%%%%%%%%%%%%%%%%%%%%%%%%%%%%%%%%%%%%%%%%%%%%%%%%%
  \begin{task}[id=training,
      title=Training Logipedia developers and users,
      lead=Bir,BirRM=2,wphases=1-48!.05]
%     The development and growth of Logipedia depends on two separate factors:
%     \begin{compactenum}
%      \item The development and maintenance of computer-checked proofs in the computer proof assistants supported by Dedukti as part of the Logipedia project.
%      \item The development and maintenance of import and export mechanisms for libraries of computer-checked proofs.
%     \end{compactenum}
%     For each of the two items,
    We are planning to provide suitable training
    for people at various stages of their career:

%     We will organize schools targeted to the different Logipedia users:
    \begin{compactitem}
    \item Master and PhD students.
      We will support the planning and running of schools
      on various computer proof assistants.
      We are also planning to run a school dedicated to the topic of Logipedia, on the translation of proofs between different systems.
%       training on formal proofs, proof
%       translation, and automated theorem proving.
      Both activities form an
      opportunity to improve the gender balance in our community by
      attracting female students.
    \item Engineers.
      Develop training and support offers for employees. Develop pedagogic innovations in the delivery of training, in support of companies.
    \item Teachers.
      We will set up a forum for exchange of tipps and tricks for the use of computer proof assistants and formal proofs in teaching such classes.
      Particular emphasis will be on the sharing of teaching materials, which we will encourage to make available under free licenses.
    \end{compactitem}

%     \textbf{Deliverables:}
%     \begin{itemize}
%      \item UniMath school 2022
%      \item Logipedia school 2023
%      \item An electronic forum for exchanging on the use of computer proof assistants in teaching and learning.
%     \end{itemize}

  \end{task}

  %%%%%%%%%%%%%%%%%%%%%%%%%%%%%%%%%%%%%%%%%%%%%%%%%%%%%%%%%%%%%%%%%%%%%%%%%%%%
  \begin{task}[id=researchers-club,
      title=Expanding the use of Logipedia in research,
      lead=Bir,BirRM=2,wphases=1-48!.05]
     To foster interactions between developers and users of Logipedia, we plan the following activities:
    \begin{compactitem}
     \item Logipedia helpdesk:
    we will establish a ``Logipedia helpdesk'', as a unique, and easy-to-reach point of contact for any researcher seeking help on the use of Logipedia.
     The helpdesk will forward any incoming request for help to a suitable expert among the Logipedia developers.
     \item User day at Logipedia meeting.
     We plan to have a ``user day'' attached to our yearly Logipedia meetings.
     During the user day, we plan to have both talks on uses of Logipedia, as well as a round table discussion featuring both developers and users, for discussing present problems and future challenges.
     \item Electronic seminar series.
     We will run an electronic seminar series, thus providing an ecological and economical way of presenting one's work and giving feedback.
    \end{compactitem}
  \end{task}

  %%%%%%%%%%%%%%%%%%%%%%%%%%%%%%%%%%%%%%%%%%%%%%%%%%%%%%%%%%%%%%%%%%%%%%%%%%%%
  \begin{task}[id=industrial-club,
    title=Expanding the use of Logipedia in the industry,
    lead=Inr,InrRM=2,wphases=1-48!.05]

    The use of formal proof systems by companies is still limited, in particular by SMEs/SMIs. The club will be an opportunity to:
    \begin{compactitem}
    \item Develop and organize a club of industrial users of Logipedia.
    \item Promote the use of Logipedia to more industrial users.
    \item Organize meetings to allow companies to present their results/experiences using Logipedia, present the results of Logipedia to companies, collect their opinions on their experience and needs with respect to Logipedia.
    \item Encourage cooperation between companies from the Logipedia
      ecosystem. For example, by sharing rules and proofs between companies using the B method.
    \end{compactitem}
  \end{task}

  %%%%%%%%%%%%%%%%%%%%%%%%%%%%%%%%%%%%%%%%%%%%%%%%%%%%%%%%%%%%%%%%%%%%%%%%%%%%
  \begin{task}[id=certifiers-club,
      title=Promoting the use of Logipedia by certification
      authorities, lead=Imt,ImtRM=2,wphases=1-48!.05]
    With the help of the French national agency of the security of
    information systems (ANSSI), we will promote use of Logipedia in
    European certification authorities.
    \begin{compactitem}
    \item \href{https://www.ssi.gouv.fr/}{ANSSI} would be interested in having one unique language for formal
proofs, rather than several ones which potentially require as many
experts as the number of languages. The consortium will have to first
evaluate if Logipedia addresses 
this need.
\item Contact other relevant European public institutions
(e.g., certification bodies) in order to introduce them to Logipedia,
and understand their potential use of Logipedia.
    \end{compactitem}
  \end{task}

  %%%%%%%%%%%%%%%%%%%%%%%%%%%%%%%%%%%%%%%%%%%%%%%%%%%%%%%%%%%%%%%%%%%%%%%%%%%%
  \begin{task}[id=teachers-club,
      title=Expanding the use of Logipedia in education,
      lead=Str,StrRM=2,wphases=1-48!.05]
    The club of users in education gathers
    teachers who are already actively using formal proof for teaching
    in computer science, mathematics and logic but are not necessarily
    members of the commmunity of researchers in formal theorem
    proving. These early adopters, will provide continuous feedback to
    the project members on the usuability and accessibility of the
    system from this particular point of view.  More than 30 persons
    have already accepted to take part in this club.

%    Along the standard intuitive descriptions of theorem statements
%    and proofs, having a formal description is crucial in the
%    education. Indeed, students are often faced to the difficulty of
%    understanding mathematical concepts and proofs based on pieces of
%    informations found in heteregoneous sources (books, lecture notes,
%    \ldots) with varying definitions, notations. The informal proofs
%    are often given omitting details or relying on implicit knowledge
%    that students may not already be familiar with.

%    Moreover, devlopment of mathematics in education, is often based
%    not on the systematic development of mathematics in the style of
%    Bourbaki, but on so-called deductive islands: sets of local
%    assumptions used in a curriculum, a lecture or an exercise.  On
%    top of the coherent foundations provided by the base libraries
%    incorporated in Logipedia, we will need to build coherent
%    collection of mathematical results suitable for being used in a
%    given classroom.

    The role of the club of users in education will be to:
    \begin{compactitem}
    \item Express the needs of students and teachers with respect to proof
      presentations and tools.
    \item Exchange their experience in teaching formal proofs or using
      theorem provers in class by participating in the ThEdu community.
    \item Provide use cases.
    \item Contribute to the dissemination of information about Logipedia.
    \end{compactitem}

  \end{task}

  %%%%%%%%%%%%%%%%%%%%%%%%%%%%%%%%%%%%%%%%%%%%%%%%%%%%%%%%%%%%%%%%%%%%%%%%%%%%
  \begin{task}[id=publishers-club,
      title=Expanding the use of Logipedia in publishing,
      lead=Zib,ZibRM=2,wphases=1-48!.05]
    We will create and animate a club of users in publishing. We will invite
    people and organization working in the publication of research
    works to join this club: publication archives (arXiv, CCSD-HAL,
    Lipics, ACM, zbMath, etc.), conference steering committees (CPP,
    ITP, POPL, CICM, etc.), journal editorial boards (JFR, JFM,
    etc.). It will be an opportunity to:
    \begin{compactitem}
    \item Discuss with them how to use Logipedia to store and check
      proofs presented or used in scientific publications.
    \item Express their needs with respect to Logipedia.
    \end{compactitem}
  \end{task}

  %%%%%%%%%%%%%%%%%%%%%%%%%%%%%%%%%%%%%%%%%%%%%%%%%%%%%%%%%%%%%%%%%%%%%%%%%%%%
  \begin{task}[id=zib,
      title=Linking scientific publications to Logipedia,
      lead=Zib,ZibRM=12,wphases=12-23]
    We will build a service for Logipedia which provides an overview
    about the development in logic and an intuitive access to the
    following classes of objects - axioms - theories/calculi -
    software (theorem provers) - applications based on logic tools
    (e.g. the metro in Paris, other SAT problems)

    The objects and the relations between the objects will be
    realized via a graph. We will do this in the following way:
    \begin{compactitem}
    \item Defining the objects of each class (we will start from the
      theorem provers, apply the publication-based approach for the
      software and search for references to axioms, theories, and
      applications).
    \item Definition of a special ontology to describe linking and
      dependencies between the objects (metadata).
    \item Add these metadata to all objects
    \item Graph visualization of the network (lattice) and development
      of a search engine.
    \end{compactitem}
    This gives the user a comprehensive entry point into an emerging
    topic.
  \end{task}

  %%%%%%%%%%%%%%%%%%%%%%%%%%%%%%%%%%%%%%%%%%%%%%%%%%%%%%%%%%%%%%%%%%%%%%%%%%%%%%
  \begin{task}[id=edukera,
      title=Web teaching interface for doing proofs at school,
      lead=Edu,EduRM=12,wphases=12-23]
    We build an web application for the education community on top of Logipedia. It will enable students to solve exercices that require
    a mathematical proof with a simple user interface.
    The development tasks are listed below:
    \begin{compactitem}
    \item Graphical web component to display a Logipedia proof (could be used in Logipedia website)
    \item Point-and-click interaction engine on top of Logipedia
    \item Interactive proof interface application (Web interface, connection to LMS, ...)
    \end{compactitem}
  \end{task}

\end{tasklist}

%%%%%%%%%%%%%%%%%%%%%%%%%%%%%%%%%%%%%%%%%%%%%%%%%%%%%%%%%%%%%%%%%%%%%%%%%%%%
\begin{wpdelivs}

%    R:  Document, report (excluding the periodic and final reports)
%  DEM:  Demonstrator, pilot, prototype, plan designs
%  DEC:  Websites, patents filing,   press & media actions, videos, etc.
% OTHER: Software, technical diagram, etc.

  % deliverables for T1 communication (com Inria Saclay)

  \begin{wpdeliv}[due=1,id=website-setting,dissem=PU,nature=DEC,lead=Inr]{Setting up project website}
  \end{wpdeliv}

  \begin{wpdeliv}[due=12,id=website-enriched,dissem=PU,nature=DEC,lead=Inr]{Enrich project website}
  \end{wpdeliv}

  % deliverables for T2 training (Benedikt Ahrens, Birmingham, with the help of IRT SystemX for training targeting engineers)

  \begin{wpdeliv}[due=24,id=training-report1,dissem=PU,nature=R,lead=Bir]{Midway report on training}
  \end{wpdeliv}

  \begin{wpdeliv}[due=48,id=training-report2,dissem=PU,nature=R,lead=Bir]{Final report on training}
  \end{wpdeliv}

  % deliverables for T3 researchers club (Benedikt Ahrens, Birmingham)

  \begin{wpdeliv}[due=24,id=researchers-club-report1,dissem=PU,nature=R,lead=Bir]{Midway report on researchers club}
  \end{wpdeliv}

  \begin{wpdeliv}[due=48,id=researchers-club-report2,dissem=PU,nature=R,lead=Bir]{Final report on researchers club}
  \end{wpdeliv}

  % deliverables for T4 industrial club (IRT System X)

  \begin{wpdeliv}[due=24,id=industrial-club-report1,dissem=PU,nature=R,lead=Inr]{Midway report on industrial club}
  \end{wpdeliv}

  \begin{wpdeliv}[due=48,id=industrial-club-report2,dissem=PU,nature=R,lead=Inr]{Final report on industrial club}
  \end{wpdeliv}

  % deliverables for T5 certification authorities (Catherine Dubois, IMT)

  \begin{wpdeliv}[due=24,id=certifiers-club-report1,dissem=PU,nature=R,lead=Imt]{Midway report on certification bodies}
  \end{wpdeliv}

  \begin{wpdeliv}[due=48,id=certifiers-club-report2,dissem=PU,nature=R,lead=Imt]{Final report on certification bodies}
  \end{wpdeliv}

  % deliverables for T6 teachers club (Julien Narboux, Unistra)

  \begin{wpdeliv}[due=24,id=teachers-club-report1,dissem=PU,nature=R,lead=Str]{Midway report of the teachers club}
  \end{wpdeliv}

  \begin{wpdeliv}[due=48,id=teachers-club-report2,dissem=PU,nature=R,lead=Str]{Final report of the teachers club}
  \end{wpdeliv}

  % deliverables for T7 publishers club (Wolfgang Dalitz, ZIB)

  \begin{wpdeliv}[due=24,id=publishers-club-report1,dissem=PU,nature=R,lead=Zib]{Midway report on publishers club}
  \end{wpdeliv}

  \begin{wpdeliv}[due=48,id=publishers-club-report2,dissem=PU,nature=R,lead=Zib]{Final report on the publishers club}
  \end{wpdeliv}

  % deliverables for T8 linking scientific publications to Logipedia (Wolfgang Dalitz, ZIB)

  \begin{wpdeliv}[due=24,id=publish-links,dissem=PU,nature=OTHER,lead=Zib]{Database and search engine for linking Logipedia formal proofs with scientific publications}
  \end{wpdeliv}

  % deliverables for T9 web interface for education (Benoit Rognier, Edukera)

  \begin{wpdeliv}[due=24,id=edu-display,dissem=PU,nature=DEM,lead=Edu]{Display Logipedia proofs}
  \end{wpdeliv}

  \begin{wpdeliv}[due=24,id=edu-app,dissem=PU,nature=OTHER,lead=Edu]{Application for education}
  \end{wpdeliv}

\end{wpdelivs}

\end{workpackage}


%%% Local Variables:
%%% mode: latex
%%% TeX-master: "../propB"
%%% mode: flyspell
%%% ispell-local-dictionary: "english"
%%% End:
