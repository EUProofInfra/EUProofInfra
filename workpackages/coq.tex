% task leader: Claudio Sacerdoti Coen
% participants: Enrico Tassi, Claudio Sacerdoti Coen
%\textbf{Budget requirements:} One one PhD student or postdoc at UBo.

\ednote{Bol: 12 MM = 44,630 euros (39,372 euros salary + travels etc.); Inr: 12 MM}

% Moved to concept & methodology
% Coq is an interactive theorem prover developed at Inria since the 1984.
% It is based on Type Theory and was used to formally verify the correctness
% of both industrially relevant software such as the CompCert C compiler and
% complex mathematical proofs such as the one of the Four Color theorem and the
% one of the Odd Order theorem. In 2013 Coq received the ACM system award.

\begin{enumerate}
% deliverable 1
\item Access Coq internal data structures to gather logical data, such as
statements and proof terms (for tasks in WP3)
% deliverable 1
\item Implement the translation of Coq terms to Dedukti terms
% deliverable 2
\item Access Coq internal data structures to gather extra-logical data,
such as the role played by a constant in the library like begin an implicit
cast from one algebraic structure to another (for tasks in WP3)
% deliverable 2
\item Make extra-logical data available in a structured and
  extensible format (for tasks in WP5)
\end{enumerate}

% Moved to concept & methodology
%A technological hurdle steps (a) and (c) have to overcome is that Coq is an actively
%developed system that is constantly evolving. Previous attempts at extracting
%data from Coq without a direct interaction with the developers of Coq resulted
%in prototypes like CoqinE that quickly became outdated.
%To overcome the problems associated with external tools such as
%CoqInE, we plan to have the required instrumentation merged in Coq
%proper and reuse it for other projects that could benefit from it so
%to amortize its development cost. More precisely E. Tassi is a core
%Coq developer acquainted with its development process and he will take
%care of the integration of the infrastructure in Coq and foster its
%reuse in third party projects with similar needs such as SerAPI,
%CoqHammer and Coq-Elpi.

% Moved to concept & methodology
% Step (b) is also problematic for two reasons. The first one is that the encoding
% in Dedukti requires some information, typically types of sub-expressions, that
% are not stored in Coq, it is transient. So the instrumentation for steps (a) and
% (c) needs to be complemented by providing not only access to existing data but
% also to log transient data or re-synthesize it on demand. Both approaches may
% be used, depending on the the tradeoff between computation time and space for
% storage. The second reason, which is more critical, is ...

%Since the type theory of Coq is extremely large, with features that
%have no corresponding representation in Dedukti, the type theory of
%Coq needs to be translated to a core one representable in Dedukti.
%This translation to a core calculus is not implemented in Coq and the
%amount and complexity of code necessary for it is very significant and
%indeed the CoqinE prototype only covers a small subset of
%Coq. Feedback from WP2 will be of guidance\ednote{Tassi: there is no
%task in WP2 about Coq's TT, maybe it is included in HoTT?} to extend
%the translation currently available CoqinE to cover a larger subset of
%Coq.

% Moved to methodology
% In step (d) we plan to take advantage of the work done by Sacerdoti Coen (UBo)
% in 2019 in exporting non trivial logical and extra-logical data from Coq to
% an XML format. Data in that format was then translated by Kohlhase et al
% (UBo + FAU) to the MMT system, another logical framework to encode different
% logics and their libraries. We plan to extend that format to include even
% more extra-logical data as well as the data gathered in step (a) and (c). We shall
% evaluate if all the data needed for step (b) can be saved in this format, and
% give us the freedom to implement step (b) in a standalone tool making no
% requests to Coq in order to re-synthesize missing data.

%%% Local Variables:
%%% mode: latex
%%% TeX-master: "../propB"
%%% End:
