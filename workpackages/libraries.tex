\begin{workpackage}[id=libraries,wphases=0-48,type=RTD,
  short=Libraries,% for Figure 5.
  title=Libraries,
  lead=Inr,
  InrRM=10,
  TumRM=36]
%TUM: 24 for Analysis library, 12 for CakeML
%resources for Makarius (25k EUR) should not go here because they do not generate overheads


\ednote{MK: We need one coordinating site. original coordinators: Georges Gonthier and
Tobias Nipkow} \ednote{MK: interested parties (add their sites and RM here): David
Deharbe, Nicola Gambino, Tobias Nipkow, Makarius Wenzel, Julien Narboux, Gaspard Férey,
François Thiré, Magnus Myreen}

\begin{wpobjectives}
  The objective of this work package is to \ldots

This includes notably:
  \begin{compactitem}
  \item Access
  \item Scalability
  \end{compactitem}
  A key aspect will be to foster \ldots

For the future: Mizar, PVS, seL4, CompCert
\end{wpobjectives}


\begin{wpdescription}
Translating the standard libraries of the systems is part of the WP1.
\end{wpdescription}

Two aspects: scalability and accessibility

{\bf Trusted chain}

A grant challenge in program verification is that of running trusworthy
software on trustworthy hardware. This challenge can be decomposed in
four steps: (1) formally verify the source code of the program,
(2) compile this code using a formally-verified compiler,
(3) launch the program on a formally-verified operating system,
(4) running the whole thing on hardware with formally-verified circuits.

Developing a program verification framework, a verified compiler, a
verified operating system, and a verified hardware, corresponds to 4
daunting tasks. Just setting up a reasonable proof-of-concept takes of
the order of 10 man-year of work. Raising the bar to a production-ready
tool takes at least Can additional order of magnitude in terms of efforts.

Considerable progress has been made on these three aspects
over the past decade, and every year the technology becomes more mature.
Yet, there is a pitfall. The tools are not all developed using the same
theorem prover, not just for historical reasons but also because different
proof assistants have appeared better-suited at different tasks.

In that setting, how one could possibly complete a trustworthy chain if
the four main pieces of the chain are not properly hooked to one another?
As a concrete example, CFML is a program verification for verifying ML code,
developed in Coq; and CakeML is a verified compiler for ML code,
developed in HOL4. How can we combine the two tools to produce verified
machine code from verified ML code?

There are two main ways to address the issue. One way is to port the
proofs of all the formalized tools into a same proof assistant.
Doing so might be possible in a far future, however given the scale and
the complexity of the tools, this approach does not provide a short-term
solution. Another, more accessible approach consists of translating
between proof assistants only the formal statements associated with the
interfaces between the 4 pieces of the verified chain.

Concretely, the interfaces involve: (1-to-2) a formal semantics for
a programming language, (2-to-3) a formal semantics for a set of
system calls (the OS API and isolation model), (2-and-3-to-4) a formal
semantics for machine code (instruction set and memory model).
Translating such formal interfaces, which consists solely of statements,
is considerably easier than translating all the verification proofs
associated with the tools.

A proposal for translating such interfaces should satisfy the following
requirements. First, the translations must to be trusworthy. In particular,
translation by hand is not an option, and developing translation tools
between every pair of provers might make the trusted code base prohibitively
large. Second, the translations should be maintainable. Indeed,
formal specifications of the aforementioned interfaces do evolve, slowly but
surely, through time, in particular for incorporating new features.
Third, the translations should produce formal definitions that are written
in a style sufficiently idiomatic with respect to the target proof assistant.
Indeed, carrying out proofs with respect to definitions in non-idiomatic
style induce prohibitive overheads.

We propose to tackle the problem by leveraging the unifying language
Dedukti in the following way. Assume that we have, for formal statements,
a bi-directional translations between each prover and Deduki. Then,
we could translate a formal interface from one system to another.
The user may then provide alternative, more idiomatic statements for
specific definitions, and prove (by hand) that the alternative definitions
are equivalent to the automatically-generated ones. If the original
interface is modified, then the automatic translations can be updated,
and the proof system would notify the user of which alternative definitions
need to be updated accordingly.

For example, we could take CakeML's semantics of its input ML language,
translate it automatically into Dedukti, then translate Dedukti's
definition into Coq, refine by hand a few definitions to make them more
idiomatic, and obtain a usable Coq semantics, with respect to which
the correctness of the CFML verification tool can be established in Coq.
Completing this case study is motivating not only because it delivers
an immediate result of relating two existing tools, but also because it
would validate the general approach of translating semantics via Dedukti
in a realistic manner.

Beyond this one example, other motivating case studies will be considered,
such as translating the formal semantics of ComCert-C into other proof
assistants, or translating the formal semantics of ARM instruction set.


\begin{tasklist}
\begin{task}[id=mathcomp,title=MathComp]
\ednote{Sophia, Saclay (Gonthier), Paris}
\end{task}

\begin{task}[id=milc,title=Revised Coq Analysis Library]
\ednote{Saclay (Boldo), Paris, Sophia}
\end{task}

%\begin{task}[id=mizar,title=The Mizar library]
%\ednote{Innsbruck, Bialystok}
%\end{task}

\begin{task}[id=afp,title=Isabelle/AFP: The Archive of Formal Proofs]
\ednote{TU München (Wenzel)}

In Feb-2020, the Archive of Formal Proofs \cite{isabelle-afp}
consisted of more than 500 entries (articles of formalized
mathematics) by 340 authors, and required approx.\ 60h CPU time for
checking (using many gigabytes of memory).

The purpose of this task is to scale up the Isabelle instrumentation
for Dedukti further, to cover major parts of this library.  This might
require a few pathological proofs to be reworked. The ultimate aim is
to export the main substance of Isabelle/AFP, without promising full
coverage: some entries with prohibitive resource requirements will be
omitted. Also note that Isabelle/AFP is continuously growing at a high
rate and thus a moving target.

\end{task}

\begin{task}[id=isaAnalysisProb,title=The Isabelle Analysis \& Probability library]
\ednote{TU München (Nipkow)}

The Isabelle Analysis and Probability Theory library consists of more than
200.000 lines of definitions and proofs, corresponding to almost 4000 printed
pages. It covers topology, homology, multivariate, functional and complex
analysis, measure and probability theory, and a dedicated library for
ordinary differential equations. It is fair to say that it is the most
advanced machine-checked library in the area of analysis and probability
theory.\footnote{The Isabelle analysis and probability theory library is
in part based on the extensive library of the HOL Light system but has
generalized it from real numbers to appropriate algebraic structures and
includes areas absent from the HOL Light library (in particular measure
theory and ordinary differential equations).} It is currently used by 60 out
of 500 articles in the Archive of Formal Proofs, a user-contributed library
of Isabelle/HOL proofs from all areas of computer science and
mathematics. Moreover, it is the only existing library in this area where
proofs are structured and readable. Due to its size and generality and
because analysis and probability theory are pervasive in any application in
the engineering and natural sciences and economics, this library will be a
KER of the project: it is a fundamental enabling resource for almost any
formal verification activity in these areas, for example autonomous driving
and mathematical finance. The purpose of this task is to structure, document
and develop this library for optimal accessibility, ease of use and
comprehensiveness. The aim is a curated library for applications.

Any library is only as useful as its structure and documentation, no matter
how sophisticated the search facilities are. Therefore the primary aim of
this task is to make this library easily accessible. At the same time the
library needs to be developed further: some material needs revising for ease
of use and some essential material needs to be added.

\textbf{Accessibility}\quad
The library is a large collection of theories with a hierarchical dependency
relation. However, this structure has grown over time and does not always
reflect the abstract mathematical dependencies. In short, the structure needs
to be modularized. This requires a significant refactoring effort.
%
At the same time we need to add metadata to the source material to turn this
structured collection of theorems and proofs into a curated library at the
Dedukti level.  This is where we need to interface with \WPref{structuring}. We
will be test drivers of the metadata infrastructure provided by that WP, in
particular the ability to describe ontologies for structuring in the large.
This will also entail annotating those statements in the library that
correspond to proper mathematical theorems as opposed to auxiliary lemmas
required only for the benefit of the theorem prover.
To increase accessibility we will also include links from the library into
Wikipedia and in particular in the other direction to raise the awareness
of Dedukti.

\textbf{Development}\quad
Although the library support for integrals is extensive, it suffers
from the (necessary) coexistence of different kinds of
integrals. This complicates proofs about integrals in applications and needs
to be unified, which entails further refactoring. The rest of the
development adds further essential material:
1. Fourier analysis because of its extreme important both for pure mathematics and for
engineering and physics (especially the Fourier
transform). A formalization of basic properties of the Laplace transform is already available in the AFP. 2. Stability theory for differential equations and
dynamical systems, in particular Lyapunov functions, because of their
relevance for the verification of cyber-physical systems.
3. Stochastic differential equations to model a large range of
dynamical systems with stochastic components, from physics to financial markets.
\end{task}

\begin{task}[id=geocoq,title=The GeoCoq library]
\ednote{Sophia (Boutry), Strasbourg, Belgrade}
\end{task}

\begin{task}[id=flyspeck,title=The Flyspeck library]
\ednote{Saclay (Grienenberger)}
\end{task}

\begin{task}[id=cakeml,title=The CakeML compiler library]
\ednote{Chalmers (Myreen), Strasbourg, TU München}
The task is to export the CakeML compiler library to Dedukti.  The
CakeML compiler library consists of the CakeML programming language
definition, its compiler and the correctness proofs about the
compiler. This is a cutting edge compiler library and one of only two
verified compilers for real languages, the other being CompCert. Its
export to Dedukti is one of the KERs of this project. Because of the
size and importance of this library, we will approach the export from
two angles.

The first approach utilizes OpenTheory-based technology. The CakeML
compiler development lives within the HOL4 prover. HOL4 can export
definitions and proofs in the OpenTheory format, which can in turn be
translated into Dedukti. This link from HOL4 via OpenTheory to Dedukti
exists but, in its current state, fails to scale to the task of
transporting something as sizeable as the CakeML compiler
development. This part of this task will rework the route via
OpenTheory to scale better, possibly taking inspiration from an
OpenTheory-like approach that scaled well for the HOL light
prover~\cite{KaliszyKrauss}.

The second approach establishes a connection from HOL4 via Isabelle to
Dedukti. The basis is a promising new approach of virtualizing HOL4
inside Isabelle \cite{ImmlerRaedleWenzel}. That is, the inference
kernel of HOL4 is replaced by that of Isabelle and the resulting
system produces Isabelle theorems instead of HOL4 theorems. This has
the invaluable additional benefit that tools, not just theorems, can
be shared. As a benchmark of the viability of this approach we plan to
export CakeML via Isabelle to Dedukti.
\end{task}

\begin{task}[id=unimath,title=The UniMath library]
\ednote{Birmingham (Ahrens)}
\end{task}

%\begin{task}[id=pvs,title=The NASA PVS library]
%\end{task}
%\begin{task}[id=sel4,title=The seL4 library]
%\end{task}

%\begin{task}[id=compcert,title=The CompCert library]
%\end{task}
\end{tasklist}

\begin{wpdelivs}
  \begin{wpdeliv}[due=3,miles=startup,id=requirements,dissem=PU,nature=DEM,lead=Inr]
      {Requirements Analysis and Synchronization}
\end{wpdeliv}
  \begin{wpdeliv}[due=36,miles=isabelle-stdlib,id=requirements,dissem=PU,nature=DEM,lead=Tum]
      {Scalable export of proof terms for major parts of Isabelle/AFP}
  \end{wpdeliv}
\end{wpdelivs}
\end{workpackage}

%%% Local Variables:
%%% mode: latex
%%% TeX-master: "../propB"
%%% End:
