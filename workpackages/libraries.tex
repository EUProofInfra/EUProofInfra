\begin{workpackage}[id=libraries,wphases=0-48,type=RTD,
  short=Libraries,% for Figure 5.
  title=Libraries,
  lead=Inr,
  InrRM=10]

\ednote{MK: We need one coordinating site. original coordinators: Georges Gonthier and
Tobias Nipkow} \ednote{MK: interested parties (add their sites and RM here): David
Deharbe, Nicola Gambino, Tobias Nipkow, Makarius Wenzel, Julien Narboux, Gaspard Férey,
François Thiré}

\begin{wpobjectives}
  The objective of this work package is to \ldots

This includes notably:
  \begin{compactitem}
  \item Access
  \item Scalability
  \end{compactitem}
  A key aspect will be to foster \ldots

For the future: Mizar, PVS, seL4, CompCert
\end{wpobjectives}


\begin{wpdescription}
Translating the standard libraries of the systems is part of the WP1.
\end{wpdescription}

Two aspects: scalability and accessibility

{\bf Trusted chain}

A grant challenge in program verification is that of running trusworthy
software on trustworthy hardware. This challenge can be decomposed in
four steps: (1) formally verify the source code of the program,
(2) compile this code using a formally-verified compiler,
(3) launch the program on a formally-verified operating system,
(4) running the whole thing on hardware with formally-verified circuits.

Developing a program verification framework, a verified compiler, a
verified operating system, and a verified hardware, corresponds to 4
daunting tasks. Just setting up a reasonable proof-of-concept takes of
the order of 10 man-year of work. Raising the bar to a production-ready
tool takes at least Can additional order of magnitude in terms of efforts.

Considerable progress has been made on these three aspects
over the past decade, and every year the technology becomes more mature.
Yet, there is a pitfall. The tools are not all developed using the same
theorem prover, not just for historical reasons but also because different
proof assistants have appeared better-suited at different tasks.

In that setting, how one could possibly complete a trustworthy chain if
the four main pieces of the chain are not properly hooked to one another?
As a concrete example, CFML is a program verification for verifying ML code,
developed in Coq; and CakeML is a verified compiler for ML code,
developed in HOL. How can we combine the two tools to produce verified
machine code from verified ML code?

There are two main ways to address the issue. One way is to port the
proofs of all the formalized tools into a same proof assistant.
Doing so might be possible in a far future, however given the scale and
the complexity of the tools, this approach does not provide a short-term
solution. Another, more accessible approach consists of translating
between proof assistants only the formal statements associated with the
interfaces between the 4 pieces of the verified chain.

Concretely, the interfaces involve: (1-to-2) a formal semantics for
a programming language, (2-to-3) a formal semantics for a set of
system calls (the OS API and isolation model), (2-and-3-to-4) a formal
semantics for machine code (instruction set and memory model).
Translating such formal interfaces, which consists solely of statements,
is considerably easier than translating all the verification proofs
associated with the tools.

A proposal for translating such interfaces should satisfy the following
requirements. First, the translations must to be trusworthy. In particular,
translation by hand is not an option, and developing translation tools
between every pair of provers might make the trusted code base prohibitively
large. Second, the translations should be maintainable. Indeed,
formal specifications of the aforementioned interfaces do evolve, slowly but
surely, through time, in particular for incorporating new features.
Third, the translations should produce formal definitions that are written
in a style sufficiently idiomatic with respect to the target proof assistant.
Indeed, carrying out proofs with respect to definitions in non-idiomatic
style induce prohibitive overheads.

We propose to tackle the problem by leveraging the unifying language
Dedukti in the following way. Assume that we have, for formal statements,
a bi-directional translations between each prover and Deduki. Then,
we could translate a formal interface from one system to another.
The user may then provide alternative, more idiomatic statements for
specific definitions, and prove (by hand) that the alternative definitions
are equivalent to the automatically-generated ones. If the original
interface is modified, then the automatic translations can be updated,
and the proof system would notify the user of which alternative definitions
need to be updated accordingly.

For example, we could take CakeML's semantics of its input ML language,
translate it automatically into Dedukti, then translate Dedukti's
definition into Coq, refine by hand a few definitions to make them more
idiomatic, and obtain a usable Coq semantics, with respect to which
the correctness of the CFML verification tool can be established in Coq.
Completing this case study is motivating not only because it delivers
an immediate result of relating two existing tools, but also because it
would validate the general approach of translating semantics via Dedukti
in a realistic manner.

Beyond this one example, other motivating case studies will be considered,
such as translating the formal semantics of ComCert-C into other proof
assistants, or translating the formal semantics of ARM instruction set.


\begin{tasklist}
\begin{task}[id=mathcomp,title=MathComp]
\ednote{Sophia, Saclay (Gonthier), Paris}
\end{task}

\begin{task}[id=milc,title=Revised Coq Analysis Library]
\ednote{Saclay (Boldo), Paris, Sophia}
\end{task}

%\begin{task}[id=mizar,title=The Mizar library]
%\ednote{Innsbruck, Bialystok}
%\end{task}

\begin{task}[id=afp,title=The Isabelle Archive of Formal Proofs]
\ednote{TU München (Wenzel), Saclay}
\end{task}

\begin{task}[id=isaAnalysisProb,title=The Isabelle Analysis \& Probability library]
\ednote{TU München (Nipkow)}
\end{task}

\begin{task}[id=geocoq,title=The GeoCoq library]
\ednote{Sophia (Boutry), Strasbourg, Belgrade}
\end{task}

\begin{task}[id=flyspeck,title=The Flyspeck library]
\ednote{Saclay (Grienenberger)}
\end{task}

\begin{task}[id=cakeml,title=The CakeML programming language library]
\ednote{Gothenburg (Myreen), Strasbourg, TU München}
\end{task}

\begin{task}[id=unimath,title=The UniMath library]
\ednote{Birmingham (Ahrens)}
\end{task}

%\begin{task}[id=pvs,title=The NASA PVS library]
%\end{task}
%\begin{task}[id=sel4,title=The seL4 library]
%\end{task}

%\begin{task}[id=compcert,title=The CompCert library]
%\end{task}
\end{tasklist}

\begin{wpdelivs}
  \begin{wpdeliv}[due=3,miles=startup,id=requirements,dissem=PU,nature=DEM,lead=Inr]
      {Requirements Analysis and Synchronization}
\end{wpdeliv}
\end{wpdelivs}
\end{workpackage}

%%% Local Variables:
%%% mode: latex
%%% TeX-master: "../propB"
%%% End:
