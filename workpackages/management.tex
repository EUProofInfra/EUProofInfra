\begin{workpackage}[id=management,type=MGT,
  short=Management,
  title=Management,
  lead=Inr,InrRM=36,BirRM=1,InnRM=1,SacRM=1,TumRM=1,IrtRM=1,LeeRM=1]
  
  \begin{wpobjectives}
    The scope of this work package is the overall management of the project activities led by the consortium. The management of the Logipedia project will ensure the necessary conditions to enable the project to achieve its objective(s) while meeting its cost, time and quality requirements. This includes the scientific, administrative, financial and legal management.

    Gilles Dowek, Inria senior researcher and professor at ENS Paris-Saclay, will be the coordinator and leader of this work package. Gilles is PI of many projects, including under the H2020 framework programme and international projects. He will also be supported by a deputy coordinator, an European project manager from the Innovation, Partnership and Transfer Office of Inria Saclay and a Chief engineer.

This therefore includes:
\begin{compactitem}
\item A scientific and technical coordination to create a vibrant scientific and technical environment within the project.
\item The overall management of the project and consortium according to the governance structure and procedures explained in section 3.2.
\item An efficient project management, as specified in 3.2, including:
  \begin{compactitem}
  \item Overall administrative and financial project management, including reporting to the European Commission.
  \item Quality management.
  \item Assessment and risk management, including conflict or dispute management.
  \end{compactitem}
\end{compactitem}
\end{wpobjectives}

\begin{tasklist}
  \begin{task}[id=coordination,title=Scientific and technical coordination,lead=Inr,InrRM=12,wphases=1-48]
    The scientific and technical coordination will be led by Inria senior researcher Gilles Dowek. He will be in charge of ensuring the implementation of the scientific strategy of Logipedia and thereby ensuring the growth of the Logipedia community. This task foresees a key role of scientific animation and to impulse the organisation of scientific activities, together with the WP leader in charge of dissemination. Gilles Dowek will supervise the ongoing scientific and technical coordination and help with the innovation management, together with the technical manager and the steering committee. The scientific coordinator will also chair the steering committee and the general assembly. The scientific coordinator will be the scientific point of contact within the consortium and for the consortium when the project needs to be represented.
  \end{task}

  \begin{task}[id=admin,title=Administrative and Financial Management,lead=Inr,InrRM=24,wphases=1-48]
    A European Project Manager (EPM) from the Transfer and Innovation team of Inria Saclay which has an extensive experience in handling innovation from research projects such as Logipedia. The consortium will therefore benefit from tailored and on-demand advice regarding the use and potential transfer of the research results during the course of the project.
The EPM will ensure the day-to-day management as it will be the administrative and financial point of contact for the consortium and a dedicated contact point for the European Commission. The meeting preparation and follow-up will be another task of the EPM and will include organising plenary meeting with the partner organisation, general assembly, minutes, review meeting with the consortium. Financial aspects will be a crucial task of the EPM and include: payment to partners; ensuring financial monitoring within the consortium and leading the financial reporting to the European Commission.
The EPM will also be responsible, together with the scientific coordinator, for ensuring the technical work and deliverables meet the technological objectives of the project according to the defined schedule. The EPM will work very closely with the work package leaders in order to monitor the progress of the technical work and to identify potential risks within each WP. The EPM also acts as a quality manager to ensure that the content of the deliverables meets the quality standards defined for the project. The EPM reports to the Scientific Coordinator.
The development and maintenance of collaborative tools will be ensured by Inria and monitored by the EPM. A common teleconference tool, storage space, reporting and intranet will be set up and detailed in the collaborative tools deliverable of M2.
  \end{task}

  \begin{task}[id=legal,title={Legal Management (data, ethics, GDPR)},wphases=1-48]
    The preparation of the Consortium and Grant agreement will be led by the European Project Manager at Inria Saclay. If a change arises during the course of the project, amendment will be prepared by the project management team, in close collaboration with Inria legal team. 
Data Protection, ethics and GDPR Compliance Management will also be ensured by INRIA. The main objective here is to provide guidance on data protection for the research activities of the project in the context of the European General Data Protection Regulation (GDPR). If at some point during the course of the project, the consortium or any scientist is unsure about how to handle a particular situation or requires advice on ethical issues, the partners or the individuals, supported by the EPM, will refer to the operational ethical committee of Inria (the COERLE) before proceeding.
  \end{task}

\end{tasklist}

\begin{wpdelivs}
  
  \begin{wpdeliv}[due=2,miles=???,id=collab-tools,dissem=PU,nature=DEC,lead=Inr]{Collaborative Tools} Document or Notice introducing the collaborative tools of the consortium
  \end{wpdeliv}

  \begin{wpdeliv}[due=3,miles=???,id=guide,dissem=PU,nature=R,lead=Inr]{Logipedia Partner Guide} In order to present the processes and governance within the consortium
  \end{wpdeliv}

  \begin{wpdeliv}[due=6,miles=???,id=data-plan,dissem=PU,nature=R,lead=Inr]{Data Management Plan}
  \end{wpdeliv}
  
\end{wpdelivs}

\end{workpackage}


%%% Local Variables:
%%% mode: latex
%%% TeX-master: "../propB"
%%% End:
