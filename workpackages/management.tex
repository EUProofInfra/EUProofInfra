\begin{workpackage}[id=management,type=MGT,wphases=1-48,
  short=Management,
  title=Management,
  lead=Inr,InrRM=40,InnRM=2,SacRM=2,TumRM=2,LieRM=2,BelRM=2,DelRM=2,FauRM=2]
  
  \begin{wpobjectives}
    The scope of this work package is the overall management of the
    project activities led by the consortium. The management of the
    Logipedia project will ensure the necessary conditions to enable
    the project to achieve its objectives while meeting its cost, time
    and quality requirements. This includes the scientific,
    administrative, financial and legal management.

    Gilles Dowek, Inria senior researcher and professor at ENS
    Paris-Saclay, will be the Coordinator and leader of this work
    package. Gilles has been PI of many projects. He will also be
    supported by a Deputy coordinator, Frédéric Blanqui, an European
    project manager from the Innovation, Partnership and Transfer
    Office of Inria Saclay and a Chief engineer.

This therefore includes:
\begin{compactitem}
\item A scientific and technical coordination to create a vibrant scientific and technical environment within the project.
\item The overall management of the project and consortium according to the governance structure and procedures explained in section 3.2.
\item An efficient project management, as specified in 3.2, including:
  \begin{compactitem}
  \item Overall administrative and financial project management, including reporting to the European Commission.
  \item Quality management.
  \item Assessment and risk management, including conflict or dispute management.
  \end{compactitem}
\end{compactitem}
\end{wpobjectives}

\begin{tasklist}
  \begin{task}[id=coordination,title=Scientific and technical coordination,shorttitle=Sci.\&tech.,lead=Inr,InrRM=20,wphases=1-48]
    The scientific and technical coordination will be led by Inria
    senior researcher Gilles Dowek. He will be in charge of ensuring
    the implementation of the scientific strategy of Logipedia and
    thereby ensuring the growth of the Logipedia community. This task
    foresees a key role of scientific animation and to impulse the
    organisation of scientific activities, together with the WP leader
    in charge of dissemination. Gilles Dowek will supervise the
    ongoing scientific and technical coordination and help with the
    innovation management, together with the Deputy coordinator, the
    Chief engineer and the steering committee. The
    Coordinator will also chair the steering committee and the general
    assembly. The Coordinator will be the scientific point of contact
    within the consortium and for the consortium when the project
    needs to be represented.

    He will be seconded and replaced by the Deputy coordinator, Frédéric
    Blanqui.
  \end{task}

  \begin{task}[id=admin,title=Administrative and Financial Management,shorttitle=Adm.\&Fin.,lead=Inr,InrRM=20,wphases=1-48]
    A European Project Manager (EPM) from the Transfer and Innovation
    team of Inria Saclay, who has an extensive experience in handling
    innovation from research projects such as Logipedia, will be working
    50\% of her or his time during the course of the project.
     The
    consortium will therefore benefit from tailored and on-demand
    advice regarding the use and potential transfer of the research
    results during the course of the project.  The EPM will ensure the
    day-to-day management as it will be the administrative and
    financial point of contact for the consortium and a dedicated
    contact point for the European Commission. The meeting preparation
    and follow-up will be another task of the EPM and will include
    organising plenary meeting with the partner organisation, general
    assembly, minutes and review meeting with the consortium. Financial
    aspects will be a crucial task of the EPM and include: payment to
    partners; ensuring financial monitoring within the consortium and
    leading the financial reporting to the European Commission.  The
    EPM will also be responsible, together with the 
    Coordinator, for ensuring the technical work and deliverables meet
    the technological objectives of the project according to the
    defined schedule. The EPM will work very closely with the work
    package leaders in order to monitor the progress of the technical
    work and to identify potential risks within each WP. The EPM also
    acts as a quality manager to ensure that the content of the
    deliverables meets the quality standards defined for the
    project. The EPM reports to the Coordinator.  The
    development and maintenance of collaborative tools will be ensured
    by Inria and monitored by the EPM. A common teleconference tool,
    storage space, reporting and intranet will be set up and detailed
    in the collaborative tools deliverable of M2.
  \end{task}

  \begin{task}[id=legal,title={Legal Management (data, ethics, GDPR)},shorttitle={Legal},
      lead=Inr,InrRM=4,wphases=1-48]
    The preparation of the Consortium and Grant agreement will be led
    by the European Project Manager at Inria Saclay. If a change
    arises during the course of the project, an amendment will be
    prepared by the project management team, in close collaboration
    with Inria legal team.  Data Protection, ethics and GDPR
    Compliance Management will also be ensured by INRIA. The main
    objective here is to provide guidance on data protection for the
    research activities of the project in the context of the European
    General Data Protection Regulation (GDPR). If at some point during
    the course of the project, the consortium or any scientist is
    unsure about how to handle a particular situation or requires
    advice on ethical issues, the partners or the individuals,
    supported by the EPM, will refer to the operational ethical
    committee of Inria (the COERLE) before proceeding.
  \end{task}

  
\end{tasklist}

\begin{wpdelivs}
  \begin{wpdeliv}[due=2,id=workplan,dissem=PU,nature=R,lead=Inr]{Detailed
      workplan} Detailed workplan for the Grant agreement
  \end{wpdeliv}

  \begin{wpdeliv}[due=2,id=collab-tools,dissem=PU,nature=R,lead=Inr]{Collaborative Tools}
    Document or Notice introducing the collaborative tools of the consortium
  \end{wpdeliv}

  \begin{wpdeliv}[due=3,id=guide,dissem=PU,nature=R,lead=Inr]{Logipedia Partner Guide}
    In order to present the processes and governance within the consortium
  \end{wpdeliv}

  \begin{wpdeliv}[due=4,id=data-plan,dissem=PU,nature=R,lead=Inr]{Data
      Management Plan}
  \end{wpdeliv}
\end{wpdelivs}

\end{workpackage}


%%% Local Variables:
%%% mode: latex
%%% TeX-master: "../propB"
%%% End:
