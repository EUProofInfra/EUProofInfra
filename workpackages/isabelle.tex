% task leader: Tobias

% participants: Makarius (Isabelle), David Matthews (Poly/ML)

% Moved to concept & methodology
% Isabelle as a logical framework \cite{paulson700} is an intermediate
% between Type-Theory provers (like Coq or Agda) and classic LCF-style
% systems (like HOL Light or HOL4). The inference kernel can already
% output proofs as $\lambda$-terms on request, but this has so far been
% only used for small examples \cite{Berghofer-Nipkow:2000:TPHOL}. The
% challenge is to make Isabelle proof terms work robustly for the basic
% libraries and reasonably big applications.  Preliminary work by Wenzel
% (2019) has demonstrated the feasibility for relatively small parts of
% Isabelle/HOL, but this requires scaling up.

\begin{enumerate}
  \item Improve the efficiency of important aspects of the
  Isabelle/HOL logic implementation, such as normalization of proofs,
  type-class reasoning, and special representation of derived rules
  and definition principles.
  \item Reduce the volume of proof terms in the Dedukti encoding.
  \item Improve memory usage of the Isabelle/ML implementation
  platform.
\end{enumerate}
