%[TU München]

Isabelle as a logical framework \cite{paulson700} is an intermediate
between Type-Theory provers (like Coq or Agda) and classic LCF-style
systems (like HOL Light or HOL4). The inference kernel can already
output proofs as $\lambda$-terms on request, but this has so far been
only used for small examples \cite{Berghofer-Nipkow:2000:TPHOL}. The
challenge is to make Isabelle proof terms work routinely for
reasonably big entries from The Archive of Formal Proofs
\cite{isabelle-afp}. Preliminary work by Wenzel (2019) has
demonstrated the feasibility for relatively small parts of
Isabelle/HOL: some orders of magnitude in scalability are still
missing.

This work package will revisit important aspects of the Isabelle/HOL
logic implementation on top of the Isabelle/Pure framework, such as
normalization of proofs, efficient type-class reasoning, special
representation of derived rules and definition principles (datatypes,
recursion, induction). The volume of proof term output may be reduced
further, by taking more structure of the target language (Dedukti)
into account and omitting certain low-level reasoning of HOL (e.g.\
for inductive types).

%[Subcontracted to Makarius Wenzel, Augsburg, Germany.]

The underlying Isabelle/ML implementation platform (on top of Poly/ML)
will be revisited as well, to improve monitoring of memory usage, and
to double the standard heap size from 16\,GB to 32\,GB (without
suffering from the full overhead of the 64\,bit
addressing).

%[Subcontracted to David Matthews, Edinburgh, UK.]
