To foster the interoperability of proof systems and the sustainability
and the cross-verification of formal proofs, we propose to collect
them in an online encyclopedia, called {\sf Logipedia}.  For each
proof, {\sf Logipedia} will indicate in which systems it can be used
and, when this is the case, it will provide a version of this proof in
the theory of these systems.

Such a project will not only foster the use of formal proofs in
research in mathematics and computer science, but also in industry, by
allowing cross-verification, sustainability, and interoperability of
formal proofs, and in education, freeing the teaching of formal proof
technology from being bound to one system.

Such a project can only have a worldwide ambition. However, as a
majority of proof systems are developed in Europe, there is a unique
opportunity for Europe to take the lead on such a project and prepare
the grounds for the economic spinoffs from the project benefiting
European industry. That is why the consortium gathers most of the
European actors active on formal proof systems, while also developing
links with non-European partners.

Building this encyclopedia is {\em per se} a networking activity
between the partners, from academia and industry, involved in the
project.  Currently, we know how to express the theories of {\sf HOL
  Light} \cite{Assaf12}, {\sf Matita} \cite{Assaf15}, {\sf Coq} and
{\sf FoCaLiZe} \cite{Cauderlier16} in {\sf Dedukti} and recheck proofs
developed in these systems.  We aim at including, in twenty years, all
formal proofs developed at that time.
In the next four years, we plan to address the theories of {\sf
  Abella}, {\sf Agda}, {\sf Atelier B}, {\sf HOL4}, {\sf Isabelle},
{\sf Minlog}, {\sf Mizar}, {\sf SMT-Lib}, {\sf TLA+}, {\sf Why3}, {\sf
  LFSC}, {\sf PVS}, and {\sf TSTP}.  Other systems, such as {\sf
  ACL2}, {\sf IMPS}, {\sf Lean}, {\sf Nuprl}, and {\sf Rodin}, are
kept for later, except if some other partners join the project (Figure
\ref{systems}). This effort will require, and contribute, to build a
network of research teams, much stronger than what we currently have.

Beyond our main focus on interactive systems, we also plan to
integrate some proofs coming from automated theorem provers, SMT
solvers, and model checkers, when these proofs have a manageable
size. We already have experience with Archsat \cite{Bury19}, iProver
\cite{Burel10}, and Zenon \cite{CauderlierHalmagrand15}. We plan to go
further in this direction, in cooperation with our colleagues working
on LFSC \cite{Stump09}. Thus beyond the teams focused on formal proofs, we
will extend this network to teams of neighbour communities, such as
the automated theorem proving community and the SAT/SMT community.

\begin{figure}[t]
\begin{framed}
\begin{itemize}
\item[LIL 1:]
The theory implemented in the system has been defined in
the $\lambda\Pi$-calculus modulo theory and in {\sf Dedukti}.

\item[LIL 2:]
The system has been instrumented so some of its proofs can be exported
and checked in {\sf Dedukti}.

\item[LIL 3:] A significant part of the library of the system has been
  exported and checked in {\sf Dedukti}.

\item[LIL 4:] A significant part of the library of the system have
  been made available in {\sf Logipedia}.

\item[LIL 5:]
A tool has been defined to analyze the {\sf Dedukti} proofs for the system,
detect those that can be expressed in a theory weaker than that of the
system, and translate those proofs into a weaker logic.

\item[LIL 6:]
All proofs of the system have been exported, translated,
and made available in {\sf Logipedia}.
\end{itemize}
\caption{The {\sf Logipedia} integration levels (LIL)\label{lil}}
\end{framed}
\end{figure}

To measure the level of integration of an existing proof system and
associated proof library in {\sf Logipedia}, we introduce a metric:
{\em the {\sf Logipedia} integration levels} (Figure \ref{lil}) that
counts six levels.  


Figure \ref{objectives} presents the various systems addressed in the
project, their current {\sc Logipedia} integration level and the
targeted integration level in four years.

\begin{figure}[t]
\begin{center}
\begin{tabular}{|l|c|c|}
\hline
System & current level & targeted level\\
\hline
{\sf Matita} & 5 & 6\\
\hline
{\sf HOL Light} & 3 & 5\\
\hline
{\sf FoCaLiZe} & 3 & 5\\
\hline
{\sf Coq} & 3 & 5\\
\hline
{\sf Agda} & 2 & 4\\
\hline
{\sf Atelier B} & 1 & 5\\
\hline
{\sf Isabelle} & 2 & 5\\
\hline
{\sf HOL4} & 1 & 5\\
\hline
{\sf TSTP} & 1 & {\color{red} ???}\\
\hline
{\sf Minlog} & 0 & 4\\
\hline
{\sf PVS} & 0 & 2\\
\hline
{\sf Abella} & 0 & 2\\
\hline
{\sf Mizar} & 0 & 4\\
\hline
{\sf TLA+} & 0 & 2\\
\hline
{\sf SMT-Lib} & 0 & {\color{red} ???}\\
\hline
{\sf Why3} & 0 & {\color{red} ???}\\
\hline
{\sf LFSC} & 0 & {\color{red} ???}\\
\hline
\end{tabular}
\end{center}
\caption{The objectives system by system \label{objectives}}
\end{figure}

These systems can be roughly divided into two groups. Those in the
first half of the array ({\sf Matita}, {\sf HOL Light}, {\sf
  FoCaLize}, {\sf Coq}, {\sf Agda}, {\sf Atelier B}, {\sf Isabelle},
{\sf HOL4}, and {\sf TSTP}) with which we already have some experience
and that we plan to bring to a very high level of integration. Those
in the second half ({\sf Minlog}, {\sf PVS}, {\sf Abella}, {\sf
  Mizar}, {\sf TLA+}, {\sf SMT-Lib}, {\sf Why3}, and {\sf LFSC}) with
which we are starting to experiment and for which our goal are less
ambitious.  These two groups of systems will be addressed in different
work packages, but both are key to the project. The first ones will
constitute {\sf Logipedia} in 2024 and the second ones will be
included between 2024 and 2028, on a potential second phase of the
project.

{\sf Logipedia} will be freely accessible through a web browser, from
any country in Europe and beyond. So the access will be trans-national
and virtual. But beyond these two objectives of a trans-national and
virtual access, an important effort will be made to make this
encyclopedia accessible to a large community of specialists and
non-specialists. This requires to structure its content into
libraries, books, chapters, etc. Some of the libraries we start with
already have a structure (modules, qualified names, etc.) that it is
important to preserve. We also need to develop ergonomic interfaces,
search engines, etc.

Finally, this project will trigger joint research activities between
its users and between its developers.  First, on the theories in the
second half of the array (Figure \ref{objectives}), that have yet
reached level 1, and that we still need to express in {\sf
  Dedukti}. Then, to bring the other theories to level 5, we need to
define algorithms to eliminate some axioms from proofs. Many such
algorithms already exist to eliminate the excluded middle, the axiom
of choice, or the universes from a given proof. We need to develop new
ones, more specific to the theories we address in this project.

In addition, each library contains definitions of natural numbers,
real numbers, etc. and, most importantly, logical connectors, that
must be aligned: structural results proved for one definition of real
numbers, for instance, must be transported to any isomorphic structure, 
regardless the way it is defined.

All these objectives contribute to building a new formal proof
community, focused on the values of knowledge exchange and
sustainability.

%%% Local Variables:
%%%   mode: latex
%%%   mode: flyspell
%%%   ispell-local-dictionary: "english"
%%% End:

