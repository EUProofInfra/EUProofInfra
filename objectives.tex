To foster the interoperability of proof systems and the sustainability
and the certification of formal proofs, we propose to build an online
encyclopedia of formal proofs called {\sc Logipedia}, that indicates
which proof can be used in which system and, when it is the case,
provides a version of the proof in the theory of this system.  Such a
project will increase networking activities between its members and
also between the academic and industrial users of this encyclopedia.
This encyclopedia will be freely accessible through a web browser from
every country in Europe and beyond. This project will also trigger
joint research activities between its users and between its
developers.

{\sc Logipedia} aims at including, in twenty years, all formal proofs
developed at that time. To measure the level of integration of an
existing library in {\sc Logipedia}, we propose to define a metric:
{\em the {\sc Logipedia} readiness levels}, that counts six levels
(level 6 being the highest).

In the next four years, we propose to focus on the theory and library
of 15 systems (Figure \ref{systems}), and increase their {\sc
  Logipedia} readiness levels, bringing 10 of them to level 5 or 6.

Such a project can only have a worldwide ambition. However, as a
majority of proof systems are developed in Europe, Europe can take the
lead on this project, so that the economic spinoffs from the project
benefit the active participants mainly based in Europe.  That is why
the consortium gathers most of the European actors active on formal
proof systems, and also proposes to develop links with non-European
partners.

%%% Local Variables:
%%%   mode: latex
%%%   mode: flyspell
%%%   ispell-local-dictionary: "english"
%%% End:

