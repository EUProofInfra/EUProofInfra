To foster the interoperability of proof systems and the sustainability
and the cross-verification of formal proofs, we propose to build an
online encyclopedia of formal proofs called {\sc Logipedia}, that
indicates which proof can be used in which system and, when this is the
case, provides a version of the proof in the theory of this system.
Such a project will increase networking activities between its members
and also between the academic and industrial users of this
encyclopedia.  This encyclopedia will be freely accessible through a
web browser from any country in Europe and beyond. This project will
also trigger joint research activities between its users and between
its developers.

{\sc Logipedia} aims at including, in twenty years, all formal proofs developed
at that time, and in four years a significant part of it. To measure the level
of integration of an existing proof system and associated proof library in {\sc
  Logipedia}, we introduce a metric: {\em the {\sc Logipedia} readiness
  levels} (Figure \ref{lrl}) that counts six levels (level 6 being the highest).
In the next four years, we shall focus on the theory and library of 15
systems (Figure \ref{systems}), and increase their {\sc Logipedia} readiness
levels, bringing 10 of them to level 5 or 6.

Such a project can only have a worldwide ambition. However, as a majority of
proof systems are developed in Europe, there is a unique opportunity for Europe
to take the lead on this project and prepare the grounds for the economic
spinoffs from the project benefitting European industry. That is why the
consortium gathers most of the European actors active on formal proof systems,
while also developing links with non-European partners.

As we shall explain, such a project will not only foster the use of
formal proofs in research in mathematics and computer science, but
also in industry, by allowing cross-verification, sustainability, and
interoperability of formal proofs, and in education, freeing the
teaching of formal proof technology from being bound to one system. 

{\color{red} more details}

%%% Local Variables:
%%%   mode: latex
%%%   mode: flyspell
%%%   ispell-local-dictionary: "english"
%%% End:

