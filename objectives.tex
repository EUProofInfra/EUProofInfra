To foster the interoperability of proof systems and the sustainability
and the cross-verification of formal proofs, we propose to collect
them in an online encyclopedia, called Logipedia.  For each proof,
Logipedia will indicate in which systems it can be used and, when this
is the case, it will provide a version of this proof in the theory of
these systems.

Such a project will not only foster the use of formal proofs in
research in mathematics and computer science, but also in industry, by
allowing cross-verification, sustainability, and interoperability of
formal proofs, and in education, freeing the teaching of formal proof
technology from being bound to one system.

Such a project can only have a worldwide ambition. However, as a
majority of proof systems are developed in Europe, there is a unique
opportunity for Europe to take the lead on such a project and prepare
the grounds for the economic spinoffs from the project benefiting
European industry. That is why the consortium gathers most of the
European actors active on formal proof systems, while also developing
links with non-European partners.

The ultimate goal of Logipedia is to have all the formal proofs
available to mankind in a single encyclopedia.  A first proof of
concept contains a few hundred lemmas, from the Matita library,
expressed in the logic of six different systems: Matita, Coq, Lean,
HOL Light, Isabelle/HOL, and PVS.  In the next four years, we plan to
address, with different ambitions, the libraries Abella, Agda, Atelier
B, Coq, FoCaLiZe, HOL Light, HOL4, Isabelle, Matita, Minlog, Mizar,
ProB, PVS, TLA+, and Why3.  Beyond our main focus on interactive
systems, we also plan to integrate some proofs coming from automated
theorem provers, SMT solvers, and model checkers, when these proofs
have a manageable size.

%Convinced that such a cloud of formal proofs could bring to the
%applications of formal proof technology the same boost that the cloud
%has brought to computing, and also that managing such a large
%encyclopedia (for instance being able to query a proof with a search
%engine) required some interdisciplinary effort, we organized, in
%January 2019, a meeting to discuss the future of this project
%\url{http://deducteam.gforge.inria.fr/seminars/190121.html}.  This
%meeting brought together 38 researchers from Austria, the Czech
%Republic, France, Italy, the Netherlands, and Poland.  Since then,
%colleagues from Belgium, Germany, Serbia, Sweden, and the United
%Kingdom, from academia and industry, have manifested interest in
%participating in this effort.  These researchers and engineers are
%ready to contribute to develop this encyclopedia, aiming at sharing
%proofs, under a creative common licence making them findable,
%accessible, interoperable, and reusable.

Building this encyclopedia is {\em per se} a networking activity
between the partners, from academia and industry, involved in the
project.

Logipedia will be freely accessible through a web browser, from any
country in Europe and beyond. So, by construction, the access will be
trans-national and virtual. But beyond these two objectives of a
trans-national and virtual access, an important effort will be made to
make this encyclopedia accessible to a large community of specialists
and non-specialists. This requires to structure its content into
libraries, books, chapters, etc. Some of the libraries we start with
already have a structure (modules, qualified names, etc.) that we will
preserve. We will also develop ergonomic interfaces, search engines,
etc.

Finally, this project will trigger joint research activities between
its users and between its developers.  First, the theories implemented
in some systems have already been understood, so that the integration
of the proofs expressed in these systems and their translation to
other theories, can be made smoothly. For others, more research work
is needed. A research effort will also be needed to analyze the proofs and
translate them to other theories. 

In addition, each library contains definitions of natural numbers,
real numbers, etc. and, most importantly, logical connectors, that
we will align: structural results proved for one definition of real
numbers, for instance, will be transported to any isomorphic structure, 
regardless the way it has been defined.

All these objectives contribute to building a new formal proof
community, focused on the values of knowledge exchange and
sustainability.


\newpage

\begin{framed}
\begin{center}
{\bf Main objectives}
\end{center}

\begin{tabular}{l}
  Integration of proof systems\\
  Integration of automated theorem proving\\
  Integration of libraries\\
  \\
  Development of the infrastructure\\
  Structuring the infrastuture\\
  \\
  New theories, new systems\\
  Proof engineering\\
\end{tabular}
\end{framed}

%%% Local Variables:
%%%   mode: latex
%%%   mode: flyspell
%%%   ispell-local-dictionary: "english"
%%% End:

