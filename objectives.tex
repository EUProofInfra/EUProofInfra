\begin{figure}
\begin{framed}
  \begin{itemize}
\item[LIL 1:]
The theory implemented in the system has been defined in
the lambda-Pi-calculus modulo theory and in Dedukti.

\item[LIL 2:]
The system has been instrumented so some of its proofs can be exported
and checked in Dedukti.

\item[LIL 3:]
A significant part of the library of the system has been exported and checked in
Dedukti.

\item[LIL 4:]
A tool has been defined to analyze the Dedukti proofs for the system,
detect those that can be expressed in a theory weaker than that of the
system, and translate those proofs into a weaker logic.

\item[LIL 5:]
Certain proofs of the system have been made available in Logipedia.

\item[LIL 6:]
All proofs of the system have been exported, translated,
and made available in Logipedia.
\end{itemize}
\caption{The Logipedia integration levels (LIL)\label{lil}}
\end{framed}
\end{figure}

To foster the interoperability of proof systems and the sustainability
and the cross-verification of formal proofs, we propose to collect
them in an online encyclopedia, called {\sc Logipedia}.  For each
proof, {\sc Logipedia} will indicate in which systems it can be used
and, when this is the case, it will provide a version of this proof in
the theory of these systems.

Such a project will not only foster the use of formal proofs in
research in mathematics and computer science, but also in industry, by
allowing cross-verification, sustainability, and interoperability of
formal proofs, and in education, freeing the teaching of formal proof
technology from being bound to one system.

Such a project can only have a worldwide ambition. However, as a
majority of proof systems are developed in Europe, there is a unique
opportunity for Europe to take the lead on such a project and prepare
the grounds for the economic spinoffs from the project benefiting
European industry. That is why the consortium gathers most of the
European actors active on formal proof systems, while also developing
links with non-European partners.

Building this encyclopedia is {\em per se} a networking activity
between the partners, from academia and industry, involved in the
project.  Currently, we know how to express the theories of {\sc HOL
  Light} \cite{Assaf12}, {\sc Matita} \cite{Assaf15}, {\sc Coq} and
{\sc FoCaliZe} \cite{Cauderlier16} in {\sc Dedukti} and recheck proofs
developed in these systems.  We aim at including, in twenty years, all
formal proofs developed at that time.
In the next four years, we plan to address the theories of {\sc
  Abella}, {\sc Agda}, {\sc Atelier B}, {\sc HOL4}, {\sc Isabelle},
{\sc Minlog}, {\sc Mizar}, {\sc SMT-Lib}, \tlaplus, {\sc Why3}, {\sc
  LFSC}, {\sc PVS}, and {\sc TSTP}.  Other systems, such as {\sc
  ACL2}, {\sc IMPS}, {\sc Lean}, {\sc Nuprl}, and {\sc Rodin}, are
kept for later, except if some other partners join the project (Figure
\ref{systems}). This effort will require, and contribute, to build a
network of research teams, much stronger than what we currently have.

Beyond our main focus on interactive systems, we also plan to
integrate some proofs coming from automated theorem provers, SMT
solvers, and model checkers, when these proofs have a manageable
size. We already have experience with Archsat \cite{Bury19}, iProver
\cite{Burel10}, and Zenon \cite{CauderlierHalmagrand15}. We plan to go
further in this direction, in cooperation with our colleagues working
on LFSC \cite{Stump09}. Thus beyond the teams focused on formal proofs, we
will extend this network to teams of neighbour communities, such as
the automated theorem proving community and the SAT/SMT community.

To measure the level of integration of an existing proof system and
associated proof library in {\sc Logipedia}, we introduce a metric:
{\em the {\sc Logipedia} integration levels} (Figure \ref{lil}) that
counts six levels.  Among the 17 systems we shall focus on we plan to
increase the {\sc Logipedia} integration levels of 10 of them to level
5 or 6.

The various systems addressed in the project are currently at those levels:

\begin{tabular}{ll}
Matita:& level 5\\
FoCaliZe:& level 3\\
HOL Light:& level 5\\
Coq:& level 2\\
Agda:& level 2\\
Atelier B:& level 2\\
Isabelle:& level 2\\
HOL4:& level 1\\
Minlog:& level 0\\
PVS:& level 0\\
Abella:& level 0\\
Mizar:& level 0\\
TLA+:& level 0\\
SMT-Lib:& level 0\\
Why3:& level 0\\
LFSC:& level 0\\
TSTP:& level 0\\
\end{tabular}

and we plan to increase these levels to 

\begin{tabular}{ll}
Matita:& from level 5 to level 6\\
HOL Light:& from level 3 to level 5\\
FoCaliZe:& from level 3 to level 5\\
Coq:& from level 2 to level 5\\
Agda:& from level 2 to level 3\\
Atelier B:& from level 2 to level 5\\
Isabelle:& from level 2 to level 5\\
HOL4:& from level 1 to level 5\\
Minlog:& from level 0 to level 3\\
PVS:& from level 0 to level 2\\
Abella:& from level 0 to level 2\\
Mizar:& from level 0 to level 3\\
TLA+:& from level 0 to level 2\\
SMT-Lib:& from level 0 to level ???\\
Why3:& from level 0 to level ???\\
LFSC:& from level 0 to level ???\\
TSTP:& from level 0 to level ???\\
\end{tabular}

This encyclopedia will be freely accessible through a web browser,
from any country in Europe and beyond. But an important effort will be
made to make this encyclopedia accessible to a large community of
specialists and non-specialists, but structuring the contents on
libraries, books, chapters, etc. Some of the libraries we start with
already have a structure (modules, qualified names, etc.) that it is
important to preserve.

We shall also develop ergonomic interfaces, search engines, etc. 


Finally, this project will trigger joint research activities between
its users and between its developers.
First, for all the systems for which we have not yet





In addition, each library contains definitions of natural
numbers, real numbers, etc.\ and, most importantly, logical connectors,
that must be aligned.  All these objectives contribute to building a
new formal proof community, focused on the values of knowledge
exchange and sustainability.

%%% Local Variables:
%%%   mode: latex
%%%   mode: flyspell
%%%   ispell-local-dictionary: "english"
%%% End:

