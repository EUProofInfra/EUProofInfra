\thispagestyle{empty}

Bugs kill, and although testing may reveal some bugs, only formal
verification can guarantee their absence.  Thus, we
should never fly in an autonomous plane driven by a piece of software
that has not been formally verified.

The trust in critical systems today relies on formal verification, in
particular formal proofs, that guarantee the safety of the people
using transportation systems (autonomous cars, subways, trains,
planes, etc.), health systems (robotic surgery, etc.), energy provided
by nuclear plants, financial applications, e-governance, etc. 

This crucial role of formal proof is highlighted by several successes,
like the correctness proofs of the automatic Paris metro line 14
\cite{metro14}, the detect-and-avoid system for unmanned aircraft
system developed by NASA \cite{Munoz16}, the operating system seL4
\cite{Klein09}, or the C compiler CompCert \cite{Leroy06}.  It has
been empirically observed that compilers and operating systems often
have bugs, but proved ones, such as seL4 or CompCert have none, or much
fewer.  This is why certification processes require proofs, and not
only testing, at the highest Evaluation Assurance Levels of the Common
Criteria security evaluation international standard, in effect since
1999.  Because formal methods are crucial in the development of the
information society, because people can die and companies go bankrupt
because of a bug, it is crucial for Europe to master this technology
and its evolution.  Proof systems are also used to develop proofs in
mathematics, and several important advances, such as the proof of the
Feit-Thompson theorem \cite{Gonthier13} and Hales' theorem (Kepler's
conjecture) \cite{Hales17}, show the wide range of applications to of
formal proofs, from safety and security of software to mathematics.

These formal proofs are developed using research infrastrucures called
``proof systems''.  These proof systems allow computer scientists,
mathematicians, engineers, and logicians to build and study formal
proofs, just like particle accelerators allow physicists to build and
study particles.  Some twenty major proof systems exist in the world
(Figure \ref{systems}).

A lot of formal proofs developed for one critical system could be used
in another.  Unfortunately, the development of formal methods is
slowed down by the large number of proof systems (and sometimes the
large number of versions, over time, of one single system) and the
lack of a common theory used by these systems.  For instance, the
Paris metro line 14 has been proved correct in Atelier B, while the
Nasa detect-and-avoid system for unmanned aircraft system has been
proved correct in PVS, the seL4 operating system has been proved
correct in Isabelle/HOL, and the compiler CompCert has been proved
correct in Coq.  Some projects, such as the proof of Hales' theorem
(Kepler's conjecture) \cite{Hales17}, have been started in different
systems and required significant integration effort for obtaining the
overall result.

Thus, the development of formal methods is slowed down by the lack of
integration of these research infrastructures.  Because of this lack
of integration, each small community is centered around one theory and
one system. Each library is specific to one proof system, or often
even to a specific version of this system. In general, a library
developed in one system cannot be used in another, and when the system
is no longer maintained, the library may disappear.  Thus,
interoperability (the possibility for one user to use a proof
developed in another system), sustainability (the possibility to use a
proof decades after it has been developed), and cross-verification
(the possibility to have a higher assurance in the correctness of some
statement by verifying its proof in a system different from that in
which it has been constructed) are restricted.

The fragmentation of these infrastucutures hinders productivity
because foundational work (for instance, developing a calculus library
with theorems about the sinus and cosinus functions, derivatives,
etc.) has to be repeated instead of being reused.

It also limits the spreading of formal proofs in non-specialist
communities. For instance, teaching formal proving to undergraduate
students in a logic course is difficult, as it requires the choice of
a specific language, a specific theory and a specific system that
orients the course towards this language, theory, and system, instead
of orienting it to fundamental principles that are useful
everywhere. The same is true for the use of formal proofs in industry
or by working mathematicians.

On more philosophical grounds, while we had in the past an (informal)
proof of Pythagoras' theorem or Fermat's little theorem, the same
proof now has different formalizations in PVS, Isabelle/HOL, Coq, etc.
Thus, the universality of logical truth itself is jeopardized.

\begin{figure}
\begin{framed}
\begin{center}
\begin{tabular}{l@{\hspace{3cm}}l}
Abella                & Acl 2\\
\underline{Agda}      & \underline{HOL Light}\\
\underline{Atelier B} & IMPS\\
\underline{Coq}       & Lean\\
\underline{FoCaliZe}  & \underline{LFSC}\\
\underline{HOL4}      & Nuprl\\
\underline{Isabelle}  & \underline{PVS}\\
\underline{Matita}    & \underline{TSTP}\\
\underline{Minlog}\\  
\underline{Mizar}\\
\underline{ProB}\\
\underline{Rodin}\\
\underline{TLA+}\\
\underline{Smart}\\
\underline{Why3}\\
\end{tabular}
\end{center}
\caption{Some major proof systems or formats. The European ones are in the first column.
  Those addressed in the project are underlined\label{systems}}
\end{framed}
\end{figure}

Making these systems interoperable would avoid
duplication of work, reduce development time, enable
cross-verification, and make formal proofs accessible to a much larger
community.  After three decades dedicated to the development of these
systems, allowing such a cooperation between these systems is the next
step in the development of the formal proof technology.

Each proof system comes with
its own library, and these libraries are also part of research
infrastructure.  To address this challenge of the cooperation between
these infrastructure, we will develop an on line encyclopedia of
formal proofs, where each proof will be expressed in all the theories
where it can be expressed, so that it can be used in as many systems
as possible. This will allow interoperability, sustainability, and
cross-verification of formal proofs.

This encyclopedia will be called Logipedia.

Logipedia is thus a research infrastructure that integrates proof
systems, through data sharing.

Such an infrastructure is, in many ways, new in the European Strategy
on Research Infrastructures. We can even say that the idea to
structure a networking activity around the construction and the use of
a large scale infrastructure is relatively new in computer science and
mathematics, even if other projects, such as {\em OpenDreamKit} and
{\em Software Heritage} do exist. Our goal is therefore also trigger a
little, but significant, evolution on the organization of research in
computer science and mathematics in Europe.

\begin{framed}
\begin{center}
{\bf \Large How do proofs contribute to safety and security of software?}
\end{center}
  
Imagine the following casino game. At the beginning, a player is given
eleven euros. At each round, she throws a dice. If the results is a
six, then game is over.  If it is a five, a four, a three, or a two,
she is given twice the amount of money she already has. If it is a one
she is taken two euros, if she has at least two.  When the game is
over, the player wins the money she got, except if she has zero, in
which case she looses one million.

This game can be modeled by programme $p$
%import random 
\begin{verbatim}
n = 11
stay = True;
while stay: 
    roll = random.randint(1,6)
    if roll == 6:
        stay = False
    else:
        if roll >= 2:
            n = n + 2 * n
        else:
            if (n >= 2):
                n = n - 2
print(n)
\end{verbatim}

To be on the safe side, the player wants to be sure, before starting
playing, that she will never finish with zero.  And indeed, in all
runs, the content of the variable {\tt n} is always an odd number,
thus it cannot be zero. This property ``nothing bad happens'' is
called the ``safety'' of this programme. This property is a
consequence of two simple theorems of arithmetic
$$\forall x~(\mbox{\it odd}(x) \Rightarrow \mbox{\it odd}(x + 2 * x))$$
$$\forall x~(\mbox{\it odd}(x) \Rightarrow \mbox{\it odd}(x - 2))$$
Hence, verifying the safety of this programme, that is that the
proposition 
${\mbox{\it safe}}(p)$ holds, 
amounts to prove these three theorems of arithmetic.

A tiny bug in the programme, for instance replacing the {\tt 2} by a
{\tt 3} in the instruction {\tt n = n + 2 * n} makes the programme unsafe
as shown by the sequence $11, 9, 7, 5, 3, 1, 4, 2, 0$. Yet, testing
this programme will, most likely, not reveal this bug, that manifests
very rarely.  In contrast, attempting to prove the correctness of this
programme will
reveal the bug as it is impossible to prove the proposition
$$\forall x~(\mbox{\it odd}(x) \Rightarrow \mbox{\it odd}(x + 3 * x))$$
\end{framed}
\pagebreak
\begin{framed}
  \begin{center} {\bf \Large Non trivial mathematics are sometime
      needed to verify that a programme is correct}
\end{center}

Prime numbers are used in a very wide variety of applications:
pseudo-random number generation, cryptographic algorithms, etc.
The correctness of complex primality testing algorithms is 
critical due to this prevalent use of prime numbers.  

To do so, one needs first to pose a definition of primality, for instance

\begin{verbatim}
Definition prime p := 1 < p /\ (forall n, 1 < n < p -> ~ (n | p)).
\end{verbatim}

To lower the cost of primality testing, we use more efficient
primality tests.  For instance, a very simple improvement over the
naive use of the definition is to test the divisibility of the
considered number, not by each of the smaller natural numbers, but
only up to its square root.  Indeed, if a natural number $p$ can be
factored into the product of two smaller natural numbers $m$ and $n$,
one of them has to be smaller than or equal to the square root of $p$.
Therefore, one could give the following more complex, but more
efficient, definition.

\begin{verbatim}
Definition prime' p :=
  1 < p /\ (forall n, (1 < n /\ (n * n) <= p) -> ~ (n | p)).
\end{verbatim}

To ensure that these definitions are equivalent one needs to prove
the following statement.

\begin{verbatim}
Theorem prime_alt : forall p, prime' p <-> prime p.
\end{verbatim}

It would have been easy to introduce a small bug
in the new definition.  For example, one could have written
\verb!(n * n) < p! instead of \verb!(n * n) <= p!.

\begin{verbatim}
Definition prime'' p :=
  1 < p /\ (forall n, (1 < n /\ (n * n) < p) -> ~ (n | p)).
\end{verbatim}
This would have resulted in accepting squares of primes as primes.

Yet, attempting to prove the equivalence of these two definitions would have 
revealed the bug, as they are not provably equivalent.
\end{framed}
\pagebreak
\begin{framed}
  \begin{center}
    {\bf \Large What is a formal proof? What is a proof system?}
    \end{center}
        
Since Antiquity, we have known that
proofs, both purely mathematical ones, as in Euclid's elements or the
recent proof of the Kepler conjecture by Thomas Hales, and proofs used
to establish the safety and security of software, can be built with a
limited number of rules, for example
\begin{itemize}
\item From $A \Rightarrow B$ and $A$, deduce $B$.
\item From $A$, deduce $A~\mbox{\it or}~B$.
\item etc.
\end{itemize}
Yet, through history, most mathematical proofs have been written in
pidgin of natural language and mathematical formulas. When proofs are
very long (as it is often of the proofs used in safety and security,
but also of some proofs in pure mathematics), mistakes in proofs are
very difficult to detect. For instance dozen of wrong proofs of
parallel postulate have been given through history, sometimes by the
best mathematicians such as Ptolémée, Proclus, al-Haytam, Tacket,
Clairaut, Legendre, etc.

In the 1960s, Robin Milner and Nicolaas De Bruijn noticed that the
correctness of a mathematical proof could be checked by a
computer. This led to the development of the two first proof systems
in history: Milner's LCF or De Bruijn's Automath.  From
the axioms
$$\forall x~(\mbox{\em philosopher}(x) \Rightarrow \mbox{\em human}(x))$$
$$\forall x~(\mbox{\em human}(x) \Rightarrow \mbox{\em mortal}(x))$$
we can deduce
$$\forall x~(\mbox{\em philosopher}(x) \Rightarrow \mbox{\em mortal}(x))$$
In the language implemented in Automath, this proof is written
$$\lambda x \lambda h~(g~x~(f~x~h))$$
\end{framed}

\begin{framed}
\begin{center}
{\bf \Large What is a theory?}
\end{center}

Deduction rules such as ``From $A \Rightarrow B$ and $A$, conclude
$B$'' are universal, but building proofs require more rules, that are
often specific to a domain of knowledge and are called
``axioms''. Examples are the axioms of geometry, the axioms of
arithmetic, etc. These axioms constitute a theory.

At the beginning of the 20th century an axiomatic theory, {\em set
  theory}, has been proposed to express all mathematical proofs. In
the first half of the 20th century a few variants of set theory, and a
few alternatives have been proposed (such as Simple type theory).
But, because these theories had not been designed for being
implemented on a computer, the rise of computer checked formal proofs
has led to a multiplication of such alternative theories. Each proof system,
such as Coq, Isabelle/HOL, Mizar, Atelier B,
etc. implements its own theory.

This is the major obstacle to interoperability.
\end{framed}

%%% Local Variables:
%%%   mode: latex
%%%   mode: flyspell
%%%   ispell-local-dictionary: "english"
%%% End:
