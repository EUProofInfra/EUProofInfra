Bugs kill, and although testing may reveal some bugs, only formal modeling
and formal verification can guarantee their absence.  Thus, we should
never fly in an autonomous plane fully driven by a piece of software
that has not been formally verified.

Today, the trust in critical systems relies on formal verification, in
particular formal proofs, that guarantee the safety of the people using
transportation systems---autonomous cars, subways, trains, planes,
etc.---, health systems---robotic surgery, etc.---, energy provided by
nuclear plants, financial applications, e-governance, etc. The crucial
role of formal proof is highlighted by several successes, like the
correctness proofs of the automatic Paris metro line 14 \cite{metro14},
the detect-and-avoid system for unmanned aircraft system developed
by NASA \cite{Munoz16}, the operating system seL4 \cite{Klein09},
or the C compiler CompCert \cite{Leroy06}. Because formal methods are crucial in the
development of the information society, because people can die and
companies go bankrupt because of a bug, it is crucial for Europe to
master this technology and its evolution.

\thispagestyle{empty}

\begin{figure}
\begin{center}
\begin{tabular}{l@{\hspace{3cm}}l}
{\sc \underline{Abella}}    & {\sc Acl2}\\
{\sc \underline{Agda}}      & {\sc HOL Light}\\
{\sc \underline{Atelier B}} & {\sc IMPS}\\
{\sc \underline{Coq}}       & {\sc Lean}\\
{\sc FoCaliZe}              & {\sc Nuprl}\\
{\sc \underline{HOL4}}      & {\sc \underline{PVS}}\\
{\sc \underline{Isabelle}}  & {\sc \underline{TSTP}}\\
{\sc \underline{Matita}}    & {\sc \underline{LFSC}}\\
{\sc \underline{Minlog}}\\
{\sc \underline{Mizar}}\\
{\sc Rodin}\\
{\sc \underline{TLAPS}}\\
{\sc \underline{Why3}}\\
{\sc \underline{SMT-Lib}}\\
\end{tabular}
\caption{Some major proof systems or formats. The European ones are in the first column.
  Those addressed in the project are underlined\label{systems}}
\end{center}
\end{figure}

A lot of formal proofs developed for one application could be used in
another.  Unfortunately, the development of formal methods is slowed
down by the large number of proof systems and the lack of a common
theory used by these systems.  Because each small community is
centered around one theory and one system, and each system has its own
library of proofs, interoperability---that is the possibility for one
user to use a proof developed in another system---,
sustainability---that is the possibility to use a proof decades after
it has been developed---, and cross-verification---that is the
possibility to have a higher assurance in the correctness of some
statement by verifying its proof in a system different from that in
which it has been defined---are restricted, even if some provers
sometimes call external tools to help building some proofs. The fragmentation of
systems of formal proof also hinders productivity because foundational
work---for instance, developing a calculus library with theorems
about derivatives, sinus, cosinus, etc.---has to be repeated instead of
being reused.  For instance, the Paris metro line 14 has been proved
correct in {\sc Atelier B}, while the Nasa detect-and-avoid system for
unmanned aircraft system has been proved correct in {\sc PVS}. Some
projects, such as the Flyspeck project, have been started in different
systems and required significant integration effort for obtaining the overall
result.

Some twenty major proof assistants exist in the world (Figure
\ref{systems}), and making these systems interoperable would avoid
duplication of work, reduce development time, and enable
cross-verification.  After three decades dedicated to the development of
these systems, allowing such a cooperation between these systems is
the next step in the development of the formal proof technology.  

In January 2019, we put online a first prototype of {\sc Logipedia},
an encyclopedia of formal proofs, expressed in the languages of five
different systems. At that time, this encyclopedia contained only a
few hundred proofs.  Convinced that such a cloud of formal proofs
could bring to the applications of formal proof technology the same
boost that the cloud has brought to computing, and also that managing
such a large encyclopedia---for instance being able to query a proof
with a search engine---required some interdisciplinary effort, we
organized a meeting to discuss the future of this project
\url{http://deducteam.gforge.inria.fr/seminars/190121.html}.  This
meeting brought together 38 researchers from Austria, the Czech Republic,
France, Italy, the Netherlands, and Poland. Since then, colleagues
from Belgium, Germany, Serbia, Sweden, and the United Kingdom have
manifested interest in participating in this effort.
These researchers are ready to contribute to
develop this encyclopedia, aiming at having in twenty years all the
formal proofs then developed, in a single encyclopedia.


{\color{red} How safety and security require formal proofs?
  Julien and Bu

  Many bugs in compilers but a proved compiler does not have bugs}

Many bugs in operation but a proved os does not have bugs}

}

  {\color{red} Informal proofs are not always correct

    What is a formal proof

    Certfification processes requiire formal proofs level EAL 6 ad 7}

  }






%%% Local Variables:
%%%   mode: latex
%%%   mode: flyspell
%%%   ispell-local-dictionary: "english"
%%% End:
