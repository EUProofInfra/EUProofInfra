\thispagestyle{empty}

Bugs kill, and although testing may reveal some bugs, only formal
modeling and verification can guarantee their absence.  Thus, we
should never fly in an autonomous plane driven by a piece of software
that has not been formally verified.

The trust in critical systems today relies on formal verification, in
particular formal proofs, that guarantee the
safety of the people using transportation systems (autonomous cars,
subways, trains, planes, etc.), health systems (robotic surgery,
etc.), energy provided by nuclear plants, financial applications,
e-governance, etc. This crucial role of formal proof is highlighted by
several successes, like the correctness proofs of the automatic Paris
metro line 14 \cite{metro14}, the detect-and-avoid system for unmanned
aircraft system developed by NASA \cite{Munoz16}, the operating system
seL4 \cite{Klein09}, or the C compiler CompCert \cite{Leroy06}.  It
has been empirically observed that compilers and operating systems
often have bugs, but proved ones, such as seL4 or CompCert have none
or much fewer.  This is why certification processes require proofs,
and not only testing, at the highest Evaluation Assurance Levels of
the Common Criteria security evaluation international standard, in
effect since 1999.  Because formal methods are crucial in the
development of the information society, because people can die and
companies go bankrupt because of a bug, it is crucial for Europe to
master this technology and its evolution.

A lot of formal proofs developed for one critical system could be used
in another.  Unfortunately, the development of formal methods is
slowed down by the large number of proof systems (and sometimes the
large number of versions, over time, of one single system) and the
lack of a common theory used by these systems.  For instance, the
Paris metro line 14 has been proved correct in Atelier B, while the
Nasa detect-and-avoid system for unmanned aircraft system has been
proved correct in PVS, the seL4 operating system has been proved
correct in Isabelle/HOL, and the compiler CompCert has been proved
correct in Coq.  Some projects, such as the proof of Hales' theorem
(Kepler's conjecture) \cite{Hales17}, have been started in different
systems and required significant integration effort for obtaining the
overall result.

Thus, these proof systems are infrastructures and these
infrastructures have not been integrated yet.  Because of this lack of
integration, each small community is centered around one theory and
one system. Each library is specific to one proof system, or often
even to a specific version of this system. In general, a library
developed in one system cannot be used in another, and when the system
is no longer maintained, the library may disappear.  Thus,
interoperability (the possibility for one user to use a proof
developed in another system), sustainability (the possibility to use a
proof decades after it has been developed), and cross-verification
(the possibility to have a higher assurance in the correctness of some
statement by verifying its proof in a system different from that in
which it has been constructed) are restricted.

The fragmentation of proof systems hinders productivity
because foundational work (for instance, developing a calculus library
with theorems about the sinus and cosinus functions, derivatives,
etc.) has to be repeated instead of being reused.

This fragmentation limits the spreading of formal proofs in
non-specialist communities. For instance, teaching formal proving to
undergraduate students in a logic course is difficult, as it requires
the choice of a specific language, a specific theory and a specific
system that orients the course towards this language, theory, and
system, instead of orienting it to fundamental principles that are
useful everywhere. The same is true for the use of formal proofs in
industry or by working mathematicians.

On more philosophical grounds, while we had in the past an (informal)
proof of Pythagoras' theorem or Fermat's little theorem, the same
proof now has different formalizations in PVS, Isabelle/HOL, Coq, etc.
Thus, the universality of logical truth itself is jeopardized.

\begin{figure}
\begin{framed}
\begin{center}
\begin{tabular}{l@{\hspace{3cm}}l}
\underline{Abella}    & Acl 2\\
\underline{Agda}      & \underline{HOL Light}\\
\underline{Atelier B} & IMPS\\
\underline{Coq}       & Lean\\
\underline{FoCaliZe}  & \underline{LFSC}\\
\underline{HOL4}      & Nuprl\\
\underline{Isabelle}  & \underline{PVS}\\
\underline{Matita}    &  \underline{TSTP}\\
\underline{Minlog}\\  
\underline{Mizar}\\
\underline{ProB}\\
Rodin\\
\underline{TLA+}\\
\underline{Smart}\\
\underline{Why3}\\
\end{tabular}
\end{center}
\caption{Some major proof systems or formats. The European ones are in the first column.
  Those addressed in the project are underlined\label{systems}}
\end{framed}
\end{figure}

Some twenty major proof assistants exist in the world (Figure
\ref{systems}), that are used both for proving properties of software
and results in pure mathematics, such as the Feit-Thompson theorem
\cite{Gonthier13} or Hales' theorem (Kepler's conjecture)
\cite{Hales17}.  Making these systems interoperable would avoid
duplication of work, reduce development time, enable
cross-verification, and make formal proofs accessible to a much larger
community.  After three decades dedicated to the development of these
systems, allowing such a cooperation between these systems is the next
step in the development of the formal proof technology.

These proof systems are research infrastructures, as they allow
computer scientists, mathematicians, engineers, and logicians to build
and study formal proofs, just like particle accelerators allow
physicists to build and study particles. Each proof system comes with
its own library, and these libraries are also part of research
infrastructure.  To address this challenge of the cooperation between
these infrastructure, we will develop an on line encyclopedia of
formal proofs, where each proof will be expressed in all the theories
where it can be expressed, so that it can be used in as many systems
as possible. This will allow interoperability, sustainability, and
cross-verification of formal proofs.

This encyclopedia will be called Logipedia.

Logipedia is thus a research infrastructure that permits the
cooperation between these proof systems, by allowing to share the data
processed by these infrastructures.

Such an infrastructure is, in many ways, new in the European Strategy
on Research Infrastructures. We can even say that the idea to
structure a networking activity around the construction and the use
of a large scale infrastructure is relatively new in computer science and
mathematics, even if other projects, such as {\em OpenDreamKit} and
{\em Software Heritage} do exist. Our goal is therefore also trigger a
little, but significative, evolution on the organization of research
in computer science and mathematics in Europe.

%%% Local Variables:
%%%   mode: latex
%%%   mode: flyspell
%%%   ispell-local-dictionary: "english"
%%% End:
