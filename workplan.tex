{\color{red} Where should this sentence go?: The Logipedia kick-off meeting
http://deducteam.gforge.inria.fr/seminars/190121.html, in January
2019, brought together 38 researchers from Austria, Czech Republic,
France, Italy, the Netherlands, and Poland. Since then, colleagues
from Belgium, Germany, Serbia, Sweden, and the United Kingdom have
manifested interest. Some of these researchers are ready to contribute
to Logipedia, that currently contains a few hundred lemmas, aiming at
having in twenty years all the formal proofs then developed, in a
single encyclopedia.}

Currently, we know how to express the theories of {\sc HOL Light},
{\sc Matita}, and {\sc FoCaliZe} in {\sc Dedukti} and recheck proofs
developed in these systems. In the next five years, we plan to address
the theories of {\sc Abella}, {\sc Agda}, {\sc Atelier B}, {\sc Coq},
{\sc HOL4}, {\sc Isabelle}, {\sc Lean}, {\sc Minlog}, {\sc Mizar},
{\sc PVS}, {\sc Rodin}, and {\sc TLA+}. Other systems, {\sc ACL2},
{\sc IMPS}, and {\sc Nuprl} are kept for later, except if some other
groups join the project.

\begin{figure}
\begin{itemize}
\item[Level 1:]
The theory implemented in the system X has been defined in
the lambda-Pi-calculus modulo theory and in Dedukti.

\item[Level 2:]
The system X has been instrumented to export of proof that
can be checked in Dedukti.

\item[Level 3:]
A significant part of the library of the system X has been exported and checked in Dedukti.

\item[Level 4:]
A tool has been defined to analyze the Dedukti X proofs,
detect those that can be expressed in a theory weaker than that of the
system X and translate those proofs in a weaker logic.

\item[Level 5:]
These proofs have been made available in Logipedia.

\item[Level 6:]
All proofs of the system X have been exported, translated
and made available in Logipedia.
\end{itemize}
\caption{The Logipedia readiness levels \label{lrl}}
\end{figure}

We propose to introduce a metric to measure the progress of the
integration of a library in Logipedia (Figure \ref{lrl})

The various systems addressed in the project are currently at this level 

\begin{tabular}{ll}
Matita:& level 5\\
FoCaliZe:& level 3\\
HOL Light:& level 3\\
Coq:& level 3\\
Agda:& level 2\\
Atelier B:& level 2\\
Isabelle:& level 2\\
HOL4:& level 1\\
Minlog:& level 1\\
Rodin:& level 1\\
Lean:& level 0 to 1\\
PVS:& level 0 to 1\\
Abella:& level 0\\
Mizar:& level 0\\
TLA+:& level 0
\end{tabular}

\subsection{Goals}

Specific, Measurable, Achievable, Relevant, Time-Bound

\begin{tabular}{ll}
Matita:& from level 5 to level 6\\
HOL Light:& from level 3 to level 5\\
FoCaliZe:& from level 3 to level 5\\
Coq:& from level 3 to level 5\\
Agda:& from level 2 to level 3\\
Atelier B:& from level 2 to level 5\\
Isabelle:& from level 2 to level 5\\
HOL4:& from level 1 to level 5\\
Minlog:& from level 1 to level 5\\
Rodin:& from level 1 to level 5\\
Lean:& from level 0 to 1 to level 3\\
PVS:& from level 0 to 1 to level 2\\
Abella:& from level 0 to level 2\\
Mizar:& from level 0 to level 5\\
TLA+:& from level 0 to level 2
\end{tabular}


Beyond our main focus on interactive systems, we also plan to
integrate some proofs coming from automated theorem provers, SMT
solvers, and model checkers, when these proofs have a reasonable
size. We already have experience with Zenon, iProver, and Archsat, but
we also plan to go in this direction, in cooperation with our
colleagues working on LFSC [Stump09].

Finally, we must also structure this encyclopedia: some of the
libraries we start with already have a structure (modules, qualified
names, etc.) that it is important to preserve.

But, in addition, each library contains a definition of natural
numbers, real numbers, etc. and, most importantly, logical connectors,
that must be aligned.  All these objectives contribute to building a
new formal proof community, focused on the values of knowledge
exchange and sustainability.


