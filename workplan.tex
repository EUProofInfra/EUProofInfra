Currently, we know how to express the theories of {\sc HOL Light},
{\sc Matita}, {\sc Coq} and {\sc FoCaliZe} in {\sc Dedukti} and recheck proofs
developed in these systems. In the next five years, we plan to address
the theories of {\sc Abella}, {\sc Agda}, {\sc Atelier B},
{\sc HOL4}, {\sc Isabelle}, {\sc Minlog}, {\sc Mizar},
{\sc PVS}, and \tlaplus. Other systems, such as {\sc Lean}, {\sc Rodin}, {\sc ACL2},
{\sc IMPS}, and {\sc Nuprl} are kept for later, except if some other
groups join the project.

We introduce a metric to measure the progress of the
integration of a mechanized proof system and its corresponding proof library in
Logipedia: the Logipedia Readiness Level (LRL), cf.\ Figure \ref{lrl}.

\begin{figure}[ht]
\begin{itemize}
\item[LRL level 1:]
The theory implemented in the system has been defined in
the lambda-Pi-calculus modulo theory and in Dedukti.

\item[LRL level 2:]
The system has been instrumented so some of its proofs can be exported
and checked in Dedukti.

\item[LRL level 3:]
A significant part of the library of the system has been exported and checked in
Dedukti.

\item[LRL level 4:]
A tool has been defined to analyze the Dedukti proofs for the system,
detect those that can be expressed in a theory weaker than that of the
system, and translate those proofs into a weaker logic.

\item[LRL level 5:]
Certain proofs of the system have been made available in Logipedia.

\item[LRL level 6:]
All proofs of the system have been exported, translated,
and made available in Logipedia.
\end{itemize}
\caption{The Logipedia readiness levels (LRL)\label{lrl}}
\end{figure}

The various systems addressed in the project are currently at those levels:

\begin{tabular}{ll}
Matita:& level 5\\
FoCaliZe:& level 3\\
HOL Light:& level 5\\
Coq:& level 2\\
Agda:& level 2\\
Atelier B:& level 2\\
Isabelle:& level 2\\
HOL4:& level 1\\
Minlog:& level 0\\
Rodin:& level 1\\
Lean:& level 1\\
PVS:& level 0\\
Abella:& level 0\\
Mizar:& level 0\\
TLA+:& level 0
\end{tabular}

\subsection{Goals}

Specific, Measurable, Achievable, Relevant, Time-Bound

\begin{tabular}{ll}
Matita:& from level 5 to level 6\\
%HOL Light:& from level 3 to level 5\\
%FoCaliZe:& from level 3 to level 5\\
Coq:& from level 2 to level 5\\
Agda:& from level 2 to level 3\\
Atelier B:& from level 2 to level 5\\
Isabelle:& from level 2 to level 5\\
HOL4:& from level 1 to level 5\\
Minlog:& from level 0 to level 3\\
%Rodin:& from level 1 to level 5\\
%Lean:& from level 0 to 1 to level 3\\
PVS:& from level 0 to level 2\\
Abella:& from level 0 to level 2\\
Mizar:& from level 0 to level 3\\
TLA+:& from level 0 to level 2
\end{tabular}


Beyond our main focus on interactive systems, we also plan to
integrate some proofs coming from automated theorem provers, SMT
solvers, and model checkers, when these proofs have a manageable
size. We already have experience with Zenon, iProver, and Archsat, but
we plan to go further in this direction, in cooperation with our
colleagues working on LFSC \cite{Stump09}.

Finally, we must also structure this encyclopedia: some of the
libraries we start with already have a structure (modules, qualified
names, etc.) that it is important to preserve.

But, in addition, each library contains definitions of natural
numbers, real numbers, etc.\ and, most importantly, logical connectors,
that must be aligned.  All these objectives contribute to building a
new formal proof community, focused on the values of knowledge
exchange and sustainability.


