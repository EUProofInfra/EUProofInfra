\chapter{Impact}\label{chap:impact}

\section{Expected impacts}\label{sec:expected-impact}

%\begin{todo}{}\color{red} Please be specific, and provide only
%  information that applies to the proposal and its
%  objectives. Wherever possible, use quantified indicators and
%  targets.
  
%  * Describe how your project will contribute to:
  
%     - each of the expected impacts mentioned in the work programme,
%     - under the relevant topic;
  
%      - where relevant, any substantial impacts not mentioned in the
%      - work programme, that would enhance innovation capacity;
%      - create new market opportunities, strengthen competitiveness
%      - and growth of companies, address issues related to climate
%      - change or the environment, or bring other important benefits
%      - for society
  
%  * Describe any barriers/obstacles, and any framework conditions
%  * (such as regulation, standards, public acceptance, workforce
%  * considerations, financing of follow-up steps, cooperation of
%  * other links in the value chain), that may determine whether and
%  * to what extent the expected impacts will be achieved. (This
%  * should not include any risk factors concerning implementation, as
%  * covered in section 3.2.)
%\end{todo}

\subsection{Wider, simplified, and more efficient access}

The shift from informal, pencil and paper, proofs to formal
computerized proof has been a major improvement on the never ending
quest for logical rigor, with a strong impact both on computer
science, where safety and security can be dramatically improved with
the use of formal methods and on mathematics, where much more complex
proofs can be built.

But this major step forward also has a negative side effect: we have
evolved from a time where we had (informal) proofs of Fermat's little
theorem, to a time where we have (formal) proofs in Coq, in Matita, in
HOL Light, in PVS, etc.  of this theorem, jeopardizing the
universality of logical truth.

This loss of universality of logical truth is the main obstacle to the
access to formal proofs in the communities of computer scientists and
mathematicians, but also researchers of other disciplines, engineers,
and students. As a consequence, the use of formal proofs is restricted
to a community of specialists, that has fortunately been growing with
time, but that is still too small, compared to the needs, for formal
proofs in the digital society.  Our long-term goal is to resurrect the
universality of logical truth, in order to build a strong formal proof
community including specialists and non-specialists, making
formal proofs findable, accessible, interoperable, and reusable.

This requires to express the theories implemented in these systems in
a common logical framework, with axioms and rewrite rules, in order to
be able to say, not that a proof is expressed in the theory of one
system or of another, but which axioms and rewrite rules are used in
this proof, as we have been used to since the development of
non-Euclidean geometries. In such a shared public encyclopedia
providing virtual access, each user, in research, industry, and
education, can find the formal proofs she needs in the logic she
wants, regardless the logic and system this proof has been developed
in.  Alignment of isomorphic structures, both across libraries, and
within each library, provides a wider and simplified access.

Because it constitutes a central place, such a encyclopedia can foster
projects that would make less sense for a specific library. For instance,

\begin{itemize}
\item indexing mechanisms for mathematical formulas, and search engines
  that are specialized to query such formulas,

  \item a structure of mathematical knowledge in theories, books,
    chapters, categories of users (high school students, researchers),
    that can be inherited from the libraries imported in 
      Logipedia, from concept alignment, and from clustering
    algorithms in the dependence graph of the encyclopedia,

  \item ergonomic user interfaces, that allows a navigation in the
    structure of the encyclopedia, and that can be specialized from
    some domains (safety, geometry...) or some category of users,

  \item programmatic access through an Application Programming
    Interface, and a package distribution system, so that the
    mathematical knowledge can be used not only by humans, but also by
    software.
\end{itemize}

Because of its size, such an encyclopedia, provides a better dataset
for machine learning algorithms, and more generally statistical
analysis of the properties of mathematical developments.

\subsection{New or more advanced research infrastructure}

Proof systems are research infrastructure. Logipedia is not yet
another infrastructure of this kind, but it is a research
infrastructure of a completely new kind, that integrates proof systems
through data sharing.  Such as new research infrastructure will, of
course, impact research in many ways.

\begin{itemize}

\item Computer scientists will be able to prove safety and security
  properties of the software they develop faster and at a lower cost
  because they can access to already developed proofs, independently
  of the system they use.

\item When building new proof systems, computer scientists will not
  need to start from scratch, but will be able to start with an already
  existing data base of proofs.

\item In Automated Theorem Proving, computer scientists will be able
  to use already proved theorems as axioms, to enhance the power of
  their tools.

\item In their publications, computer scientists often provide access
  to formal proofs of the results they present (as required or
  suggested by several conferences and journals). They will be able to
  use Logipedia as a universal repository for such proofs, some of
  which would be available across systems, and these proofs being
  available for a long time.

\item Computer scientists conducting research on machine learning
  using mathematical dataset will have access to a wider database.

\item When they are unsure of the correctness of a proof,
  mathematicians will be able to formalize it faster and at a lower
  cost because they can access to already developed proofs,
  independently of the system they use.

\item Mathematicians who refer to a formal proof in one or their
  publication will be able to use Logipedia as a universal
  repository for such proofs, some of which would be available across
  systems.

\item Mathematicians interested in reverse mathematics will be able to
  analyze, in an easier way, the axioms used in each proof, when they
  have access to this proof formalized.

\item The results will be archived for a long time increasing the
  reproducibility of results, both in mathematics and computer science.

\item As mathematics and software are in any part of modern science,
  Logipedia will also have an impact on other sciences. In
  particular because proving properties of a piece of software driving
  a car or piloting an aircraft require to formalize part of the
  physical world in which this piece of software evolves.
\end{itemize}
  
\subsection{Operators develop synergies}

The formal proof community is currently a scattered community, each
sub-community being centered around its own system, its own theory,
and its own library.

To build a stronger community, where the researchers develop
strategies, it is not sufficient to talk to each other, or to organize
conferences, but these sub-communities must exchange data and work
together on a joint project to exchange those data.
Expressing these data in a common encyclopedia will
lead the developers of various systems to express the theories they
implement in a common logical framework, yielding a better understanding
to the similarities and differences between these theories.
Developing synergies will induce less work duplication and will increase
the efficiency of the community as a whole.

Working on common projects will not only increase the communication
between relatively close communities, such as the Coq and 
  Agda communities that meet every year at the TYPES conference, but
also to more distant communities, such as the TYPES community, the HOL
community (that already meet around the OpenTheory standard),
the B and TLA+ communities, and the Mizar community.

More importantly, this data exchange between researchers and
engineers, and this evolution towards a standard, will allow a better
cooperation between research and industry and suppress one of the main
obstacles to the diffusion of formal proofs in industry.

This evolution towards a standard and this resurrection of the
universality of logical truth will also suppress one of the main
obstacles to the diffusion of formal proofs in the community of
working mathematicians and in and education, just like the development
of the Html standard induced a renewal of document sharing in general
and the definition of predicate logic induced a renewal of logic in
the 1930's.

Sharing a logical framework will also allow new synergies between 
the formal proof community and the Automated Theorem Proving community.

It will also allow a better communication with machine learning community.

We have already organized two Logipedia workshops that have proven to
be very valuable to develop joint project and synergies.  This project
will permit to organize wider international events on this topic of
sharing formal proofs, in academia, industry, education, and
publishing.

\subsection{Innovation is fostered through a reinforced partnership of
research infrastructures with industry}

Formal methods are now an important part of some advanced industrial
projects. Mastering formal methods is key to give Europe a competitive
advantage in conquering the market of autonomous cars, trains, planes
or drones, or the block-chains and crypto-currencies. But this
penetration of formal methods in industry hits the same obstacle that
researchers often promote one method, theory or system, while their
industrial partners are in search of universality. We expect to make
formal proofs more accessible to industry by avoiding each project to
redevelop elementary proofs, but instead benefit of the formalization
work shared with other communities.

Several European companies are member of the project,
as contributors or as members of the future Logipedia
club of industrial users.

From the industry point of view the key exploitable results are threefold:

\begin{enumerate}
\item Cross-verification.

In the current state of affairs, when a company proves a piece of
software correct using a system $X$, its client, or the certification
authorities can check the proof developed by this company, but then
need to use the same system $X$ to check this proof, so they need to
trust the system $X$, which is a limitation. Several actors want to be
able to check the proofs using an independent proof system, and even
one they have developed themselves.

A side effect of the construction of the Logipedia platform is
to incent all the proof systems to be able to produce proofs in a
common language. Such proofs can then be all be checked by
Dedukti, and several other systems.

Moreover, the certification authorities can develop their own
proof-checker (the development of such a proof-checker takes a few
weeks) so that they do not need to trust anyone else.

\item Towards standardization.

The lack of standards is currently a major obstacle to the development
of formal methods in industry. Although we consider that starting a
standardization process is still premature, this project will allow us
to experiment with a common language that we shall improve until we
reach the point where it can be proposed as a standard.

\item Sustainability.

When an industrial project is over, it is sometimes difficult,
specially for small companies, to keep an archive of their work over
a long period of time. In formal methods, most of the projects are a
two-stages rocket. The first stage of the rocket contains basic
developments that can be shared between the company and its
competitors. The second contains developments that are specific to the
project and that often contain industrial secrets.

Sharing the first stage on a public encyclopedia, while keeping the
second secret, contributes to the sustainability of the
developments. They can, for instance, be reused decades later, and
cannot be lost by the company.

\item Interoperability.

  Some industrial programme verification tools rely only on one single
  proof system and any missing feature of its library forces the
  end-user to prove complicated theorems that probably exist in other
  proof systems. Logipedia will make it possible to access all
  standard libraries and proofs of all systems. It makes the overall
  proof of programmes much less costly.

  Other industrial programme verification tools use several proof
  systems. First because some proof systems are better for some
  application domains and other are better for others, and also
  because these companies hire researchers and engineers that have
  different cultures and are more efficient using the tools they know.

  A side effect of the construction of the Logipedia platform is that
  such development are made interoperable. First because a proof
  developed in one system can be translated into another. Second
  because proofs developed in different systems can be combined in
  Logipedia itself.
\end{enumerate}

The participation of industry to the project is twofold.  First, CEA
LIST, Clearsy, Edukera, MED-EL, OCamlPro, Prove\&Run, and SystemX, are
full partners of the project and will contribute to its work packages
and tasks.

Second, we will build over the project duration time a {\em club of
  industrial users of Logipedia}. This club already contains
18 members.

\begin{framed}
\begin{center}
  {\bf \Large The current members of the club of industrial users}
\end{center}
\begin{itemize}
\item Alstom,
\item CEA LIST,
\item ClearSy,
\item Edukera,
\item Facebook France,
\item IBM Research,
\item MED-EL
\item Mitsubishi Electric R\&D Centre Europe,
\item Nomadic Labs,
\item OCamlPro,
\item Origin Labs,
\item Prove\&Run,
\item TrustInSoft,
\item RATP,
\item Siemens,
\item System X,
\item Systerel,
\item TrustInSoft.
\end{itemize}

ANSSI?

\end{framed}

These companies are working in the area of transportation, health
care, energy, cybersecurity, block-chain...

This club will organize two of meetings every year where the
industrial members will:
\begin{itemize}
\item learn about the advancement and new features of the infrastructure,
\item give feedback on the use of the infrastructure,
\item provide tests sets and proofs to include into Logipedia,
\item propose new features, services or research directions for the projet,
\item participate to the project self assessment.
\end{itemize}
From the point of view of its members, this club is a unique
opportunity for technology watch. Our empirical observations show that
many companies, working in safety critical areas, would like to be
more involved in the development of formal methods, but that the first
step into using such methods is often too costly, and that we must
offer them a smoother way to get into this technology. Such a club
where the industrial users will be able to develop a base culture in
formal methods, share experience with other companies, and conduct
technology watch on a regular basis is an efficient way to disseminate
formal methods in the European industry.

\begin{framed}
{\bf \Large Impact of Logipedia on the transportation industry}

{\color{red} David}  
\end{framed}

\begin{framed}
{\bf \Large Impact of Logipedia on the health care industry}

{\color{red} Cezary}  
\end{framed}

\begin{framed}
{\bf \Large Impact of Logipedia on the energy industry}

\medskip

Energy, specially nuclear energy, is a one of the key industrial sectors 
where safety and security are of prime importance.

The certification process in the energy industry is very specific
depending on the country. Indeed they don't share common certification
practice, unlike in the aeronautic industry. So a company who
certified critical software used in a nuclear plant in one country
needs to redo the certification in other country using different
tools. The tools recognized by one certification authority are
different in another.

Logipedia by allowing to share proof and models,
would ease the adaptation to another certification authority.
\end{framed}

\begin{framed}
{\bf \Large Impact of Logipedia on the block chain industry}

{\color{red} Raphaël}  
\end{framed}

\subsection{Closer interaction between a larger number of researchers}

We have already discussed lengthily the impact of Logipedia on the
community of academic and industrial researchers in formal methods
and, more generally, in logic.
The European Union has already invested a lot in logic and formal
methods, and we believe it is our duty to propose a project to
integrate the results of all this research.

Scientific research and education at all levels are concerned with the
discovery, verification, communication, archival and usage of
mathematical results.  These tasks have been supported by physical
books, conferences and other means.

The project can lead to a societal breakthrough opening the use of
formal proofs by a larger group of users, from experts users coming
from the formal proof community to a group of non expert users
(mathematicians, education and researchers in other science using
mathematical statements).

We want to insist on this last point: everywhere mathematics and
computer science are used (in physics, in some parts of biology and
social sciences, in engineering, in particular through simulation,
etc.) a quest for higher level of rigor should pave the way for a
development of formal proofs, but that this development is slowed down
by the multiplicity of theories, systems and libraries that make the
first step difficult for beginners. A common reference infrastructure
should simplify the access of non-specialists to formal methods.  For
instance, scientists willing to formalize hybrid systems should have
access to a good analysis library, whatever system they use.

Among the communities of researchers, one on which we can have a real
impact during the project is the community of working mathematicians.
Some of them have started using proof systems and we must take care that they
can have access to the best libraries of basic mathematics, whatever system
they use.


\begin{framed}
\begin{center}
  {\bf \Large The current members of the club of academic users}
\end{center}
  
Assia Mahboubi
  
Karl Palmskog <palmskog@gmail.com> 

Paul Jackson (not yet contacted)

Simon Foster <simon.foster@york.ac.uk> (Burkhart)

Achim D. Brucker <A.Brucker@exeter.ac.uk> (Burkhart)

Angeliki Koutsoukou-Argyraki 

Kevin Buzzard (not yet contacted)

Sébastien Gouëzel (not yet contacted)
\end{framed}

\subsection{Better management of the continuous flow of data}

Organizing a continuous flow of data is at the center of a project
like Logipedia. We already instead on the aspects of interoperability,
sustainability, cross verification, and ergonomy of the interfaces.

We want also to insist on the fact that Logipedia will be a service to
certification authorities, specially in security, as witnessed by the
presence of a representative of ANSSI in the advisory board,
Prove\&Run among the partners, and several other companies focused on
security in the club of industrial users.

\subsection{Socio-economic impact}

One aspect of the socio-economic impact of formal method in general
and integration activities such as Logipedia, is that making the 
digital society safer and more secure contributes to a safer and more 
secure society in general.

We are also interested in the recent trend to add a third pillar to
safety and security: ethics. For instance, proving properties of
software including the respect of privacy or, in the case of
electronic voting systems, secret of vote, auditability, etc.  is not
directly part of our project, but the development of formal methods
contributes to promote these values.

\subsection{Education}

The availability of a formal online encyclopedia which propose in a
single place the communication, archival and verification of
mathematical knowledge will be of prime importance for teachers.


\paragraph*{At university}
Education to formal methods in computer science and to formal proofs
in mathematics always hits the same obstacle: the need to choose a
specific theory or system, the need to focus on foundational issues in
contradiction with the claimed universality of logical truth.

Education to formal methods and formal proofs will gain in
universality once it will be demonstrated that this choice amounts to
include, or not, a few axioms and rewrite rules.  

This will allow teaching formal methods at a lower level of university
than what is currently done.

\paragraph*{In secondary education}
The usage of formal proofs in the class room is also slown down by the
lack of a large and well organized library of results.  Logipedia,
through a specific interface, can be used to teach mathematics, for
instance geometry, in secondary school.  It can also be used to write
textbooks promoting mathematical rigor by having in a textbook only
theorems that have formal proofs, even if the proofs in the textbook
are not presented formally. 

This is why we have included, in the project, a company that has 
already experimented the development of software to teach rigorous proof
in high school. 

We also defend that this renewal of logic education in secondary education
is of prime importance in our ``post-truth era''.

Because several of us are involved in the renewal of teaching
mathematics and computer science in secondary education and in the
first years of university, we felt the need to include in the project
a club of users of Logipedia in education.

\begin{framed}
\begin{center}
  {\bf \Large The current members of the club of users in education}
\end{center}



\end{framed}


\subsection{Publishing}

\subsection{Open Science}


%%% Local Variables:
%%%   mode: latex
%%%   mode: flyspell
%%%   ispell-local-dictionary: "english"
%%% End:


\section{Measures to maximise impact}

\begin{todo}{}\color{red}
  See participant portal FAQ on how to address \underline{dissemination and exploitation} in Horizon 2020.

  Provide a draft {\bf plan for the dissemination and exploitation of the project's results}. In the case of Integrating Activities, these are typically the results of the joint research activities to improve the infrastructure services, the enhanced access provision, and/or common standards, protocols etc. resulting from networking activities. Please note that such a draft plan is an \underline{admissibility condition}, unless the work programme topic explicitly states that such a plan is not required. 

Show how the proposed measures will help to achieve the expected impact of the project. 

The plan, should be proportionate to the scale of the project, and should contain measures to be implemented both during and after the end of the project. 

Your plan for the dissemination and exploitation of the project's results is key to maximising their {\bf impact}. This plan should describe, in a concrete and comprehensive manner, the {\bf area} in which you expect to make an impact and {\bf who} are the potential users of your results.  Your plan should also describe {\bf how} you intend to use the appropriate channels of dissemination and interaction with potential users.

Consider the full range of potential users and uses, including research, commercial, investment, social, environmental, policy-making, setting standards, skills and educational training where relevant.

Your plan should give due consideration to the possible {\bf follow-up} of your project, once it is finished. Its exploitation could require additional investments, wider testing or scaling up. Its exploitation could also require other pre-conditions like regulation to be adapted, or value chains to adopt the results, or the public at large being receptive to your results.

Include a business plan where relevant.

As relevant, include information on how the participants will manage the research data generated and/or collected during the project, in particular addressing the following issues:

    - What types of data will the project generate/collect?

    - What standards will be used?
    
    - How will this data be exploited and/or shared/made accessible for verification and re-use? If data cannot be made available, explain why.
    
    - How will this data be curated and preserved?
    
    - How will the costs for data curation and preservation be covered?

Actions under Horizon 2020 participate in the extended ‘Pilot on Open Research Data in Horizon 2020 ('open research data by default'), except if they indicate otherwise ('opt-out'.)1. Once the action has started (not at application stage) those beneficaries which do not opt-out, will need to create a more detailed Data Management Plan for making their data findable, accessible, interoperable and reusable (FAIR).

You will need an appropriate consortium agreement to manage (amongst other things) the ownership and access to key knowledge (IPR, research data etc.). Where relevant, these will allow you, collectively and individually, to pursue market opportunities arising from the project's results.

The appropriate structure of the consortium to support exploitation is addressed in section 3.3.

    - Include information about any open source software used or developed by the project.

    - Outline the strategy {\bf for knowledge management and protection}. Include measures to provide {\bf open access} (free on-line access, such as the ‘green’ or ‘gold’ model) to peer-reviewed scientific publications which might result from the project.

    Open access must be granted to all scientific publications resulting from Horizon 2020 actions (in particular scientific peer reviewed articles). Further guidance on open access is available in the H2020 Online Manual on the Participant Portal. This obligation does not apply to trans-national access users.

    Open access publishing (also called 'gold' open access) means that an article is immediately provided in open access mode by the scientific publisher. The associated costs are usually shifted {\bf away from readers, and instead (for example) to the university or research institute to which the} researcher is affiliated, or to the funding agency supporting the research. Gold open access costs are fully eligible as part of the grant. Note that if the gold route is chosen, a copy of the publication has to be deposited in a repository as well.

    Self-archiving (also called 'green' open access) means that the published article or the final peer-reviewed manuscript is archived by the researcher - or a representative - in an online repository before, after or alongside its publication. Access to this article is often - but not necessarily - delayed (‘embargo period’), as some scientific publishers may wish to recoup their investment by selling subscriptions and charging pay-per-download/view fees during an exclusivity period.
\end{todo}

The measures to maximize impact are at the heart of the organization
of the work package dissemination and communication.
A first draft of the {\em Plan for dissemination and exploitation of
  results} will be delivered at month 4 of the project and will
be updated during the course of the project.
Here are the first elements to build this plan.
Its objectives are:
\begin{compactitem}
\item Increase the awareness and use of Logipedia in academia.
\item Use Logipedia as a way to increase the cooperation between academia and
industry. 
\item Prepare the sustainability and exploitation of Logipedia before the
  end of the project, illustrating the philosophy of 
the 
\href{http://roadmap2018.esfri.eu/media/1048/rm2018-part1-20.pdf}{2018
  roadmap} of the European Strategy Forum on Research Infrastructures
(ESFRI): 
``A robust long-term vision is essential to successfully and
sustainably develop, construct and operate Research Infrastructures.''
even if Logipedia is still in its ``incubation phase''.
\end{compactitem}



%%%%%%%%%%%%%%%%%%%%%%%%%%%%%%%%%%%%%%%%%%%%%%%%%%%%%%%%%%%%%%%%%%%%%%%%%%%%%%
\subsection*{(a) Plan for dissemination and exploitation of results}
\label{sec:dissemination}


\subsubsection*{Dissemination}

\begin{longtable}{|p{0.55\textwidth}|p{0.12\textwidth}|p{0.15\textwidth}|}
\hline
{\bf Action}
&
{\bf Target}
&
{\bf Indicator and schedule}
\\
%%%%%%%%%%%%%%%%%%%%%%%%%%%%%%%%%%%%%%%%%%%%%%%%%%%%%%%%%%%%%%%%%%%%%%%%%%%%%%
\hline
{\bf 1. Participation to conferences.}
In computer science, publishing in conference
proceedings are often favored over journal publication. We have targeted
more than ten conferences:
% in alphabetical order
CADE (Conference on Automated Deduction),
CICM (Intelligent Computer Mathematics),
CPP (Certified Programs and Proofs),
CSL (Computer Science Logic),
FSCD (Formal Structures for Computation and Deduction),
FROCOS (Frontiers of Combining Systems),
ICALP (International Colloquium on Automata, Languages, and Programming),
%ICFP (Functional Programming),
IJCAR (International Joint Conference on Automated Reasoning),
ITP (Interactive Theorem Proving),
LFMTP (Logical Frameworks and Meta-pLanguages: Theory
and Practice), 
LICS (Logic in Computer Science),
LPAR (Logic Programming and Automated Reasoning),
PxTP (Proof eXchange for Theorem Proving).
&
Researchers.
&
10 papers per year.
\\
\hline
{\bf 2. Organisation of conferences.}
We shall organize our own event, with a conference,
specialized workshops, and the general assembly.
&
Researchers, industrials.
&
100 participants each year.
\\
\hline
{\bf 3. Organizing summer schools}
open to anyone and not only the partners.
We shall organize several training sessions targeting the
different communities of users: master and PhD students to teach them
the foundations of Logipedia, teachers to help them use interactive
theorem provers at school and university, and to engineers to help
them use formal methods tools in their work. Such training sessions
are key dissemination events that will accompany the growing of the
Logipedia community and contribute to educate a new generation of researchers,
teachers and engineers.
&
Master and PhD students
&
2
\\
\hline
    {\bf 4.
Advising PhD students} 
to educate a new generation of researchers and engineers, 
Some will have academic and industrial co-advisors.
&
PhD students.
&
3 PhD students start each year.
\\
\hline
{\bf 5. Co-building the Logipedia strategy}
by participating to joint meetings, such as
the clubs and advisory board.
&
Researchers, industrials.
&
Participation to the meetings
every year.
\\
\hline
{\bf 6. Disseminate Logipedia in relevant communities.}
The fours clubs contribute to the dissemination of
Logipedia in their own ecosystem, by organising talks,
courses, meetings.
&
Research, industry,
education, publishing.
&
At least one event organised by each club.
\\
\hline
{\bf 7. Delivering teaching material}
co-produced by 
the partners of the project and the members of
the club of users in education. 
&
Students
&
At least one textbook during the project.
\\
\hline
{\bf 8. Using Logipedia to increase reproducibility in science}
by referencing formal proofs in a single place. 
&
Publishers and researchers.
&
Researchers outside the consortium use Logipedia as a reference.
\\
\hline
{\bf 9. Initiate a discussion with certification authorities}
about the use of a common language across the European Union.
&
Certification agencies and the industry. 
&
Meetings are organised with several European certification agencies.
\\
\hline
{\bf 10. Use a free licence for data and software} to make 
the data is findable, accessible, interoperable and reusable
and develop Open data / Open science / Open innovation.
&
Scientists and innovators.
&
Delivery of Logipedia at month 14.
\\
\hline
{\bf 11. Publish in Open access venues.}
&
Scientists and students.
&
All the publications of the partners are open.
\\
\hline
\end{longtable}


\subsubsection*{Exploitation}

\begin{longtable}{|p{0.30\textwidth}|p{0.30\textwidth}|p{0.30\textwidth}|}
\hline
{\bf Action}
&
{\bf Stakeholders}
&
{\bf Indicator and schedule}
\\
\hline
%{\bf Monitoring the innovation during the course of the project.}
%&
%The steering commitee, the European project manager, the 
%transfer, innovation, and partnership department of Inria Saclay and
%any relevant member of our partners' institution.
%&
%Meetings organized with the transfer and innovation department
%upon request and at least once a year.
%\\
%\hline
{\bf 1. Build an organization in charge of managing Logipedia
after the end of the project}.
&
The project management team, volunteer members of the consortium,
after consulting the advisory board.
&
The structure is created at month 48 at the latest.
\\
\hline
{\bf 2. Raise funds to manage Logipedia}.
&
The project management team, after consulting the advisory board.
&
Sponsors and fundings have been identified at month 36 at the latest.
\\
\hline   
{\bf 3. Find a server to permanently host Logipedia}.
&
The project management team, volunteer members of the consortium,
after consulting the advisory board.
&
Logipedia is kept functional.
\\
\hline
{\bf 4. Continue developping Logipedia}
&
New developers taking over.
&
New libraries (for instance that of ACL2 or Nuprl) are integrated.
New features are added to Logipedia.
\\
\hline
{\bf 5. Generate new projects, such as ``Logipedia, security, and
certification'', ``Logipedia and automated theorem proving'',
``Logipedia and the B method''...}
&
Special interest groups within Logipedia. 
&
New communities adopt Logipedia.
\\
\hline
\end{longtable}

%%%%%%%%%%%%%%%%%%%%%%%%%%%%%%%%%%%%%%%%%%%%%%%%%%%%%%%%%%%%%%%%%%%%%%%%%%%%%%
\subsection*{(b) Communication activities}

Communication activities aim at raising the awareness about Logipedia
to potential stakeholders that would not be concerned by our
dissemination actions. It will ensure the growth of the Logipedia
ecosystem and be a way to advertise the work achieved by the partners
and the clubs of users. It also serves the purpose of informing the
European citizen of the research findings she has been financially
contributing to.

Six person-months of an experienced communication officer, from the
Inria Saclay communication team, are dedicate to the sole task of
external communication.  She will work in close cooperation with the
all dissemination, communication, and exploitation work package leader
and the project management team.

In this project, we have the ambition to communicate to the general
public, even if, in the past, these communication activities have been
considered as less important than the dissemination towards the
research, industry, and education communities.

\begin{longtable}{|p{0.30\textwidth}|p{0.30\textwidth}|p{0.30\textwidth}|}
\hline
{\bf Action}
&
{\bf Target audience}
&
{\bf Indicator and schedule}
\\
\hline
{\bf Promoting the existence of the project}
though the creation of a a web site, flyers, and 
a visual identity. 
&
Research, industry, and education.
&
Website at M3.
\\
\hline
{\bf Promoting the results and the values of the project} through 
outreach actions: 
publication of articles in  popular science magazines
online videos, and live events such that the
European Researchers Night or {\em La Fête de la Science}, 
focusing, as it is our habit. 
&
General public.
&
At least five partners organise an outreach activity in their country.
\\
\hline
\end{longtable}

%%% Local Variables:
%%%   mode: latex
%%%   mode: flyspell
%%%   ispell-local-dictionary: "english"
%%% End:




%%% Local Variables: 
%%% mode: LaTeX
%%% TeX-master: "propB"
%%% End: 

% LocalWords:  ednote
