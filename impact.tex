\chapter{Impact}\label{chap:impact}

%\ednote{Maximum length for the whole of Section 3 –-- ten pages}

\section{Expected impacts}\label{sec:expected-impact}

%\begin{todo}{}\color{red} Please be specific, and provide only
%  information that applies to the proposal and its
%  objectives. Wherever possible, use quantified indicators and
%  targets.
  
%  * Describe how your project will contribute to:
  
%     - each of the expected impacts mentioned in the work programme,
%     - under the relevant topic;
  
%      - where relevant, any substantial impacts not mentioned in the
%      - work programme, that would enhance innovation capacity;
%      - create new market opportunities, strengthen competitiveness
%      - and growth of companies, address issues related to climate
%      - change or the environment, or bring other important benefits
%      - for society
  
%  * Describe any barriers/obstacles, and any framework conditions
%  * (such as regulation, standards, public acceptance, workforce
%  * considerations, financing of follow-up steps, cooperation of
%  * other links in the value chain), that may determine whether and
%  * to what extent the expected impacts will be achieved. (This
%  * should not include any risk factors concerning implementation, as
%  * covered in section 3.2.)
%\end{todo}

As discussed above, the shift from informal, pencil and paper, proofs
to formal computerized proof is a major improvement on the never
ending quest for logical rigor, with a strong impact both on
mathematics, where much more complex proofs can be built, and computer
science, where safety and security can be dramatically improved with
the use of formal methods. But this major step forward also has a
negative side effect: we have moved from a time where we had
(informal) proofs of Pythagoras' theorem or Fermat's little theorem,
to a time where we have (formal) proofs in Coq, in Matita, in HOL
Light, in PVS, etc. of these theorems, jeopardizing the universality
of mathematical truth.

We see this loss of universality of mathematical truth as the main
obstacle to the diffusion of the notion of formal proof, in the
communities of mathematicians and computer scientists, but also
engineers and students. Our long-term goal is to resurrect the
universality of mathematical truth in order to build a strong formal
proof community including specialists and non-specialists such as
working mathematicians, engineers and students.

This requires to express the theories implemented in these systems in
a common logical framework, each with a finite number of axioms and
reduction rules, in order to be able to say, not that a proof is
expressed in one system or in another, but to say which axioms and
reduction rules it uses, as we have been used to since the development
of non-Euclidean geometries.

Having a standard for expressing theories and proofs and resurrecting
this way the universality of mathematical truth will also make proof
systems interoperable and will allow the construction of an on-line
system-independent encyclopedia. More importantly, this will suppress
one of the main obstacles to the diffusion of formal proofs in
mathematics, computer science, industry, and education, just like the
development of the html standard induced a renewal of document sharing
in general and the definition of predicate logic induced a renewal of
logic in the 1930's.

Innovation Formal methods are now an important part of some advanced
industrial projects. For instance, mastering formal methods is key to
give Europe a competitive advantage in conquering the market of
autonomous cars, trains, planes, and drones. But this penetration of
formal methods in industry hits the same obstacle that researchers
often promote one method, theory or system, while their industrial
partners are in search of universality. We expect to make formal
proofs more accessible to industry by avoiding each project to
redevelop elementary proofs, but instead benefit of the formalization
work shared with other communities.


\paragraph{Key exploitable results.}

- Logipedia in itself from TRL 3 to TLR 4

- Representation of theories implemented in various systems

- Rechecking formal proofs for higher Evaluation Assurance Level




\section{Key Exploitable results}

Logipedia is a Key Exploitable Results, each domain specific library
(on real analysis, probability...) is also a Key Exploitable Result,
each specific development made available to all provers (CompCert?,
CakeML?...) is a Key Exploitable Result, as well as each proof in
Logipedia. Formalized datastructures and algorithms shared accros
provers are a Key Exploitable Result.

Indexing algorithms and seach engines are Key Exploitable Results.

User interfaces are Key Exploitable results.

API and package distribution systems are Key Exploitable Results.

Logipedia as a repository for proofs referenced in publications, 
in mathematics, in computer science and in other sciences is a
Key Exploitable Result.

- Each import feature to Logipedia and each export
feature from  Logipedia is a 
Key Exploitable Result.



\section{Measures to maximise impact}

{\bf (a) Dissemination and exploitation of results}

\begin{todo}{}\color{red}
  See participant portal FAQ on how to address \underline{dissemination and exploitation} in Horizon 2020.

  Provide a draft {\bf plan for the dissemination and exploitation of the project's results}. In the case of Integrating Activities, these are typically the results of the joint research activities to improve the infrastructure services, the enhanced access provision, and/or common standards, protocols etc. resulting from networking activities. Please note that such a draft plan is an \underline{admissibility condition}, unless the work programme topic explicitly states that such a plan is not required. 

Show how the proposed measures will help to achieve the expected impact of the project. 

The plan, should be proportionate to the scale of the project, and should contain measures to be implemented both during and after the end of the project. 

Your plan for the dissemination and exploitation of the project's results is key to maximising their {\bf impact}. This plan should describe, in a concrete and comprehensive manner, the {\bf area} in which you expect to make an impact and {\bf who} are the potential users of your results.  Your plan should also describe {\bf how} you intend to use the appropriate channels of dissemination and interaction with potential users.

Consider the full range of potential users and uses, including research, commercial, investment, social, environmental, policy-making, setting standards, skills and educational training where relevant.

Your plan should give due consideration to the possible {\bf follow-up} of your project, once it is finished. Its exploitation could require additional investments, wider testing or scaling up. Its exploitation could also require other pre-conditions like regulation to be adapted, or value chains to adopt the results, or the public at large being receptive to your results.

Include a business plan where relevant.

As relevant, include information on how the participants will manage the research data generated and/or collected during the project, in particular addressing the following issues:

    - What types of data will the project generate/collect?

    - What standards will be used?
    
    - How will this data be exploited and/or shared/made accessible for verification and re-use? If data cannot be made available, explain why.
    
    - How will this data be curated and preserved?
    
    - How will the costs for data curation and preservation be covered?

Actions under Horizon 2020 participate in the extended ‘Pilot on Open Research Data in Horizon 2020 ('open research data by default'), except if they indicate otherwise ('opt-out'.)1. Once the action has started (not at application stage) those beneficaries which do not opt-out, will need to create a more detailed Data Management Plan for making their data findable, accessible, interoperable and reusable (FAIR).

You will need an appropriate consortium agreement to manage (amongst other things) the ownership and access to key knowledge (IPR, research data etc.). Where relevant, these will allow you, collectively and individually, to pursue market opportunities arising from the project's results.

The appropriate structure of the consortium to support exploitation is addressed in section 3.3.

    - Include information about any open source software used or developed by the project.

    - Outline the strategy {\bf for knowledge management and protection}. Include measures to provide {\bf open access} (free on-line access, such as the ‘green’ or ‘gold’ model) to peer-reviewed scientific publications which might result from the project.

    Open access must be granted to all scientific publications resulting from Horizon 2020 actions (in particular scientific peer reviewed articles). Further guidance on open access is available in the H2020 Online Manual on the Participant Portal. This obligation does not apply to trans-national access users.

    Open access publishing (also called 'gold' open access) means that an article is immediately provided in open access mode by the scientific publisher. The associated costs are usually shifted {\bf away from readers, and instead (for example) to the university or research institute to which the} researcher is affiliated, or to the funding agency supporting the research. Gold open access costs are fully eligible as part of the grant. Note that if the gold route is chosen, a copy of the publication has to be deposited in a repository as well.

    Self-archiving (also called 'green' open access) means that the published article or the final peer-reviewed manuscript is archived by the researcher - or a representative - in an online repository before, after or alongside its publication. Access to this article is often - but not necessarily - delayed (‘embargo period’), as some scientific publishers may wish to recoup their investment by selling subscriptions and charging pay-per-download/view fees during an exclusivity period.
\end{todo}

\subsection{Dissemination and/or Use of Project Results, and Management of Intellectual Property}\label{sec:dissemination}


- Dissemination  (Cofernce, Summer schools...)

- Clubs

- Textbooks for working mathematicians and math students...

- co-avised PhDs on Logipedia, between Academia and Industry

- proof editing tools

- use cases from indutry

- cooperation with certification authorities

- APIs

- Press real




{\bf (b) Communication activities}

\begin{todo}{}\color{red}
  Describe the proposed communication measures for promoting the project and its findings during the period of the grant. Measures should be proportionate to the scale of the project, with clear objectives.  They should be tailored to the needs of different target audiences, including groups beyond the project's own community.

  See participant portal FAQ on how to address communication activities in Horizon 2020.

  For further guidance on communicating EU research and innovation for project participants, please refer to the H2020 Online Manual on the Participant Portal.
\end{todo}

\section{Related projects}

Section title? Keep here or move elsewhere?

{\color{red} Compare with OpenDreamKit, Software Heritage, ProofPeer,
  Wolfram alfa, etc.}

{\color{red} To be more detailed and compared to Logipedia.}

% from François Bobot:
The H2020 \href{https://www.decoder-project.eu}{Decoder} aims at
providing a unified interface for storing and querying any kind of
information related to a given software project, from initial
requirements to code, to formal specifications and analysis results,
including proof artifacts. It would of course be very beneficial for
both projects to agree on a common exchange format for such objects, and
coordination with the Decoder consortium (in which CEA acts as technical
leader) will seek to achieve that.

The \href{https://formalabstracts.github.io/}{Formal Abstracts}
project, headed by Thomas Hales (UK), will establish a formal abstract
service that will express the results of mathematical publications in
a computer-readable form (the Lean syntax) that captures the semantic
content of publications.

The \href{http://www.proofpeer.net/}{ProofPeer} project, headed by
Jacques Fleuriot (UK), aims at designing a framework for (massively)
collaborative theorem proving, using some extension of simple type
theory and set theory to be defined compatible with HOL Light.

The
\href{https://www.cl.cam.ac.uk/~lp15/Grants/Alexandria/}{Alexandria}
project, headed by Larry Paulson (UK), aims at creating a proof
development environment attractive to working mathematicians,
utilising the best technology available across computer science, based
on Isabelle/HOL and set theory.

The \href{https://deepspec.org/}{DeepSpec} project, headed by Andrew
Appel, aims at eliminating software "bugs" that can lead to security
vulnerabilities and computing errors by improving the formal methods
-- or the mathematically based techniques -- by which software is
developed and verified. It is based on the Coq language.

The \href{https://xenaproject.wordpress.com/}{Xena} project, headed by
Kevin Buzzard (UK) aims at introducing mathematicians to formal proof
verification software, and to digitise the undergraduate mathematics
curriculum of Imperial College London.







%%% Local Variables: 
%%% mode: LaTeX
%%% TeX-master: "propB"
%%% End: 

% LocalWords:  ednote
