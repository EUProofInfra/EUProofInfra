\subsection{Related projects}

Section title? Keep here or move elsewhere?

{\color{red} Compare with OpenDreamKit, Software Heritage, ProofPeer,
  Wolfram alfa, etc.}

{\color{red} To be more detailed and compared to Logipedia.}

% from François Bobot:
The H2020 \href{https://www.decoder-project.eu}{Decoder} aims at
providing a unified interface for storing and querying any kind of
information related to a given software project, from initial
requirements to code, to formal specifications and analysis results,
including proof artifacts. It would of course be very beneficial for
both projects to agree on a common exchange format for such objects, and
coordination with the Decoder consortium (in which CEA acts as technical
leader) will seek to achieve that.

The \href{https://formalabstracts.github.io/}{Formal Abstracts}
project, headed by Thomas Hales (UK), will establish a formal abstract
service that will express the results of mathematical publications in
a computer-readable form (the Lean syntax) that captures the semantic
content of publications.

The \href{http://www.proofpeer.net/}{ProofPeer} project, headed by
Javques Fleuriot (UK), aims at designing a framework for (massively)
collaborative theorem proving, using some extension of simple type
theory and set theory to be defined compatible with HOL Light.

The
\href{https://www.cl.cam.ac.uk/~lp15/Grants/Alexandria/}{Alexandria}
project, headed by Larry Paulson (UK), aims at creating a proof
development environment attractive to working mathematicians,
utilising the best technology available across computer science, based
on Isabelle/HOL and set theory.

The \href{https://deepspec.org/}{DeepSpec} project, headed by Andrew
Appel, aims at eliminating software "bugs" that can lead to security
vulnerabilities and computing errors by improving the formal methods
-- or the mathematically based techniques -- by which software is
developed and verified. It is based on the Coq language.

The \href{https://xenaproject.wordpress.com/}{Xena} project, headed by
Kevin Buzzard (UK) aims at introducing mathematicians to formal proof
verification software, and to digitise the undergraduate mathematics
curriculum of Imperial College London.
