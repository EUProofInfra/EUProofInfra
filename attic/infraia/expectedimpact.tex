The proposal is in coherence with the goals of the call INFRAIA-02-2020:
Integrating activities for starting communities (listed on page 56 of
the part 4 of Horizon 2020 Work Programme 2018-2020). It also clearly
refers to one of the mentioned cross-cutting activities: open science.

\paragraph{Wider and more efficient access --- New research infrastructure}
In a shared public encyclopedia providing virtual access, each user,
in research, industry, and eucation, can find the formal proofs she
needs in the logic she wants, regardless the logic and system this
proof has been developed in.

\paragraph{Synergy --- Open research data.}
Instead of having a scattered community, each group developing a
library for its own logic and its own system, researchers and
engineers will be able to work together on common developments,
reusing proofs developed in other systems and in other communities.

%In terms of networking, we have already organized one logipedia
%meeting and the funding will insure we can continue to organize
%large-scale international meetings on a regular basis. The first
%logipedia event has proven to be very valuable in terms of exchange of
%best practices.

\paragraph{Partnership with industry}

Several European and non European companies are member of the project,
as contributors or as members of the future {\sc Logipedia} user's group.

\paragraph{Education --- Closer interaction between a larger number of
researchers.}
Education to formal methods in computer science and to formal proofs
in mathematics always hits the same obstacle: the need to choose a
specific theory or system, in contradiction with the claimed
universality of logical truth. Education to formal methods and formal
proofs will gain in universality once it will be demonstrated that
this choice amounts to include, or not, a few axioms and reduction
rules. We also defend that this renewal of logic education at
university level and before is of prime importance in our ``post-truth
era''.

\paragraph{Better management of the continuous flow of data.}
A shared encyclopedia allows a better sustainability of the formal
proofs developed over time. Too many formal proofs developed in the
past are not available any more.



-------------------------

As discussed above, the shift from informal, pencil and paper, proofs
to formal computerized proof is a major improvement on the never
ending quest for logical rigor, with a strong impact both on
mathematics, where much more complex proofs can be built, and computer
science, where safety and security can be dramatically improved with
the use of formal methods. But this major step forward also has a
negative side effect: we have moved from a time where we had
(informal) proofs of Pythagoras' theorem or Fermat's little theorem,
to a time where we have (formal) proofs in Coq, in Matita, in HOL
Light, in PVS, etc. of these theorems, jeopardizing the universality
of mathematical truth.

We see this loss of universality of mathematical truth as the main
obstacle to the diffusion of the notion of formal proof, in the
communities of mathematicians and computer scientists, but also
engineers and students. Our long-term goal is to resurrect the
universality of mathematical truth in order to build a strong formal
proof community including specialists and non-specialists such as
working mathematicians, engineers and students.

This requires to express the theories implemented in these systems in
a common logical framework, each with a finite number of axioms and
reduction rules, in order to be able to say, not that a proof is
expressed in one system or in another, but to say which axioms and
reduction rules it uses, as we have been used to since the development
of non-Euclidean geometries.

Having a standard for expressing theories and proofs and resurrecting
this way the universality of mathematical truth will also make proof
systems interoperable and will allow the construction of an on-line
system-independent encyclopedia. More importantly, this will suppress
one of the main obstacles to the diffusion of formal proofs in
mathematics, computer science, industry, and education, just like the
development of the html standard induced a renewal of document sharing
in general and the definition of predicate logic induced a renewal of
logic in the 1930's.

Innovation Formal methods are now an important part of some advanced
industrial projects. For instance, mastering formal methods is key to
give Europe a competitive advantage in conquering the market of
autonomous cars, trains, planes, and drones. But this penetration of
formal methods in industry hits the same obstacle that researchers
often promote one method, theory or system, while their industrial
partners are in search of universality. We expect to make formal
proofs more accessible to industry by avoiding each project to
redevelop elementary proofs, but instead benefit of the formalization
work shared with other communities.


\paragraph{Key exploitable results.}

- Logipedia in itself from TRL 3 to TLR 4

- Representation of theories implemented in various systems

- Rechecking formal proofs for higher Evaluation Assurance Level

