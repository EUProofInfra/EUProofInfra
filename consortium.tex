Building an infrastructure like Logipedia requires to have most of the
major proof systems in the project. We currently have eighteen out of
the twenty four systems presented Figure 1, and fourteen out of the
sixteen European ones.  The only reason why we do not have them all is
because of budget constraints.

This explains why the consortium has to be large.  We have twenty-nine
partners in the consortium which is almost all the groups working on
formal proof technology in Europe.  A smaller consortium would not
reach the universality which is a key factor of success.  Being such a
large consortium is a strategic decision.

Each partner brings a different expertise to the project.  Eighteen
partners bring their expertise on one specific system.  Eight bring
their expertise on different aspects of automated theorem
proving. Four bring their expertise on some specific library. Twelve
bring their expertise on a transverse issue: access (three partners),
proof engineering (nine partners), and publishing (one partner).


Finally, among the twenty nine partners, eight are industrial.
Two of them are industrial research centers, six of them are enterprises.
Three are transverse and five are
focused on one industrial sector: transportation, health care, energy,
security, and education. These six enterprises are small and
medium-sized enterprises. In contrast the enterprises in the club of
industrial users are very different sizes, from small and medium-sized
enterprises to very large enterprises such as Alstom or Siemens.

\begin{longtable}{|p{0.3\textwidth}|p{0.15\textwidth}|p{0.1\textwidth}|p{0.1\textwidth}|p{0.1\textwidth}|p{0.1\textwidth}|}
%%%%%%%%%%%%%%%%%%%%%%%%%%%%%%%%%%%%%%%%%%%%%%%%%%%%%%%%%%%%%%%%%%%%%%%%%%%%%%
\hline
{\bf Partner}
&
{\bf Systems}
&
{\bf ATP}
&
{\bf Libraries}
&
{\bf Transverse}
&
{\bf Industrial sector}
\\
\hline
Institut National de Recherche en Informatique et Automatique
&
HOL Light, PVS, Coq, TLA+, HoTT
&
&
Flyspeck
&
Proof engineering
&
\\
\hline
Université de Strasbourg
&
&
&
GeoCoq
&
Proof engineering
&
\\
\hline
Institut National Polytechnique de Toulouse
&
Rodin
&
&
&
&
\\
\hline
Universität Innsbruck
&
Mizar
&
x
&
&
Proof engineering
&
\\
\hline
Université de Liège
&
&
x
&
&
&
\\
\hline
Alma Mater Studiorum --- Università di Bologna
&
Matita, Coq
&
&
&
Proof engineering
&
\\
\hline
Faculty of Mathematics, University of Belgrade
&
&
x
&
&
Proof engineering
&
\\
\hline
Technische Universität München
&
Isabelle/HOL
&
&
AFP, Proba / analysis
&
&
\\
\hline
Technische Universiteit Delft
&
Agda
&
&
&
&
\\
\hline
Université Paris-Saclay
&
&
x
&
&
Proof engineering
&
\\
\hline
Friedrich-Alexander Universität Erlangen-Nürnberg
&
&
&
&
Access, Proof engineering
&\\
\hline
University of Leeds
&
HoTT
&
&
&
Proof engineering
&
\\
\hline
Göteborgs Universitet
&
Agda
&
&
&
&
\\
\hline
Chalmers Tekniska Högskola
&
HOL4
&
&
CakeML 
&
&
\\
\hline
Ludwig-Maximilians-Universität München
&
Minlog
&
&
&
&
\\
\hline
Institut Mines-Télécom
&
Atelier B, FoCaLiZe
&
x
&
&
Proof engineering
&
\\
\hline
Uniwersytet w Białymstoku
&
Mizar
&&&&\\
\hline
OCamlPro
&
&
x
&
&
&
\\
\hline
ClearSy
&
Atelier B
&
&
&
&
Transportation
\\
\hline
University of Birmingham
&
HoTT
&
&
&
&
\\
\hline
Commissariat à l’Energie Atomique et aux Energies Alternatives
&
Why3
&
x
&
&
&
Energy
\\
\hline
Duale Hochschule Baden-Württemberg
&
&
x
&
&
&
\\
\hline
Institut de Recherche Technologique System X
&
&
&
&
Access
&
\\
\hline
Edukera
&
&
&
&
Access
&
Education
\\
\hline
MED-EL Elektromedizinische Geraete GmbH
&
&
&
&
&
Health care
\\
\hline
Prove \& Run
&
ProvenTools
&
&
&
&
Security
\\
\hline
Konrad-Zuse-Zentrum für Informationstechnik Berlin
&
&
&
&
Publishing
&
\\
\hline
Universitatea Alexandru Ioan Cuza din Iasi
&
K Prover
&
&
&
&
\\
\hline
Runtime Verification SRL
&
K Prover
&
&
&
&
\\
\hline
\end{longtable}

%%% Local Variables:
%%%   mode: latex
%%%   mode: flyspell
%%%   ispell-local-dictionary: "english"
%%% End:
