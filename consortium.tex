\begin{todo}{from the proposal template}
  Describe how the participants collectively constitute a consortium capable of achieving
  the project objectives, and how they are suited and are committed to the tasks assigned
  to them. Show the complementarity between participants. Explain how the composition of
  the consortium is well-balanced in relation to the objectives of the project.

  If appropriate describe the industrial/commercial involvement to ensure exploitation of
  the results. Show how the opportunity of involving SMEs has been addressed
\end{todo}

The project partners of the \pn project have a long history of successful collaboration;
Figure~\ref{tab:collaboration} gives an overview over joint projects (including proposals) and
joint publications (only international, peer reviewed ones).

\jointorga{Fau,Bol}% CICM
\jointorga{Inn,Bol}% CICM
\jointpub{Fau,Bol}% CICM paper
\jointpub{Tum,Bol}% CICM paper
\jointpub{Fau,TUM}%
%\jointsup{Fau,}
\jointsoft{Fau,Tum}% Isabelle Extension
\jointsoft{Fau,Bol}% Coq exporter
\jointpub{Inr,Bol}% ELPI
\jointproj{Inr,Bol}% MoWGLI


\jointpub{Pra,Stu}% CADE 2015 paper
\jointOrga{Inr,Stu}% 3rd PAAR, 5th PAAR

\coherencetable

\subsection{Subcontracting}\label{sec:subcontracting}

\begin{todo}{from the proposal template}
  If any part of the work is to be sub-contracted by the participant responsible for it,
  describe the work involved and explain why a sub-contract approach has been chosen for
  it.
\end{todo}

The tasks \taskref{instrumentation}{isabelle},
\taskref{libraries}{afp} (both handled by \site{Tum}) and part of
task~\taskref{structuring}{strontorepml} (handled by \site{Fau}) will
be carried out by subcontracting Dr.\ M.\ Wenzel.  Each subcontract
will cover roughly the equivalent of $3$ person-months.  Concretely,
it concerns the export of proof terms and other data from the Isabelle
system.  Wenzel is the main Isabelle developer and has spent the
last $10$ years building the technological prerequisites for the
required work and is the natural person to carry it out.  However, he
has left academia and started his own company that specializes on
Isabelle kernel development and routinely carries out subcontracts for
Isabelle-related research projects.  Therefore, a subcontract is the
best option as developing the necessary expertise in-house would take
an additional 6-12 person-months per task.  \site{Fau} has already worked with
Wenzel in similar subcontracts twice before (including the
OpenDreamKit EU infrastructure project), and the collaborations have
been very effective and efficient. Wenzel obtained his Ph.D. at
\site{Tum} advised by Nipkow.

The task \taskref{instrumentation}{isabelle} (handled by \site{Tum})
requires special assistance by Dr.\ David Matthews (PROLINGUA LTD,
Edinburgh): As provider of the underlying Poly/ML infrastructure,
Matthews is in a unique position to provide extra scalability of ML
heap management, and thus allow Isabelle to export more library
material.


\subsection{Other Countries}\label{sec:other-countries}
\begin{todo}{from the proposal template}
  If a one or more of the participants requesting EU funding is based outside of the EU
  Member states, Associated countries and the list of International Cooperation Partner
  Countries\footnote{See CORDIS web-site, and annex 1 of the work programme.}, explain in
  terms of the project’s objectives why such funding would be essential.
\end{todo}

\subsection{Additional Partners}\label{sec:assoc-partner}
\begin{todo}{from the proposal template}
  If there are as-yet-unidentified participants in the project, the expected competences,
  the role of the potential participants and their integration into the running project
  should be described
\end{todo}

%%% Local Variables:
%%% mode: latex
%%% TeX-master: "propB"
%%% End:
