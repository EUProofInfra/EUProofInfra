\subsection*{Innovation potential}

Europe is a leader in the adoption of formal methods for
industrial use. Yet, the multiplicity of approaches used by the
different actors impedes the wider application of such
techniques. Thus, the wide diffusion of scientific innovation requires
building bridges between communities. Europe's leadership will be
strengthened by Logipedia that will enable cross-fertilization of
results and resources between the different formal methods
communities.

This project to integrate the scientific and technological effort
around formal proofs in Europe is thus a way to foster the development
of formal methods in industry, as the economic spinoffs from the
project will benefit the European industry, mainly by reducing the
cost of this technology. 

First, reusing a proof produced with a particular tool in another, 
will become possible in a much more systematic way and at a
lower cost. In the uncommon case where a proof relies on a theory
which is not compatible with the target tool, it will be easier to
understand why and determine whether adapting it is manageable.
Furthermore, as an infrastucture, Logipedia will help users
in finding existing proofs of properties, making their verification
process faster.

Second, checking a proof will remain possible over time. Today, it
requires either to maintain proofs along new versions of the tools,
which can represent a significant maintenance cost, or to archive
them together with a version of the tool used to produce it. In this
situation, Logipedia will help on two aspects. First, the
common format will guarantee that proofs can be checked by any tool
implementing it, thus reducing proof maintenance cost. Second, by
providing a common proof database, general interest proofs can be
stored and maintained in the infrastructure, allowing industrial users
to focus their resources for their specific needs only.

Finally, Logipedia will increase the trust in formal proofs. For
example, in a certification context, the use of a particular tool for
the verification of the candidate system must be approved. If the
certification body is not familiar with the tool, producing a
justification for it can represent a significant amount of time. For a
new tool, adoption will be made faster, as not only certification
bodies but also potential users could question its soundness despite
the potential advantages it could provide. The common, co-built,
format offered by Logipedia will provide an answer to this lack of
trust. It will help the certification bodies, as they will not have to
learn about a new tool for each new certification process, as any tool
implementing the format will be suitable for them to check the
proof. Thus they only have to trust one. Consequently, it will also
help industrial users, as justifying the use of a particular tool will
be easier as long as they can provide a proof in the right format. And
finally, it will encourage the development of new tools, as potential
users will not have to trust these tools directly, but only the proofs
that they provide, which can be checked easily.

Each industrial partner in the project of course has their own agenda.


\begin{longtable}{|p{0.10\textwidth}|p{0.85\textwidth}|}
\hline
{\bf CEA}
&
CEA has been developing Frama-C since 2005, which is used for the analysis and
the verification of safety critical applications, which ranges from airplanes
to nuclear plants or railway systems. In these domains, the confidence
into the proofs produced by the analysis tools is an important concern.
\\

&
\hspace{0.4cm}
For deductive verification of C code, Frama-C relies on
a complex proof architecture that first simplifies the properties to prove,
and then transmit them to the Why3 tool, which will itself perform other
simplifications before transmitting them to different SMT solvers. All
these steps, while bringing the benefit to get the best of each tool, comes
with a risk of error during simplifications, or inconsistency when passing
from a tool to another, thus questioning the confidence one can have into
the obtained proof.
\\

&
\hspace{0.4cm}
Furthermore, it is common that lemmas developed for a particular use
cases are useful in many other use cases. Logipedia is thus an opportunity
to create a common database of lemmas that could be reused by the different
users of Frama-C.
\\

&
\hspace{0.4cm}
Logipedia will foster the innovation potential of CEA on three aspects:
\\

& $\blacktriangleright$ Frama-C will be able to gather the proof
traces produced by the tools it relies on, and the traces produced by
its own simplification process. Thus enabling the possibility to check
the complete traces, increasing the confidence into the verification.\\
& $\blacktriangleright$ The availability of these traces will also
benefit to the validation process of the Frama-C tool itself, using
Dedukti as a cross validator of the tests results.\\
& $\blacktriangleright$ Sharing of lemmas through Logipedia will
benefit to the users of both Frama-C and other tools, avoiding to
develop proofs for already known results, thus making their proof
process more efficient.

\\
\hline
{\bf Clearsy}
&
In B software development, each
industrial user enriches Atelier B's provers with lemmas which must be
qualified by tools or proof-reading.  It is often the case that lemmas
are qualified multiple times, even within the same company.  Clearsy
has been involved for years in projects to allow the integration into
AtelierB of proof tools developed by other industrial or academic teams
(Bware, LCHIP, DISCONT) using modern proof techniques.  Eventually,
industrial integration raises issues related to the qualification of
the tools in relation to industrial safety standards (Common Critera,
DO-178, 50129, etc).
\\

&
\hspace{0.4cm}
Logipedia is a unique opportunity for CLEARSY to consolidate relationships with leading European actors in the research field for machine-based
theorem proving. The tools used by CLEARSY and its client companies, namely Atelier B and its provers, will benefit directly from the base of lemmas
that will be developed and maintained by Logipedia.
\\

&
\hspace{0.4cm}
For example, a typical railway application requires that tens of thousands of proofs are performed. The rate of automation being typically around 80 \%,
this leaves hundreds if not thousands of proofs that have to be conducted with a highly specialized human interaction. Such interactions sometimes require
more than one thousand manual steps and the creation of lemmas that have to be later verified by a third-party expert.
Logipedia, by enabling both the application of new proof tools in this context, would both increase the ratio of automatic proof and facilitate the verification
of manually introduced lemmas. Industrial development based on the B method should improve significantly by both cost reduction
and increased reliability.
\\

& The following challenges met by CLEARSY are
directly addressed in Logipedia:\\

& $\blacktriangleright$ leverage advances in proof techniques
developed in other communities due to the disparity of expressiveness
in different languages,\\

& $\blacktriangleright$ certifying the results of the tools used for
proof (including CLEARSY's own tools),\\


& $\blacktriangleright$ integration of new technologies in an
industrial context demanding qualification with respect to safety
standards.  \\

\hline

{\bf Edukera}&

\hspace{0.4cm} Over the years Edukera has experienced an increasing
need for formal methods in education. This is due to the emergence of
new industrial activities (cryptocurrency and the verification of
smart contracts, software dependant autonomous vehicles, etc.). This
is why Edukera considers the existence of a solid open-source
education-dedicated interface for formal proofs, to be critical to the
upcoming industrial ecosystem.\\

&Edukera has developed an online educational application
to teach mathematics which relies on the use of the formal proof
assistant Coq. However, several issues in the design of the current
Edukera application prevent a larger user base and need to be solved:
developers in formal methods need to create their own theories and
teachers must be able to develop their exercises; plus the core
Edukera solution should be open-sourced for anyone to use, fix and
improve when necessary.  \\

&
\hspace{0.4cm} The use of Logipedia as the mathematical foundation of
the Edukera application can solve these critical issues: indeed
Logipedia federates a large scientific community around a unique
language, and a large developer community around a unique open-source
mathematical framework, which is beyond the financial means and
traction of the Edukera company on its own.  \\

&
\hspace{0.4cm} Students may already access the Edukera application and
exercises. Hence the existence of an open-source version of the
Edukera application will not affect the business model which is based
on the billing of class administration features (activity reports,
homework schedule, connection to the LMS, etc.).  \\

\hline

{\bf MED-EL}
&
Given strict and specific quality requirements in the health care
domain, medical software needs already a different approach to testing
and much stricter verification of requirements than a traditional
software. Health care software should provide for reliable data
exchange, save patients and professionals time and effort on routine
procedures, show the stable performance, and securely deal with the
sensitive data. Therefore, such a software should be validated from
the perspectives of interoperability, usability, performance, and
compliance with requirements, risk factors as well as industry
regulations.\\

&
\hspace{0.4cm}
MED-EL considers that a traditional software testing
might be still enhanced with methods assuring that under every
condition medical software provides desired results. Hence by adding
formal verification to medical software testing toolkit we could
considerably increase probability of delivering new functionalities
without unexpected behaviors.\\
&
\hspace{0.4cm}
Such development of formal proofs must be facilitated in the 
medical software industry by being able to reuse formal proofs developed
for other purposes.\\

\hline

{\bf OCamlpro} &

Logipedia will foster the innovation
potential of OCamlpro in two directions.\\

&
$\blacktriangleright$
The developpement of proof trace
outputs for automatic theorem prover, allowing an ever increasing
confidence in the results of formal verification is a major milestone
toward their usefulness and actual usage in concrete industrial
projects. It also helps mathematicians by allowing more automation in
their formalization of mathematic in assistant theorem provers.
\\

&
$\blacktriangleright$
OCamlpro also supports the dissemination of source code thanks to
the opam open-source package manager, already used to distribute packages
for the ocaml programming language and the Coq assistant theorem prover.
With the addition of Dedukti proofs in this project, this will demonstrate
the capacity of opam to be language-agnostic and help disseminate knowledge.
\\

\hline
{\bf Prove \& Run} &

Prove \& Run uses its innovative verification framework ProvenTools
to develop highly-secure bricks in order to help its customers resolve
the security challenges linked to the large-scale deployment of
connected devices and of the Internet of Things. Formal methods
are used to ensure the highest security requirements possible, such
as CC EAL7. As such, Prove \& Run's interest in Logipedia is
twofold.\\

&
$\blacktriangleright$
In order to obtain an EAL7 Common Criteria certificate, formal
verification by itself is not enough and a lot of other constraints
including documentation, testing and software development hygiene come
into play. The formal proofs themselves must be auditable and the underlying
framework documented at length to ensure it is used in a consistent fashion.
Logipedia would bring the ability to cross-check proofs obtained in a
framework with Dedukti, as well as other frameworkds compatible with
Dedukti, and thus bring much more confidence in our proofs and streamline
the evaluation process.\\

&
$\blacktriangleright$
When formally verifying low-level bricks of critical systems, one
particularly tedious and somewhat error-prone task is to get the models
of underlying architecture right. They are hard to design and hard to
test against actual hardware. Being able to share such models with
other key players, via translation to Dedukti, would improve the confidence
in models and speed up the inclusion of new hardware in formal developments.\\
    
\hline

{\bf Runtime Verification} &

Runtime Verification is a research and development start-up advocating for the industrial
adoption of formal programming language design using Matching Logic (ML), a
general purpose logic well-suited for programming language specification.
Runtime Verification promotes a "one semantics to rule them all" philosophy; that is, the usage
of the (testable, executable) operational semantics of a language to derive
formal analysis and verification tools for that language.

An important aspect of the formal program verification process is the
generation of (checkable) proof objects as justification for the verification
conditions.
Logipedia will foster the innovation potential of Runtime Verification in several directions:\\
&
$\blacktriangleright$
As part of Runtime Verification's involvement in the Logipedia project, the infrastructure of 
Runtime Verification's ML prover will be adapted to allow the generation of proof traces which
can be converted into proof objects.
\\
&
$\blacktriangleright$
Program verification crucially depends on SMT solvers for discharging
path conditions. The fact that Logipedia aims to incorporate several such SMT
solvers and the fact that ML itself will be integrated into Logipedia will
allow to combine ML proofs with the SMT solver proofs into a common proof
object.\\

& Moreover, since many logics / calculi / models, especially those
important in the study of programming languages, can be defined as
theories in ML, and since these logics are to be incorporated into
Logipedia, Dedukti can be used as a vehicle to smoothly integrate
proofs from these logics into more complex combined proof objects.\\ &
$\blacktriangleright$ Integrating ML with Logipedia will allow for the
proof objects to be exported and checked in all logics which are part
of the project, thus increasing the confidence in the proof itself by
allowing one to use their trusted proof infrastructure to explore and
check the proofs.  \\ \hline
\end{longtable}

The members of the club of industrial users also focus on
various aspect of Logipedia. Many of the, including 
{\bf Alstom}, {\bf Mitsubishi Electric}, {\bf
  Nomadic Labs}, {\bf IBM Research},  {\bf Origin Labs}, {\bf RATP},
etc.  insist on the importance of mutualizing formal proofs and
reusing existing ones.
Some
others, for instance {\bf Systerel}, are more interested in 
increasing the trust in formal developments, in particular by allowing
an external verification of the automated theorem provers and SMT
solvers.

Some others, for instance {\bf Trusted Labs} and {\bf Onera} insist
on the transformation of the certification process.

{\bf Arm} insists
on the importance of having system independent specifications of
pieces of software and hardware,   {\bf Facebook Artificial
  Intelligence Reseach} on the importance of having large
databases of proofs to train machine learnin algorithms, and  {\bf
  Siemens} on the fact that Logipedia is also a communication tool to
to remain up to date with the latest developments.

\newcommand\couic[1]{}

\couic{
  \subsubsection*{Logipedia will foster the innovation potential of the 
members of the club of industrial users}

{\bf Alstom} is a major rolling stock and railways signalling systems
supplier with more than 35,000 employees all over the world. It
provides its customers in more than 25 countries with safety-critical
systems that prevent hazardous situations for passengers, staff and
equipment while ensuring fluent and cost-effective operation.

For more than 30 years, Alstom has been developing or verifying with
formal methods some of its safety-critical signalling systems. He is
therefore an intensive end-user of proof systems (Atelier B, S3, Why3,
Isabelle) and highly concerned and interested in all projects that can
improve the quality and efficiency of its proof activity. This is the
reason why Alstom is highly interested in the Logipedia project and
wants to be an active member of its industrial user group.

{\bf Arm} is the world's largest provider of semiconductor IP and
is the architecture of choice for more than 90\% of the smart
electronic products being designed today: Arm designs have found their
way into more than 20,000,000,000 devices in 2018, for instance.
Increasingly, the Arm architecture is also being deployed in
supercomputers and servers, with Amazon’s AWS recently announcing its
Graviton line of Arm-based high-performance servers.

As well as the 32- and 64-bit CPU cores, their hardware products
extend to GPUs, DSP cores, cell libraries, memory compilers and system
components. Arm also produces a wide array of software products, many
of which are extremely security- sensitive: low-level firmware,
privileged security monitors, operating system kernel patches,
cryptography libraries, and transport layer security protocol
implementations are all developed and actively maintained by Arm
engineers on behalf of their wider ecosystem, for example. Arm
engineers also play an active role in developing and standardising
novel cryptographic protocols and ciphers through various
international bodies.

Given the ubiquity of the Arm architecture, Arm has an
obvious interest in ensuring that various functional and security
properties hold, and also that its microprocessor designs are indeed
correct instantiations of this architecture. As a result,
semi-automated formal techniques have long been used within the
company to ensure the correctness of our architecture and of our
microprocessor designs.

Whilst many of their verification flows have been built around
commercial tools that they license ``off-the-shelf'', they also
developed their own formal tools using SAT and SMT technology, and
many of these tools are in active use by our product engineering
groups. Moreover, we are increasingly experimenting with:

\begin{compactitem}
\item The deployment of formal techniques as an enhanced bug-finding
  mechanism for especially sensitive software products, using
  bounded-model checking technology and SMT-based pre/postcondition
  checking, using tools such as the C-bounded model checker (CBMC) and
  Frama-C.
\item Model checking novel locking mechanisms for highly concurrent
  code, to spot potential deadlocks, using the TLA model checker.
\item Formally documenting and semi-automatically finding security
  flaws in cryptographic protocols using dedicated model checking
  tools, such as Tamarin.
\end{compactitem}

However, often they would like to establish deeper properties of the
architecture and of their software implementations than can reasonably
be obtained using the semi-automated formal approaches described
above. For this reason, they have ongoing experiments with the use of
interactive theorem proving, using both Coq and Isabelle/HOL, two
popular tools that have shown promise in academia on a range of
software and hardware verification projects.

From Arm’s perspective, ideally there would be some way of
transferring definitions and proofs between some of these tools so
that we can maximise reuse, avoid wasted engineering effort, and also
potentially share the models that we develop and associated proofs
with the many communities surrounding these tools for use in academic
projects. Unfortunately, to our knowledge no robust mechanism exists
at present that can achieve this and therefore writing a formal model
in Coq almost certainly precludes the Isabelle/HOL community from
using that model without significant duplication of engineering
effort, for example, and vice versa.

In light of the above, Arm is especially excited by the Logipedia
project which aims to facilitate the sharing of definitions and proofs
between many different interactive theorem proving systems,
considering the problem a long-standing issue that is holding back
commercial adoption of interactive theorem proving technology.
Moreover, Arm wishes to keep track of the project as it progresses,
and potentially provide insight from their industrial use of formal
methods technology. For this reason, they are committing to joining
the Logipedia industrial users club.


{\bf Facebook} AI Research (FAIR) is the fundamental research lab in
Artificial Intelligence of Facebook. Its largest research center in
Europe is in Paris and has been employing over 70 people, mixing
researchers, engineers and students for over four years now. FAIR is
recognized as one of the top leading AI labs in the world and is proud
of doing fundamental research. FAIR actively engages with the research
community through publications, open source software, participation in
technical conferences and workshops, and collaborations with
colleagues in academia. FAIR has hundreds of academic partners across
the world, such as Inria.

FAIR Paris has a team working on deep learning for theorem proving, a
long-term effort in academic research. It uses existing formal proofs
as its learning datasets and is interested by the Logipedia project as
a possible source of training data. FAIR Paris also has a strong
interest in automatic language translation and would be keen to apply
this to formal proof languages.

Facebook is therefore willing to join the club of industrial users
that the Logipedia project wants to create.

{\bf IBM}'s Thomas J. Watson Research Center serves as the
headquarters of IBM Research -- one of the largest industrial research
organizations in the world -- with 12 labs on 6 continents. Scientists
at T. J. Watson, and at IBM around the globe, are pioneering
scientific breakthroughs across today's most promising and disruptive
technologies including the future of artifial intelligence, blockchain
and quantum computing. In all of these areas, formal proofs can bring
benefits ranging from reducing bugs to having a deep and fundamental
understanding of the studied objects.

The Logipedia project is of strong interest for IBM
Research. Developing proofs requires a huge effort. Being able to
reuse formally proved properties could reduce the development time and
thus allows to tackle larger problems. Moreover, each proof assistant
requires its own expertise. The ability to reuse proofs coming from
any system in the proof assistant of our choice is therefore a great
benefit.

That is why IBM Research would like to join the club of industrial
users of Logipedia. IBM Research would also be happy to contribute to
the project with Q*cert, a data base query compiler developed in Coq,
and could then be translated to other proof assistants thanks to
Dedukti and the tools developed around Dedukti.

{\bf Mitsubishi Electric} is one of the world's leading names in the
manufacture and sales of electrical and electronic products and
systems used in a broad range of fields and applications, in
particular energy and electric systems, industrial automation,
information and communication systems, electronic devices, and home
appliances. One of Mitsubishi Electric's objectives consist in
creating new value through innovation and promoting R\&D that persues
sustainable growth. For that purpose, Mitsubishi Electric devotes 5\%
of its revenue to its research and development budget, contributing to
realizing Society 5.0 and achieving the goals of the United Nation's
sustainable development goals.

Mitsubishi Electric R\&D Centre Europe believes that for achieving
those sustainable goals, one of the key factors is the confidence one
can have in software, that drives people and more people's life,
industry, etc. That's why it has studied for more than ten years
different formal methods theories and tools that can help designing,
specifying, implementing and maintaining software. Its objectives is
to fill the gap between state of the art academic work and industrial
objectives and processes, to bring the power of formal methods to
regular engineers. Among those formal methods, formal proofs allow to
reach the highest level of confidence in software, but they are also
the hardest to manipulate for non-specialists. Mitsubishi Electric
R\&D Centre Europe sees the Logipedia project as an important
milestone for widening access to formal proofs, and helping their
dissemination in industry for two reasons. First, by giving to an
encyclopedia of off-the-shelf and ready-to-use formal proofs, avoiding
to prove many times the same properties. Second, by also strengthening
the European formal proofs community and helping them to provide a
kind of unified interface to industry problems.

For those reasons, Mitsubishi Electric R\&D Centre Europe strongly
supports the Logipedia project proposal and intents to participate to
its club of industrial users.

{\bf Nomadic Labs} is a R\&D company that contributes to the
development of the Tezos protocol. It gathers experts in formal
methods, distributed systems, and programming language theory and
practice. It puts a particular emphasis on formal verification, and
uses and develops related tools, applying them to the Tezos codebase,
algorithms, and smart contracts. Nomadic Labs uses mostly the Coq and
F* proof assistants, and supports the development of Coq and other
tools dedicated to formal verification through its partnerships with
premier research institutes.

The Logipedia project aims to create a library of formal proofs in a
common pivot language called Dedukti. The existence of such a
populated library, and the possibility of bringing proofs from one
proof system to another one, will favor the general use of formal
verification by avoiding the duplication of efforts, and encouraging
proof reuse.

Nomadic Labs is of course very interested by such a project and is
willing to be a member of the club of its industrial users.


{\bf Onera} has research activities in the development and application
of formal methods for critical systems and software for aerospace
applications. We believe formal techniques are useful for risk
analyses, safety and security assessment of critical systems,
development and verification of critical software. We apply formal
techniques for aerospace applications (aircrafts, UAVs, robotics). We
have also been part of the international group that defined DO-333
(formal methods technical supplement to DO-178, software certification
standard for aeronautics).

Onera agrees on the significance of the problems addressed by the
Logipedia infrastructure, and has a serious interest in the approach
of the Logipedia project to tackle these problems.

Onera wants to be involved in the project by becoming a member of the
industrial club. As a member of this group, Onera will follow the
progress of the project and provide feedback and advice. It also would
like to contribute in formulating challenges that are important to
Onera, especially in relationship with certification constraints.

{\bf Origin Labs} develops blockchain applications and tooling for
blockchains. It is currently developing the Dune Network blockchain, a
public blockchain that has been running since June 2019. One of its
main concerns is security and safety, because its developments are
often used to store and manage sensitive information and
assets. Formal methods are a corner stone of its strategy around
security and safety, and Origin Labs plans to start developing soon a
framework for formal verification of smart contracts for Dune Network.

The Logipedia project looks very interesting for Origin Labs. Formal
verification is often a difficult task, even for what looks like
simple programs. Being able to reuse existing work, and translating
such works for different provers and tools, are important challenges,
to ease the development of industrial tooling for formal
verification. For these reasons, Origin Labs would like to give his
support to the Logipedia project, and be part of its club of
industrial users.

{\bf RATP} is well known for more than 30 years in the railway
industry to use formal methods in industrial applications. They
participated at the end of the 90's to the birth of the B method with
the Atelier B. Since then, RATP still promotes formal methods both for
our suppliers and our safety software assessment studies.

Mutualizing the proof concepts and axioms of different languages is
kind of a logic path from their point of view to ensure
sustainability, share and improve fundamental knowledge which can be
applied in their future applications.

{\bf Siemens} is one of the world's leading suppliers of control
systems for automatic urban transport. Siemens Mobility relies on
formal methods in order to guarantee the safety of the systems they
provide.

Participating in the club of industrial users of Logipedia is
interesting for Siemens Mobility because it would help them remain up
to date with the latest research developments in the subject of formal
proof, and more precisely, the possibility of exchange of proof
certificates among different proof assistant software. Formal
mathematical proof is an important part of the formal methods that
Siemens Mobility uses.

{\bf Systerel} is an SME specialized in the specification, design,
development, verification and validation, and assessment of real-time
and safety-critical systems. Systerel's main achievements thus
concern: on-board systems with hard real time or safety requirements;
safety related tools (data preparation, data validation, system
maintenance, etc.); formal specification of complex industrial
systems; evaluation of the RAMS (Reliability / Availability /
Maintainability / Safety) level of dependable systems.

Systerel applies formal methods to industrial systems everyday (mainly
using Systerel Smart Solver, B and Event-B). Systerel had also been in
charge of the maintenance of the Rodin platform for more than 10
years. It is thus quite naturally that Systerel got interested in the
Logipedia project. The purpose of the project is very relevant to its
business for several reasons: Firstly, the usage of Dedukti would
allow us to put better trust in automated theorem provers and SMT
solvers, by allowing an external verification of their
proofs. Secondly, they are sometimes in the need to prove some general
well-known theorem (in order to apply it to a specific case), and the
library approach of the Logipedia project would prove very useful by
permitting proof reuse, rather than having to carry one again a manual
proof (which can be quite costly).

This explains that Systerel wants to join the club of industrial users 
and participate to the advisory board of the Logipedia project.

{\bf Trusted Labs} is a global expert in security consulting and
evaluation with the connected ecosystem, and is the world leader in
security certification scheme definition. Trusted Labs is also an
accredited ITSEF lab in France, and conducts Common Criteria
evaluations.


Since more than 15 years, Trusted Labs acquired a great experience in
formal methods and support customers to get high level certifications
for their products. They also conduct or support industrials on
evaluations for common criteria certifications.

They provide their expertise to a large variety of customers, in
sectors such as health care, automotive, energy and connected
devices. Hence, Trusted Labs is very interested to be part of the
Logipedia industrial users club, and bring its contribution on formal
proofs.
}


%%% Local Variables:
%%%   mode: latex
%%%   mode: flyspell
%%%   ispell-local-dictionary: "english"
%%% End:

