\subsection{Progress beyond current achievements}

\subsubsection{Networking activities}

{\color{red} Read part D in the call (110 or such)}

This project will include several types of networking activities, to
foster a culture of cooperation between scientific communities, that
are today often too centered around one system and one theory.

First, the development of a common infrastructure is \emph{per se} a
networking activity, as each member of the project must investigate
how the theory she implements compares to other theories, implemented
in other systems.

In the same way, the use of a common infrastructure incentivizes each
proof developer to investigate in which theories and systems her proof
can be used.

This effort will also help to develop the formal proof community in
countries where this community is still small.

This will also foster a culture of cooperation between scientific
communities and the communities of users: teachers and engineers.
This is why the project includes a club of industrial users, and
a club of teachers. In particular, this will lead to new ways to teach
formal proofs to a new audience, freeing the teacher from the need to
chose a specific system.

The development and maintenance of {\sc Logipedia} will eventually lead
to the discussion of standards for proof languages, even if such an
effort is premature today.

A yearly workshop of {\sc Logipedia} developers and users will be
organized, continuing the effort started on the January 2019 meeting,
and pursued on the January 2020 meeting, where the project was
finalized. The research directions will also be discussed
every year with the advisory board.

{\color{red} Summer schools etc.}

\subsubsection{Transnational access}

Since our encyclopedia will be available online, it will of course be
accessible from every country in Europe and beyond.

The licence chosen for the proofs in {\sc Logipedia} and its interface
will allow a free access, a findable, accessible, interoperable, and
reusable data management.

Finally several interfaces must be developed for various audience: a
secondary school student looking for a theorem in geometry requiring a
different interface from a engineer looking for the correctness proof
of an algorithm.

This project will also help to abolish internal European borders as,
today, some researchers and engineers often use a system developed in
their own country, only a few major systems having an international
community of users.

\subsubsection{Joint research activities}

The project includes two types of joint research activities.  First,
as any infrastructure, it will allow joint research projects
between the users of this infrastructure that will be able to develop
new proofs together using different systems.

Second, as any infrastructure, {\sc Logipedia} raises new research
problems. Some of them have already been solved in the past and
require to be implemented jointly in a first version of the
encyclopedia. Others, such as automatic concept alignment, the
structure of the encyclopedia, or the development of interfaces are
newer and will trigger new cooperation between the teams of the
project.

\subsection{Innovation potential}

Formal methods are at a turning point. Several academic and
industrial successes have proved the readiness of the technology, but
this technology takes too much time to be adopted.

Analyzing this phenomenon, it is clear that the redundancy of the
efforts to develop proof systems and the lack of common theories,
benchmarks, and standards for these systems is a limiting factor.

This project to integrate the scientific and technological effort
around formal proofs in Europe is also a way to foster the
development of formal methods in industry, as the economic spinoffs
from the project will benefit the European industry.

{\color{red}
François B.}


%%% Local Variables:
%%%   mode: latex
%%%   mode: flyspell
%%%   ispell-local-dictionary: "english"
%%% End:
