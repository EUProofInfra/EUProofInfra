\subsection*{Progress beyond state of the art}


\begin{longtable}{|p{0.2\textwidth}|p{0.75\textwidth}|}
\hline
{\bf Networking activities}
&
This project will include several types of networking activities, to
foster a culture of cooperation between scientific communities, that
are today often too centered around one system and one theory.\\
&
\hspace{0.4cm} First, the development of a common infrastructure is
\emph{per se} a networking activity, as the twenty nine partners of the
project have to join their forces towards a common goal (which is much
more an integrating activity than, for instance, organizing a common
conference). Then as the partners will also use this infrastructure
for their own developments, they will have to understand what the
other partners are doing, and how this can help in their own work.\\ &
\hspace{0.4cm}
This effort will also help to develop the formal proof community in
countries where it is still at an early stage of development.\\
&
\hspace{0.4cm}
This will also strengthen a culture of cooperation between scientific
researchers and engineers in academia and in industry. Currently this
cooperation is too often point to point (for instance a company uses a
one system and develops links this the group of academic researchers
developing this system). But we need to develop a more comprehensive
culture of cooperation, around common languages and a common research
infrastructure.\\
&
\hspace{0.4cm}
Our action in this direction is twofold. First, some companies that
already have a strong commitment in formal methods (Clearsy, CEA,
MED-EL, etc.) are included in the project as partners. May be, more
importantly, other companies, that could not be part of the project
because the project is too small, or that have a less developed
culture in formal methods are members of the ``club of industrial
users''. This club, among other goals, participates to develop a
culture of technology watch in formal methods in industry, which is a
stepping stone for developing a strong industrial awareness in safety
and security in European industry. This club also is currently too
small, but we are confident that organizing this club will only be a
first step, that will be followed by others.\\
&
\hspace{0.4cm}
It is important to develop a similar culture of technology watch for
researchers who are outside, or at the border of formal methods.  This
is why we have also organized a ``club of academic users'' that
gathers colleagues from our community that could not participate to
this project and colleagues outside the formal methods community, in
particular working mathematicians who are understanding the long term
impact formal proofs will have on mathematics.\\
&
\hspace{0.4cm}
It is also important to develop a similar culture of technology watch
for teachers, including high school teachers who are investigating how
a library of formal proofs can be used with younger students.  This is
why we have also organized a ``club of users in education''.  Here we
do not advocate teaching formal proofs to pre-schoolers (we should not
repeat the mistakes of the past), but we claim that having rigorous
statements of theorems in high-school textbooks (including all the
corner cases that are often omitted) and a clear dependency of which
theorem is used in the proof of which is a way to foster a culture of
rigor in high-school teaching, which is both useful for the students
who will take science at University and to those who will not. In
particular, we want, in this way, to contribute modestly to the
renewal of the culture of logical thinking, which is of prime
importance in our ``post-truth era''.\\
&
\hspace{0.4cm}
We also believe that an early exposition to formal statements and / or
formal proofs may contribute to compensate the shameful gender balance
in our research community. Here also, we must remain modest, as
achieving a decent gender-balance will require more than a single
action, but is is obviously much more effective to try to promote
contemporary scientific ideas, and encourage scientific carriers, to
women at the high school level rather that at the university level,
because at the university level there are already very few women in
our auditoriums.\\
&
\hspace{0.4cm}
We also plan to develop networking activities with publishers
(Elsevier, Springer, etc. but also, ArXiv, HAL, Wikipedia, etc.)  in
such a way that formal proofs mentioned in research papers and in
other encyclopedia can be made accessible, in Logipedia, for a long
period ot time, while currently, they are often just made accessible
on the web page of the authors.\\
&
\hspace{0.4cm}
These four clubs of users in industry, research, education, and
publishing, together with the partners of the project, will also
prepare a larger community that will develop Logipedia beyond the end
of the project, so that Logipedia contains all the formal proofs then
developed in twenty years.\\
&
\hspace{0.4cm}
Finally, Logipedia also prepares another kind of networking activities
beyond the duration of the project, as this effort will eventually
lead to the discussion of standards for proof languages.\\
&
\hspace{0.4cm}
We, of course, plan to organize the usual networking activities,
such as a yearly conferences with associated workshops on applications
in industry, research, education and publishing, a general assembly
where the research directions can be discussed collectively. And
summer schools, specially at the beginning of the project in order to
train the new participants (doctoral students, post-docs, etc.) to the
technology developed in Dedukti and Logipedia.
(See Section Dissemination and Communication.)\\  
\hline
{\bf Trans-national access and virtual access}
&
Logipedia will be accessible online. It will therefore be accessible
at no cost, and without identification, from every country in Europe
and beyond, just like, for instance, Wikipedia is.\\
&
\hspace{0.4cm}
The licence chosen for the Logipedia proofs needs not be the same for
all proofs, because some proofs already have a licence before being
imported in Logipedia and, in some cases, this licence must be
preserved.  Yet, in general we will favor a creative common licence
and in particular cc-by.  Such a licence allows a free
access, a findable, accessible, interoperable, and reusable
data management. It will contribute to the development of the Open
data / Open science / Open innovation philosophy.\\
&
\hspace{0.4cm}
Being a central infrastructure, Logipedia will contribute to abolish
internal European borders as, today, researchers and engineers often
use a system developed in their own country (only a few systems having
an international community of users), and libraries of formal proofs
specific to this system.\\
&
\hspace{0.4cm}
As explained above, our effort on access goes beyond providing a
trans-national and virtual access, as accessibility depends also on
developing an ergonomic web interface, a package distribution system,
a search engine, and an ontology of mathematical concepts. The public
targeted by these interfaces also has to be taken into account, a
secondary school student looking for a theorem in geometry needs a
different interface from a engineer looking for the correctness proof
of an algorithm.\\
\hline
{\bf Joint research activities}
&
The project includes two types of joint research activities.  First,
as any infrastructure, it will allow joint research projects between
the users of this infrastructure that will be able to develop new
proofs together using different systems.\\
&
\hspace{0.4cm}
Second, as any infrastructure, Logipedia raises new research
problems. Some of them have already been solved in the past and
require to be implemented jointly. Others are newer.\\
&
\hspace{0.4cm}
First, as we have explained, using a common infrastructure, requires
to describe the theories implemented in the different systems in a
common logical framework. We have already discussed how the expression
of geometry, arithmetic, and set theory in predicate logic has
permitted a renewal of logic at the end of the 1920's, allowing all
the logicians to speak the same language, and we can expect a similar
renewal here, fostering new joint research through the sharing, not
only of a common infrastructure, but also of a common language.\\
&
\hspace{0.4cm}
The development of a common encyclopedia, such as Logipedia, also
raises completely new problems such as automatic concept alignment,
structuring a large body of knowledge, or the development of
interfaces. These problems will trigger new cooperation between the
partners of the project and beyond.\\
\hline
\end{longtable}

\subsection*{Innovation potential}

Formal methods are at a turning point. Several academic and
industrial successes have proved the readiness of the technology,
in particular in critical systems where it has helped in
dramatically improving the quality of the systems. But this
technology takes too much time to be adopted in a broader context.

Limiting factors, probably the main ones, are the redundancy of the
efforts to develop proof systems, the lack of a common theory or at
least a common language to express the various theories, the lack of
common benchmarks, and the lack of standards for these systems.  For
industry, at least three key aspects slow down the adoption of formal
methods.

First, reusing a proof produced with a particular tool in another
can only be done at a high cost, when it is even possible.
Logipedia will help reducing this cost by unifying tools
around a common format, enabling the possibility to share proofs
between tools. In the uncommon case where a proof relies on a theory
which is not compatible with the target tool, it will be easier to
understand why and determine whether adapting it is managable.
Furthermore, as an infrastucture, Logipedia will help users
in finding existing proofs of properties, making their verification
process faster.

Second, checking a proof must remain possible over time. Today, it
requires either to maintain proofs along new versions of the tools,
which can represents a significant maintenance cost, or to archive
them together with a version of the tool used to produce it. In this
situation, Logipedia will help on two aspects. First, the
common format will guarantee that proofs can be checked by any tool
implementing it, thus reducing proof maintenance cost. Second, by
providing a common proof database, general interest proofs can be
stored and maintained in the infrastructure, allowing industrial users
to focus their resources for their specific needs only.

Finally, a proof or verification tool can be mistrusted. For example,
in a certification context, the use of a particular tool for the
verification of the candidate system must be approved. If the
certification body is not familiar with the tool, producing a
justification for it can represent a significant amount of time. For
a new tool, adoption is even slower, as not only certification
bodies but also potential users could question its soundness despite
the potential advantages it could provide. The common format offered
by Logipedia will answer to this lack of trust. It will help
the certification bodies, as they will not have to learn about a new
tool for each new certification process, as any tool implementing
the format will be suitable for them to check the proof. Thus they
only have to trust one. Consequently, it will also help industrial
users, as justifying the use of a particular tool will be easier as
long as they can provide a proof in the right format. And finally, it
will encourage the development of new tools, as potential users will
not have to trust these tools directly, but only the proofs that they
provide, which can be checked easily.

This project to integrate the scientific and technological effort
around formal proofs in Europe is thus a way to foster the
development of formal methods in industry, as the economic spinoffs
from the project will benefit the European industry, mainly reducing
the cost of this technology.

%%% Local Variables:
%%%   mode: latex
%%%   mode: flyspell
%%%   ispell-local-dictionary: "english"
%%% End:
