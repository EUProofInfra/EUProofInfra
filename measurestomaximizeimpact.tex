The measures to maximize impact are at the heart of the organization
of the work package dissemination and communication. They will be made
more precise in the {\em Plan for dissemination and exploitation of
results} that will be delivered at month 4 of the project.
Here are the first elements to build this plan.

The objectives of this plan are:
\begin{compactitem}
\item Increase the awareness and use of Logipedia in academia.
\item Use Logipedia as a way to increase the cooperation between academia and
industry. 
\item Prepare the sustainability and exploitation of Logipedia before the
  end of the project.
\end{compactitem}


%%%%%%%%%%%%%%%%%%%%%%%%%%%%%%%%%%%%%%%%%%%%%%%%%%%%%%%%%%%%%%%%%%%%%%%%%%%%%%
\subsection*{(a) Plan for dissemination and exploitation of results}
\label{sec:dissemination}

{\color{red} remove schedule ?}


\begin{longtable}{|p{0.12\textwidth}|p{0.50\textwidth}|p{0.10\textwidth}|
p{0.10\textwidth}|}
\hline
{\bf Target}
&
{\bf Action}
&
{\bf Indicator}
&
{\bf Schedule}
\\
%%%%%%%%%%%%%%%%%%%%%%%%%%%%%%%%%%%%%%%%%%%%%%%%%%%%%%%%%%%%%%%%%%%%%%%%%%%%%%
\hline
Researchers.
&
{\bf Participation to conferences.}
In computer science, publishing in conference
proceedings are often favored over journal publication. We have targeted
more than ten conferences:
% in alphabetical order
CADE (Conference on Automated Deduction),
CICM (Intelligent Computer Mathematics),
CPP (Certified Programs and Proofs),
CSL (Computer Science Logic),
FSCD (Formal Structures for Computation and Deduction),
FROCOS (Frontiers of Combining Systems),
ICALP (International Colloquium on Automata, Languages, and Programming),
%ICFP (Functional Programming),
IJCAR (International Joint Conference on Automated Reasoning),
ITP (Interactive Theorem Proving),
LFMTP (Logical Frameworks and Meta-pLanguages: Theory
and Practice), 
LICS (Logic in Computer Science),
LPAR (Logic Programming and Automated Reasoning),
PxTP (Proof eXchange for Theorem Proving).
&
10 papers.
&
Every year.
\\
\hline

Researchers, industrials.
&
{\bf Organisation of conferences.}
We shall organize our own event, with a conference,
specialized workshops, and the general assembly.
&
100 participants.
&
Every year.
\\
\hline
Master and PhD students
&
{\bf Organizing summer schools}
open to anyone and not only the partners.
We shall organize several training sessions targeting the
different communities of users: master and PhD students to teach them
the foundations of Logipedia, teachers to help them use interactive
theorem provers at school and university, and to engineers to help
them use formal methods tools in their work. Such training sessions
are key dissemination events that will accompany the growing of the
Logipedia community and contribute to educate a new generation of researchers,
teachers and engineers.
&
2
&
Every two years.
\\
\hline
PhD students.
&
{\bf Educate a new generation of researchers and engineers}, 
by advising 
PhD
students. Some will have academic and industrial co-advisors.
&
2 PhD students start
&
Every year.
\\
\hline
Researchers, industrials.
&
{\bf Co-building the Logipedia strategy}
by participating to joint meetings, such as
the clubs and advisor advisory board.
&
Participation to the meetings
&
Every year.
\\
\hline
\end{longtable}


\paragraph*{Clubs of users}

The second, less classical, type of dissemination activity, is the
organization of four clubs of users, which is a first step towards
extending the community beyond the sole partners of
the project, to all stakeholders of the Logipedia ecosystem.

Each club will organize a yearly workshop, possibly co-located with
international conferences in order to maximize its visibility
and the dissemination of Logipedia results.

More dissemination activities will take place in these clubs,
for instance the redaction of teaching material in the club of users in
education, new reproducible reference practices in the club of users
in publishing, and new use cases and new certification processes in the
club of industrial users.

\paragraph*{Management of research data generated and/or collected}

The dissemination and exploitation of our data is different from many
other projects, as the key to the success of such an infrastructure is
to make the data as open and as free as possible so that the data is
findable, accessible, interoperable and reusable. Such data cannot be
subject to a patent. It should not be subject of intellectual property
and should be distributed under a free licence, such as cc-by. The only
restriction to this is that we may need to include, in Logipedia,
libraries that already have a licence that must be preserved.  For the
formal proofs that come to Logipedia without a pre-existing licence,
we suggest to use a cc-by licence or another free licence.

We will of course comply to the GDPR, but as Logipedia, just like 
Wikipedia, does not collect personal data, this issue should be simple.

We will participate in the extended Pilot on Open Research Data as
will be formalized in the data management plan we will release at
month 6.

\paragraph*{Open source software used and developed by the project}

Most academic provers and libraries (Agda, Coq, Isabelle, Why3,
Dedukti, etc.) used by the project are open software. Many of them are
easily accessible on free platforms like \url{http://www.github.com/}.

The tools that will be developed for Logipedia will also be released
as open source software under some free license.

\paragraph*{Knowledge management and protection}

As for our publications, we will favor free on-line access and
self-archiving (also called ``green'' open access publishing), for
instance, on \url{http://hal.inria.fr/} and \url{https://arxiv.org/}.

\paragraph*{Exploitation}

During the project, the Inria Saclay transfer, innovation, and
partnership department, will contribute to the innovation management.
Similar departments in the organisations of the other beneficiaries of
the project will also contribute. Of course, our seven industrial
partners already contribute to this innovation management.

As stated in the
\href{http://roadmap2018.esfri.eu/media/1048/rm2018-part1-20.pdf}{2018
  roadmap} of the European Strategy Forum on Research Infrastructures
(ESFRI): ``A robust long-term vision is essential to successfully and
sustainably develop, construct and operate Research Infrastructures.''
So even if Logipedia is still in its ``incubation phase'' our ambition
is that Logipedia, that will be hosted by Inria Saclay and Technische
Universität München, remains accessible long after the project is
finished.

Two key factors determine the sustainability of the infrastructure,
that in turn determines the sustainability of the proofs it contains.
\begin{compactitem}
\item First, its architecture must be simple enough so that it
  requires little maintenance to preserve the proofs it contains, and
  even to allow more theories and more proofs to be integrated.
\item Second, it must contain enough proofs so that its usefulness has
  been demonstrated, and it keeps momentum. Beyond Inria and TUM, we
  should include several other stakeholders in a structure in charge of
  collecting funds to ensure the sustainability of Logipedia.
\end{compactitem}

If we succeed in making Logipedia big enough, useful, and easy to
maintain, we are confident that Inria and TUM will continue to
maintain it even after the end of this project and that we will find
other sources of finance for maintaining Logipedia, in the end of the
project.

Of course, we do not just want the encyclopedia to be maintained, but
there still will be more to be done after the end of the project. The
roadmap to continue to develop Logipedia after the end of the project
is twofold. First, there will be some work similar to that conducted
during the project, such as integrating more libraries (for instance,
the ACL2, or the Nuprl libraries that are not addressed in the
project), developing better search engines for formulas, etc.

Then, some themes emerged during the preparation of this proposal such
as ``Logipedia, security, and certification'', ``Logipedia and
automated theorem proving'', ``Logipedia and the B method''... that
could be independent projects once this project is over.

This roadmap will be made more precise in an update of 
the {\em Plan for dissemination and exploitation of results} 
we shall write in the third year of the project.

%%%%%%%%%%%%%%%%%%%%%%%%%%%%%%%%%%%%%%%%%%%%%%%%%%%%%%%%%%%%%%%%%%%%%%%%%%%%%%
\subsection*{(b) Communication activities}

Communication activities address several groups: the general public, through
institutional communication and outreach, and the consortium itself through
internal communication.

\paragraph*{Institutional communication}

Logipedia will have a visual identity, a logo, flyers presenting the
project, and a web site to be identified, and defined, by all partners,
stakeholders, and institutions we work with.

These communication activities are included in the budget under the
work package ``Dissemination and communication''.

\paragraph*{Outreach}

As the European citizen must be informed of the progress of science
she funds, we will, as it is our habit, publish popular science
articles in magazines, online videos, and live events such that the
European Researchers Night or {\em La Fête de la Science}, focusing
both on the scientific results and on the values, such a project
carries.


\paragraph*{Internal communication}

The goal of the internal communication is to allow each partner to
know what the others are doing. Actions exist at different levels.

\begin{compactitem}
\item The yearly event is a way to communicate the main scientific and
  technological progress of each partner.
\item The steering committee meets on a regular basis, through a
  teleconference, and exchanges about managerial, scientific, and
  technological issues.
\item Within each work package, the task leaders meet, through a
  teleconference, on a regular basis to coordinate the action of the
  work package.
\item Within each task, the partners communicate on teleconference, to
  coordinate the action of the work package. Some tasks have a budget
  to organize meetings, depending on their needs.
\end{compactitem}

\paragraph*{Monitoring communication activities}

Different types of communication activities are monitored in
different ways.  The institutional communication activities are
monitored by the leader of the work package ``Dissemination and
communication''. The internal communication activities are monitored
by the project coordinator and each work package leader. The communication
in conferences and workshops is monitored by the
the leader of the work package ``Dissemination and
communication''. And the outreach activities are monitored by the
project coordinator.

%%% Local Variables:
%%%   mode: latex
%%%   mode: flyspell
%%%   ispell-local-dictionary: "english"
%%% End:
