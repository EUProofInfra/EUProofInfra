{\color{red} Speak about TRL}

- Likn to WP dissemination and communication

- Link with other projects (in Part 1).

- Dissemnination and communication plan Month 3

\subsection{Dissemination and exploitation of results}
\label{sec:dissemination}

\subsubsection*{Publications, conferences, etc.}

- Publication : give a list of journals 

- Conferences (give names, places, dates)

- Summer schools...

- co-avised PhDs on Logipedia, between Academia and Industry

- popular science

\subsubsection*{Clubs of users}

- use cases from industry

- Certification authorities

- Teachers

- Publishing

- Textbooks for working mathematicians and math students...



\subsubsection*{Open access}

- Open access

- Open reseach data pilot (we are in, and we are making our data available,
data managdement plan at M3, but already a skeleton: Licence, etc.

Intelectual property

comply to the GSPR no personal data)

(check that we are in the ORDP on the website of the EU)

No backgroud IP, except on some libraries
we shall preserve.

We shall write a consortium agreement, that will be signed before the
project starts [[[demander à Inria quel modèle utiliser (DESCA ?)]]]
Talk about IP here. No IP : open source, no patents on algorithms.
The consortim sill follow the EU guidelines for IP rights.
Everything will be published. IP will be mentioned in the General assembly.

The details about all these will be given in the consortium agreement.
 
\subsubsection*{Exploitation}

- Exploitation plan (in the last third of the project)

- Inria and TUM are going to make sure the infrastructure remains active


- Exploitation: how will the results be exploited, how will the project
continue (road map), what will remain to be done

- Here we can speak about after 2024 (only place in the document)

\subsection{Communication activities}


\begin{todo}{}\color{red}
  Describe the proposed communication measures for promoting the project and its findings during the period of the grant. Measures should be proportionate to the scale of the project, with clear objectives.  They should be tailored to the needs of different target audiences, including groups beyond the project's own community.

  See participant portal FAQ on how to address communication activities in Horizon 2020.

  For further guidance on communicating EU research and innovation for project participants, please refer to the H2020 Online Manual on the Participant Portal.
\end{todo}

Key messages

Tools

Timing

Geographical level

Web site

Objective of communication. Why do you want to communicate?

What aare the results you want to communicate.

Who is goind to do it (every WP has to do it as well)

Visual identity, logos...

Website

Events and conferences

How to evaluate / monitor these actions?

{\color{red} a small table for each}


%%% Local Variables: 
%%% mode: LaTeX
%%% TeX-master: "propB"
%%% End: 

% LocalWords:  ednote
