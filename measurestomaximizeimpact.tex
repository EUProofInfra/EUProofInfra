The measures to maximize impact are at the heart of the organization
of the work package dissemination and communication. They will be made
more precise in the {\em Dissemination and communication} that will be
delivered at month 3 of the project. Their overall objective is to
prepare the sustainability of Logipedia after the end of this
integrating phase of our starting community.

\subsection*{(a) Dissemination and exploitation of results}
\label{sec:dissemination}

\subsubsection*{Publications, conferences, etc.}

The first dissemination activity is, of course, the publication of
scientific papers. In computer science publication in conference
proceedings are often favored over journal publication. The
conferences relevant to our area of research are FSCD, (Formal
Structures for Computation and Deduction), LICS (Logic in Computer
Science), ICALP (International Colloquium on Automata, Languages, and
Programming), CSL (Computer Science Logic), CADE (Conference on
Automated Deduction), and IJCAR (International Joint Conference on
Automated Reasoning). Note that the first paper on the $\lambda
\Pi$-calculus implemented in Dedukti was published in TLCA (now FSCD)
and the first invited lecture was delivered at ICALP. We should also
mention two smaller, but more specific venues, LFMTP (Logical
Frameworks and Meta-Languages: Theory and Practice) and PxTP (Proof
eXchange for Theorem Proving), that the development of Logipedia
should promote.

We shall also organize our own yearly event, with a conference,
specialized workshops, and the general assembly.

A second classical type of dissemination event is the organization of
summer schools. We shall organize two. A first one during the first
year of the project and a second one during the third year. Such summer
schools, especially the first one, are key dissemination events, that
will accompany the growing of the Logipedia community in the first
year of the project.

The project will also increase the cooperation between academia and
industry, both with the participation to shared events, such as the
Logipedia conference and workshops, but also by co-advising PhD students
and participating in joint meetings, such as those of the advisory
board.

\subsubsection*{Clubs of users}

The second, less classical, type of dissemination activity, is the
organization of four clubs of users, which is a first step towards
extending the community to all stakeholders, beyond the partners of
the project.

Each club will organize a yearly workshop at the annual Logipedia
event. More dissemination activities will take place in these clubs,
for instance the redaction of teaching material in club of users in
education, new reproducible reference practices in the club of users
in publishing, and new use cases and new certification processes in the
club of industrial users.

\subsubsection*{Open access and intellectual property}

The dissemination and exploitation of our data is different from many
other projects, as the key to the success of such an infrastructure is
to make the data as open and as free as possible so that the data is
findable, accessible, interoperable and reusable. Such data cannot be
subject to a patent. It should not be subject of intellectual
property and should not distributed under a too restrictive
licence.  The only limitation to this is that we may need to include,
in Logipedia, libraries that already have a licence that must be
preserved.  For the formal proofs that come to Logipedia without a
pre-existing licence, we suggest to use a cc-by licence or another 
free licence.

{\color{red} No personal data, like Wikipedia}

Moreover, the data we shall collect contains no personal data, so this
simplifies the ethical issues and the compliance to the GDPR.

We will comply to the Open Research data pilot, as will be formalized
in the data management plan we shall release at month 3 and in the
consortium agreement that will be signed before the project starts.

\subsubsection*{Exploitation}


As stated in the roadmap ({\tt
  http://roadmap2018.esfri.eu/media/1048/rm2018-part1-20.pdf}) of the
European Strategy Forum on Research Infrastructures: ``A robust
long-term vision is essential to successfully and sustainably develop,
construct and operate Research Infrastructures.''  So even if
Logipedia is still in its ``incubation phase'' our ambition is that
Logipedia, that will be hosted by Inria Saclay and by die Technische
Universität München, remains accessible long after the project is
finished, an encyclopedia having to remain accessible for a very long
period of time.

Two key factors determine the sustainability of the infrastructure,
that in turn determines the sustainability of the proofs it contains.
\begin{itemize}
\item First, its architecture must be simple enough so that it requires
  little maintenance to preserve the proofs it contains, and even to
  allow more theories and more proofs to be integrated.
\item Second, it must contain enough proofs so that its usefulness has been
demonstrated, and it becomes ``too big to fail''.
\end{itemize}

If we succeed in making Logipedia big enough, useful, and easy to
maintain, we are confident Inria and TUM will continue to maintain it
even after the end of this project.

{\color{red} On va chercher de nouveaux financement en fin de projet
  pour la maintenance (communauté de dev --> asso)}

Of course, we do not just want the encyclopedia to be maintained, but
there still will be more to be done after the end of the project. The
roadmap to continue to develop Logipedia after the end of the project
is twofold. First, there will be some work similar to that conducted
during the project, such as integrating more libraries (for instance,
the ACL2, or the Nuprl libraries that are not addressed in the
project), developing better search engines for formulas, etc.

Then, some themes emerged during the preparation of this proposal such
as ``Logipedia, security, and certification'', ``Logipedia and
automated theorem proving'', ``Logipedia and the B method''... that
could be independent projects once this project is over.

This road-map will be made more precise in the exploitation plan
we shall write in the third year of the project.

{\color{red} Inria Saclay transfer, innovation, and partnetship
  department will contribute to the innovation management}


\subsection*{(b) Communication activities}

Communication activities address several needs.

\subsubsection*{Institutional communication}

As any project, Logipedia will need to have a visual identity, a logo,
flyers presenting the project, and a web site.
This way, the project can be identified, and defined, by all partners,
stakeholders, and institutions we work with.

These communication activities are included in the budget under the
work package ``Dissemination and communication''.

\subsubsection*{Internal communication}

The goal of the internal communication is to allow each partner to
know what the others are doing. Actions exist at different levels.

\begin{itemize}
\item The yearly event is a way to communicate the main scientific and
  technological progress of each partner.

\item The steering committee meets on a regular basis, through a
  teleconference, and exchanges about managerial, scientific, and
  technological issues.

\item Within each work package, the task leaders meet, through a
  teleconference, on a regular basis to coordinate the action of the
  work package.

\item Within each task, the partners communicate on teleconference, to
  coordinate the action of the work package. Some tasks have a budget
  to organize meetings, depending on their needs.
\end{itemize}

\subsubsection*{Open conference and workshops}

To permit the growth of the community beyond the partners and the
members of the clubs, the yearly conference and workshop will be
advertised and open for communications to academic and industrial
researchers who are not members of the project.

\subsubsection*{Outreach}

Finally, as the European citizen must be informed of the progress of
science she funds, we shall use our habit to publish popular science
articles in magazines, online videos, and live events such that the
European Researchers Night or {\em La Fête de la Science}, focusing
both on the scientific results and on the values, such a project
carries.

\subsubsection*{Monitoring communication activities}

Different types of communication activities are monitored in
different ways.  The institutional communication activities are
monitored by the leader of the work package ``Dissemination and
communication''. The internal communication activities are monitored
by the project coordinator and each work package leader. The communication
in conferences and workshops is monitored by the
the leader of the work package ``Dissemination and
communication''. And the outreach activities are monitored by the
project coordinator.

%%% Local Variables:
%%%   mode: latex
%%%   mode: flyspell
%%%   ispell-local-dictionary: "english"
%%% End:
