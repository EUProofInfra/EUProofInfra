{\color{red} Speak about TRL}

The measures to maximize impact are at the heart of the organization
of the work package dissemination and communication. They will be made
more precise in the {\em Dissemnination and communication} that will
be delivered at month 3 of the project. Their overall objective is to
prepare the sustainability of Logipedia after the end of this integrating
phase of our starting community.

\subsection{Dissemination and exploitation of results}
\label{sec:dissemination}

\subsubsection*{Publications, conferences, etc.}

The first dissemination activity is, of course, the publication of
scientific papers. In computer science publication in conference
proceedings are often favored over journal publication. The
conferences relevant to our area of research are FSCD, (Formal
Structures for Computation and Deduction), LICS (Logic in Computer
Science), ICALP (International Colloquium on Automata, Languages, and
Programming), CSL (Computer Science Logic), CADE (Conference on
Automted Deduction) / IJCAR (International Joint Conference on
Automated Reasonning). Note that the first paper on the $\lambda
\Pi$-calculus implemented in Dedukti was published in TLCA (now FSCD)
and the first invited lecture was delivered at ICALP. We should also
mention two smaller, but more specific venues, LFMTP (Logical
Frameworks and Meta-Languages: Theory and Practice) and PxTP (Proof
eXchange for Theorem Proving), that the development of Logipedia
should promote.

We shall also organize our own yearly event, with a conference,
specialized workshops, and the general assembly.

A second classical type of dissemination event is the organization of
summer schools. We shall organize two. A first one during the first
year of the project and a second one durin the third year. Such summer
schools, specially the first one is a key dissemination event, that
will accompany the growing of the Logipedia community in the first
year of the project.

The project will also increase the cooperation between academia and
industry, both with the participation to the same events, such as the
Logipedia conference, workshops, but also by co-advising PhD students
and participating in joint meetings, such as those of the advisory
board.

Finally, we want to mention another type of dissemination action, that
is more communication than dissemination but that fits in this
section: popular science, trough our participation to popular science
magazines and on-line videos.

\subsubsection*{Clubs of users}

The second, less classical, type of dissemination activity, is the
organization of four clubs of users that is a first step towards
extending the community to all stakeholders, beyond the partners of
the project.

Each club will organize a yearly workshop at the annual Logipedia
event. More dissemination activities will take place in these clubs,
for instance the redaction of teaching material in club of users in
education, new reproducible reference practices in the club of users
in publishing, new use cases and new certification processes in the
club of industrial users.



\subsubsection*{Open access}

- Open access

- Open reseach data pilot (we are in, and we are making our data available,
data managdement plan at M3, but already a skeleton: Licence, etc.

Intelectual property

comply to the GSPR no personal data)

(check that we are in the ORDP on the website of the EU)

No backgroud IP, except on some libraries
we shall preserve.

We shall write a consortium agreement, that will be signed before the
project starts [[[demander à Inria quel modèle utiliser (DESCA ?)]]]
Talk about IP here. No IP : open source, no patents on algorithms.
The consortim sill follow the EU guidelines for IP rights.
Everything will be published. IP will be mentioned in the General assembly.

The details about all these will be given in the consortium agreement.
 
\subsubsection*{Exploitation}

- Exploitation plan (in the last third of the project)

- Inria and TUM are going to make sure the infrastructure remains active


- Exploitation: how will the results be exploited, how will the project
continue (road map), what will remain to be done

- Here we can speak about after 2024 (only place in the document)

\subsection{Communication activities}


\begin{todo}{}\color{red}
  Describe the proposed communication measures for promoting the project and its findings during the period of the grant. Measures should be proportionate to the scale of the project, with clear objectives.  They should be tailored to the needs of different target audiences, including groups beyond the project's own community.

  See participant portal FAQ on how to address communication activities in Horizon 2020.

  For further guidance on communicating EU research and innovation for project participants, please refer to the H2020 Online Manual on the Participant Portal.
\end{todo}

Key messages

Tools

Timing

Geographical level

Web site

Objective of communication. Why do you want to communicate?

What aare the results you want to communicate.

Who is goind to do it (every WP has to do it as well)

Visual identity, logos...

Website

Events and conferences

How to evaluate / monitor these actions?

{\color{red} a small table for each}


%%% Local Variables:
%%%   mode: latex
%%%   mode: flyspell
%%%   ispell-local-dictionary: "english"
%%% End:
