The measures to maximise impact are at the heart of the organisation
of the work package dissemination and communication.
A first draft of the {\em Plan for dissemination and exploitation of
  results} will be delivered at month 4 of the project and will
be updated during the course of the project.
Here are the first elements to build this plan.
Its objectives are:
\begin{compactitem}
\item Increase the awareness of Logipedia in academia and its number of users.
\item Use Logipedia as a way to increase the cooperation between academia and
industry. 
\item Prepare the sustainability and exploitation of Logipedia before the
  end of the project, illustrating the philosophy of 
the 
\href{http://roadmap2018.esfri.eu/media/1048/rm2018-part1-20.pdf}{2018
  roadmap} of the European Strategy Forum on Research Infrastructures
(ESFRI): 
``A robust long-term vision is essential to successfully and
sustainably develop, construct and operate Research Infrastructures.''
even if Logipedia is still in its ``incubation phase''.
\end{compactitem}



%%%%%%%%%%%%%%%%%%%%%%%%%%%%%%%%%%%%%%%%%%%%%%%%%%%%%%%%%%%%%%%%%%%%%%%%%%%%%%
\subsection*{(a) Plan for dissemination and exploitation of results}
\label{sec:dissemination}


\subsubsection*{Dissemination}

\begin{longtable*}{|p{0.55\textwidth}|p{0.12\textwidth}|p{0.15\textwidth}|}
\hline
{\bf Action}
&
{\bf Target}
&
{\bf Indicator and schedule}
\\
%%%%%%%%%%%%%%%%%%%%%%%%%%%%%%%%%%%%%%%%%%%%%%%%%%%%%%%%%%%%%%%%%%%%%%%%%%%%%%
\hline
{\bf 1. Participation to conferences.}
In computer science, publishing in conference
proceedings are often favoured over journal publication. We have targeted
more than ten conferences:
% in alphabetical order
CADE (Conference on Automated Deduction),
CICM (Intelligent Computer Mathematics),
CPP (Certified Programs and Proofs),
CSL (Computer Science Logic),
FSCD (Formal Structures for Computation and Deduction),
FROCOS (Frontiers of Combining Systems),
ICALP (International Colloquium on Automata, Languages, and Programming),
%ICFP (Functional Programming),
IJCAR (International Joint Conference on Automated Reasoning),
ITP (Interactive Theorem Proving),
LFMTP (Logical Frameworks and Meta-Languages: Theory
and Practice), 
LICS (Logic in Computer Science),
LPAR (Logic Programming and Automated Reasoning),
PxTP (Proof eXchange for Theorem Proving).
&
Researchers.
&
10 papers per year.
\\
\hline
{\bf 2. Organisation of conferences.}
We shall organise our own yearly event, with a conference, parallel
specialised workshops, and the general assembly, and a final
conference to draw the conclusions of the past four years and discuss the
road map for the exploitation of the infrastructure after the end of the
project.
&
Researchers, industrials.
&
100 participants each year.
\\
\hline
{\bf 3. Organising summer schools}
open to anyone and not only the partners.
We shall organise several training sessions targeting the
different communities of users: master and PhD students to teach them
the foundations of Logipedia, teachers to help them use interactive
theorem provers at school and university, and to engineers to help
them use formal methods tools in their work. Such training sessions
are key dissemination events that will accompany the growing of the
Logipedia community and contribute to educate a new generation of researchers,
teachers and engineers.
&
Master and PhD students
&
2
\\
\hline
    {\bf 4.
Advising PhD students} 
to educate a new generation of researchers and engineers, 
Some will have academic and industrial co-advisors.
&
PhD students.
&
3 PhD students start each year.
\\
\hline
{\bf 5. Co-building the Logipedia strategy}
by participating to joint meetings, such as
the clubs and advisory board.
&
Researchers, industrials.
&
Participation to the meetings
every year.
\\
\hline
{\bf 6. Disseminate Logipedia in relevant communities.}
The fours clubs contribute to the dissemination of
Logipedia in their own ecosystem, by organising talks,
courses, meetings.
In particular, we will organize once a year an event, 
with the workshops of all the clubs to discuss our main achievements,
future developments, and exploitation.
&
Research, industry,
education, publishing.
&
At least one event organised by each club every year.
\\
\hline
{\bf 7. Delivering teaching material}
co-produced by 
the partners of the project and the members of
the club of users in education. 
&
Under- graduate and secondary education students
&
At least one textbook during the project.
\\
\hline
{\bf 8. Using Logipedia to increase reproducibility in science}
by referencing formal proofs in a single place. 
&
Publishers and researchers.
&
Researchers outside the consortium use Logipedia as a reference.
\\
\hline
{\bf 9. Initiate a discussion with certification authorities}
about the use of a common language across the European Union.
&
Certification agencies and the industry. 
&
Meetings are organised with several European certification agencies.
\\
\hline
{\bf 10. Use a free licence for data and software} to make 
the data is findable, accessible, interoperable and reusable
and develop Open data / Open science / Open innovation.
&
Scientists and innovators.
&
Delivery of Logipedia at month 14.
\\
\hline
{\bf 11. Publish in Open access venues.}
&
Scientists and students.
&
All the publications of the partners are open.
\\
\hline
\end{longtable*}


\subsubsection*{Exploitation}

\begin{longtable*}{|p{0.30\textwidth}|p{0.30\textwidth}|p{0.30\textwidth}|}
\hline
{\bf Action}
&
{\bf Stakeholders}
&
{\bf Indicator and schedule}
\\
\hline
%{\bf Monitoring the innovation during the course of the project.}
%&
%The steering commitee, the European project manager, the 
%transfer, innovation, and partnership department of Inria Saclay and
%any relevant member of our partners' institution.
%&
%Meetings organized with the transfer and innovation department
%upon request and at least once a year.
%\\
%\hline
{\bf 1. Build an organisation in charge of managing Logipedia
after the end of the project}.
&
The project management team, volunteer members of the consortium,
after consulting the advisory board.
&
The structure is created at month 48 at the latest.
\\
\hline
{\bf 2. Raise funds to manage Logipedia}.
&
The project management team, after consulting the advisory board.
&
Sponsors and fundings have been identified at month 36 at the latest.
\\
\hline   
{\bf 3. Find a server to permanently host Logipedia}.
&
The project management team, volunteer members of the consortium,
after consulting the advisory board.
&
Logipedia is kept functional.
\\
\hline
{\bf 4. Continue developing Logipedia}
&
New developers taking over.
&
New libraries (for instance that of ACL2 or Nuprl) are integrated.
New features are added to Logipedia.
\\
\hline
{\bf 5. Generate new projects, such as ``Logipedia, security, and
certification'', ``Logipedia and automated theorem proving'',
``Logipedia and the B method''...}
&
Special interest groups within Logipedia. 
&
New communities adopt Logipedia.
\\
\hline
\end{longtable*}

%%%%%%%%%%%%%%%%%%%%%%%%%%%%%%%%%%%%%%%%%%%%%%%%%%%%%%%%%%%%%%%%%%%%%%%%%%%%%%
\subsection*{(b) Communication activities}

Communication activities aim at raising the awareness about Logipedia
to potential stakeholders that would not be concerned by our
dissemination actions. It will ensure the growth of the Logipedia
ecosystem and be a way to advertise the work achieved by the partners
and the clubs of users. It also serves the purpose of informing the
European citizen of the research findings she has been financially
contributing to.

Six person-months of an experienced communication officer, from the
Inria Saclay communication team, are dedicate to the sole task of
communication.  She will work in close cooperation with the
all dissemination, communication, and exploitation work package leader
and the project management team.

In this project, we have the ambition to communicate to the general
public, even if, in the past, these communication activities have been
considered as less important than the dissemination towards the
research, industry, and education communities.

\begin{longtable*}{|p{0.30\textwidth}|p{0.30\textwidth}|p{0.30\textwidth}|}
\hline {\bf Action} & {\bf Target audience} & {\bf Indicator and
  schedule} \\
\hline {\bf Promoting the existence of the project}
including defining the project visual identity,
creating a web site,
designing flyers, posters, and videos, and
publishing press releases.
&
Research, industry, and education.
& Website at month 3.
\\

\hline
{\bf Promoting the results and the values of the project} through
outreach actions: publication of articles in popular science
magazines, online videos\footnote{such as {\tt
https://www.facebook.com/TheatreLaReineBlanche/videos/518698965681202},
a one-minute video hosted by le Th\'e\^atre de la Reine Blanche,
presenting in
French the example given at the beginning of this document.}, and live
events such that the European Researchers Night or {\em La Fête de la
  Science}, focusing, as it is our habit.  & General public.  & At
least five partners organise an outreach activity in their country.
\\ \hline
\end{longtable*}

%%% Local Variables:
%%%   mode: latex
%%%   mode: flyspell
%%%   ispell-local-dictionary: "british"
%%% End:
