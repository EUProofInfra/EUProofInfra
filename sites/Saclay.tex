\begin{sitedescription}{Sac}

Université Paris-Saclay is a newly created university since the 1st of
January 2020. Created from the merger of Université Paris-Sud and the
community of universities and institutions "Université Paris-Saclay", it
is recognized for its top level in fundamental science.  Since 2006,
scientists from the University were awarded two Fields medals, one Nobel
Prize and a number of other international and national prizes.
University Paris Saclay has a complete array of competences, ranging
from the purest of exact sciences to clinical practices in medicine,
covering life and health sciences, legal sciences and economics.
Research at the University of Paris-Saclay, an essential part of
academic understanding, is completed by research activities with a high
valorization potential. Located on the Paris-Saclay site, at the heart
of the most influential private financial and research areas in Europe,
Université Paris-Saclay is a significant driving force in the
development of industries, particularly in high-tech and technology
fields. 300 laboratories constitute the research potential of the
Université Paris-Saclay. They cover all scientific disciplines that
mobilize over 15,000 researchers, and PhD students.

Inside Université Paris-Saclay, the Laboratory for Computer Science
(LRI) covers a wide spectrum of computer science. Its VALS team works in
the Area of Verification and Validation of Algorithms, Languages and
Systems, right in the heart of the scientific field of Formal Methods.
Its group working on \pn will include Chantal Keller and Burkhart Wolff.

\paragraph{Main tasks:}

\begin{itemize}
\item WP4: \dots
\item WP6: \dots
\item WP7: \dots
\end{itemize}

\begin{compactitem}
\item\ednote{specify the main tasks and reference the respective work packages} 
\end{compactitem}

\paragraph{Publications, products or services:}

\ednote{a list of up to 5 relevant publications, and/or products,
  services (including widely-used datasets or software), or other
  achievements relevant to the call content}

\begin{itemize}
\item An automatic tool to import proofs between the two interactive
  theorem provers HOL Light and Coq, companion
  paper:~\cite{DBLP:conf/itp/KellerW10}
\item The SMTCoq project: \url{https://smtcoq.github.io} (companion
  papers:~\cite{DBLP:conf/cpp/ArmandFGKTW11,DBLP:conf/cav/EkiciMTKKRB17})
\item Isabelle/DOF : Design and Implementations.
 Sources/Papers: \cite{brucker_achim_d_2019_3370483,brucker.ea:isabelle-ontologies:2018}
\item Applications of Ontologies in the Context of Formal Software Engineering.
 \cite{brucker.ea:ontologies-certification:2019}
\end{itemize}

\paragraph{Previous projects or activities:}

\ednote{a list of up to 5 relevant previous projects or activities,
  connected to the subject of this proposal}

\begin{itemize}
\item The ANR-project Paral-ITP (\url{www.lri.fr/~wolff/projects/ANR-Paral-ITP/}) with the  
      partners INRIA Roquencourt, INRIA Saclay (Bruno Barras) 
      and U-PSud/LRI (B.Wolff, Project Leader) was targeting 
      of advanced parallelisation techniques for Coq and Isabelle.
\item The EU-Project EUROMILS (Oct. 2012 - Sept. 2015 ) and 
      ANR Project PST (\url{http://www.irt-systemx.fr/en/project/pst/})
      were both targeting at applying Formal Methods in an industrial
      effort aiming at a high-level certification (Common Criteria EAL6, 
      CENELEC SIL4). The latter project incited the development of 
      Isabelle/DOF\cite{brucker_achim_d_2019_3370483}.
\end{itemize} 

\paragraph{Infrastructures or technical equipments:}

\ednote{a description of any significant infrastructure and/or any major
  items of technical equipment, relevant to the proposed work}

The group has a mainstream experience in interoperability between proof
systems. It is in particular the leader of the SMTCoq project, a tool to
benefit from both worlds of interactive and automatic theorem provers.
This tool allows one to check automatic theorem provers with great
confidence, as well as to enjoin proof automation in the Coq proof
assistant.

Concept alignment is fundamental in this project, to relate Coq
mathematical objects with the built-in theories of automatic provers,
and thus benefit from the best possible automation. The necessity to
align concepts for proof interoperability was pioneered by one member of
the group, in a project to import proofs between the two interactive
theorem provers HOL Light and Coq.

The group also has experience in the representation of proofs as
high-level documents and tool integration at the implementation level
of Coq and Isabelle. 

\paragraph{Persons primarily responsible for carrying out the proposed
  activities:}

\begin{itemize} % in alphabetical order

\item {\bf Dr.\ Chantal Keller} is Associate Professor at Université
  Paris-Saclay since 2015. She obtained her PhD (2013) at École
  polytechnique. She is a pioneer of interoperability between proof
  systems as the developer of one of the first tools to import proofs
  between two interactive theorem provers with very different underlying
  logics: HOL Light and Coq. In this work, the alignment of mathematical
  concepts was the key to re-usability of proofs. She is the leader of
  the SMTCoq project that links the Coq proof assistant with external
  automatic theorem provers. She has strong expertise in interactive
  (conception and use) and automatic (mainly use) theorem provers, and
  interoperability between them. She is one of the two coordinators of
  WP4.

\item {\bf Pr.\ Dr.\ Burkhart Wolff} is full professor at Université
  Paris-Saclay since 2008. He has a strong background in interactive
  theorem proving for the modeling of languages semantics as well as
  applications to model-based testing. He pioneered an Ontological
  Framework, which is currently mostly used for document ontologies and
  ontologies imposing a particular theory structure and typed meta-data.
  He implemented this framework in Isabelle/HOL, together with Prof.
  A.D. Brucker. In this EU project, his role is to supervise the
  conception and implementation of an Dedukti-oriented ontological
  framework. He will contribute to the conception of ontologies used for
  prover interoperability as well as domain-specific ontologies
  supporting advanced search and access mechanisms. As such, he has the
  role of a mediator between the technical needs for (efficient)
  alignments on the one hand and end-users of the Logipedia libraries
  needing advanced search mechanisms in order to access its mathematical
  knowledge. He is one of the two coordinators of WP7.

\end{itemize}

\end{sitedescription}


%%% Local Variables:
%%% mode: latex
%%% TeX-master: "../propB"
%%% End:
