\begin{sitedescription}{Sac}

\logo{Saclay}

Université Paris-Saclay is a newly created university since the 1st of
January 2020. Created from the merger of Université Paris-Sud and the
community of universities and institutions "Université Paris-Saclay", it
is recognized for its top level in fundamental science.  Since 2006,
scientists from the University were awarded two Fields medals, one Nobel
Prize and a number of other international and national prizes.
University Paris Saclay has a complete array of competences, ranging
from the purest of exact sciences to clinical practices in medicine,
covering life and health sciences, legal sciences and economics.
Research at the University of Paris-Saclay, an essential part of
academic understanding, is completed by research activities with a high
valorization potential. Located on the Paris-Saclay site, at the heart
of the most influential private financial and research areas in Europe,
Université Paris-Saclay is a significant driving force in the
development of industries, particularly in high-tech and technology
fields. 300 laboratories constitute the research potential of the
Université Paris-Saclay. They cover all scientific disciplines that
mobilize over 15,000 researchers, and PhD students.

Inside Université Paris-Saclay, the Laboratory for Computer Science
(LRI) covers a wide spectrum of computer science. Its VALS team works in
the Area of Verification and Validation of Algorithms, Languages and
Systems, right in the heart of the scientific field of Formal Methods.
Its group working on \pn will include Chantal Keller and Burkhart Wolff.

\paragraph*{Main tasks:}

\begin{compactitem}
\item Chantal Keller is one of the two coordinators of \WPref{atpetc}. She has
strong expertise in interoperability between interactive and automatic
theorem provers (ATPs). She is the leader of the SMTCoq project that
links the Coq proof assistant with external ATPs. In this EU project,
she will contribute to make automatic theorem provers interact with
Logipedia: proofs will be exchange to and fro, ATPs will be used to
check proofs from other systems and increase interaction between them,
and to detect, organize and discharge theory alignments
(\taskref{alignment}{aligntheories} of \WPref{alignment}).

\item Burkhart Wolff is one of the two coordinators of \WPref{structuring}. He
pioneered an Ontological Framework, which is currently mostly used for
document ontologies and ontologies imposing a particular theory
structure and typed meta-data. He implemented this framework in
Isabelle/HOL, together with Prof. A.D. Brucker. In this EU project, his
role is to supervise the conception and implementation of a
Dedukti-oriented ontological framework. He will contribute to the
conception of ontologies used for prover interoperability as well as
domain-specific ontologies supporting advanced search and access
mechanisms. As such, he has the role of a mediator between the technical
needs for (efficient) alignments on the one hand and end-users of the
Logipedia libraries needing advanced search mechanisms in order to
access its mathematical knowledge.
\end{compactitem}

\paragraph*{Publications, products or services:}

\begin{compactitem}
\item ``Importing HOL Light into Coq'', by Chantal Keller and Benjamin
  Werner, First International Conference on Interactive Theorem Proving,
  2010.
\item ``A Modular Integration of SAT/SMT Solvers to Coq through Proof
  Witnesses'', by Michaël Armand, Germain Faure, Benjamin Grégoire,
  Chantal Keller, Laurent Théry and Benjamin Werner, First International
  Conference on Certified Programs and Proofs, 2011.
\item ``SMTCoq: A Plug-In for Integrating SMT Solvers into Coq'', by
  Burak Ekici, Alain Mebsout, Cesare Tinelli, Chantal Keller, Guy Katz,
  Andrew Reynolds and Clark W. Barrett, 29th International Conference on
  Computer Aided Verification, 2017.
\item ``Using The Isabelle Ontology Framework: Linking the Formal with
  the Informal'', by Achim D. Brucker, Idir Ait-Sadoune, Paolo
  Crisafulli and Burkhart Wolff, 11th Conference on Intelligent Computer
  Mathematics, 2018.
\item ``Using Ontologies in Formal Developments Targeting
  Certification'', by Achim D. Brucker and Burkhart Wolff, 15th
  International Conference on Integrated Formal Methods, 2019.
\end{compactitem}

\paragraph*{Previous projects or activities:}

\begin{compactitem}
\item Burkhart Wolff was a member of the ANR project
  \href{http://www.lri.fr/~wolff/projects/ANR-Paral-ITP/}{Paral-ITP}
  with the partners INRIA Roquencourt, INRIA Saclay (Bruno Barras) and
  U-PSud/LRI (B.Wolff, Project Leader), which was targeting of advanced
  parallelisation techniques for Coq and Isabelle.
\item Burkhart Wolff was a member of the EU-Project EUROMILS (Oct. 2012
  - Sept. 2015 ) and the ANR project
  \href{http://www.irt-systemx.fr/en/project/pst/}{PST}, which were both
  targeting at applying Formal Methods in an industrial effort aiming at
  a high-level certification (Common Criteria EAL6, CENELEC SIL4). The
  latter project incited the development of
  Isabelle/DOF\cite{brucker_achim_d_2019_3370483}.
\end{compactitem} 

\paragraph*{Infrastructures or technical equipments:}

\begin{compactitem}
\item Université Paris-Saclay participates in the development of the Coq
  and Isabelle interactive theorem provers.
\item Université Paris-Saclay is the leader of the SMTCoq project since
  2015.
\item Université Paris-Saclay participates in the development of the
  Isabelle/DOF implementation of the Ontological Framework.
\end{compactitem}

\paragraph*{Persons primarily responsible for carrying out the proposed
  activities:}

\begin{compactitem} % in alphabetical order

\item {\bf Chantal Keller} is Associate Professor at Université
  Paris-Saclay since 2015. She obtained her PhD (2013) at École
  polytechnique. Her research focus on the development, democratization,
  use and interoperability of formal methods. She is a pioneer of
  interoperability between proof systems as the developer of one of the
  first tools to import proofs between two interactive theorem provers
  with very different underlying logics: HOL Light and Coq.

\item {\bf Burkhart Wolff} is Full Professor at Université
  Paris-Saclay since 2008. He has a strong background in interactive
  theorem proving for the modeling of languages semantics as well as
  applications to model-based testing. He recently proposed an
  ontological framework for document consistency and meta-data, which is
  implemented on top of Isabelle/HOL.

\end{compactitem}

\end{sitedescription}


%%% Local Variables:
%%% mode: latex
%%% TeX-master: "../propB"
%%% End:
