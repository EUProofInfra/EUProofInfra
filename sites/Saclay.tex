\begin{sitedescription}{Sac}

Université Paris-Saclay is a public research university located south of
Paris. With 275 laboratories shared with the mainstream French research
institutes, Université Paris-Saclay represents 13\% of the French
research potential.

The Laboratory for Computer Science (LRI) covers a wide spectrum of
computer science. Its VALS team works in the Area of Verification and
Validation of Algorithms, Languages and Systems, right in the heart of
the scientific field of Formal Methods. Its group working on \pn will
include Chantal Keller and Burkhart Wolff.

\paragraph{Main tasks:}

\begin{itemize}
\item WP4: \dots
\item WP6: \dots
\item WP7: \dots
\end{itemize}

\begin{compactitem}
\item\ednote{specify the main tasks and reference the respective work packages} 
\end{compactitem}

\paragraph{Publications, products or services:}

\ednote{a list of up to 5 relevant publications, and/or products,
  services (including widely-used datasets or software), or other
  achievements relevant to the call content}

\begin{itemize}
\item An automatic tool to import proofs between the two interactive
  theorem provers HOL Light and Coq, companion
  paper:~\cite{DBLP:conf/itp/KellerW10}
\item The SMTCoq project: \url{https://smtcoq.github.io} (companion
  papers:~\cite{DBLP:conf/cpp/ArmandFGKTW11,DBLP:conf/cav/EkiciMTKKRB17})
\end{itemize}

\paragraph{Previous projects or activities:}

\ednote{a list of up to 5 relevant previous projects or activities,
  connected to the subject of this proposal}

\paragraph{Infrastructures or technical equipments:}

\ednote{a description of any significant infrastructure and/or any major
  items of technical equipment, relevant to the proposed work}

The group has a mainstream experience in interoperability between proof
systems. It is in particular the leader of the SMTCoq project, a tool to
benefit from both worlds of interactive and automatic theorem provers.
This tool allows one to check automatic theorem provers with great
confidence, as well as to enjoin proof automation in the Coq proof
assistant.

Concept alignment is fundamental in this project, to relate Coq
mathematical objects with the built-in theories of automatic provers,
and thus benefit from the best possible automation. The necessity to
align concepts for proof interoperability was pioneered by one member of
the group, in a project to import proofs between the two interactive
theorem provers HOL Light and Coq.

The group also has experience in the representation of proofs as
high-level documents. \dots

\paragraph{Persons primarily responsible for carrying out the proposed
  activities:}

\begin{itemize} % in alphabetical order

\item {\bf Dr.\ Chantal Keller} is Associate Professor at Université
  Paris-Saclay since 2015. She obtained her PhD (2013) at École
  polytechnique. She is a pioneer of interoperability between proof
  systems as the developer of one of the first tools to import proofs
  between two interactive theorem provers with very different underlying
  logics: HOL Light and Coq. In this work, the alignment of mathematical
  concepts was the key to re-usability of proofs. She is the leader of
  the SMTCoq project that links the Coq proof assistant with external
  automatic theorem provers. She has strong expertise in interactive
  (conception and use) and automatic (mainly use) theorem provers, and
  interoperability between them. She is one of the two coordinators of
  WP4.

\item {\bf Pr.\ Burkhart Wolff} \dots He is one of the two coordinators
  of WP7.

\end{itemize}

\end{sitedescription}


%%% Local Variables:
%%% mode: latex
%%% TeX-master: "../propB"
%%% End:
