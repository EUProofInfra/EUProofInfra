\begin{sitedescription}{Cea}

%\ednote{a description of the legal entity}

\logo{CEAList}

\href{http://www.cea.fr/default_gb.htm}{CEA} is a public multidisciplinary research organization whose research fields
range from nuclear industry to biosciences and fundamental physics. It is made
of several research centres located in France. CEA repre- sents: 15024
employees, 2.7 B Euros budget, 1689 patents registered or active, 1300 contracts
signed with industry, 83 new companies created since 1984 in high technologies
sectors, 9 research centres. In HORIZON
2020, CEA is already involved in more than 100 projects. The CEA LIST
(“Laboratoire d'Intégration de Systèmes et des technologies”) department is part
of CEA TECH, the CEA Technological Research Division. CEA LIST combines basic
research and industrial R\&D and is primarily concerned with the development of
technologies that combine software and hardware to form highly integrated
complex systems. The research activities are structured into three major themes:
em- bedded systems, interactive systems and sensors and signal processing. CEA
LIST focuses on methods and tools for the design of embedded systems with
appropriate architectures, software, and an optimal level of safety. Within CEA
LIST, the LSL (“Laboratoire de Sureté et de Sécurité des Logiciels”, Software
Safety and Secu- rity Lab) is focused on the verification \& validation of
software and hardware components. The LSL devel- ops methods and tools for the
static analysis and test case generation of safety-critical applications. CEA
LIST has participated to several international and national research projects in
the above mentioned fields, within the FP6, FP7, RNTL, ANR and ITEA Programmes.
In particular, LSL develops Frama-C since 2005. Since the initial release of the
platform in 2008, LSL actively maintains and extends the kernel and a growing
set of analysis plugins. An important part of Frama- C's development is done
through close interaction with users during collaborative projects. In
particular, one can cite ANR projects CAT and U3CAT, and FP7 project STANCE.
Generally, when successful mature technologies emerge, they are transferred from
CEA to its partners by means of the following mechanisms: 1) Joint Laboratories,
consisting of specific contracts aiming at transfer- ring some well-defined
intellectual property from CEA to industry, possibly using a team of dedicated
per- sonnel, 2) patents and intellectual properties sale, and 3) the creation of
start-up companies. Concerning the methods and tools developed by LSL , once
validated on representative industrial applications, they are disseminated
through 1) an open source distribution platform (such as OPEES1), 2) by direct
industrial contracts or 3) transferred to companies commercialising such tools.
%such as TrustinSoft, a spin-off from CEA LIST and also part of VESSEDIA.

\paragraph*{Main tasks:}
% CEA LIST has extensive experience in R&D projects, in the scope of FP6, FP7,
% ARTEMIS and national French programs. CEA LIST participated to the FP6 project
% OpenTC, developing a secure Linux OS and is coordinating the FP7 project STANCE,
% developing tools for analysing security of C++ and Java code. This experience
% and corresponding developments – especially the Frama-C toolkit developed in
% STANCE – will be used as a starting point of the VESSEDIA project.

% CEA LIST LSL coordinates and develops the Frama-C toolkit in order to handle the
% analysis of the C++ programming language and to handle the security analyses
% through Frama-C. Some research will be involved in finding solutions to security
% problem specific to CIs. Thus, CEA LIST LSL will develop dedicated plug- ins to
% analyse the security properties inherent to the applications of the project. CEA
% LIST LSL will coordi- nate the VESSEDIA project and will lead WP2 and WP3.


% ....

% Expected results from VESSEDIA
% One of the main roles of CEA is bridge the gap between research and industry, using the mechanisms de- scribed above. The expected results for CEA LIST are mainly improvements and extensions of toolboxes and applications.
% The results for CEA LIST LSL will be a mature and improved Frama-C platform – especially to address new connected and dynamic applications, on which rely several projects in the LSL Laboratory as well as our clients and partners. We expect VESSEDIA to allow us to provide the partners a mature safety and security analysis framework Frama-C.
% ....

%\ednote{its main tasks, with an explanation of how its profile matches the tasks in the proposal}

\begin{compactitem}
\item\ednote{specify the main tasks and reference the respective work packages}
\end{compactitem}

\paragraph*{Publications, products or services:}

\ednote{a list of up to 5 relevant publications, and/or products, services (including widely-used datasets or software), or other achievements relevant to the  call content}

% Dale: Picking only 5 publications for all of Inria will be difficult.  I
% suggest one but do not insist that it is used.

\begin{compactitem}
\item ``Why3: Shepherd Your Herd of Provers'', by F. Bobot and J-C. Filli\^atre and
C. March\'e and A. Paskevich, Boogie 2011, First International Workshop on
Intermediate Verification Languages, p53-64
\end{compactitem}

\paragraph*{Previous projects or activities:}

%\ednote{a list of up to 5 relevant previous projects or activities, connected to
  the subject of this proposal}

\begin{compactitem}
\item The H2020 \href{https://www.decoder-project.eu}{DECODER} project aims at
providing a unified interface for storing and querying any kind of
information related to a given software project, from initial
requirements to code to formal specifications and analysis results,
including proof artifacts. It would of course be very beneficial for
both projects to agree on a common exchange format for such objects, and
coordination with the DECODER consortium (in which CEA acts as technical
leader) will seek to achieve that.
\end{compactitem}

\paragraph*{Infrastructures or technical equipments:}

%\ednote{a description of any significant infrastructure and/or any major items of technical equipment, relevant to the proposed work}

\begin{compactitem}
\item The CEA are the main developper of the Frama-C plateform for C verification.
\end{compactitem}

\paragraph*{Persons primarily responsible for carrying out the proposed activities:}

\begin{compactitem} % in alphabetical order

\item{\bf Allan Blanchard} is an engineer-researcher at CEA since 2019. He is
  interested in the analysis of concurrent code using formal methods and more
  precisely using deductive verification.

  He obtained his PhD in Computer Science from the University of Orléans in
  2016. He prepared his PhD at the Software Reliability Laboratory of the CEA
  LIST. His most recent research work focused on applying formal verification to
  an operating system for internet of things, using the Frama-C analysis
  platform.

\item{\bf François Bobot} is an engineer-researcher at CEA since 2012. He work
  at different steps of formal methods using deductive verification techniques:
  from helping industrial partners to use formal tools, designing and extending
  verification tools (Why3, Frama-C) to improving automatic solvers (Alt-Ergo,
  CVC4, COLIBRI).
\end{compactitem}

\end{sitedescription}

%%% Local Variables:
%%% mode: latex
%%% TeX-master: "../propB"
%%% End:

% LocalWords:  site-jacu.tex clange sitedescription emph compactitem pn semmath
% LocalWords:  prosuming-flexiformal KohSuc asemf06 GinJucAnc alsaacl09 StaKoh
% LocalWords:  tlcspx10 KohDavGin psewads11 ednote Radboud Bia ystok CALCULEMUS
% LocalWords:  textbf keypubs OntoLangMathSemWeb uwb Deyan Ginev Stamerjohanns
% LocalWords:  searchability
