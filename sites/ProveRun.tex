\begin{sitedescription}{Pro}

\newcommand{\provenrun}[0]{Prove\ \&\ Run}

\logo{ProveRun}

\provenrun's mission is to help its customers resolve the security
challenges linked to the large-scale deployment of connected devices
and of the Internet of Things. Thanks to an innovative software
development toolchain (called ProvenTools) based on state of the art
proof techniques, \provenrun{} can deal with the development of the
most sensitive software components (microkernels, hypervisors, OSes,
secure bootloaders, etc) and meet the highest security requirements
(such as CC EAL7) in a cost effective manner, taking also into account
time-to-market and required skill levels constraints.

Using ProvenTools, \provenrun{} has developed two unique critical
off-the-shelf software bricks:
\begin{compactitem}
\item ProvenCore: a next generation ultra-secure OS (TEE) available
  for ARM® Cortex®-A, Cortex®-M and RISC-V processors, certified at
  the EAL7 level according to the Common Criteria evaluation scheme.
\item ProvenVisor: an ultra-secure hypervisor available for ARM
  Cortex-A processors.
\end{compactitem}

\provenrun{} was founded in 2009 and today employs 40 engineers. It is
independent and privately owned.

\paragraph*{Main tasks:}

\begin{compactitem}
\item \WPref{theories}, task \taskref{theories}{smart}: Translate
  models and proofs from ProvenTools to Dedukti.  Stéphane Lescuyer
  will be in charge of this task. He is the main architect and
  developer of ProvenTools' internal verification condition generator
  and automated prover.
\item \WPref{dissemination}, task
  \taskref{dissemination}{certif-club}: Expanding the use of Logipedia
  within certification authorities. Guillaume Dufay is a senior
  security manager who has been in charge of many Common Criteria
  evaluations, including ProvenCore's EAL7.
\end{compactitem}

\paragraph*{Publications, products or services:}

\begin{compactitem}
\item ``Security Filters for IoT Domain
  Isolation'', by Dominique Bolignano, Embedded Conference, 2018
\item ``Formally Proven and Certified
  Off-The-Shelf Software Components'', by Dominique Bolignano, C\&SAR,
  2016
\item ``Proven Security for the Internet of
  Things'', by Dominique Bolignano, Embedded Conference, 2016
\item ``ProvenCore: Towards a Verified Isolation
  Micro-Kernel'', by Stéphane Lescuyer, 10th HiPEAC Conference, 2015
\end{compactitem}

\paragraph*{Previous projects or activities:}

\begin{compactitem}
\item EPI: The European Processor Initiative (EPI) is a project
  financed by the European Commission, whose aim is to design and
  implement a roadmap for a new family of low-power European
  processors for extreme scale computing, high-performance Big-Data
  and a range of emerging applications. \provenrun{} is the Security
  Leader of this project.
\item CPS4EU: The CPS4EU project aims to arm Europe with an extensive
  value chain across key sectors by: 1. Strengthening Cyber-Physical
  Systems (CPS) technology providers, mainly European SMEs, to
  increase their market share and their competitiveness to become
  world leaders, 2. Improve design efficiency and productivity and
  enable secure certification, 3. Enabling the creation of innovative
  European CPS products that will strengthen the leadership and
  competitiveness of Europe by both large groups and SMEs, 4. Large
  dissemination of CPS technologies.
\item SECREDAS: European collaborative research project aimed at
  developing an innovative solution for the safety, security, and
  privacy of automated systems, including a reference architecture,
  powerful components, and common approaches to integration and
  verification in the automotive, health and rail sectors, led by
  NXP.
\item MuSiC: European collaborative research project aimed at
  providing a scalable and certifiable security solution for the mid
  to high data-rate cost-effective devices in order to secure against
  application, OS, web, and network based threats and protect critical
  services on shared networks, led by STMicroelectronics.
\end{compactitem}

%\paragraph*{Infrastructures or technical equipments:}

\paragraph*{Persons primarily responsible for carrying out the proposed activities:}

\begin{compactitem} % in alphabetical order

\item {\bf Dominique Bolignano} led the formal methods'
  group of Bull until 1996, prior to taking charge of all the
  technology transfer initiatives of the Dyade GIE (created by INRIA
  and Bull) within the formal methods and security areas. In 1999 he
  created Trusted Logic, a start-up of INRIA, which he led for eleven
  years, until its sale to Gemalto in 2009. With more than one hundred
  experts, researchers and engineers, Trusted Logic became the world
  leader in its field: operating systems and middleware security for
  smart cards and mobile terminals. In 2009 he founded \provenrun{} and
  is currently its CEO. In parallel with these activities, he has
  maintained many academic positions. In particular he was Associate
  Professor at the Paris Dauphine University for nine years and then a
  member of the Scientific Council of CNRS in the field of engineering
  and computer science for four years, until September 2010. Most
  recently he chaired the AERES evaluation committee for the INRIA
  Rennes Bretagne and led the quadrennial assessment of its research
  laboratories.

\item {\bf Guillaume Dufay} (PhD, CISSP) has more than 15 years of
  experience in security architecture and security evaluation for
  connected and embedded devices. This experience was acquired from
  academic research and security consulting missions in various
  vertical domains (Mobile, Financial services, IoT, Transport, DRM,
  Enterprise). He is the author of ARM PSA Certified security
  certification scheme and ARM IoT Protection Profiles; co-author of
  the GlobalPlatform TEE, Java Card and (U)SIM Protection Profiles;
  and the author of security targets based on Java Card, secure
  signature, payment, TPM, car-to-car and e-passport Protection
  Profiles. He managed several Common Criteria evaluations of
  smartcards and similar devices products including the interactions
  with ITSEFs and certification bodies.

\item {\bf David Garnier} holds a Master's Degree in Engineering from
  the Ecole des Mines de Nantes and a Master's Degree in Computer
  Science from the University of Nantes. He worked for four years at
  Trusted Logic, leading studies in the field of secure embedded
  computing for customers such as Orange and SFR. He then created the
  Integration Group within Trusted Labs and led the implementation of
  prototypes of secure embedded system in collaboration with the DGA
  and RATP, among others. Prior to joining \provenrun{}, he obtained an
  MBA from the Rotterdam School of Management.

\item {\bf Stéphane Lescuyer} graduated from the Ecole
  Polytechnique and is a State Engineer of the Mines, a technical
  corps of the French state. He earned a PhD degree from University
  Paris-Sud in the field of automation of formal proofs in the Coq
  proof assistant, as part of the INRIA Saclay - Ile-de-France. He
  has ten years of experience of working on formal methods and
  program verification at \provenrun, and is the lead architect
  in charge of ProvenCore's design and proofs.
\end{compactitem}

\end{sitedescription}

%%% Local Variables:
%%% mode: latex
%%% TeX-master: "../propB"
%%% End:
