\begin{sitedescription}{Inr}

%\ednote{a description of the legal entity}

\logo{Inria}

% From Inria administration:

Established in 1967, Inria is the only French public research body fully dedicated to computational sciences. It is a national operator in research in digital sciences and is a primary contact point for the French Government on digital matters. Under its founding decree as a public science and technology institution, jointly supervised by the French ministries for research and industry, Inria's missions are to produce outstanding research in the computing and mathematical fields of digital sciences and to ensure the impact of this research on the economy and society in particular. Inria covers the entire spectrum of research at the heart of these activity fields and works on digitally-related issues raised by other sciences and by actors in the economy and society at large. Beyond its structures, Inria's identity and strength are forged by its ability to develop a culture of scientific innovation, to stimulate creativity in digital research.
Throughout its 8 research centres and its 220 project teams, Inria has a workforce of 2 400 employees (including 1600 researchers) with an annual budget of 231 million euros, 25\% of which coming from its own resources.
Inria’s mission is to pursue excellent research in computer science and applied mathematics in order to play a major role in resolving scientific, societal and industrial challenges. Therefore, Inria actively collaborates with public and private bodies including strategic partnerships with large firms, SME’s technology platforms and industrial clusters. Technology transfer is further enhanced by helping to launch new companies (since 1984, about 160 companies have stemmed from Inria) and by forming partnerships with innovative SMEs.

The institute is strongly involved in European programmes aimed at fostering scientific excellence, such as the European Research Council (58 Grants) or the Marie S. Curie Actions (24 projects in Horizon 2020).

Inria makes a firm commitment to Horizon 2020, with which the institute’s strategic plan is aligned. The objective is to combine scientific excellence with a more focused consideration of major European and global societal challenges to which Inria can bring a key contribution. Inria is currently involved in more than 140 H2020 funded projects.

Inria is also playing a lead role in the development of the Knowledge and Innovation Community (KIC) EIT Digital as host of the French node. EIT Digital’s ambition is to create for Europe a structure dedicated to technology transfer and innovation in the digital field. Besides EIT Digital, Inria is also a core partner of the KIC EIT Health.

\paragraph*{Main tasks:}

%\ednote{its main tasks, with an explanation of how its profile matches the tasks in the proposal}

%\ednote{specify the main tasks and reference the respective work packages}
\begin{compactitem}
\item Gilles Dowek is the coordinator of the project and leads
  \WPref{management}. He developed the Dedukti proof format and
  coordinated tools for translating other proof format to/from
  Dedukti.
\item Frédéric Blanqui leads the \WPref{dissemination}. He is a member of the
  steering committees of the Logic in Computer Science (LICS) and
  TYPES conferences, and of the International School on Rewriting
  (ISR).
\item Stephan Merz leads the task \taskref{theories}{tla}.
  He is one of the designers and developers of the \tlaplus Proof System.
\item Bruno Barras leads task \taskref{theories}{hott}.
\item Enrico Tassi participates to \taskref{instrumentation}{coq} and \taskref{alignment}{alignproofs}. He is a core
developer of the Coq system and of the ELPI programming language (used in \taskref{alignment}{alignproofs}).
\end{compactitem}

\paragraph*{Publications, products or services:}

%\ednote{a list of up to 5 relevant publications, and/or products, services (including widely-used datasets or software), or other achievements relevant to the  call content}

\begin{compactitem}
\item ``Dedukti: a logical framework based on the $\lambda\Pi$-calculus modulo theory'', by A. Assaf, G. Burel, R. Cauderlier, D. Delahaye, G. Dowek, C. Dubois, F. Gilbert, P. Halmagrand, O. Hermant and R. Saillard, Technical report, 2019.
\item ``A generalization of the Takeuti–Gandy interpretation'', by B. Barras, T. Coquand and S. Huber. In Mathematical Structures in Computer Science, vol 25(5), pp 1071-1099, 2015.
\item ``\tlaplus Proofs'', by D.\ Cousineau, D.\ Doligez, L.\ Lamport, S.\ Merz, D.\  Ricketts, H.\ Vanzetto. 18th Intl.\ Symp.\ Formal Methods (FM 2012).
\item ``ELPI: fast, Embeddable, $\lambda$Prolog Interpreter'', by Cvetan Dunchev, Ferruccio Guidi, Claudio Sacerdoti Coen, Enrico Tassi. Proceedings of LPAR, Nov 2015.
\end{compactitem}

\paragraph*{Previous projects or activities:}

%\ednote{a list of up to 5 relevant previous projects or activities, connected to the subject of this proposal}

\begin{compactitem}
\item Stephan Merz is a core member of the team designing and developing the \tlaplus Proof System at the Joint Microsoft Research-Inria Centre. He also is a senior collaborator in the project \emph{Matryoshka} (Principal investigator: Jasmin Blanchette, Vrije Universiteit Amsterdam, ERC Starting Grant 2016, Grant agreement no.\ 713999) on reducing the gap between efficient automated reasoning systems and more expressive interactive proof assistants.
\item Bruno Barras has been a member of the ANR project
  \href{http://www.lri.fr/~wolff/projects/ANR-Paral-ITP/}{Paral-ITP}, led by Burkhart Wolff (University Paris-Saclay), which was targeting of advanced
  parallelisation techniques for Coq and Isabelle.
\end{compactitem}

\paragraph*{Infrastructures or technical equipments:}

%\ednote{a description of any significant infrastructure and/or any major items of technical equipment, relevant to the proposed work}

\begin{compactitem}
\item Inria develops tools around the Dedukti proof format since 2009.
\item Since 1984, Inria develops the Coq proof assistant which
  received in 2013 the ACM SIGPLAN Programming Languages Sofware and
  the ACM Software System awards.
\item Since 2006 Inria has supported the development of the Mathematical Components
  library. Initially created to support the computer proof of
  the Odd Order Theorem, since 2012 the library is publically distributed
  and has been used in many third party verification projects.
\item Since 2008 Inria has supported the development of the \tlaplus Proof System.
\item Inria hosts the Grid5K infrastructure for running large-scale computing experiments.
\end{compactitem}

\paragraph*{Persons primarily responsible for carrying out the proposed activities:}

\begin{compactitem} % in alphabetical order

\item{\bf Bruno Barras} is a research scientist at Inria since
  2000. His research focuses on the formal study of the
  metatheoretical properties of Type Theory and on designing efficient
  algorithms for type-checking and conversion modulo rewriting. He
  obtained his PhD in 1999 at the University Denis Diderot (Paris
  7). In 2012, he participated to the IAS Special Year on Univalent
  Foundations in Princeton, which has been an important stepstone in
  the dissemination of Homotopy Type Theory. He received with other
  colleagues the ACM Software System Award 2013 for his contribution
  to the implementation of Coq.

\item{\bf Frédéric Blanqui} leads the \WPref{dissemination}. He is a research scientist at Inria since 2003. His research interests are centered on the termination of programmes and the use of rewriting techniques in logics and proof assistants. He is the coordinator of the developments of the Dedukti interface and tactic language. He obtained his PhD (2001) at the University Paris-Sud and his habilitation (2012) at the University Denis-Diderot. He actively contributed to the development of the CPF language for termination certificates used in the international termination competition. He published about 35 papers in peer-rewieved international conferences and journals, and has been PC member of international conferences such as PPDP, RTA, FOSSACS, FSCD, ICTAC and CICM. He is now member of the steering committees of ISR, TYPES and LICS, and organized the 11th International School on Rewriting in Paris in 2019.

\item{\bf Valentin Blot} is a research scientist at Inria since 2019. His research focuses on the semantic interpretation of proofs and programs, in particular through realizability. He obtained his PhD in 2014 at \'Ecole Normale Sup\'erieure de Lyon. He published papers in the most recognized international conferences in the field, served on the program committee of national and international conferences and published papers in international journals.

\item{\bf Pierre Boutry} is a postdoctoral researcher in the STAMP Inria team at the research center in Sophia Antipolis - M\'editerran\'ee since 2019.
He obtained his PhD (2018) at the University of Strasbourg.
His main research interest is in the formalization of foundation of geometry.
He published 9 papers in peer-rewieved international conferences and journals.
He has been an invited speaker at the Logic Colloquium in 2019.

\item{\bf Arthur Charguéraud} is a research scientist at Inria since 2013.
His research interests range from interactive programme verification and mechanized
semantics of programming languages, to multicore programming.
He obtained his PhD (2010) at the University Paris Diderot. He published 27 papers
in peer-reviewed international conferences and journals, and has been PC member
of international conferences such as PPOPP, OOPSLA, ITP, and ESOP. He is the main
developer of the TLC Coq library and of the CFML programme verification tool.
He was recently funded by Inria for an ``Exploratory Action'' to develop a framework
for user-guided interactive source-to-source optimizations, with formal correctness
guarantees.
He co-organizes and designs tasks for the french version of the Bebras Contest,
attended each year by 700.000+ pupils aged 8 to 18.
%He also co-organized the Summer School for Young Reseachers (EJCP) in Strasbourg in 2019.

\item{\bf Gilles Dowek} is the coordinator of the project.
He is a researcher at Inria and professor at the École
normale supérieure de Paris-Saclay. He has previously been a professor
at the École polytechnique and a consultant for the NASA Langley
research center. He has been active in several types of
outreach activities, especialy by writinb books, articles in several
magazines, a column in {\em Pour la Science}, and different types
of live activities.
He has published several textbooks, popular science
books, and philosophy of science books. Some of them have been
translated to Chinese, English, German, Greek, Italian, Korean,
Romanian, and Spanish. He has been a member of two ethics committees:
the CERNA, and then the CNPEN. 
He has been active in the promotion of
Computer science education in K-12 in France. He has been the
President of the Scientific Board of the French informatics society.

His work adresses the formalization of mathematics, the proof
processing systems, the physics of computation, the safety of
aerospace systems, and the epistemology and ethics of informatics.
Together with Denis Cousineau, he is at the origin of the
$\lambda\Pi$-calculus modulo theory. He is the head of the Deducteam
group, where the logical framework Dedukti has been developed.

\item{\bf Hugo Herbelin} is a member of the $\pi r^2$ Inria team
located in Paris. Former coordinator of the Coq development team,
his research mixes contributions to proof theory and logical
foundations and contributions to the development of the Coq proof
assistant. His interests are moving towards the design and development
of proof assistant features targetting mathematics the way
mathematicians think about them.

\item{\bf Dominique M\'ery}  is a professor of computer science at the
  University of Lorraine (Telecom Nancy)  and head of the
  \href{https://mosel.loria.fr}{MOSEL} team at LORIA in Nancy.
  He is a member of the Veridis project.
His research interests are centered on the   correct-by-construction
design  of software-based systems  using refinement and on the
verification   of distributed algorithms, in particular proofs of safety and liveness
properties. Among other contributions, he implemented an interactive
proof assistant using Isabelle  for deriving safety and liveness
properties of SDL programs (CAV 1992) and  the closed-loop model of
a pacemaker with a heart model.
He obtained his PhD (1983) and Thèse d'État (1993) degrees at the
University of Nancy and joined University of Lorraine in 1993. He published  papers
in peer-reviewed international conferences and journals and has been a PC co-chair
of conferences such as FM, iFM, TASE, and ABZ. He is member of the IFIP
Working Group 1.3 on Foundations of System Specification and has been a junior member of IUF from 1995 till 2000.

\item{\bf Stephan Merz} is a senior research
scientist and head of the \href{https://team.inria.fr/veridis/}{VeriDis}
research group, as well as the deputy for science, at Inria Nancy\,--\,Grand Est.
His research interests are centered on the formal specification and verification
of distributed algorithms, in particular proofs of safety and liveness
properties, as well as refinement relations between specifications expressed at
different levels of abstraction. He is a main contributor to TLAPS, the \tlaplus
Proof System. He obtained his PhD (1992) and habilitation (2002) degrees at LMU
M\"unchen and joined Inria in 2002. He has published more than 100 papers
in peer-reviewed international conferences and journals and has been a PC chair
of conferences such as iFM, ICFEM, and ITP. He co-founded the FRiDA (Formal
Reasoning in Distributed Algorithms) series of workshops as well as the VTSA
(Verification Technology, Systems, and Applications) summer school. He has been
a member of the scientific directorate of Schloss Dagstuhl, as well as a member
of the Inria Evaluation Committee and of the National Committee of Scientific
Research in France.

\item{\bf Pierre Senellart} is head of the Inria team Valda at the Inria
  research center in Paris. He
  received his PhD in 2007 from Université Paris-Sud. He has been a
    Professor in the Computer Science Department at the École normale
    supérieure (ENS, PSL University) in Paris, France since 2016, when he
    founded the Inria Valda team focusing on extracting \emph{Value from
    data}. Before joining ENS, he was an Associate Professor (2008–2013)
    then a Professor (2013–2016) at Télécom Paris. He also held secondary
    appointments as Lecturer at the University of Hong Kong in 2012–2013,
    and as Senior Research Fellow at the National University of Singapore
    from 2014 to 2016. Pierre Senellart is currently holding a Chair
    within the Paris Artificial Intelligence Research Institute
    (PRAIRIE), and is a Research Fellow at the Centre on Regulation in
    Europe (CERRE).
    His research interests
    focus around practical and theoretical aspects of Web data
    management, including Web crawling and archiving, Web information
    extraction, uncertainty management, Web mining, and intensional data
    management.

\item{\bf Enrico Tassi} is a member of the STAMP Inria team at the research
center in Sophia-Antipolis. He obtained his PhD (2008) at the University of
Bologna and he joined Inria as research scientist in 2012.
He is interested in the design and implementation of interactive
provers and in high level languages for extending their functionalities.
He is a core developer of the Coq system, the Mathematical Components library
and the Elpi programming language. He regularly serves as PC member of
international conferences such as ITP, CPP, PADL, MKM. He regularly organizes
schools on Coq and the Mathematical Components library.

\item {\bf Pierre-Yves Strub}
is Teaching Assistant in Computer Sciences at École
Polytechnique, France since 2016. During that time, and during time
spent as a post-doctoral researcher at the IMDEA software institute
(Madrid, Spain) and Microsoft/Inria Joint Lab (Paris, France), he has
been contributing to the development and application of formal methods
to security in general, and cryptography in particular. Starting with
his PhD, Dr.~Pierre-Yves Strub also works on the theory and practice
of proof assistants with applications to security and mathematics
formalization.
%
He has a strong and consistent track record of strong academic
publications on this topic, and has applied the formal techniques and
tools whose development he contributes to various aspects of
cryptography, formally verifying security proofs for standard
primitives and constructions.  He has also contributed to the
development of new techniques and tools for formally reasoning about
differential privacy and side-channel countermeasures.
%
He is one of the two main developers of the EasyCrypt tool (an
interactive framework for verifying the security of cryptographic
constructions in the computational model) and is strongly involved in
the development of Jasmin (a framework for developing high-speed and
high-assurance cryptographic software).

\end{compactitem}

\end{sitedescription}

%%% Local Variables:
%%% mode: latex
%%% TeX-master: "../propB"
%%% End:
