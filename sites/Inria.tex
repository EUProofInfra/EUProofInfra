\begin{sitedescription}{Inr}

\paragraph{Organization:}
\ednote{Give a one-paragraph run-down of the site and the team there. }

\paragraph{Main tasks:}

\begin{compactitem}
\item\ednote{specify the main tasks and reference the respective work packages} 
\end{compactitem}


\paragraph{Relevant previous experience:}

\ednote{give an overview over previous work and projects that add to the \pn project}

\paragraph{Specific expertise:}

\begin{compactitem}
\item \ednote{give three to five specific areas of expertise that pertain to the \pn project}
\end{compactitem}

\paragraph{Staff members undertaking the work:}

\textbf{Dr.\ Great Leader}\ednote{describe the site leader and his expertise}
\textbf{Joe Implementor}\ednote{and more of them. }
\ednote{provide the key publications below}
\keypubs{providemore}

Émilio Gallego, Hugo Herbelin, Théo Zimmerman


Bruno Barras,
Frédéric Blanqui,
Valentin Blot,
Kaustuv Chaudhuri,
Gilles Dowek,
Georges Gonthier,
Jean-Pierre Jouannaud,

Stephan Merz (co-leader of work package \WPref{2}) is a senior research
scientist and head of the \href{https://team.inria.fr/veridis/}{VeriDis}
research group, as well as the deputy for science, at Inria Nancy\,--\,Grand Est.
His research interests are centered on the formal specification and verification
of distributed algorithms, in particular proofs of safety and liveness
properties, as well as refinement relations between specifications expressed at
different levels of abstraction. He is a main contributor to TLAPS, the \tlaplus
Proof System. He obtained his PhD (1992) and habilitation (2002) degrees at the
University of Munich and joined Inria in 2002. He published more than 100 papers
in peer-reviewed international conferences and journals and has been a PC chair
of conferences such as IFM, ICFEM, and ITP. He co-founded the FRiDA (Formal
Reasoning in Distributed Algorithms) series of workshops as well as the VTSA
(Verification Technology, Systems, and Applications) summer school. He has been
a member of the scientific directorate of Schloss Dagstuhl, as well as a member
of the Inria Evaluation Committee and of the National Committee of Scientific
Research in France.


Dale Miller (co-leader of work package \WPref{5}) is a senior research scientist
and former head of the Parsifal at Inria Saclay\,--\,\^Ile-de-France.  Miller
received his Ph.D. in Mathematics from Carnegie Mellon University in 1983.  He
has been on the faculty of the University of Pennsylvania (1983-1997), Penn
State University (1997-2002), and the Ecole Polytechnique (2002-2006). Miller
has been a two-term Editor-in-Chief of the ACM Transactions on Computational
Logic (ToCL) and is currently serving on the editorial board of the Journal of
Automated Reasoning. He is presently the General Chair of the ACM/IEEE Symposium
on Logic in Computer Science (LICS) and has served as a program committee chair
on a number of conferences in the general area of computational logic.  Miller
has twice received the LICS Test-of-Time Award for papers published in 1991 and
in 1994.  Miller received an ERC Advanced Grant titled “ProofCert: Broad
Spectrum Proof Certificates” for the years 2012-2016.  Miller's main research
interests are in computational logic, in particular, with the design of
automated and interactive theorem provers, the design and semantics of logic
programming and functional programming languages, and the design and
applications of proof certificates.


Lutz Straßburger

Yves Bertot, Pierre Boutry, Cyril Cohen

\end{sitedescription}
%%% Local Variables: 
%%% mode: latex
%%% TeX-master: "../propB"
%%% End: 

% LocalWords:  site-jacu.tex clange sitedescription emph compactitem pn semmath
% LocalWords:  prosuming-flexiformal KohSuc asemf06 GinJucAnc alsaacl09 StaKoh
% LocalWords:  tlcspx10 KohDavGin psewads11 ednote Radboud Bia ystok CALCULEMUS
% LocalWords:  textbf keypubs OntoLangMathSemWeb uwb Deyan Ginev Stamerjohanns
% LocalWords:  searchability
