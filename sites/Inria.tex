\begin{sitedescription}{Inr}

\ednote{a description of the legal entity}

Established in 1967, Inria is the only French public research body fully dedicated to computational sciences. It is a national operator in research in digital sciences and is a primary contact point for the French Government on digital matters. Under its founding decree as a public science and technology institution, jointly supervised by the French ministries for research and industry, Inria's missions are to produce outstanding research in the computing and mathematical fields of digital sciences and to ensure the impact of this research on the economy and society in particular. Inria covers the entire spectrum of research at the heart of these activity fields and works on digitally-related issues raised by other sciences and by actors in the economy and society at large. Beyond its structures, Inria's identity and strength are forged by its ability to develop a culture of scientific innovation, to stimulate creativity in digital research.
Throughout its 8 research centres and its 220 project teams, Inria has a workforce of 2 400 employees (including 1600 researchers) with an annual budget of 231 million euros, 25\% of which coming from its own resources.
Inria’s mission is to pursue excellent research in computer science and applied mathematics in order to play a major role in resolving scientific, societal and industrial challenges. Therefore, Inria actively collaborates with public and private bodies including strategic partnerships with large firms, SME’s technology platforms and industrial clusters. Technology transfer is further enhanced by helping to launch new companies (since 1984, about 160 companies have stemmed from Inria) and by forming partnerships with innovative SMEs.

The institute is strongly involved in European programmes aimed at fostering scientific excellence, such as the European Research Council (58 Grants) or the Marie S. Curie Actions (24 projects in Horizon 2020).

Inria makes a firm commitment to Horizon 2020, with which the institute’s strategic plan is aligned. The objective is to combine scientific excellence with a more focused consideration of major European and global societal challenges to which Inria can bring a key contribution. Inria is currently involved in more than 140 H2020 funded projects.

Inria is also playing a lead role in the development of the Knowledge and Innovation Community (KIC) EIT Digital as host of the French node. EIT Digital’s ambition is to create for Europe a structure dedicated to technology transfer and innovation in the digital field. Besides EIT Digital, Inria is also a core partner of the KIC EIT Health.

\paragraph{Main tasks:}

\ednote{its main tasks, with an explanation of how its profile matches the tasks in the proposal}

\begin{compactitem}
\item\ednote{specify the main tasks and reference the respective work packages}
\end{compactitem}

\paragraph{Publications, products or services:}

\ednote{a list of up to 5 relevant publications, and/or products, services (including widely-used datasets or software), or other achievements relevant to the  call content}

% Dale: Picking only 5 publications for all of Inria will be difficult.  I
% suggest one but do not insist that it is used.

- ``A semantic framework for proof evidence'' by Zakaria Chihani, Dale
  Miller, and Fabien Renaud. Journal of Automated Reasoning, 59(3),
  pp. 287-330.  DOI 10.1007/s10817-016-9380-6.

\paragraph{Previous projects or activities:}

\ednote{a list of up to 5 relevant previous projects or activities, connected to the subject of this proposal}

% Dale: Picking only 5 projects for all of Inria will be difficult.  I
% suggest one but do not insist that it is used.

- ERC Advanced Grant titled “ProofCert: Broad Spectrum Proof
  Certificates” for 2.2 million euros for the five years
  2012-2016. Miller was the Principle Investigator.

\paragraph{Infrastructures or technical equipments:}

\ednote{a description of any significant infrastructure and/or any major items of technical equipment, relevant to the proposed work}

\paragraph{Persons primarily responsible for carrying out the proposed activities:}

\begin{itemize} % in alphabetical order

%%%%%%%%%%%%%%%%%%%%%%%%%%%%%%%%%%%%%%%%%%%%%%%%%%%%%%%%%%%%%%%%%%%%%%%%%%%%%%
\item{\bf Frédéric Blanqui} is a research scientist at Inria since 2003. His research interests are centered on the termination of programs and the use of rewriting techniques in logics and proof assistants. He is the coordinator of the developments of the Dedukti interface and tactic language. He obtained his PhD (2001) at the University Paris-Sud and his habilitation (2012) at the University Denis-Diderot. He actively contributed to the development of the CPF language for termination certificates used in the international termination competition. He published about 35 papers in peer-rewieved international conferences and journals, and has been PC member of international conferences such as PPDP, RTA, FOSSACS, FSCD, ICTAC and CICM. He is now member of the steering committees of ISR, TYPES and LICS, and organized the 11th International School on Rewriting in Paris in 2019.

%%%%%%%%%%%%%%%%%%%%%%%%%%%%%%%%%%%%%%%%%%%%%%%%%%%%%%%%%%%%%%%%%%%%%%%%%%%%%%
\item{\bf Arthur Charguéraud} is a research scientist at Inria since 2013.
His research interests range from interactive program verification and mechanized
semantics of programming languages, to multicore programming.
He obtained his PhD (2010) at the University Paris Diderot. He published 27 papers
in peer-reviewed international conferences and journals, and has been PC member
of international conferences such as PPOPP, OOPSLA, ITP, and ESOP. He is the main
developer of the TLC Coq library and of the CFML program verification tool.
He was recently funded by Inria for an ``Exploratory Action'' to develop a framework
for user-guided interactive source-to-source optimizations, with formal correctness
guarantees.
He co-organizes and designs tasks for the french version of the Bebras Contest,
attended each year by 700.000+ pupils aged 8 to 18.
%He also co-organized the Summer School for Young Reseachers (EJCP) in Strasbourg in 2019.


%%%%%%%%%%%%%%%%%%%%%%%%%%%%%%%%%%%%%%%%%%%%%%%%%%%%%%%%%%%%%%%%%%%%%%%%%%%%%%
\item{\bf Stephan Merz} (co-leader of work package \WPref{theories}) is a senior research
scientist and head of the \href{https://team.inria.fr/veridis/}{VeriDis}
research group, as well as the deputy for science, at Inria Nancy\,--\,Grand Est.
His research interests are centered on the formal specification and verification
of distributed algorithms, in particular proofs of safety and liveness
properties, as well as refinement relations between specifications expressed at
different levels of abstraction. He is a main contributor to TLAPS, the \tlaplus
Proof System. He obtained his PhD (1992) and habilitation (2002) degrees at the
University of Munich and joined Inria in 2002. He published more than 100 papers
in peer-reviewed international conferences and journals and has been a PC chair
of conferences such as IFM, ICFEM, and ITP. He co-founded the FRiDA (Formal
Reasoning in Distributed Algorithms) series of workshops as well as the VTSA
(Verification Technology, Systems, and Applications) summer school. He has been
a member of the scientific directorate of Schloss Dagstuhl, as well as a member
of the Inria Evaluation Committee and of the National Committee of Scientific
Research in France.

%%%%%%%%%%%%%%%%%%%%%%%%%%%%%%%%%%%%%%%%%%%%%%%%%%%%%%%%%%%%%%%%%%%%%%%%%%%%%%
\item {\bf Dale Miller} (co-leader of work package \WPref{reversemath}) is a senior research scientist
and former head of the Parsifal at Inria Saclay\,--\,\^Ile-de-France.  Miller
received his Ph.D. in Mathematics from Carnegie Mellon University in 1983.  He
has been on the faculty of the University of Pennsylvania (1983-1997), Penn
State University (1997-2002), and the Ecole Polytechnique (2002-2006). Miller
has been a two-term Editor-in-Chief of the ACM Transactions on Computational
Logic (ToCL) and is currently serving on the editorial board of the Journal of
Automated Reasoning. He is presently the General Chair of the ACM/IEEE Symposium
on Logic in Computer Science (LICS) and has served as a program committee chair
on a number of conferences in the general area of computational logic.  Miller
has twice received the LICS Test-of-Time Award for papers published in 1991 and
in 1994.  Miller received an ERC Advanced Grant titled “ProofCert: Broad
Spectrum Proof Certificates” for the years 2012-2016.  Miller's main research
interests are in computational logic, in particular, with the design of
automated and interactive theorem provers, the design and semantics of logic
programming and functional programming languages, and the design and
applications of proof certificates.

%%%%%%%%%%%%%%%%%%%%%%%%%%%%%%%%%%%%%%%%%%%%%%%%%%%%%%%%%%%%%%%%%%%%%%%%%%%%%%
\item{\bf Lutz Straßburger} is head of the Inria team PARTOUT at the research
center in Saclay. He received his PhD in 2003 from TU Dresden.
% After that he held a postdoc positions at Inria in Nancy and in the
% Programming Systems Lab in Saarbrücken.
In 2005 he joined Inria as junior research scientist in the PARSIFAL
team. Lutz Straßburger obtained his HDR in 2011 from Université Paris
Diderot.  He supervised 6 Master’s students, 2 PhD students, and 11
postdocs. He held 3 ANR grants and one Inria Action de Recherche
Collaborative (ARC).  His main research interests lie in deep
inference, Boolean categories, and, most recently, efficient proof
presentations using combinatorial methods.

%%%%%%%%%%%%%%%%%%%%%%%%%%%%%%%%%%%%%%%%%%%%%%%%%%%%%%%%%%%%%%%%%%%%%%%%%%%%%%
\item{\bf Enrico Tassi} is a member of the STAMP Inria team at the research
center in Sophia-Antipolis. He obtained his PhD (2008) at the University of
Bologna and he joined Inria as research scientist in 2012.
He is interested in the design and implementation of interactive
provers and in high level languages for extend their functionalities.
He is a core developer of the Coq system, the Mathematical Components library
and the Elpi programming language. He regularly serves as PC member of
international conferences such as ITP, CPP, PADL, MKM. He regularly organizes
schools on Coq and the Mathematical Components library.



\end{itemize}

\end{sitedescription}

%%% Local Variables:
%%% mode: latex
%%% TeX-master: "../propB"
%%% End:

% LocalWords:  site-jacu.tex clange sitedescription emph compactitem pn semmath
% LocalWords:  prosuming-flexiformal KohSuc asemf06 GinJucAnc alsaacl09 StaKoh
% LocalWords:  tlcspx10 KohDavGin psewads11 ednote Radboud Bia ystok CALCULEMUS
% LocalWords:  textbf keypubs OntoLangMathSemWeb uwb Deyan Ginev Stamerjohanns
% LocalWords:  searchability
