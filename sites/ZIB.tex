\begin{sitedescription}{Zib}

%\ednote{a description of the legal entity}

\logo{ZIB}

The Zuse Institut Berlin (ZIB) is an interdisciplinary research institute
for applied mathematics and data-intensive high-performance computing.
Its research focuses on modeling, simulation and optimization
with scientific cooperation partners from academia and industry.
The institute is located at the interface between mathematical method
development and the analysis and management of large data sets.
It is a link and a communication amplifier between the applied sciences and
mathematics.  Together with FIZ Karlsruhe we develop and maintain swMATH,  a freely
accessible information platform for mathematical software.

\paragraph*{Main tasks:}

%\ednote{its main tasks, with an explanation of how its profile matches the tasks in the proposal}

	
\begin{compactitem}
	\item 	 \taskref{dissemination}{publishers-club}: Expanding the use of  Logipedia in publishing
    \item 	 \taskref{dissemination}{zib}: Linking scientific publications to Logipedia 
\end{compactitem}

 
% \ednote{specify the main tasks and reference the respective work packages}

%\begin{compactitem}
%\item 
%\end{compactitem}

\paragraph*{Publications, products or services:}

% \ednote{a list of up to 5 relevant publications, and/or products, services (including widely-used datasets or software), or other achievements relevant to the  call content}

\begin{compactitem}
\item "alsoMATH - A Database for Mathematical Algorithms and Software" 
by Wolfgang Dalitz, Wolfram Sperber, Moritz Schubotz, Hagen Chrapary, 
Workshop Papers at 12th Conference on Intelligent Computer Mathematics CICM 2019, 
12th Conference on Intelligent Computer Mathematics (CICM 2019), Prague (CZ), July 8-12, 2019 

\item  "Software Products, Software Versions, Archiving of Software, and swMATH",
by Hagen Chrapary, Wolfgang Dalitz, 
in Mathematical Software - ICMS 2018
6th International Conference, South Bend, IN, USA, July 24-27, 2018, Proceedings 

\end{compactitem}

\paragraph*{Previous projects or activities:}

%\ednote{a list of up to 5 relevant previous projects or activities, connected to the subject of this proposal}

\begin{compactitem}
\item swMATH (www.swmath.org) is a freely accessible, innovative information service for mathematical software. 
swMATH not only provides access to an extensive database of information on mathematical software, 
but also includes a systematic linking of software packages with relevant mathematical publications.
\end{compactitem}

%\paragraph*{Infrastructures or technical equipments:}

%\ednote{a description of any significant infrastructure and/or any major items of technical equipment, relevant to the proposed work}

\paragraph*{Persons primarily responsible for carrying out the proposed activities:}

\begin{compactitem} % in alphabetical order

\item{\bf Wolfgang Dalitz} is a  researcher at ZIB working in the field of Scientific Information Services. 
He leads the working group 'Web Technology and Multimedia' in the division 'Scientific Information System' at ZIB.


\end{compactitem}

\end{sitedescription}

%%% Local Variables:
%%% mode: latex
%%% TeX-master: "../propB"
%%% End:
