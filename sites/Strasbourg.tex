\begin{sitedescription}{Str}

\paragraph{Organization:}
\ednote{Give a one-paragraph run-down of the site and the team there. }

Located in the heart of Europe, the University of Strasbourg is heir to a great tradition born of
the humanism of the 16 th century.

On 1 January 2009 the University of Strasbourg was born - a unique and pioneering
example of merging universities in France: Louis Pasteur, Marc Bloch and Robert Schuman.
European by nature and international by design, the University’s fundamental training and
research goals include forging partnerships with European and international universities.
Located on 4 campuses spread all over the city, the University of Strasbourg is one of the
largest universities in France, with nearly 51 000 students (including 20 \% of international
students).

Certified Excellence Initiative (IdEx) - obtained in 2012 and definitively confirmed in 2016
by the national programme “Investissements d’Avenir - the University of Strasbourg
strengthens its position as an internationally attractive university. Implementing innovative
projects that foster excellence, the University of Strasbourg is involved in supporting its
researchers and students.
As a leading European centre for training and research, the University of Strasbourg has
developed a strong French-German cooperation and is now a privileged partner among the
Upper-Rhine universities.

The University is involved in national and European research projects within various
programmes. Since 2009, the University of Strasbourg obtained 74 FP7 projects, 76 H2020
projects, 30 INTERREG IV projects, 29 INTERREG V projects and 375 projects under the
French National Research Programme (ANR). Presently 107 ANR projects, 50 H2020
projects, 23 INTERREG V projects and 19 Erasmus+ projects (1 European University, 2
Erasmus Mundus master degrees, 5 Erasmus + strategic partnerships, 1 knowledge alliance, 1
capacity building project, 1 Erasmus + Sport project and 6 Jean Monnet actions - including 1
Centre of excellence) are active. It currently coordinates 42 EU projects and is preparing and
awaiting the evaluation of approximately 40 proposals.

ICube: Created in 2013, the laboratory brings together researchers of the University of Strasbourg, the CNRS(Centre National de la Recherche Scientifique), in the fields of engineering science and computer science.
In this context the IGG team focuses on geometric modeling, visualization, constraint solving and formalization of geometry. The member of the project focus on the formal definition of the geometric universe, proof of properties, automatic generation of geometric objects defined by a specification and deriving certified geometric algorithms. We work on computer science methods allowing to assist proofs, guarantee the correctness and the feasibility and, when possible, to insure automatically some task using Coq tactics or, geometric constraint solving. The results of these researches can be exploited in geometric modeling, computational geometry, pure geometry, mathematics teaching.

\paragraph{Main tasks:}

\begin{compactitem}
\item\ednote{specify the main tasks and reference the respective work packages} 
\item Integration of the GeoCoq library in Logipedia
\item Concept alignement for geometry.
\item Reverse mathematics for geometry.
\item User interface for interactive theorem proving for the education.
\end{compactitem}

\paragraph{Relevant previous experience:}

\ednote{give an overview over previous work and projects that add to the \pn project}

Members of the group have expertise in the field of interactive and automated theorem proving in geometry.
They have been involved in several national, bilateral and international projects (the French ANR project Galapagos, Serbian-French Technlology Co-Operation grant, etc).


\paragraph{Specific expertise:}

\begin{compactitem}
\item \ednote{give three to five specific areas of expertise that pertain to the \pn project}
\item Interactive theorem proving in geometry (Magaud, Narboux, Schreck).
\item Automatic theorem proving in geometry (Magaud, Narboux, Schreck).
\item Axiomatization of geometry (Schreck, Narboux).
\item Change of data representation in proofs (Magaud). 
\end{compactitem}

\paragraph{Staff members undertaking the work:}


\textbf{Dr.\ Julien Narboux}\ednote{describe the site leader and his expertise}
Julien Narboux is an associate professor at the Department of Computer Science, University of Strasbourg, France since 2007. He received a doctorate from University of Orsay in 2006 about “Formalization and automation of geometric reasoning”. After that he held a postdoc positions at TUM.
He published about 30 papers in peer-rewieved international conferences and journals, and has been PC member of international conferences and workshops such as ADG, AISC, SCSS, FVPS, ThEdu. He is the head of the steering committee of the Automatic Deduction in Geometry conference. Julien Narboux is the leader of the GeoCoq project.


\textbf{Nicolas Magaud} is an associate professor at the Department of
Computer Science, University of Strasbourg, France since 2005. He
received a PhD from the University of Nice Sophia-Antipolis, France in
2003. His thesis subject was ``changing data representation in the
calculus of constructions''. Before being hired by University of
Strasbourg, he was a senior research associate at the University of
New South Wales, Sydney, Australia. In Strasbourg, Nicolas Magaud
has been working on formalizing various aspects of geometry using Coq, spanning from
computational geometry algorithms to exact real computations applied to
discrete geometry. He published about 15 papers in peer-rewieved
international conferences and journals.  


\textbf{Pascal Schreck} is full professor in computer science since 2002. He is interested in the formalization of various geometries from the rule and compass constructions to finite incidence geometry including geometric algebras, Tarski and Wu's geometries \emph{etc.} He studied some applications of theses formal geometries mainly in mechanical CAD and computer aided education.


\ednote{provide the key publications below}
\keypubs{providemore}

\begin{itemize}
\item Nicolas Magaud. \emph{Changing Data Representation within the Coq System.} In TPHOLs'2003, volume 2758 of LNCS. Springer-Verlag, 2003
\item Pierre Boutry, Gabriel Braun, Julien Narboux. \emph{Formalization of the Arithmetization of Euclidean Plane Geometry and Applications.} Journal of Symbolic Computation, Elsevier, 2019, Special Issue on Symbolic Computation in Software Science, 90, pp.149-168.
\item Michael Beeson, Julien Narboux, Freek Wiedijk. \emph{Proof-checking Euclid.} Annals of Mathematics and Artificial Intelligence, Springer Verlag, 2019, pp.53.
\item Julien Narboux, David Braun. \emph{Towards A Certified Version of the Encyclopedia of Triangle Centers.} Mathematics in Computer Science, Springer, 2016
\item David Braun, Nicolas Magaud, Pascal Schreck, \emph{Two Cryptomorphic Formalizations of Projective Incidence Geometry}, Annals of Mathematics and Artificial Intelligence, Springer Verlag
\end{itemize}

\textbf{Software:}
Michael Beeson, Pierre Boutry, Gabriel Braun, Charly Gries, Julien Narboux. GeoCoq. 2018, ⟨swh:1:dir:97ce53176b7d5e89d069bc60f49c3fa186831307⟩

\end{sitedescription}
%%% Local Variables: 
%%% mode: latex
%%% TeX-master: "../propB"
%%% End: 

% LocalWords:  site-jacu.tex clange sitedescription emph compactitem pn semmath
% LocalWords:  prosuming-flexiformal KohSuc asemf06 GinJucAnc alsaacl09 StaKoh
% LocalWords:  tlcspx10 KohDavGin psewads11 ednote Radboud Bia ystok CALCULEMUS
% LocalWords:  textbf keypubs OntoLangMathSemWeb uwb Deyan Ginev Stamerjohanns
% LocalWords:  searchability
