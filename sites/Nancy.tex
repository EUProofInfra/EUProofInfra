\begin{sitedescription}{INa}

\paragraph{Organization:}
%\ednote{Give a one-paragraph run-down of the site and the team there. }

Inria Nancy\,--\,Grand Est (INa) is one of eight centers of Inria, a public
research institute in computer science and applied mathematics in France. INa
hosts 22 research groups. The \href{https://team.inria.fr/veridis/}{VeriDis}
research group headed by \emph{Dr.\ Stephan Merz}, joint to Inria, CNRS,
University of Lorraine and Max-Planck Institute for Informatics, works on formal
techniques of modeling and verification, in particular on automatic and
interactive theorem proving techniques, applied to distributed algorithms and
systems. The members of VeriDis participating to \pn include Dominique M\'ery
and Stephan Merz.

\paragraph{Main tasks:}

\begin{compactitem}
\item\ednote{specify the main tasks and reference the respective work packages} 
\end{compactitem}


\paragraph{Relevant previous experience:}
%\ednote{give an overview over previous work and projects that add to the \pn project}

VeriDis has strong experience with the (Event-)B and \tlaplus formalisms for
modeling software systems and the associated verification toolkits. In
particular, it plays a leading role in developing TLAPS, the \tlaplus Proof
System~\cite{cousineau:tla-proofs}, a proof assistant for reasoning about
\tlaplus specifications, and it has contributed to an SMT
backend~\cite{deharbe:smt-rodin} to the Rodin platform for verifying Event-B
models. VeriDis, together with the site at University of Liège, has strong
expertise in SMT (satisfiability modulo theory) techniques and develops the SMT
solver \veriT that serves both as a testbed for experimenting new ideas and can
produce detailed proofs that can be certified by skeptical proof assistants. As
such, it is used as a backend prover for Coq, Isabelle/HOL, and Rodin.

\paragraph{Specific expertise:}

%\ednote{give three to five specific areas of expertise that pertain to the \pn project}
\begin{compactitem}
\item automatic and interactive theorem proving,
\item generating and checking certificates for automatic theorem provers,
\item formal models and proofs for (distributed) algorithms and systems.
\end{compactitem}

\paragraph{Staff members undertaking the work:}

\textbf{Dr.\ Great Leader}\ednote{describe the site leader and his expertise}
\textbf{Joe Implementor}\ednote{and more of them. }
\ednote{provide the key publications below}
\keypubs{providemore}
\end{sitedescription}
%%% Local Variables: 
%%% mode: latex
%%% TeX-master: "../propB"
%%% End: 

% LocalWords:  site-jacu.tex clange sitedescription emph compactitem pn semmath
% LocalWords:  prosuming-flexiformal KohSuc asemf06 GinJucAnc alsaacl09 StaKoh
% LocalWords:  tlcspx10 KohDavGin psewads11 ednote Radboud Bia ystok CALCULEMUS
% LocalWords:  textbf keypubs OntoLangMathSemWeb uwb Deyan Ginev Stamerjohanns
% LocalWords:  searchability
