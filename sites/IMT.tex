\begin{sitedescription}{Imt}

\logo{imt_logo}

IMT (Institut Mines-T\'el\'ecom) is a French institute that groups together 13 engineering and management graduate schools, around 12,300 students from Bachelor to PhD degrees.
ENSIIE is an engineering school centered on applied mathematics and computer science; this school is associated to the IMT group. SAMOVAR is a research unit under the IMT umbrella that gathers researchers from T\'el\'ecom SudParis (which is part of IMT) and ENSIIE. Its key domains are Applied Mathematics,  Computer Science, Networks, IT applications and uses, Signal and image processing. EUProofInfra participants are members of the METHODES team, whose research activities are devoted to
optimization, verification and performance evaluation.

\paragraph*{Main tasks:}

\begin{compactitem}
\item Translate ATP traces into Dedukti \WPref{atpetc}. Guillaume Burel leads Task \taskref{atpetc}{tracetodedukti}; he is one
of the experts on the translation of SAT-solver traces into Dedukti and on the reconstruction of TSTP traces.
\item Automatic detection of alignments \WPref{alignment}. Stefania Dumbrava leads Task \taskref{alignment}{aligntheories}; she will be 
in charge of investigating approaches to aligning theories based on ontology reasoning methods. She is one of the experts on automated reasoning and databases.
\item Export B models to Dedukti \WPref{instrumentation}. Catherine Dubois leads the Task \taskref{instrumentation}{atelier-b}. She developed the translator of FoCaLiZe into Dedukti and 
partipated in the Bware project, concerning Dedukti-based proofs and B proof obligations. 
\item Expand the use of Logipedia within certification authorities \WPref{dissemination}. Catherine Dubois also leads the Task \taskref{dissemination}{certif-club}. She has already worked 
with the French security agency, ANSII, to study the security of functional languages.
\end{compactitem}

\paragraph*{Publications, products or services:}

\begin{compactitem}
 \item ``CoqInE: Translating the Calculus of Inductive Constructions
   into the $\lambda{}\Pi{}$-calculus Modulo'', by M. Boespflug and
   G. Burel. 2nd Intl. Workshop on Proof Exchange for Theorem Proving,
   2012.
 \item ``Translating HOL to Dedukti'', by A. Assaf and G.Burel. 4th Intl. Workshop on Proof eXchange for Theorem Proving, 2015.
 \item ``ML Pattern-Matching, Recursion, and Rewriting: From FoCaLiZe to Dedukti'', by R. Cauderlier and C. Dubois. Proceedings of ICTAC, 2016.
 \item ``FoCaLiZe and Dedukti to the Rescue for Proof Interoperability'', by R. Cauderlier and C. Dubois. Proceedings of ITP, 2017.
 \item ``Certifying Standard and Stratified Datalog Inference Engines in SSReflect'', by V. Benzaken, E. Contejean, and S. Dumbrava. Proceedings of ITP, 2017.
\end{compactitem}

\paragraph*{Previous projects or activities:}

\begin{compactitem}
\item ``BWare, A Proof-Based Mechanized Plate-Forme for the Verification of B Proof Obligations'',
funded by French National Research Agency, project manager: David
Delahaye, 08/2012 -- 12/2016
\end{compactitem}

\paragraph*{Infrastructures or technical equipments:}

\begin{compactitem}
\item Production of Dedukti proofs by automated theorem provers (\href{http://www.ensiie.fr/~guillaume.burel/blackandwhite_iProverModulo.html.en}{iProverModulo})
\item Reconstruction of proof traces
  (\href{https://github.com/gburel/lrat2dk}{lrat2dk} and \href{https://github.com/Deducteam/ekstrakto}{Ekstrakto})
\item Translation of proofs into Dedukti. (\href{http://www.ensiie.fr/~guillaume.burel/blackandwhite_coqInE.html.en}{Coq
  proofs}, \href{http://deducteam.gforge.inria.fr/holide/}{HOL
  proofs}, \href{http://deducteam.gforge.inria.fr/focalide/}{FoCaLiZe proofs})
  \item Interoperability of HOL and Coq through FoCaLiZe and Dedukti (proof of concept)
\end{compactitem}

\paragraph*{Persons primarily responsible for carrying out the proposed activities:}

\begin{compactitem} % in alphabetical order

\item{\bf Guillaume Burel}
 is
assistant professor at the ENSIIE since 2010. He defended his PhD,
entitled ``Good Proofs in Deduction Modulo'', in March 2009. From
September 2009 to August 2010, he was post-doctoral fellow at the Max
Planck Institute for Informatics in Saarbr\"ucken, Germany, in the
research group on Automation of Logic. From September 2010 to December
2015, he was a member of the CEDRIC laboratory of the Cnam. Since
January 2016, he is member of the SAMOVAR laboratory (UMR 5157 CNRS
T\'el\'ecom Sud Paris), where he belongs to the METHODES team. He was
temporarily assigned to Inria-team Deducteam from September 2017 to
August 2019.  He has 4
publications in peer-reviewed international journals and 12 in
international conferences ; he co-supervised 2 PhD theses, and he is
co-supervising 1 PhD student and 1 postdoctoral fellow. He is the
main developer of
\href{http://www.ensiie.fr/~guillaume.burel/blackandwhite_iProverModulo.html.en}{iProverModulo}.

\item {\bf Catherine Dubois} is a professor at ENSIIE since 2000. She defended her PhD, entitled ``Static determination of types for the SetL language'', in 1989 and her Habilitation in 2000 entitled a
\emph{A journey from programming to proof}. She is a
member of the Samovar laboratory (T\'el\'ecom
SudParis), in the METHODES team. Before, she was the leader of a research team at the CEDRIC laboratory (CNAM). Her research activities concern the application of formal methods (in particular with Coq, FoCaLiZe and B) and their combination with testing techniques. She supervised 10 Phd thesis and is currently supervising one PhD student. She participated to the translation of FoCaLiZe into Dedukti and the interoperability of Coq and HOL proofs through Dedukti.
She is one of the task leaders of \WPref{dissemination}.

\item{\bf Stefania Dumbrava}
is an assistant professor at ENSIIE since 2019. She is also a permanent researcher in the SAMOVAR laboratory (UMR 5157 CNRS T\'el\'ecom
Sud Paris), where she belongs to the METHODES team. She obtained her PhD at Universit\'e Paris-Saclay in 2016 with a thesis on the formalization
of relational and deductive databases. Her fields of expertise are theorem proving, in particular with the Coq proof assistant, and databases.
This expertise is relevant for the research concerning automated alignment detection using reasoning engines. She is one of the task leaders of \WPref{alignment}.
\end{compactitem}

% \keypubs{boespflug12coqine,burel13shallow,assaf15translating}
% \keypubs{CauderlierDubois17,CauderlierDubois17,CauderlierDubois16,DuboisG18}
% \keypubs{BCD14,BCD17,BDA18}

\end{sitedescription}
%%% Local Variables:
%%% mode: latex
%%% TeX-master: "../propB"
%%% End:

% LocalWords:  site-jacu.tex clange sitedescription emph compactitem pn semmath
% LocalWords:  prosuming-flexiformal KohSuc asemf06 GinJucAnc alsaacl09 StaKoh
% LocalWords:  tlcspx10 KohDavGin psewads11 ednote Radboud Bia ystok CALCULEMUS
% LocalWords:  textbf keypubs OntoLangMathSemWeb uwb Deyan Ginev Stamerjohanns
% LocalWords:  searchability
