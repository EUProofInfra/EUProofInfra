\begin{sitedescription}{Lmu}

\newcommand\inquotes[1]{``#1''}

\logo{LMU}

Ludwig-Maximilians-Universit\"at (LMU) is a public research university
located in Munich, Germany.  LMU consists of 18 faculties which
accommodate various departments and institutes including the
Mathematisches Institut, where the research group of mathematical
logic is headed by Prof.\ Dr.\ Helmut Schwichtenberg.  The research
areas of the group are such as proof theory, realizability
interpretation, programme extraction, constructive analysis,
constructive algebra, and proof assistant.  In particular in the
research area of proof assistant, the logic group has been actively
developing the Minlog system since early 1990's.

\paragraph*{Main tasks:}

\begin{compactitem}
  \item Encoding the underlying theory of Minlog in Dedukti. \WPtref{theories} \taskref{theories}{minlog}
  \item Implementing the encoder, so that Minlog's libraries and formal proofs of constructive analysis is available in Dedukti with proof checking.
The Logipedia integration level of Minlog is increased to 3 from 0. \WPtref{theories} \taskref{theories}{minlog}
  \item Making Minlog's classical extraction available within Dedukti.  Alignment for concepts in constructive analysis, in particular (co)induction/(co)recursion and predicativity. \WPtref{alignment} \taskref{alignment}{alignlogic}.  

\end{compactitem}

\paragraph*{Publications, products or services:}

\begin{compactitem}
\item ``Refined program extraction from classical proofs'',
  by U.~Berger, W.~Buchholz, and H.~Schwichtenberg.
  Annals of Pure and Applied Logic, 114:3--25, 2002.

\item ``Dialectica interpretation of well-founded induction'',
  by H.~Schwichtenberg.
Math. Logic. Quarterly, 54(3):229--239, 2008.

\item ``Realizability interpretation of proofs in constructive analysis'',
  by H.~Schwichtenberg.
 Theory of Computing Systems, 43(3):583--602, 2008.

\item ``Basic Proof Theory'',
  by A.~S. Troelstra and H.~Schwichtenberg.
 Cambridge University Press, second edition, 2000.

\item ``Proofs and Computations'',
  by H.~Schwichtenberg and S.~S. Wainer.
Perspectives in Logic. Association for Symbolic Logic and Cambridge
  University Press, 2012.
\end{compactitem}

\paragraph*{Previous projects or activities:}

\begin{compactitem}
  \item 1997-2006, Speaker of the DFG-Graduiertenkolleg 301
    \inquotes{Logik in der Informatik}

\item 2004-2008, LMU Coordinator of the  EST (Early Stage Traning) Programme
  \inquotes{MathLogAps} (MEST-CT-2004-504029) of the EU, together with the
universities of Leeds, Manchester, Lyon and ENS Lyon

\item 2009-2013, LMU Coordinator of the ITN (Network for Initial
    Training) Programme PITN-GA-2009-238381 \inquotes{MALOA} of the
    EU, together with the universities of Leeds, Manchester, Oxford,
    CNRS, Paris 7, M\"unster, Prague

\item 2017-2021, LMU Coordinator of the 731143-CID project of LMU

\item 05/2018-08/2018, Co-organizer (with D.~Bridges, M.~Rathjen and
  P.~Schuster) of a Trimester on \inquotes{Types, Sets, Constructions}
  at the Hausdorff Institute for Mathematics, Bonn
\end{compactitem}

%% \paragraph*{Specific expertise:}
%% \begin{compactitem}
%% \item Implementation of the proof assistant Minlog.
%% \item Foundation for constructive mathematics accommodating partial functionals and realizability.
%% \item Constructive mathematics and programme extraction.
%% %% \item \ednote{give three to five specific areas of expertise that pertain to the \pn project}
%% \end{compactitem}

\paragraph*{Infrastructures or technical equipments:}

\begin{compactitem}
\item The logic group at LMU has developed the Minlog proof assistant since 1990.
\item The Minlog library for constructive analysis has been developed since 2004
and corecursion and coinduction have been involved since 2010.
\item The Minlog feature for classical extraction has been developed since 2002.
\end{compactitem}

\paragraph*{Persons primarily responsible for carrying out the proposed activities:}

\begin{compactitem}

\item \textbf{Josef Berger} is a Privatdozent at LMU.  He earned his
  Doctoral degree in 2002 from LMU, in Nonstandard stochastics,
  supervised by Horst Osswald, and his Habilitation in 2014 at LMU
  with a thesis on "Perspectives in Constructive Reverse Mathematics".

\item \textbf{Nils K\"opp} is a teaching assistant and a PhD student
  at LMU.  Master thesis 2017 on "Automatically verified programme
  extraction from proofs with applications to constructive analysis".

\item \textbf{Franz Merkl} is a professor and the chair of stochastics
  at LMU.  He has supervised some Diploma theses on subjects in
  probability theory, which were formalized in Mizar.  He himself has
  also worked with Mizar and published in the "Journal of Automated
  Reasoning", where only papers checked by Mizar are accepted.

\item \textbf{Kenji Miyamoto} is a teaching assistant and a postdoc
  researcher at LMU.  Doctorate 2013 at LMU with a thesis "Programme
  extraction from coinductive proofs and its application to exact real
  arithmetic".  Worked as Postdoc and teaching assistant at LMU and in
  Innsbruck (with Georg Moser).

\item \textbf{Iosif Petrakis} is a lecturer and a postdoc researcher
  at LMU.  Doctorate 2015 at LMU with a thesis "Constructive Topology
  of Bishop Spaces".  Presently preparing his Habilitation in
  Mathematics.

\item \textbf{Helmut Schwichtenberg} is a professor (emeritus) of
  Mathematics at LMU.  Book (with Stanley Wainer) on Proofs and
  Computations, Cambridge University Press, 2012.  Book (with Anne
  Troelstra) "Basic Proof Theory", Cambridge University Press, 2nd
  ed. 2000.  Coorganizer (with Douglas Bridges, Michael Rathjen and
  Peter Schuster) of the Hausdorff Trimester on Sets, Types and
  Constructions at the Hausdorff Institute, Universit\"at Bonn,
  May-August 2018.  Coorganizer (with Klaus Mainzer and Peter
  Schuster) of the annual Autumn School on Proofs and Computations.

\item \textbf{Franziskus Wiesnet} is a PhD student co-supervised by
  Peter Schuster (Verona) and Helmut Schwichtenberg (LMU).  Master
  thesis "Konstruktive Analysis mit exakten reellen Zahlen" 2017 at
  LMU.  He is supported by a Marie Sk{\l}odowska-Curie fellowship of
  the Istituto Nazionale di Alta Matematica

\item \textbf{Chuangjie Xu} is a postdoc researcher at LMU, holding a
  Humboldt grant.  PhD 2015 in Birmingham under the supervision of
  Martin Escardo.  Half of the theses consisted of an Agda
  implementation of the theoretical results achieved.

\end{compactitem}
%% \textbf{Dr.\ Great Leader}\ednote{describe the site leader and his expertise}
%% \textbf{Joe Implementor}\ednote{and more of them. }

%%\keypubs{providemore}

%%Helmut Schwichtenberg, Kenji Miyamoto

\end{sitedescription}
%%% Local Variables: 
%%% mode: latex
%%% TeX-master: "../propB"
%%% End: 

% LocalWords:  site-jacu.tex clange sitedescription emph compactitem pn semmath
% LocalWords:  prosuming-flexiformal KohSuc asemf06 GinJucAnc alsaacl09 StaKoh
% LocalWords:  tlcspx10 KohDavGin psewads11 ednote Radboud Bia ystok CALCULEMUS
% LocalWords:  textbf keypubs OntoLangMathSemWeb uwb Deyan Ginev Stamerjohanns
% LocalWords:  searchability
