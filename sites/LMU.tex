\begin{sitedescription}{Lmu}

\newcommand{\inquotes}[1]{``#1''}
\paragraph{Organization:}

Ludwig-Maximilians-Universit\"at (LMU) is a public research university
located in Munich, Germany.  LMU consists of 18 faculties which
accommodate various departments and institutes including the
Mathematisches Institut, where the research group of mathematical
logic is headed by Prof.\ Dr.\ Helmut Schwichtenberg.  The research
areas of the group are such as proof theory, realizability
interpretation, programme extraction, constructive analysis,
constructive algebra, and proof assistant.  In particular in the
research area of proof assistant, the logic group has been actively
developing the Minlog system since early 1990's.

\paragraph{Main tasks:}

\begin{compactitem}
%\item\ednote{specify the main tasks and reference the respective work packages} 
  \item Encoding the underlying theory of Minlog in Dedukti. \WPtref{theories} \taskref{theories}{minlog}
  \item Implementing the encoder, so that Minlog's libraries and formal proofs of constructive analysis is available in Dedukti with proof checking.
The Logipedia integration level of Minlog is increased to 3 from 0. \WPtref{theories} \taskref{theories}{minlog}
  \item Contributing to \WPtref{alignment}.  The main targets are concepts in constructive analysis and (co)induction/(co)recursion.

%% how to use \taskref{2}{3} ?
\end{compactitem}

\paragraph{Relevant previous projects:}

\ednote{give an overview over previous work and projects that add to the \pn project}
\begin{itemize}
  \item 1997-2006, Speaker of the DFG-Graduiertenkolleg 301
    \inquotes{Logik in der Informatik}

\item 2004-2008, LMU Coordinator of the  EST (Early Stage Traning) Programme
  \inquotes{MathLogAps} (MEST-CT-2004-504029) of the EU, together with the
universities of Leeds, Manchester, Lyon and ENS Lyon

\item 2009-2013, LMU Coordinator of the ITN (Network for Initial
    Training) Programme PITN-GA-2009-238381 \inquotes{MALOA} of the
    EU, together with the universities of Leeds, Manchester, Oxford,
    CNRS, Paris 7, M\"unster, Prague

\item 2017-2021, LMU Coordinator of the 731143-CID project of LMU

\item 05/2018-08/2018, Co-organizer (with D.~Bridges, M.~Rathjen and
  P.~Schuster) of a Trimester on \inquotes{Types, Sets, Constructions}
  at the Hausdorff Institute for Mathematics, Bonn
\end{itemize}
\paragraph{Specific expertise:}
\begin{compactitem}
\item Implementation of the proof assistant Minlog.
\item Foundation for constructive mathematics accommodating partial functionals and realizability.
\item Constructive mathematics and programme extraction.
%% \item \ednote{give three to five specific areas of expertise that pertain to the \pn project}
\end{compactitem}

\paragraph{Staff members undertaking the work:}

\begin{itemize}

\item \textbf{Josef Berger} (male)\\
Doctoral degree 2002 from LMU, in Nonstandard
stochastics, supervised by Horst Osswald.  Habilitation 2014 at LMU
with a thesis on "Perspectives in Constructive Reverse Mathematics".
Currently he is Privatdozent at LMU.

\item \textbf{Nils K\"opp} (male)\\
Master thesis 2017 on "Automatically verified programme
extraction from proofs with applications to constructive analysis".
Presently teaching assistant and PhD student.

\item \textbf{Franz Merkl} (male)\\
Chair of stochastics at LMU.  Has supervised some
Diploma theses on subjects in probability theory, which were
formalized in Mizar.  He himself has also worked with Mizar and
published in the "Journal of Automated Reasoning", where only papers
checked by Mizar are accepted.

\item \textbf{Kenji Miyamoto} (male)\\
Doctorate 2013 at LMU with a thesis "Programme
extraction from coinductive proofs and its application to exact real
arithmetic".  Worked as Postdoc and teaching assistant at LMU and in
Innsbruck (with Georg Moser).  Presently teaching assistant at LMU.

\item \textbf{Iosif Petrakis} (male)\\
Doctorate 2015 at LMU with a thesis  "Constructive
Topology of Bishop Spaces".  Presently lecturer at LMU and preparing
his Habilitation in Mathematics.  

\item \textbf{Helmut Schwichtenberg} (male)\\
Professor (emeritus) of Mathematics at LMU.
Book (with Stanley Wainer) on Proofs and Computations, Cambridge
University Press, 2012.  Book (with Anne Troelstra) "Basic Proof
Theory", Cambridge University Press, 2nd ed. 2000.  Coorganizer (with
Douglas Bridges, Michael Rathjen and Peter Schuster) of the Hausdorff
Trimester on Sets, Types and Constructions at the Hausdorff
Institute, Universit\"at Bonn, May-August 2018.  Coorganizer (with
Klaus Mainzer and Peter Schuster) of the annual Autumn School on
Proofs and Computations.

\item \textbf{Franziskus Wiesnet} (male)\\
Master thesis "Konstruktive Analysis mit exakten
reellen Zahlen" 2017 at LMU.  Currently PhD student co-supervised by
Peter Schuster (Verona) and Helmut Schwichtenberg (LMU).  He is
supported by a Marie Sk{\l}odowska-Curie fellowship of the Istituto
Nazionale di Alta Matematica

\item \textbf{Chuangjie Xu} (male)\\
PhD 2015 in Birmingham under the supervision of Martin
Escardo.  Half of the theses consisted of an Agda implementation of
the theoretical results achieved.  Presently he holds a Humboldt grant
to do research at LMU.

\end{itemize}
%% \textbf{Dr.\ Great Leader}\ednote{describe the site leader and his expertise}
%% \textbf{Joe Implementor}\ednote{and more of them. }

\ednote{provide the key publications below}

\paragraph{Key publications:}

\begin{itemize}
\item U.~Berger, W.~Buchholz, and H.~Schwichtenberg.
\newblock Refined program extraction from classical proofs.
\newblock {\em Annals of Pure and Applied Logic}, 114:3--25, 2002.

\item H.~Schwichtenberg.
\newblock Dialectica interpretation of well-founded induction.
\newblock {\em Math. Logic. Quarterly}, 54(3):229--239, 2008.

\item H.~Schwichtenberg.
\newblock Realizability interpretation of proofs in constructive analysis.
\newblock {\em Theory of Computing Systems}, 43(3):583--602, 2008.

\item A.~S. Troelstra and H.~Schwichtenberg.
\newblock {\em Basic Proof Theory}.
\newblock Cambridge University Press, second edition, 2000.

\item H.~Schwichtenberg and S.~S. Wainer.
\newblock {\em Proofs and Computations}.
\newblock Perspectives in Logic. Association for Symbolic Logic and Cambridge
  University Press, 2012.
\end{itemize}

\paragraph{Infrastructure}

Rechenzentrum des Mathematischen Instituts der LMU 

\keypubs{providemore}

Helmut Schwichtenberg, Kenji Miyamoto

\end{sitedescription}
%%% Local Variables: 
%%% mode: latex
%%% TeX-master: "../propB"
%%% End: 

% LocalWords:  site-jacu.tex clange sitedescription emph compactitem pn semmath
% LocalWords:  prosuming-flexiformal KohSuc asemf06 GinJucAnc alsaacl09 StaKoh
% LocalWords:  tlcspx10 KohDavGin psewads11 ednote Radboud Bia ystok CALCULEMUS
% LocalWords:  textbf keypubs OntoLangMathSemWeb uwb Deyan Ginev Stamerjohanns
% LocalWords:  searchability
