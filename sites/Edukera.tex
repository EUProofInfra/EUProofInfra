\begin{sitedescription}{Edu}

\ednote{a description of the legal entity}

\logo{Edukera}

The \hyperlink{https://www.edukera.com}{edukera} software company, created in 2013, develops the eponym application, an interactive prover dedicated to education.

\paragraph{Main tasks:}

\ednote{its main tasks, with an explanation of how its profile matches the tasks in the proposal}

The task of Edukera is to develop the web interface for doing proofs at school. Edukera
has 8 years of experience in designing interactive prover interface for education.
Since its launch in 2016, the edukera interactive prover has enabled more than 8000 students
(117 classes) from L1 to M2 to solve more than 250 000 exercises (half a million attempts).
The goal is to develop an enhanced and open source version of the current edukera interface
and theories, and replace them with the Logipedia's ones.

\ednote{specify the main tasks and reference the respective work packages}

\begin{compactitem}
\item \taskref{access}{edukera}: Web interface for doing proofs at school.
\end{compactitem}

\paragraph{Publications, products or services:}

\ednote{a list of up to 5 relevant publications, and/or products, services (including widely-used datasets or software), or other achievements relevant to the  call content}

\begin{compactitem}
\item
\end{compactitem}

\paragraph{Previous projects or activities:}

\ednote{a list of up to 5 relevant previous projects or activities, connected to the subject of this proposal}

\begin{compactitem}
\item
\end{compactitem}

\paragraph{Infrastructures or technical equipments:}

\ednote{a description of any significant infrastructure and/or any major items of technical equipment, relevant to the proposed work}

\paragraph{Persons primarily responsible for carrying out the proposed activities:}

\begin{itemize} % in alphabetical order
\item{\bf M. Benoit Rognier}

\end{itemize}

\end{sitedescription}

%%% Local Variables:
%%% mode: latex
%%% TeX-master: "../propB"
%%% End:
