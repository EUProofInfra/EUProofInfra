\begin{sitedescription}{Edu}

\ednote{a description of the legal entity}

\logo{Edukera}

The \hyperlink{https://www.edukera.com}{Edukera} software company, created in 2013,
develops the eponym application, an interactive prover dedicated to education.

Edukera is an online application for students to solve mathematical exercices that require to
formulate a mathematical proof. The proof is automatically and instantly verified by the application. Professors
have access to detailed activity reports of the class. They can select exercices in the exercices database
for training and scheduled homeworks.

Since its commercial launch in 2016, the Edukera interactive prover has enabled more than 8000 students
(117 classes) from L1 to M2 to solve more than 250 000 exercises (half a million attempts). It currently
provides 911 exercises free of charge for students, dispatched in various domains:
\begin{compactitem}
\item 193 formalisation exercises
\item 148 Logic exercises (Connectors/Quantifiers)
\item 226 Set Theory exercises (level L1)
\item 344 Algebra exercises (level High School)
\end{compactitem}

\paragraph{Main tasks:}

\ednote{its main tasks, with an explanation of how its profile matches the tasks in the proposal}

\begin{compactitem}
\item \taskref{dissemination}{edukera}\denote{Web interface for doing proofs at school}
\end{compactitem}

\paragraph{Publications, products or services:}

\ednote{a list of up to 5 relevant publications, and/or products, services (including widely-used datasets or software), or other achievements relevant to the  call content}

\begin{compactitem}
    \item Edukera: an interactive web prover for education
\end{compactitem}

\paragraph{Previous projects or activities:}

\ednote{a list of up to 5 relevant previous projects or activities, connected to the subject of this proposal}

\begin{compactitem}
\item
\end{compactitem}

\paragraph{Infrastructures or technical equipments:}

\ednote{a description of any significant infrastructure and/or any major items of technical equipment, relevant to the proposed work}

\paragraph{Persons primarily responsible for carrying out the proposed activities:}

\begin{itemize} % in alphabetical order
\item{\bf M. Benoit Rognier} is the co-founder and CEO at Edukera.
Before Edukera, he worked in the software industry in the domain of Artifical Intelligence
as a software engineer (2000-20005 at KXEN inc.), presales engineer (2005-2010 at KXEN inc.
and SmartFocus inc.) and director of innovation (2010-2012 at Probance inc.).
He specialized in applying Artifial Intelligence techniques in marketing automation solutions.
He graduated (2000) from the Institut Supérieur de la Matière et du Rayonnement
with a major in computer science.

\item{\bf Guillaume Duhamel} is the co-founder and CTO at Edukera.
Before Edukera, he worked in the software industry in the domain of Artificial Intelligence
as a software engineer (2010-2012 at Probance inc.). He specialized in developping full-stack web applications.
He graduated (2008) from EPITA with a major in Artifical Intelligence.

\end{itemize}

\end{sitedescription}

%%% Local Variables:
%%% mode: latex
%%% TeX-master: "../propB"
%%% End:
