\begin{sitedescription}{Irt}

%\ednote{a description of the legal entity}

\logo{SystemX}


SystemX is one of the 8 French Institutes of Technological Research founded in 2012 by a pool of industrial and academic members in order to enhance the national industry competitiveness. The Institute focuses on complex systems and systems of systems in the fields of Mobility and Autonomous Transport, Industry of the Future, Defense and Security and Environment and Sustainable Systems. The IRT SystemX accelerates the digital transformation of the French industry by:

\begin{compactitem}
\item Building {\bf collaborative} and {\bf collocated} research projects first fueled by its industrial and academic needs and resources and second supplemented by SystemX’ own resources and technological platforms. More than 60 R\&I projects have been launched in 7 years for an amount of 120 full time equivalents per year.
\item Valorizing all the acquired knowledge via software and hardware {\bf platforms} which can be reused and mutualized by SystemX in research projects. CHESS and CTI’ platforms are among them in order to tackle respectively the future cyber security and autonomous transport safety challenges.
\item Animating and federating targeted {\bf innovative ecosystems} via specific actions (e.g. Start@SystemX to find out promising young SMEs and startups), technological workshops and scientific conferences. In the Paris-Saclay ecosystem, SystemX has a natural role of industrial hub gathering inside the same room, the best academics and the edge of the industry
\item Piloting national public-private {\bf transformation accompaniment initiatives} in order to help industrial sectors to tackle today’s challenges (e.g. SAAS Academy, SIMSEO)
\end{compactitem}

Based on a strategic balanced public private partnership, IRT SystemX is a non-profit Research and Technology Organization which federates more than 100 industry and economical partners in the field of transport, security, energy.



\paragraph{Main tasks:}

IRT SystemX will have two main contributions in Logipedia. First, SystemX will contribute to the WP9 in order to set up and maintain the project platform. Moreover, with its experience in terms of collaboration with academic laboratories and industrial partners, SystemX will contribute to the dissemnitation WP by expanding the use of Logipedia in the industry.




%\ednote{its main tasks, with an explanation of how its profile matches the tasks in the proposal}

%\ednote{specify the main tasks and reference the respective work packages}

\begin{compactitem}
\item \taskref{access}{archi} : Defining the functional and software architecture
\item \taskref{access}{infra} : Defining the hardware architecture for the infrastructure
\item \taskref{access}{web} : Giving access to the infrastructure on the world-wide web
\item \taskref{access}{transfer} : Transfer for the sustainability of the system
\item WP8: SystemX will contribute to tasks T4 and T5
\end{compactitem}




\paragraph{Relevant expertise/experience}

IRT SystemX is a French industry hub, hosting and operating collaborative research project involving French industrial partners, academics and SystemX own resources into the same room. SystemX is known for providing its partners two fundamental aspects:

\begin{compactitem}
\item {\bf Bringing talents together.} The institute brings together all the partners involved in its projects under one roof, thus creating a melting pot of interaction between stakeholders in the public and industrial research sectors.
\item {\bf Pooling of skills and platforms.} SystemX is consolidating its technological platforms by pooling the components and infrastructures of its research projects and is developing expertise in the service of its public and private partners.
\end{compactitem}




\paragraph{Publications, products or services:}

%\ednote{a list of up to 5 relevant publications, and/or products, services (including widely-used datasets or software), or other achievements relevant to the  call content}

\begin{compactitem}
\item Adel Djoudi, Sébastien Bardin, Éric Goubault. "Recovering High-Level Conditions from Binary Programs", 21st International Symposium on Formal Methods (FM 2016), November 2016, Limassol, Cyprus
\item Sergio Bezzecchi, Paolo Crisafulli, Charlotte Pichot, Burkhart Wolff. “Making Agile Development Processes fit for V-style Certification Procedures”, 9th European Congress on Embedded Real Time Software and Systems (ERTS 2018), Jan 2018, Toulouse, France
\end{compactitem}





\paragraph{Previous projects or activities:}

%\ednote{a list of up to 5 relevant previous projects or activities, connected to the subject of this proposal}

Among the 50+ projects of the institute, the following are particularly relevant:


\begin{compactitem}
\item PST, IRT SystemX : Launched in 2016 for three years, PST (transport systems performance) aims to develop solid expertise in the assessment of the performance and operational safety of tomorrow’s transport (railway and aviation). For this, the project provides methodologies of systems and software engineering as well as embedded technology bricks that will contribute to improving the performance of these systems.

\end{compactitem}






\paragraph{Infrastructures or technical equipments:}

%\ednote{a description of any significant infrastructure and/or any major items of technical equipment, relevant to the proposed work}


SystemX’s platforms are developed by SystemX in conjunction with their partners and the National Cybersecurity Agency of France (ANSSI). The platforms have been designed as a flexible environment to support, among other activities, cybersecurity research experimentation, training, technology deployment, technology transfer, and technology evaluation and/or certification.  Backed with state-of-the-art facilities, they allow dynamic allocation and management of infrastructural resources (e.g. IT, OT, SCADA, Cloud, etc.) to setup use-case demonstrators. One of the main advantages in using those platforms is the seamless and synergetic integration of software and hardware resources, strengthening the performance, the stability and the scalability of the overall system. The platforms integrate also a large range of cybersecurity capabilities to protect the complete lifecycle of use-cases demonstrator.





\paragraph{Expected contribution to impact}
At the crossroads of Industry and Academics, SystemX is active in and proud member of several national and international ecosystems allowing the RTO to adequately disseminate the LOGIPEDIA results.

SystemX is an associated member of the {\bf Big Data Value Association cPPP}. The more than 200 members association manage 2 European events per year, the Big Data Value (BDV) PPP Summit and the European Big Data Value Forum (EBDVF). One is dedicated to members and the other one is open to non-members. http://www.bdva.eu/

SystemX has a historical and strong and strategic partnership with the {\bf French national cluster Systematic Paris-Région} focusing on DeepTech technologies. With more than 900 members among the best French Industry and Academics players and the best performing techno-providers. Systematic is also a French Digital Innovation Hub. https://systematic-paris-region.org/fr/

SystemX it a proud member of {\bf ASAM} (Association for Standardization of Automation and Measuring Systems), which is a non-profit organization that promotes standardization for tool chains in automotive development and testing. The assocation’s members are international car manufacturers, suppliers, tool vendors, engineering service providers and research institutes from the automotive industry. https://www.asam.net/

The Institute contributes to the {\bf ETSI ITS WorkingGroup 5} in charge of leading the drive to achieve global standards for Co-operative ITS, which offers enormous potential through vehicle-to-vehicle and vehicle-to-roadside communication. Applications include road safety, traffic control, fleet and freight management and location-based services, providing driver assistance and hazard warnings and supporting emergency services. https://www.etsi.org/

SystemX is member of the {\bf French Platform for Automotive (PFA)}. The association gathers the full french ecosystem of the french automotive industry, i.e. 4000 organisations from car manufacturers, equipment/solution providers, mobility players and professional associations. It defines and set up the strategy for the full branch in terms of innovation, competitiveness, employment and capabilities. The association is also owner of the Paris Global Automotive fair. https://pfa-auto.fr/

The {\bf CAR 2 CAR Communication Consortium} represents Vehicle and Infrastructure companies, authorities, C-ITS related organisations and research institutes. The Consortium aims on ensuring the interoperability of cooperative systems, spanning all vehicle classes across brands and borders. It has been founded in 2002 with the objective of developing European standards for C-ITS, as prerequisite for interoperability of systems improving road safety and road efficiency. Moreover, the CAR 2 CAR members discuss realistic deployment strategies, a roadmap to deployment and business models to speed-up the market penetration. In close collaboration with international stakeholders, especially from the US and Japan, the Consortium pushes the harmonisation of V2X communication standards world-wide. https://www.car-2-car.org/

Several {\bf valorization initiatives} have been launched since SystemX started its activities. 2 main regular events are organized each year:
\begin{compactitem}
\item {\bf FIT forum} the French Institutes of Technology is an association federating each of the 8 French IRTs. The Forum is the only moment during the year that allows to show and demonstrate all the capabilities and project result of each IRTs in the same place.
\item {\bf DigiHall Day} is co-organised with SystemX and the CEA List and is a yearly meeting involving all of the internal SystemX projects and partners. Demo tour, conferences and concrete results are at the heart of the event.
\end{compactitem}




\paragraph{Persons primarily responsible for carrying out the proposed activities:}

\begin{itemize} % in alphabetical order

\item{\bf Adel DJOUDI (male)} is a research engineer at IRT SystemX, he is interested the application of formal methods for safety and security of critical autonomous systems. Before that, he prepared his Ph.D in the Software Safety \& Security Lab of CEA and École polytechnique (Paris Area, France) where he worked on the application of formal methods to software security, with a strong focus on binary-level security analyses such as vulnerability detection \& assessment and reverse engineering.
\item{\bf Sébastien GRIPON (male)} is R\&D manager and leader of the Software \& DevOps team at IRT SystemX. Graduated of the French school INSA de Lyon as an engineer in computer science and information technology, he has an experience of more than 15 years in software development, including coding, project management and people management. Before joining IRT SystemX, he was head of software department at Quadient Bagneux R\&D, a French multinational company where he continuously implemented software development best practices.
\item{\bf Paolo CRISAFULLI (male)} is a senior software engineer, with 20+ years of experience in the field, mainly focusing on methodology and tools. Alumnus of the French Ecole Polytechnique and of Télécom Paris, he joined IRT SystemX in 2013 as a senior research engineer. Since then, after defining and setting up the collaborative software development environment, he moved to the field of systems engineering and dependability. His contributions to the SystemX’ PST project have been published in several scientific articles and contributed to improve the AADL and seL4 tooling.

\end{itemize}

\end{sitedescription}

%%% Local Variables:
%%% mode: latex
%%% TeX-master: "../propB"
%%% End:
