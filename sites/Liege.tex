\begin{sitedescription}{Lie}

\logo{ULiege}

The \href{http://www.uliege.be}{University of Liège} (ULiege) is located in the Fédération Wallonie-Bruxelles of Belgium in the Euregio region. ULiege is the only public and complete university institution of the French-speaking region of Belgium. The ULiege counts 2977 lecturers-researchers and 24688 students (incl. 2095 PhD students). 23\% of the students at ULiege are foreign students from 127 different countries. A wide variety of fundamental and applied research projects have emerged from about 43 Faculty and 11 interfaculty Research Units. On the international level, the University of Liege is actively involved in research projects with more than seventy countries worldwide. ULiege has been involved in 191 European FP7 and H2020 projects and is active in 8 H2020 INFRA projects. At the end of 2018, 2093 research agreements were in progress, of which 1458 involved an international partner. In parallel, ULiege has developed an active policy in terms of technology transfer, resulting in the creation of more than 144 spin-off companies and in the ownership of 834 patents.

The Montefiore Institute is the electricity, electronics and computer science
department of the Faculty of Applied Sciences of the University of Liège.  It
was founded in 1883.  Research in the Software Reliability and Security group of
the Montefiore Institute focuses on symbolic techniques for verification of
systems.  One objective is to study the theoretical properties of symbolic data
structures based on finite-state automata and logical formulas.  Another line of
research, connected to the first, relates to automated reasoning, and more
specifically, the satisfiability checking problem for large logical formulas, in
particular those expressed in a combination of theories.  Its main goal consists
in engineering tools known as Satisfiability Modulo Theories (SMT) solvers,
whose application field spans several areas of computer science, including
verification.  Automated reasoning is strongly linked to this project.

\paragraph*{Main tasks:}

\begin{compactitem}
\item \WPref{atpetc}, task \taskref{atpetc}{instrumenting}: Instrumenting ATPs to produce traces, FOL provers and SMT (Boigelot, Fontaine)
\item  \WPref{atpetc}, task \taskref{atpetc}{tracetodedukti}: Translate ATP traces into Dedukti, FOL provers and SMT (Fontaine)
\item \WPref{atpetc}, task \taskref{atpetc}{deduktitoatp}: Translate Dedukti statements into ATPs inputs, TPTP and SMT-LIB (Fontaine)
\end{compactitem}

\paragraph*{Publications, products or services:}

\begin{compactitem}
\item Haniel Barbosa, Jasmin Christian Blanchette, Pascal Fontaine:
Scalable Fine-Grained Proofs for Formula Processing. CADE 2017: 398-412

\item Clark Barrett, Pascal Fontaine, Cesare Tinelli: The SMT-LIB Standard: Version 2.6 (2017)

\item Bernard Boigelot, Isabelle Mainz:
Efficient Symbolic Representation of Convex Polyhedra in High-Dimensional Spaces. ATVA 2018: 284-299

\item Maria Paola Bonacina, Pascal Fontaine, Christophe Ringeissen, Cesare Tinelli:
Theory Combination: Beyond Equality Sharing. Description Logic, Theory Combination, and All That 2019: 57-89

%\item David Déharbe, Pascal Fontaine, Yoann Guyot, Laurent Voisin:
%Integrating SMT solvers in Rodin. Sci. Comput. Program. 94: 130-143 (2014)

\item Pascal Fontaine, Jean-Yves Marion, Stephan Merz, Leonor Prensa Nieto, Alwen Fernanto Tiu:
  Expressiveness + Automation + Soundness: Towards Combining SMT Solvers and Interactive Proof Assistants. TACAS 2006: 167-181
\end{compactitem}

\paragraph*{Previous projects or activities:}

% SC-SQUARE
Members of the group have expertise in the field of automated theorem proving, notably SMT solving, and automata based symbolic techniques for arithmetic.  They have been part of several national and international projects, including
\begin{compactitem}
\item ANR-DFG SMArT (Programmes blancs 2013): 800k€ French-German project on Satisfiability Modulo Arithmetic Theories (2013-2017).  Pascal Fontaine was leader.
\item H2020-FETOPEN-2015-CSA SC-SQUARE: 350k€ Coordination and Support Action on Satisfiability Checking and Symbolic Computation (2016-2018).  Pascal Fontaine was a principal investigator.
\item European Research Council (ERC) Starting Grant 2016 Matryoshka (Grant
  agreement No. 713999): Fast Interactive Verification through Strong
  Higher-Order Automation (2017-2022).  Pascal Fontaine is a senior
  collaborator.
\end{compactitem}

\paragraph*{Infrastructures or technical equipments:}

% veriT
% LASH
% SMT-LIB

\begin{compactitem}
\item David Déharbe, Pascal Fontaine, Haniel Barbosa.  The SMT solver veriT.
\item Clark Barrett, Pascal Fontaine, Cesare Tinelli. The SMT-LIB language reference and library.
%\item Bernard Boigelot.  LASH
\end{compactitem}

\paragraph*{Persons primarily responsible for carrying out the proposed activities:}

\begin{compactitem}
\item{\bf Bernard Boigelot} is professor at the University of Liège
  since 1999.  His research interests mainly focus on computer-aided
  verification, particularly reachability analysis of infinite-state
  systems, and symbolic data structures and automata-based procedures
  for mixed integer and real arithmetic reasoning.  He has designed
  the LASH toolset for representing infinite sets and exploring
  infinite state spaces. He has been PC member of international
  conferences such as TACAS, ATVA, IJCAR and RP, and workshops such as
  SPIN and INFINITY. He is a regular co-organizer of the annual VTSA
  Summer School on Verification Technology, Systems \& Applications.

\item{\bf Pascal Fontaine} (co-leader of work package \WPref{atpetc}) is a
  professor at the University of Liège since 2019.  He obtained his PhD in 2004
  in Liège and was maître de conférence at the University of Loraine in the
  Inria team VeriDis between 2004 and 2019.  He obtained is habilitation (2019)
  at the University of Lorraine.  His research interests focus on automated
  reasoning, and particularly on satisfiability modulo theories.  He was PC
  member of international conferences such as CADE, FroCoS, IJCAI, IJCAR, SAT
  and Tableaux.  He has been PC chair of the international conferences CADE and
  FroCoS, and the workshops PAAR, SC-square and SMT.  Fontaine was co-founder of
  the PxTP (Proof eXchange for Theorem Proving) series of workshops.  He is a
  member of the steering committees of CADE and SMT.  He was co-organizer of the
  international Summer School on SAT and SMT, in Vienna 2014.  He is one of the
  main developers of the veriT SMT solver, which, among its strong features,
  provides detailed unsatisfiability proofs.  He is one of the three
  coordinators of the SMT-LIB initiative.
\end{compactitem}

\end{sitedescription}
%%% Local Variables: 
%%% mode: latex
%%% TeX-master: "../propB"
%%% End: 

% LocalWords:  site-jacu.tex clange sitedescription emph compactitem pn semmath
% LocalWords:  prosuming-flexiformal KohSuc asemf06 GinJucAnc alsaacl09 StaKoh
% LocalWords:  tlcspx10 KohDavGin psewads11 ednote Radboud Bia ystok CALCULEMUS
% LocalWords:  textbf keypubs OntoLangMathSemWeb uwb Deyan Ginev Stamerjohanns
% LocalWords:  searchability
