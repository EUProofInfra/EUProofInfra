\begin{sitedescription}{Tum}

\logo{TUM}

The 
Technische Universität München
(TUM) is characterized by a unique profile with its core domains natural sciences, engineering, life sciences and medicine. The institutional strategy is focused on strengthening the excellence of disciplinary core competences in research, teaching and learning, but is also targeted towards the promotion of ground-breaking, interdisciplinary research. TUM is committed toward the major challenges facing society in the 21st century in areas such as energy, climate, and environment, natural resources, health and nutrition, communication and information, mobility and infrastructure.
The student body of TUM is currently more than 41 000 students and is constantly rising. TUM is regularly among the best national performers in international rankings. For the fifth time in a row, TUM took the first place among the German universities in the renowned QS World University Ranking (rank 55 worldwide). Looking at the contributions published in the particularly renowned academic journals of the "Nature" Group and the "Science" Group, TUM is positioning itself as number 42 and 1st in Germany. TUM was ranked 6th in the Global University Employability Ranking in which companies worldwide evaluate the quality of university graduates. The World University Ranking has rated the
Technische Universität München
(TUM) as one of the four best technical universities in Europe. TUM placed second in comparison to all other universities in Germany and was ranked number 43 worldwide. 
In 2012 and 2019, TUM has again secured the title ``University of Excellence''.

Research and Training Programmes
Previous Involvement in Research and Training Programmes:
During the last two Framework Programmes for Research and Technological Development of the EC (FP7 and Horizon2020), TUM was and is involved in more than 500 EU research projects and has participated in over 100 ERC grants in total.
Current involvement in Research and Training Programmes: 
Currently, TUM is involved in more than 200 Horizon 2020 projects, for more than 75 of which TUM has a coordinating role. That includes 28 ERC Starting Grants, 28 ERC Consolidator Grants, 14 ERC Advanced Grants, 5 ERC Proof of Concept Grants and 1 ERC Synergy Grant.

\paragraph*{Main tasks:}

\begin{compactitem}
\item Nipkow leads \WPref{libraries} and task 
  \taskref{libraries}{isaAnalysisProb}. 
He heads the Isabelle project at TUM where the Analysis and 
Probability Theory library is being developed and he is one of the 
founding editors of the AFP. 
\item Wenzel is a subcontractor for \WPref{instrumentation}, task
  \taskref{instrumentation}{isabelle} and \WPref{libraries}, task
  \taskref{libraries}{afp}.
  He has been the chief technologist for Isabelle since 2008.
\end{compactitem}

\paragraph*{Publications, products or services:}

\begin{compactitem}
\item ``Isabelle/HOL --- A Proof Assistant for Higher-Order Logic'',
  by T. Nipkow, L. Paulson and M. Wenzel. Springer LNCS 2283, 2002.
\item ``Mining the Archive of Formal Proofs'', by J. Blanchette,
  M. Haslbeck, D. Matichuk and T. Nipkow. Springer LNCS 9150, pp.\ 3-17, 2015.
\item ``Interactive Theorem Proving --- from the perspective of Isabelle/Isar'',
by M. Wenzel. In: \emph{All about Proofs, Proofs for All.} Ed.\ by B. Woltzenlogel Paleo and D. Delahaye. Vol.\ 55. Studies in Logic. College Publications, 2015.
\item ``From LCF to Isabelle/HOL'', by T. Nipkow, L. Paulson and M. Wenzel.
\emph{Formal Aspects of Computing} 31, pp.\ 675-698, 2019.
\end{compactitem}

\paragraph*{Previous projects or activities:}

\begin{compactitem}
\item A string of national and European projects to develop and use the Isabelle
system. Most recently the EUR 1.25 million DFG Koselleck grant Verified Algorithm Analysis.
\end{compactitem}

\paragraph*{Infrastructures or technical equipments:}

\begin{compactitem}
\item The open-source theorem prover
  \href{https://isabelle.in.tum.de}{Isabelle} jointly developed by
  Nipkow, Paulson and Wenzel.
\item The \href{http://www.isa-afp.org}{Archive of Formal Proofs}, an online
open-access collection of proof libraries and larger scientific
developments for the theorem prover Isabelle. It is organized in the
way of a scientific journal.  Nipkow is a co-founder, editor and maintainer.
\end{compactitem}

\paragraph*{Persons primarily responsible for carrying out the proposed activities:}

\begin{compactitem}
\item \textbf{Tobias Nipkow} (leader of work package
\WPref{libraries}) is a full professor for Logic and
Verificatiuon at TUM. He received his Ph.D. in Computer Science
from the University of Manchester in 1987.  He has been a
lecturer at the University of Manchester (1984--1987),
post-doctoral associate at MIT (1988--1989) and at Cambridge
University (1989-1992). He was appointed associate professor for Theory of Programming at
TUM in 1992 and promoted to his current position in 2011. Since 2008
he has been Editor-in-Chief of the  Journal of Automated Reasoning
and is currently serving on the editorial board of Logical Methods in
Computer Science. He founded the steering committee for the
conference Interactive Theorem Proving in 2007 and served as its
chair until 2017.
He has served as a programme committee chair
on a number of conferences in the general area of computational logic.  Nipkow's main research
interests are in computational logic, in particular, the design of
interactive theorem provers (he is one of the designers of the
Isabelle theorem prover), the design and semantics of programming
languages and in partticular the verification of functional and
imperative programmes.

\item \textbf{Makarius Wenzel} participates as a subcontractor. He is an independent provider of Isabelle
  prover technology located in Augsburg (since Sep-2014). Before, he
  has spent 4.5 years (2010--2014) at LRI / Univ.\ Paris-Sud to work on
  the Paral-ITP project, with Pr.~Burkhart Wolff and further
  colleagues from the Coq development team. Earlier, he has worked
  many years at TU München with Prof.\ Tobias Nipkow in the Isabelle
  development team (1994--2010, exluding a few intermediate years). He
  received the title of Dr.~rer.~nat.\ in Feb-2002 from the Institut
  für Informatik, TU München. He served as the chief technologist,
  coordinator and release manager for Isabelle since 2008.
\end{compactitem}

\end{sitedescription}
%%% Local Variables: 
%%% mode: latex
%%% TeX-master: "../propB"
%%% End: 

% LocalWords:  site-jacu.tex clange sitedescription emph compactitem pn semmath
% LocalWords:  prosuming-flexiformal KohSuc asemf06 GinJucAnc alsaacl09 StaKoh
% LocalWords:  tlcspx10 KohDavGin psewads11 ednote Radboud Bia ystok CALCULEMUS
% LocalWords:  textbf keypubs OntoLangMathSemWeb uwb Deyan Ginev Stamerjohanns
% LocalWords:  searchability
