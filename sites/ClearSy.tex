\begin{sitedescription}{Cle}

\logo{ClearSy}

CLEARSY is an SME specialised in the development of safety critical software and systems in the fields
of railways (main focus), microelectronics, information systems, defence, and automotive. Engineering 
activities include:
\begin{compactitem}
\item The realisation of worldwide projects committed to achieving results in the design and/or validation of systems and software.
\item A technical support activity in the fields of formal methods and operational safety..
\end{compactitem}

CLEARSY engineers are skilled in various engineering domains (systems, mechanics, electronics, software,
operational safety) and apply IT tools and an electronic laboratory to create prototypes and
conduct trials. Collaborations with laboratories and industrial partnerships ensure the production of
the various systems components (sensors and interfaces).

CLEARSY developed and has been maintaining for the past 20 years proof tools addressing the logic of B and Event-B. These tools are 
packaged in the Atelier B and Rodin platforms. They are used routinely by large European players in the railways domain to assist the
development of safety-critical software such as automatic train control. CLEARSY also provides these actors technical assistance for the formal development of software and system, including proof-centric activities. 

Specific expertise:
\begin{compactitem}
\item Formal methods for the development of software and systems.
\item Automatic theorem proving.
\item Development of proof rule libraries and their validation.
\end{compactitem}

\paragraph*{Main tasks:}

\begin{compactitem}
\item WP1: task B-method.
\item WP4: task \emph{pp} theorem prover and connections to Zenon, ProB, SMT-Lib.
\item WP8: dissemination to industrial and certification actors.
\end{compactitem}


\paragraph*{Publications, products or services:}

\begin{compactitem}
\item ``Applying a Formal Method in Industry: A 25-Year Trajectory'' by Thierry Lecomte, David Déharbe, Étienne Prun and Erwan Mottin.
SBMF 2017: 70-87.
\item ``Web Service Compensation at Runtime: Formal Modeling and Verification Using the Event-B Refinement and Proof Based Formal Method'', by Guillaume Babin, Yamine Aït Ameur and and Marc Pantel, IEEE Trans. Services Computing, Vol. 10, Num 1, p. 107--120, 2017.
\item ``Teaching an Old Dog New Tricks - The Drudges of the Interactive Prover in Atelier B'' by Lilian Burdy and David Déharbe, ABZ 2018 Proceedings, p. 415-419, 2018.
\item ``Interfacing Automatic Proof Agents in Atelier B: Introducing "iapa"'' Lilian Burdy, David Déharbe and Étienne Prun, F-IDE@FM 2016 Proceedings, p. 82-90, 2016.
\item ``Typechecking in the lambda-Pi-Calculus Modulo : Theory and Practice'', by Ronan Saillard. Mines ParisTech, France, 2015.
\end{compactitem}

\paragraph*{Previous projects or activities:}

CLEARSY has been involved in several collaborative research projects:
\begin{compactitem}
\item EU R\&D projects: Reaims (1994-1995), FMERail (1998-2001), Matisse (2000-2003), Pussee (2001-2004), Rodin (2004-2007), and Deploy (2008-2012).
\item French R\&D projects: Forcoment (2001-2006), Equast (2002-2004), Verbatim (2003-2007), Rimel (2007-2010), Cercles-2 (2011-2014), DEPARTS (2012-2016)and BWare (2012-2015).
\end{compactitem}

These projects are dedicated to the introduction of a formal method (B or Event-B) in the industry and through the development
of dedicated tools and methods are addressing software and electronic based system development.

\paragraph*{Infrastructures or technical equipments:}

\begin{compactitem}
\item CLEARSY maintains and develops Atelier B, an integrated development environment for software development with the B method and system modelling with Event-B.
\item As part of Atelier B, CLEARSY maintains the theorem provers \textbf{pr}, based on conditional rewriting, and \textbf{pp}, based on tableaux calculus.
\item The \textbf{pr} theorem prover is extensible with proof rules, it is being distributed with more than 3000 such rules, targetting primarily the expression language of the B method.
\item Also as part of Atelier B, CLEARSY has developed proof obligation generators for different third-party proof systems, i.e. Why3, SMT-Lib, ProB.
\end{compactitem}

\paragraph*{Persons primarily responsible for carrying out the proposed activities:}

\begin{compactitem}
\item\textbf{David Déharbe} will be the site leader for CLEARSY. He obtained his PhD degree in Computer Science from Université Grenoble Alpes 
(France). He has held a software engineer position at CLEARSY since 2015, following an 18 year long academic career in UFRN (Brazil), 
where he was a key actor in the creation of the graduate studies in Computer Science, and a 2-year visiting research position at CMU 
(USA), where he developed a model checker for VHDL. He has published 40 conference papers and 18 journal papers, and has been involved 
in several national- and international-level research projects. He has been in the programme committees of many scientific events. His 
research interests include formal methods and automatic proof techniques, and their application in industrial contexts. Gender: male.

\item\textbf{Guillaume Babin} is a formal methods engineer at CLEARSY. After obtaining a PhD in Computer Science from Université de 
Toulouse (France), Guillaume joined CLEARSY to apply formal methods to safety-critical software systems in the transportation industry. 
He is interested in tooling, automation and the application of formal methods in industrial systems.

\item\textbf{Lilian Burdy} is an expert in safety critical software. He has been participating to several safety critical software 
development since 1996, mainly in railway domain, but also in smart card domain. He has notably been working for Siemens, Gemplus, 
Alstom, Thales, RATP as employee or sub-contractor, being architect or developing safety critical parts of automatic train controllers, 
side-way equipments, etc. He has participated to several formal tools development, notably AtelierB for Clearsy or Jack for INRIA. He 
has published 12 conference papers and 3 journal papers, and has been involved in several national- and international-level research 
projects.

\item\textbf{Maximilien Colange} holds a PhD in Computer Science from Université Pierre et Marie Curie (France). He followed an 
academic career in Switzerland and France during 5 years, during which he published a dozen conference papers. His research interests
include formal methods, especially model-checking of both discrete and timed systems, and synthesis of reactive programmes. He now holds 
a software engineer position at CLEARSY, with a focus on formal methods tools.

\item\textbf{Thierry Lecomte} is R\&D Project Director. He has been involved in several formal methods oriented, R\&D projects at European 
and French levels. His current subjects of interest include formal methods with proof, safety critical applications, safety computers. 
Gender: male.

\item\textbf{Etienne Prun} is Activity Manager for CLEARSY. He was project manager in several industrial projects in property-driven 
software analysis and property-Driven systems analysis. He has managed several European and French R\&D projects. He has been AtelierB 
development coordinator for 8 years. He was involved in teaching B methods in engineering school and for corporate training. His current 
research interests include safety system, safety software, with use of formal method with proof (automatic or not) in industrial 
context.

\item\textbf{Ronan Saillard} holds a PhD in Computer Science from Mines ParisTech (France) where he worked on both theoretical and praticable 
aspects of the implementation of Dedukti, a typechecker for the $\lambda\Pi$-calculus modulo. He has held a software engineer position at 
CLEARSY since 2015. His research interests include programming languages, formal methods and their application in industrial contexts.
\end{compactitem}

\end{sitedescription}
%%% Local Variables: 
%%% mode: latex
%%% TeX-master: "../propB"
%%% End: 

% LocalWords:  site-jacu.tex clange sitedescription emph compactitem pn semmath
% LocalWords:  prosuming-flexiformal KohSuc asemf06 GinJucAnc alsaacl09 StaKoh
% LocalWords:  tlcspx10 KohDavGin psewads11 ednote Radboud Bia ystok CALCULEMUS
% LocalWords:  textbf keypubs OntoLangMathSemWeb uwb Deyan Ginev Stamerjohanns
% LocalWords:  searchability
