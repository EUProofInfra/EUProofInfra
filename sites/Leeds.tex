\begin{sitedescription}{Lee}

\logo{Leeds}

The University of Leeds (UNIVLEEDS) is acclaimed world-wide for the quality of its teaching and research, and is ranked 93rd in the QS World University Rankings 2019. Leeds is in the top 10 universities in the UK (Times/Sunday Times, 2018). The results of the most recent Research Excellence Framework exercise (REF) identified that 82.76\% 
of its research activity has a top quality rating of either `world leading' or `internationally excellent' which makes it a constant member of the UK's prestigious Russell Group of research intensive universities. 

In 2017/18 it had an annual income of \pounds 715m and its annual research income exceeded \pounds 175m, of which 15.2\% was derived from EU awards. The University includes the School of Mathematics,
which hosts the Logic group, one of the strongest internationally, with expertise across the whole
spectrum of logic and good links with the School of Computing.




\paragraph*{Main tasks:}

\begin{compactitem}
\item Nicola Gambino leads \taskref{alignment}{alignlogic}  and participates in \taskref{theories}{hott}. He is an expert on type theory,
including Homotopy Type Theory, and categorical logic, with experience in computer-assisted proof-checking. 
\item Michael Rathjen participates in \taskref{alignment}{alignlogic}. He is a leading figure at international level
on proof theory. 
\item Paul Shafer participates to  \taskref{alignment}{alignlogic}. He is an expert in reverse mathematics
and computability theory.
\end{compactitem}
The combination of expertise available at Leeds makes the team uniquely placed to develop  \taskref{alignment}{alignlogic},
as the task will require relating type theories, investigating their proof-theoretic properties and analyse the strength of some
statements via reverse mathematics. Dr Gambino's experience in HoTT makes him ideally suited to help in~\taskref{theories}{hott}.


\paragraph*{Publications, products or services:} 

\begin{compactitem}
\item ``Homotopy-initial algebras in type theory'', by S. Awodey, N. Gambino and K. Sojakova, 
{\em Journal of the Association for Computing Machinery}, 63 (6), 2017, 45pp.
\item ``The identity type weak factorisation system'' by N. Gambino and R. Garner, 
{\em Theoretical Computer Science} 409 (1), 2008, pp. 94-109.
\item ``Relativized ordinal analysis: The case of Power Kripke-Platek set theory'', by M. Rathjen, 
{\em Ann. Pure Appl. Logic}, 165(1), 2014, pp.~316-339 
\item ``Constructive Zermelo-Fraenkel Set Theory, Power Set, and the Calculus of Constructions'',
by M. Rathjen, {\em Epistemology versus Ontology}, 2012, pp.~313-349. 
\item ``The reverse mathematics of the Tietze extension theorem'' by P. Shafer,  
{\em Proceedings of the American Mathematical Society}, 144, 2016, pp.~5359-5370.
\end{compactitem}

\paragraph*{Previous projects or activities:}
 
\begin{compactitem}
\item From Mathematical Logic To Applications (MALOA), EU ITN Network (FP7-PEOPLE), October 2009 -- September 2013, Value: EUR 4.3M.
\item EPSRC Standard Grant, ``Homotopy Type Theory: Programming and Verification'', joint project with the University of Nottingham and the University of Strathclyde, March 2015 -- September 2019, Value: GBP 1.2M
\item EPSRC Standard Grant, ``Homotopical inductive types'', May 2013 -- June 2016, Value: GBP 283K.
\end{compactitem}

%\paragraph*{Specific expertise:}
%
%\begin{compactitem}
%\item Type theory,
%\item Proof theory,
%\item Reverse mathematics.
%\end{compactitem}

\paragraph*{Persons primarily responsible for carrying out the proposed activities:}

\begin{compactitem}
\item \textbf{Nicola Gambino} is Associate Professor in Pure Mathematics at the University of Leeds. His publication record includes papers leading journals in both mathematics (e.g. Memoirs of the AMS,  Journal of the LMS) and computer science (e.g. Journal of the ACM, Theoretical Computer Science). He was a plenary invited speaker at the International Conference in Category Theory in 2016 and Logic Colloquium in 2000. His research has been consistently funded by EPSRC and the US Air Force for Scientific Research. He successfully supervised 4 PhD students and 1 PDRA. He serves on the editorial boards of Mathematical Structure in Computer Science and Applied Categorical Structures.
Nicola Gambino's research focuses on type theory, categorical logic and category theory. He is one of the leading experts in Homotopy Type Theory, a subject to which he made fundamental contributions.

\item \textbf{Michael Rathjen} is Professor of Pure Mathematics at the University of Leeds. His publication record includes about 100 papers. He has been an invited speaker at the International Congress of Mathematicians in 2006 and Logic Colloquium (6 times, most recently in 2019), as well as many
other conferences in mathematical logic. His research has been consistently funded by the German Science Foundation, NSF, EPSRC,
Leverhulme Trust and the Templeton Foundation. He successfully supervised 17 PhD students and 5 PDRAs. He serves on the editorial boards of  Notre Dame Journal of Formal Logic, Oxford University Press Logic Guides, and Documenta Mathematica.
\item \textbf{Paul Shafer} is Lecturer in Mathematical Logic at the University of Leeds.  His publication record includes papers some of the top journals in mathematics (e.g., Transactions of the AMS, Proceedings of the AMS, Transactions of the LMS) and mathematical logic (e.g., Journal of Symbolic Logic, Annals of Pure and Applied Logic).  He is frequently invited to speak at major meetings in mathematical logic (e.g., Logic Colloquium, ASL North American Annual Meeting).  He has received prestigious fellowships from the Fondation Sciences Mathématiques de Paris (France) and the Fonds Wetenschappelijk Onderzoek (Belgium) as well as travel and exchange grants from EPSRC and the Royal Society.  He has successfully supervised 1 PhD student.
\end{compactitem}

% \keypubs{providemore}{1}



\end{sitedescription}
%%% Local Variables: 
%%% mode: latex
%%% TeX-master: "../propB"
%%% End: 

% LocalWords:  site-jacu.tex clange sitedescription emph compactitem pn semmath
% LocalWords:  prosuming-flexiformal KohSuc asemf06 GinJucAnc alsaacl09 StaKoh
% LocalWords:  tlcspx10 KohDavGin psewads11 ednote Radboud Bia ystok CALCULEMUS
% LocalWords:  textbf keypubs OntoLangMathSemWeb uwb Deyan Ginev Stamerjohanns
% LocalWords:  searchability
