\begin{sitedescription}{Pra}

\logo{Prague}

  
\paragraph*{Organization:}
\ednote{Give a one-paragraph run-down of the site and the team there. }
 \textbf{Czech Institute of Informatics, Robotics, and Cybernetics
  (CIIRC)} is part of the Czech Technical University (CTU), which was
founded in 1707 and is one of the oldest technical universities in
Europe and currently the major technical university in the Czech
Republic. Offering high quality education and long tradition of
cutting edge science and engineering, CTU counts approx. 1700 members
of academic staff, 18500 students, 8 faculties, and 5 institutes. One
of the youngest of them, the Czech Institute of Informatics, Robotics,
and Cybernetics (CIIRC)\footnote{\url{http://www.ciirc.cvut.cz/?lang=en}} that
will participate in the project, was founded in 2013, starting its
operation in newly built facility in 2017.  Transfer of technology
from academia to industry is an important commitment for CIIRC. It
aims to concentrate an excellent research in the fields of AI,
robotics, automated reasoning, intelligent, distributed and complex
systems, automatic control, computer-aided manufacturing,
bioinformatics, biomedicine and assistive technologies. CIIRC supports
horizontal collaboration among all parts (faculties and institutes) of
CTU and opens the space for mutually beneficial cooperation with other
universities, with the Academy of Sciences of the Czech Republic, with
industrial companies and international institutions.

% CIIRC is a successful holder of national and international
% projects. In terms of success in European grant competitions (FP7,
% Horizon 2020), informatics, robotics and cybernetics teams at CTU
% belong to the strong national average. CIIRC managed to obtain grants
% from EU resources (mainly FP7, ERC, and H2020) worth more than \euro
% 15 M since its creation. Currently, CIIRC is proud to be the
% coordinator of H2020 project, RICAIP no. 857306. CIIRC hosts the H2020
% ERC Consolidator grant AI4REASON no. 649043 receiving funding from the
% European Research Council. The institute has also gained a broad
% industrial experience from providing research, development, and
% training services, and customized solutions to both local and
% international industrial partners, including Airbus, Rockwell
% Automation, Škoda Auto, Volkswagen, FORD, RWE GasNet, Eaton, and
% others. Moreover, CIIRC is the coordinator of the National Centre for
% Industry 4.0 enabling transfer of knowledge to industry and
% vice-versa. Recently, CIIRC became a full member of EIT Manufacturing
% and is also involved in a project of EIT Digital.  The Institute
% coordinates the national initiative on AI (AICZECHIA) and is involved
% in European networks on AI (CLAIRE\footnote{\url{https://claire-ai.org/}}, ELLIS\footnote{\url{https://ellis.eu/}}).

The research will be done within the
Automated Reasoning
Group
% \footnote{\url{http://arg.ciirc.cvut.cz/}} !!! CAREFUL - links are forbidden!!!
(ARG) of the 
Czech Institute of Informatics, Robotics, and Cybernetics (CIIRC)
%\footnote{\url{http://www.ciirc.cvut.cz/?lang=en}} 
at the Czech Technical University (CTU) in Prague. The group carries
out research in Automated Reasoning and its combinations with Machine
Learning and other AI methods. The group is largely responsible for
developing the area of combining Machine Learning and Automated
Reasoning, and for a number of major results in this area.  Its systems
regularly place high in the annual world championship for Automated
Theorem Proving (CASC), and it is the home of the ERC Consolidator
grant AI4REASON, running from September 1 2015 to August 31 2020. The
group currently comprises of three senior researchers, three postdocs
and five PhD students working with a number of ATP systems (E,
Vampire, Prover9, leanCoP, Satallax), ITP systems (Mizar, HOL, Coq,
Isabelle), machine learning systems and also SAT, QBF and SMT
solvers. The group has also close local connections with interested
mathematicians, in particular Dr. Stanovsk\'y from the Department of
Algebra at the Charles University in Prague. 

% A number of international
% collaborations with groups around the world are in place -- about 40
% international visitors were hosted in the past four years. The group has co-established and co-organizes the annual conference on AI and Theorem Proving (AITP) and a number of smaller events.
% The group members have access to large computational resources at CIIRC, comprising in total about 1000 CPUs and about 50
% high-performance GPU cards.  Other research groups at CIIRC (with some of them
% we already collaborate) cover a wide array of AI topics, such as
% pattern recognition and machine learning, computer vision, distributed
% artificial intelligence, mobile robotics, intelligent industrial
% systems, biomedical engineering and informatics, etc.
% The proposed project matches very well the research profile
% of the group. Implementation of the project will strengthen the
% group's position at the top of the AI-based research for 
% Automated Reasoning, and strengthen its international collaborations
% and status.


\paragraph*{Main tasks:}

\begin{compactitem}
\item %\ednote{specify the main tasks and reference the respective work packages} 
\end{compactitem}


\paragraph*{Relevant previous experience:}

% \ednote{give an overview over previous work and projects that add to the \pn project}

\begin{compactitem}
    \item \textbf{AI4REASON} -- Artificial Intelligence for Large-Scale Computer-Assisted Reasoning (ERC - Consolidator Grant 649043, 2015 -- 2020, \url{http://ai4reason.org/}). PI J. Urban.
    \item \textbf{AI\&Reasoning} -- Artificial Intelligence and Reasoning. (Excellent Research Teams within the Operational Programme Research, Development, and Education and Ministry of Education, Youth and Sport of the Czech Republic \begin{math}CZ.02.1.01/0.0/0.0/15_003/0000466\end{math}. PI J. Urban.
    \item \textbf{Knowledge-based Automated Reasoning}, NWO Free Competition, 9/12 - 8/15. EUR 205000, PI J. Urban.
    \item \textbf{POSTMAN} -- Powering SMT Solvers by Machine Learning. ERC CZ grant by the Czech Ministry of Education.  PI Mikolas Janota. (2020 -- 2024).
    \item \textbf{Powering Automatic Theorem Provers by Machine Learning}. Junior grant of the Czech Science Foundation (GACR). PI M. Suda (2020 -- 2022). 
    \end{compactitem}


\paragraph*{Specific expertise:}

\begin{compactitem}
\item Automated theorem proving
\item Combinations of learning and reasoning
\item Formalization of mathematics
  \ednote{give three to five specific areas of expertise that pertain to the \pn project}
\end{compactitem}

\paragraph*{Staff members undertaking the work:}

% \textbf{Dr.\ Great Leader}\ednote{describe the site leader and his expertise}
% \textbf{Joe Implementor}\ednote{and more of them. }
%\ednote{provide the key publications below}
%\keypubs{providemore}

% Martin Suda, Josef Urban

\begin{compactitem}
\item
\textbf{Josef Urban} is a Principal Researcher at the Czech Institute of Informatics, Robotics, and Cybernetics (CIIRC) of the Czech Technical University (CTU) in Prague where he is heading the ERC-funded project
AI4REASON. His interests include Automated Reasoning, Formal Verification and Machine Learning. In particular, he is interested in development of combined inductive/learning and deductive/reasoning
``strong AI'' methods and systems over large formal (fully semantically specified) knowledge bases. Examples are large corpora of formally stated definitions, theorems and proofs in mathematics, software
verification and related fields.  He has made such corpora available to the AI methods, created the first benchmarks, and developed first approaches and systems combining learning and reasoning over such
corpora in various feedback loops.  The systems developed by him and his colleagues have won several competitions and the methods today assist formal verification in proof assistants. He has also
co-developed first learning/reasoning systems for automated formalisation of informal mathematics, and co-founded the conference on Artificial Intelligence and Theorem Proving (AITP).
He received his MSc in Mathematics and PhD in Computers Science from
the Charles University in Prague, worked as an assistant professor in
Prague, and as a researcher at the University of Miami and Radboud
University Nijmegen.

% \item \textbf{Dr. Thibault Gauthier} is a researcher at CIIRC. He is the author
% of the first successful tactical theorem prover using Monte-Carlo search methods -- TacticToe for HOL4~\cite{Gauthier18}.
% He has also implemented strong \emph{hammer-style} methods for HOL4~\cite{hh4h4}. Both these systems have been integrated into the main HOL4 branch and serve HOL4 users in development of proofs in HOL4.
% He is also the author of methods for drawing analogies between heterogeneous proof libraries~\cite{DBLP:journals/jsc/GauthierK19}
% and deep reinforcement learning methods for interactive provers~\cite{DBLP:journals/corr/abs-1910-11797}. Dr. Gauthier received his PhD in Computer Science from the University of Innsbruck.


\item \textbf{Martin Suda}
Dr.~Martin Suda, has worked since 2008 in the field of
Automated Theorem Proving and related fields.  He is an author of a number of research
results in these fields, many of them published in top conferences such
as IJCAR (CORE A*), LPAR (CORE A), CADE (CORE A), SAT (CORE A) and TACAS (CORE A).  His paper
with Benjamin Kiesl \emph{A Unifying Principle for Clause Elimination
  in First-Order Logic}~\cite{DBLP:conf/cade/Kiesl017} won the best
paper award in CADE 2017.
Dr.~Suda has worked with a number of ATP systems, including
SPASS~\cite{WeidenbachDFKSW09}, Vampire and E~\cite{Schulz13}, and has recently implemented the first practically convincing neural machine learning guidance in the saturation-style ATP setting~\cite{abs-1903-03182}. He has
been an active developer of Vampire since 2014, and has been since
then a part of the team that won a number of first places in automated
theorem proving competitions with Vampire.
He has been an active member of the Automated Reasoning and AI
community, being a programme committee member of conferences such as
CADE'19, SAT'19, IJCAI-ECAI'18. He has been a programme co-chair of IWIL'18 and he is a
conference co-chair of CICM'19.
\end{compactitem}


\begin{compactitem}
 
%    \item Thibault Gauthier, Cezary Kaliszyk, Josef Urban, Ramana Kumar, Michael Norrish:
%Learning to Prove with Tactics. CoRR abs/1804.00596 (2018)
\item K. Chvalovsky, J. Jakubuv, M. Suda, J. Urban: ENIGMA-NG: Efficient Neural and Gradient-Boosted Inference Guidance for E. CADE 2019: 197-215
  \item Giles Reger, Martin Riener, Martin Suda:
    Symmetry Avoidance in MACE-Style Finite Model Finding. FroCos 2019: 3-21
% \item 	Jan Jakubuv, Josef Urban:
%   Hammering Mizar by Learning Clause Guidance. ITP 2019: 34:1-34:8
%   \item Mikolas Janota, Martin Suda:
% Towards Smarter MACE-style Model Finders. LPAR 2018: 454-470
% \item Giles Reger, Martin Suda, Andrei Voronkov:
% Unification with Abstraction and Theory Instantiation in Saturation-Based Reasoning. TACAS (1) 2018: 3-22
\item C. Kaliszyk, J. Urban, H. Michalewski, M. Olsak: Reinforcement Learning of Theorem Proving. NeurIPS 2018: 8836-8847
% \item Thibault Gauthier, Cezary Kaliszyk, Josef Urban:
%      TacticToe: Learning to Reason with HOL4 Tactics. LPAR 2017: 125-143
   \item Benjamin Kiesl, Martin Suda:
A Unifying Principle for Clause Elimination in First-Order Logic. CADE 2017: 274-290
    \item J. Harrison, J. Urban, F. Wiedijk: History of Interactive Theorem Proving. Computational Logic 2014: 135-214
    \item J. C. Blanchette, C. Kaliszyk, L. C. Paulson, J. Urban:
      Hammering towards QED. J. Formalized Reasoning 9(1): 101-148 (2016)
%           \item 	Krystof Hoder, Giles Reger, Martin Suda, Andrei Voronkov:
% Selecting the Selection. IJCAR 2016: 313-329
%\item Thibault Gauthier, Cezary Kaliszyk:
%Premise Selection and External Provers for HOL4. CPP 2015: 49-57
%     \item C. Kaliszyk, J. Urban:
% Learning-Assisted Automated Reasoning with Flyspeck. J. Autom. Reasoning 53(2): 173-213 (2014)
% \item Jasmin Christian Blanchette, David Greenaway, Cezary Kaliszyk, Daniel Kühlwein, Josef Urban:
% A Learning-Based Fact Selector for Isabelle/HOL. J. Autom. Reasoning 57(3): 219-244 (2016)
% \item Alexander A. Alemi, François Chollet, Niklas Eén, Geoffrey Irving, Christian Szegedy, Josef Urban:
% DeepMath - Deep Sequence Models for Premise Selection. NIPS 2016: 2235-2243
% \item Cezary Kaliszyk, Josef Urban:
% MizAR 40 for Mizar 40. J. Autom. Reasoning 55(3): 245-256 (2015)
% \item Josef Urban, Piotr Rudnicki, Geoff Sutcliffe:
% ATP and Presentation Service for Mizar Formalizations. J. Autom. Reasoning 50(2): 229-241 (2013)
% \item Josef Urban, Geoff Sutcliffe, Petr Pudlák, Jirí Vyskocil:
% MaLARea SG1- Machine Learner for Automated Reasoning with Semantic Guidance. IJCAR 2008: 441-456
% \item Josef Urban:
% MPTP 0.2: Design, Implementation, and Initial Experiments. J. Autom. Reasoning 37(1-2): 21-43 (2006)
\end{compactitem}



\end{sitedescription}
%%% Local Variables: 
%%% mode: latex
%%% TeX-master: "../propB"
%%% End: 

% LocalWords:  site-jacu.tex clange sitedescription emph compactitem pn semmath
% LocalWords:  prosuming-flexiformal KohSuc asemf06 GinJucAnc alsaacl09 StaKoh
% LocalWords:  tlcspx10 KohDavGin psewads11 ednote Radboud Bia ystok CALCULEMUS
% LocalWords:  textbf keypubs OntoLangMathSemWeb uwb Deyan Ginev Stamerjohanns
% LocalWords:  searchability
