\begin{sitedescription}{Oca}

\ednote{a description of the legal entity}

\logo{OCamlPro}

The software company OCamlPro was created in 2011. They harness their OCaml expertise and formal methods know-how to design, prototype, and build high quality software in demanding projects. Their team of PhD-level engineers also contributes open source development tools for the programming language OCaml, helping to improve the efficiency and usability of the OCaml compiler and tools (the free and open-source OCaml package manager OPAM, the optimizing compiler flambda, the SMT Solver Alt-Ergo, etc.).

\paragraph{Main tasks:}

\ednote{its main tasks, with an explanation of how its profile matches the tasks in the proposal}

\ednote{specify the main tasks and reference the respective work packages}
\begin{compactitem}
\item \taskref{atpetc}{instrumenting}: implement a proof trace output for the SMT solver Alt-Ergo.
\item \taskref{atpetc}{deduktitoatp}: task leader on the translation of dedukti statements into input format for automatic tools.
\item \taskref{access}{opam}: provide access to proofs using the opam package manager.
\end{compactitem}

\paragraph{Publications, products or services:}

\ednote{a list of up to 5 relevant publications, and/or products, services (including widely-used datasets or software), or other achievements relevant to the  call content}

\begin{compactitem}
\item The Alt-Ergo solver\cite{ae2.2} is an SMT solver developped and maintained by OCamlPro.
  It is used behind software verification tools such as Frama-C, SPARK, Why3,
  Atelier-B and Caveat.
\item Guillaume Bury's PhD thesis\cite{BURY19} presented a new automated theorem
  prover, named Archsat, capable of generating formal dedukti proofs. To date,
  Archsat and the tableaux-based theorem prover Zenon are the only two automated
  theorem provers able to produce dedukti proofs.
\item Opam\cite{OPAM} is a source-based package manager developped by OCamlPro,
  which has been successfully used by the OCaml community since 2012, where
  it manages 2585 versioned packages for a total of 13196 combinations of package
  and version, guaranteeing its ability to connect people across large communities.
  Furthermore, opam is meant to provide management capabilities not only to
  OCaml, but to any language, which is why it is already used as a proof manager
  by the Coq community where it has been proven to be reliable and suited to
  managing formal proofs.
\end{compactitem}

\paragraph{Previous projects or activities:}

\ednote{a list of up to 5 relevant previous projects or activities, connected to the subject of this proposal}

\begin{compactitem}
\item French R\&D projects: FUI LCHIP (2017-2020), ANR Vocal (2015-2020), ANR BWare (2013-2016), FUI HILITE (2010-2013)
\end{compactitem}

\paragraph{Infrastructures or technical equipments:}

\ednote{a description of any significant infrastructure and/or any major items of technical equipment, relevant to the proposed work}

\paragraph{Persons primarily responsible for carrying out the proposed activities:}

\begin{itemize} % in alphabetical order
  \item{\bf Dr.\ Raja Boujbel} Raja holds a PhD in software deployment and
    multi-agent systems from University of Toulouse. Previously, she had studied
    functional programming and compiler design at Université Pierre et Marie
    Curie, then worked on the Opa language among MLstate’s distribution team.
    She joined OCamlPro in March 2018 as a lead maintainer for opam, an
    open-source package manager for OCaml.
\item{\bf Dr.\ Guillaume Bury} Guillaume holds a research Master in computer
    science from Ecole Normale Supérieure in Paris, France, and has studied the
    integration of rewriting techniques inside SMT solvers during his PhD
    obtained under the direction of Gilles Dowek and David Delahaye in
    Deducteam at ENS Cachan. He joined OCamlPro in October 2018 and works in
    the Flambda team, on optimizations passes for the OCaml compiler.
\item{\bf Dr.\ Albin Coquereau} Albin has a PhD in computer science,
    which he obtained for his work on improving the performance of the SMT
    solver Alt-Ergo. He also helped adding a support for the SMT-LIB standard
    in Alt-Ergo allowing it to participate to the SMTCOMP 2018.
\item{\bf Dr.\ Mattias Roux} Mattias Roux holds a PhD in computer science
    for his work on the model checker Cubicle, with an extension of the backward
    reachability algorithm. He now works at OCamlPro on the Alt-ergo theorem
    prover.
\end{itemize}

\end{sitedescription}

%%% Local Variables: 
%%% mode: latex
%%% TeX-master: "../propB"
%%% End: 
