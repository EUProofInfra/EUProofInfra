\begin{sitedescription}{Oca}

\ednote{a description of the legal entity}

\logo{OCamlPro}

The software company OCamlPro was created in 2011. They harness their OCaml expertise and formal methods know-how to design, prototype, and build high quality software in demanding projects. Their team of PhD-level engineers also contributes open source development tools for the programming language OCaml, helping to improve the efficiency and usability of the OCaml compiler and tools (the free and open-source OCaml package manager OPAM, the optimizing compiler flambda, the SMT Solver Alt-Ergo, etc.).

\paragraph{Main tasks:}

\ednote{its main tasks, with an explanation of how its profile matches the tasks in the proposal}

\ednote{specify the main tasks and reference the respective work packages}
\begin{compactitem}
\item \taskref{atpetc}{instrumenting} and \taskref{atpetc}{tracetodedukti}: implement a Dedukti output for the SMT solver Alt-Ergo.
\item \taskref{access}{opam}: provide access to proofs using the opam package manager.
\end{compactitem}

\paragraph{Publications, products or services:}

\ednote{a list of up to 5 relevant publications, and/or products, services (including widely-used datasets or software), or other achievements relevant to the  call content}

\begin{compactitem}
\item
\end{compactitem}

\paragraph{Previous projects or activities:}

\ednote{a list of up to 5 relevant previous projects or activities, connected to the subject of this proposal}

\begin{compactitem}
\item
\end{compactitem}

\paragraph{Infrastructures or technical equipments:}

\ednote{a description of any significant infrastructure and/or any major items of technical equipment, relevant to the proposed work}

\paragraph{Persons primarily responsible for carrying out the proposed activities:}

\begin{itemize} % in alphabetical order
\item{\bf Dr.\ Guillaume Bury}

\end{itemize}

\end{sitedescription}

%%% Local Variables: 
%%% mode: latex
%%% TeX-master: "../propB"
%%% End: 
