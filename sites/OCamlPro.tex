\begin{sitedescription}{Oca}

\logo{OCamlPro}

The software company OCamlPro was created in 2011. They harness their OCaml expertise and formal methods know-how to design, prototype, and build high quality software in demanding projects. Their team of PhD-level engineers also contributes open source development tools for the programming language OCaml, helping to improve the efficiency and usability of the OCaml compiler and tools (the free and open-source OCaml package manager OPAM, the optimising compiler flambda, the SMT Solver Alt-Ergo, etc.).

\paragraph*{Main tasks:}

\begin{compactitem}
\item \taskref{atpetc}{instrumenting}: Implement a proof trace output for
  the SMT solver Alt-Ergo. As the maintainers and main developers of Alt-Ergo,
  OCamlPro is uniquely competent in modifying the source code of Alt-Ergo
  for such a purpose. Additionally, Guillaume Bury already has experience
  generating formal proofs from an SMT solver as described in his PhD
  thesis\cite{BURY19}.
\item \taskref{atpetc}{deduktitoatp}: Task leader on the translation of
  Dedukti statements into input format for automatic tools. As the developer
  of the \href{https://github.com/Gbury/dolmen}{Dolmen} library for parsing input format for automatic
  tools, Guillaume Bury already has experience manipulating such formats,
  and is thus suited to leading this task aimed at translating Dedukti
  statements into such formats.
\item \taskref{access}{opam}: provide access to proofs using the open source opam
  package manager, now used as the official package manager of the OCaml community. 
  As the author, developer and maintainer of opam,
  OCamlPro is again uniquely suited to leading this task given its
  unparalleled expertise on the opam package manager.
\end{compactitem}

\paragraph*{Publications, products or services:}

\begin{compactitem}
\item Guillaume Bury's PhD thesis presented a new automated theorem
  prover, named Archsat, capable of generating formal Dedukti proofs. To date,
  Archsat and the tableaux-based theorem prover Zenon are the only two automated
  theorem provers able to produce Dedukti proofs.
\item Dolmen is an OCaml library developed by Guillaume Bury,
  that deals with parsing and type-checking most input languages used in
  the automated theorem prover community.
\item \href{https://opam.ocaml.org/}{Opam} is a source-based package manager developed by OCamlPro,
  which has been successfully used by the OCaml community since 2012, where
  it manages 2585 versioned packages for a total of 13196 combinations of package
  and version, guaranteeing its ability to connect people across large communities.
  Furthermore, opam is meant to provide management capabilities not only to
  OCaml, but to any language, which is why it is already used as a proof manager
  by the Coq community where it has been proven to be reliable and suited to
  managing formal proofs.
\end{compactitem}

\paragraph*{Previous projects or activities:}

French R\&D projects:
\begin{compactitem}
\item FUI LCHIP (2017-2020)
\item ANR Vocal (2015-2020)
\item ANR BWare (2013-2016)
\item FUI HILITE (2010-2013)
\end{compactitem}

\paragraph*{Infrastructures or technical equipments:}

\begin{compactitem}
\item The Alt-Ergo solver\cite{ae2.2} is an SMT solver developed and
  maintained by OCamlPro. It is used behind software verification tools
  such as Frama-C, SPARK, Why3, Atelier-B and Caveat.
\end{compactitem}

\paragraph*{Persons primarily responsible for carrying out the proposed activities:}

\begin{compactitem} % in alphabetical order
  \item{\bf Raja Boujbel} Raja holds a PhD in software deployment and
    multi-agent systems from University of Toulouse. Previously, she had studied
    functional programming and compiler design at Université Pierre et Marie
    Curie, then worked on the Opa language among MLstate’s distribution team.
    She joined OCamlPro in March 2018 as a lead maintainer for opam, an
    open-source package manager for OCaml.
\item{\bf Guillaume Bury} Guillaume holds a research Master in computer
    science from Ecole Normale Supérieure in Paris, France, and has studied the
    integration of rewriting techniques inside SMT solvers during his PhD
    obtained under the direction of Gilles Dowek and David Delahaye in
    Deducteam at ENS Cachan. He joined OCamlPro in October 2018 and works in
    the Flambda team, on optimisations passes for the OCaml compiler.
\item{\bf Sylvain Conchon} Sylvain is a Professor of Computer
    Science at Université Paris-Saclay and currently on a sabbatical
    leave at OCamlPro. His research interests are at the crossroads of
    Model Checking, SMT solving, functional programming, and
    compilation techniques. He currently focuses on the design and
    development of the SMT solver Alt-Ergo and the SMT-based model
    checker Cubicle.
\item{\bf Albin Coquereau} Albin has a PhD in computer science,
    which he obtained for his work on improving the performance of the SMT
    solver Alt-Ergo. He also helped adding a support for the SMT-LIB standard
    in Alt-Ergo allowing it to participate to the SMTCOMP 2018.
\item{\bf Mattias Roux} Mattias Roux holds a PhD in computer science
    for his work on the model checker Cubicle, with an extension of the backward
    reachability algorithm. He now works at OCamlPro on the Alt-ergo theorem
    prover.
\end{compactitem}

\end{sitedescription}

%%% Local Variables: 
%%% mode: latex
%%% TeX-master: "../propB"
%%% End: 
