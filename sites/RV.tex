\begin{sitedescription}{Run}

\ednote{a description of the legal entity}

\logo{RV}

Runtime Verification Inc. is a startup company aimed at using runtime verification-based
techniques to improve the safety, reliability, and correctness of software systems.

Runtime Verification SRL is a research and development branch of RV Inc. based
in Bucharest Romania.  Its main activities include the design and development of
the symbolic execution and program verification backend of the K framework,
as well as offering consultance with modeling and proving properties about
consensus protocols (in the presence of Byzantine faults),
with a focus of those used in Blockchains.

\paragraph*{Main tasks:}

\ednote{its main tasks, with an explanation of how its profile matches the tasks in the proposal}

RV will lead the task of instrumenting the K Prover (WP1) to emit enough information
about the proving process, in order to allow extracting matching logic (ML) proofs from
program verification problems.  Additionally, RV will assist the UAIC team with
translating Kore (the language of ML) into Dedukti.
These will be achieved through the following activities:

\ednote{specify the main tasks and reference the respective work packages}

\begin{compactitem}
\item Generating proof traces for the K Prover
\item Translating Kore into Dedukti 
\end{compactitem}

\paragraph*{Publications, products or services:}

\ednote{a list of up to 5 relevant publications, and/or products, services (including widely-used datasets or software), or other achievements relevant to the  call content}

\begin{compactitem}
\item K Framework Tools \url{https://github.com/kframework/k} is an open source
 rewrite-based executable semantic framework in which programming languages,
 type systems and formal analysis tools can be defined using configurations,
 computations and rules. Originally designed and prototyped by Grigore Roșu and 
 Traian Șerbănuță, it is currently maintained by a team spread between teh US and 
 Romanian branches of RV. 
\item Kore \url{https://github.com/kframework/kore} provides a language for
expressing matching logic, as well as a rewrting-based engine for perfoming 
matching-logic deduction.  Kore currently serves as a symbolic execution and 
program verification backend for the K framework. Most of the team developing Kore
is part of RV Romania.
\item Kasampalis, T., Guth, D., Moore, B., Șerbănuță, T. F., Zhang, Y., Filaretti, D., ... \& Roşu, G. (2019, October). IELE: A rigorously designed language and tool ecosystem for the blockchain. In International Symposium on Formal Methods (pp. 593-610). Springer, Cham.
\item Rosu, G., Serbanuta, T. F., Moore, B., Mereuta, R., Ciobaca, S., \& Stefanescu, A. (2019). All-Path Reachability Logic. Logical Methods in Computer Science, 15.
\item Chen, X., \& Roşu, G. (2019, June). Matching $\mu$-Logic. In 2019 34th Annual ACM/IEEE Symposium on Logic in Computer Science (LICS) (pp. 1-13). IEEE.
\end{compactitem}

%\paragraph*{Previous projects or activities:}

%\ednote{a list of up to 5 relevant previous projects or activities, connected to the subject of this proposal}

%\begin{compactitem}
%\item 
%\end{compactitem}

\paragraph*{Infrastructures or technical equipments:}

\ednote{a description of any significant infrastructure and/or any major items of technical equipment, relevant to the proposed work}
RV Romania already possesses the infrastructure required for carrying out the tasks proposed.

\paragraph*{Persons primarily responsible for carrying out the proposed activities:}

\begin{itemize} % in alphabetical order

\item{\bf Andrei Burdușa} is a software engineer at RV Romania.
His research interests lay in functional programming, type theory and logic.
His Bachelor's degree project at the University of Bucharest was focused on Haskell.
He is currently pursuing two Master’s degrees: one in Security and Applied Logic,
and another one in Analytic Philosophy, both from the University of Bucharest.

\item{\bf Ana Pantilie} is a software engineer at RV Romania, involved with the
research and development of the symbolic execution and program verification
backend of the K framework.
She enjoys learning about functional programming languages and formal methods.
In addition to her duties at RV, Ana is pursuing a Master's degree in Software
Engineering at the University of Bucharest.
She completed her Bachelor's degree project in Clojure, developing a music
composition application.

\item{\bf Grigore Roșu} is a full professor of computer science at the
University of Illinois and the founder of Runtime Verification, Inc., and of 
Runtime Verification Romania. He is interested in programming languages,
formal methods and software engineering, and especially in how to combine
these to increase the safety, security and dependability of computing systems.
He was offered the NSF CAREER award, the UIUC outstanding junior award,
the Dean's award for excellence in research, and several best paper awards.
Grigore got his Ph.D. from the University of California at San Diego.

\item{\bf Traian Florin Șerbănuță} is an associate professor of computer science
at the University of Bucharest and a research consultant for RV, where he serves
as the K Technical Lead.
Traian completed his Ph.D. at UIUC, working with Grigore Rosu
on the first prototype of K, which serves as a basis for the semantics-based execution
and semantics-based program verification tools developed by RV.
Additionally, Traian designed a maximal causal model for sequential consistency
which serves as a basis for runtime verification of concurrent programs in tools
such as RV-Predict.

\end{itemize}

\end{sitedescription}

%%% Local Variables:
%%% mode: latex
%%% TeX-master: "../propB"
%%% End:
