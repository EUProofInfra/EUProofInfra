\begin{sitedescription}{Run}

%\ednote{a description of the legal entity}

\logo{RV}

Runtime Verification Inc. is a startup company aimed at using runtime verification-based
techniques to improve the safety, reliability, and correctness of software systems.

Runtime Verification SRL is a research and development branch of RV Inc. based
in Bucharest Romania.  Its main activities include the design and development of
the symbolic execution and program verification backend of the K framework,
as well as offering consultance with modeling and proving properties about
consensus protocols (in the presence of Byzantine faults),
with a focus on those used in Blockchains.

\paragraph*{Main tasks:}

%\ednote{its main tasks, with an explanation of how its profile matches the tasks in the proposal}

%\ednote{specify the main tasks and reference the respective work packages}

\begin{compactitem}
\item \taskref{theories}{matching}: RV SRL will lead the task of
  instrumenting the K Prover to emit enough information about the
  proving process, in order to allow extracting matching logic (ML)
  proofs from program verification problems
\item \taskref{theories}{matching}: RV SRL will assist the UAIC team
  with translating Kore (the language of ML) into Dedukti
\end{compactitem}

\paragraph*{Publications, products or services:}

%\ednote{a list of up to 5 relevant publications, and/or products, services (including widely-used datasets or software), or other achievements relevant to the  call content}

\begin{compactitem}
\item ``A rigorously designed language and tool ecosystem for the blockchain'', by Kasampalis, T., Guth, D., Moore, B., Șerbănuță, T. F., Zhang, Y., Filaretti, D., \ldots \& Roşu, G. (2019, October). In International Symposium on Formal Methods (pp. 593-610). Springer, Cham.
\item ``All-Path Reachability Logic'', by Rosu, G., Serbanuta, T. F., Moore, B., Mereuta, R., Ciobaca, S., \& Stefanescu, A. (2019). Logical Methods in Computer Science, 15.
\item ``Matching $\mu$-Logic'', by Chen, X., \& Roşu, G. (2019, June). In 2019 34th Annual ACM/IEEE Symposium on Logic in Computer Science (LICS) (pp. 1-13).
\item ``From Hybrid Modal Logic to Matching Logic and back'', by Leustean, I., Moanga, N., \& Serbanuta, T. F. (2019). In Working Formal Methods Symposium (FROM), EPTCS 303, 16-31. arXiv:1907.05029.
\item ``An overview of the K semantic framework'', by Roșu, G., \& Șerbănută, T. F. (2010).  Journal of Logic and Algebraic Programming, 79(6), 397-434.
\end{compactitem}

\paragraph*{Previous projects or activities:}

%\ednote{a list of up to 5 relevant previous projects or activities, connected to the subject of this proposal}

RV SRL is young company and therefore it was not yet involved in EU/national funded projects.
However, besides contributing to the developement of the infrastructure described below,
here are some activities in which RV SRL played an important role:
\begin{compactitem}
\item The \href{https://github.com/runtimeverification/casper-cbc-proofs}{formal modelling}
(using Coq) of the full and light nodes of Casper-CBC, as well as the partial
modelling of a new VLSM model proposed by Vlad Zamfir, as part of a
project funded by Casper Labs.
\item The design and development of a gas model for
\href{https://github.com/runtimeverification/iele-semantics}{IELE}, a new
virtual machine language for the blockchain aimed at improving security and
verification-readiness. IELE was developed as part of a project funded
by Input Output Hong Kong.
\end{compactitem}

\paragraph*{Infrastructures or technical equipments:}

%\ednote{a description of any significant infrastructure and/or any major items of technical equipment, relevant to the proposed work}

\begin{compactitem}
\item \href{https://github.com/kframework/k}{K Framework} is an open source
 rewrite-based executable semantic framework in which programming languages,
 type systems and formal analysis tools can be defined using configurations,
 computations and rules. Originally designed and prototyped by Grigore Roșu and 
 Traian Șerbănuță, it is currently maintained by a team spread between the US and 
 Romanian branches of RV. 
\item \href{https://github.com/kframework/kore}{Kore} provides a language for
expressing matching logic, as well as a rewrting-based engine for perfoming 
matching-logic deduction.  Kore currently serves as a symbolic execution and 
program verification backend for the K framework. Most of the team developing Kore
is part of RV SRL.
\end{compactitem}

\paragraph*{Persons primarily responsible for carrying out the proposed activities:}

\begin{compactitem} % in alphabetical order

\item{\bf Ana Pantilie} is a software engineer at RV SRL, involved with the
research and development of the symbolic execution and program verification
backend of the K framework.
She enjoys learning about functional programming languages and formal methods.
In addition to her duties at RV, Ana is pursuing a Master's degree in Software
Engineering at the University of Bucharest.
She completed her Bachelor's degree project in Clojure, developing a music
composition application.

\item{\bf Grigore Roșu} is a full professor of computer science at the
University of Illinois and the founder of Runtime Verification, Inc., and of 
Runtime Verification SRL. He is interested in programming languages,
formal methods and software engineering, and especially in how to combine
these to increase the safety, security and dependability of computing systems.
He was offered the NSF CAREER award, the UIUC outstanding junior award,
the Dean's award for excellence in research, and several best paper awards.
Grigore got his Ph.D. from the University of California at San Diego.

\item{\bf Traian Florin Șerbănuță} is an associate professor of computer science
at the University of Bucharest and a research consultant for RV, where he serves
as the K Technical Lead.
Traian completed his Ph.D. at UIUC, working with Grigore Rosu
on the first prototype of K, which serves as a basis for the semantics-based execution
and semantics-based program verification tools developed by RV.
Additionally, Traian designed a maximal causal model for sequential consistency
which serves as a basis for runtime verification of concurrent programs in tools
such as RV-Predict.

\end{compactitem}

\end{sitedescription}

%%% Local Variables:
%%% mode: latex
%%% TeX-master: "../propB"
%%% End:
