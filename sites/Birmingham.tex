\begin{sitedescription}{Bir}

\logo{Birmingham}

% \paragraph*{Organization:}
The \textbf{University of Birmingham} (UoB) is one of the leading research-based Universities in the United Kingdom. A distinctive
characteristic of the University is the wide breadth of research expertise. The last UK research assessment in 2014 confirmed that
87\% of the University’s research has global reach, meaning it is recognized internationally in terms of its originality, significance and rigor. The University is 79th in the 2019 QS World University Rankings, cementing its position in the top 100 universities globally and placing it 14th out of the 24 Russell Group universities to feature in the ranking. The School of Computer Science (CS) at UoB is world leading for its research. 
The School of Computer Science is home to one of world's leading theoretical computer science groups.

\paragraph*{Main tasks:}

\begin{compactitem}
\item Contribute to the import of the UniMath library of univalent mathematics into Dedukti, and to the specification of 2LTT and Cubical Type Theory in Dedukti, cf.\ \taskref{theories}{hott}.
\item Training on Logipedia for students, researchers, and teachers, cf.\ \taskref{dissemination}{training} and \taskref{dissemination}{researchers-club}

\end{compactitem}

\paragraph*{Publications, products or services:}

\begin{compactitem}
 \item ``UniMath --- a computer-checked library of univalent mathematics'', by V.~Voevodsky, B.~Ahrens, D.~Grayson, and others, available from \url{https://github.com/UniMath/UniMath}
 \item ``Univalent categories and the Rezk completion'', by B.~Ahrens, K.~Kapulkin, and M.~Shulman, Mathematical Structures in Computer Science 25 (2015), pp.~1010--1039
 \item Organization of the the School and Workshop on Univalent Mathematics, previously in 2017 and 2019, next planned for 2020
 \item Coorganization of the Midlands Graduate School in the Foundations of Computing,  yearly one-week school for participants from academia and industry, 1999-present
\end{compactitem}

\paragraph*{Previous projects or activities:}

\begin{compactitem}
 \item EPSRC New Investigator Award ``A theory of type theories'', PI Benedikt Ahrens, ongoing
\end{compactitem}

\paragraph*{Infrastructures or technical equipments:}
\begin{compactitem}
 \item UniMath --- a computer-checked library of univalent mathematics
 \item Introduction to Univalent Foundations of Mathematics with Agda: textbook with accompanying computer-checked proofs in Agda, by Martín Hötzel Escardó
\end{compactitem}


% \paragraph*{Relevant previous experience:}
% The Theory group at Birmingham features several internationally recognized experts on (homotopy) type theory, its semantics, and its use in verification of mathematics and computer systems.
% Members of the group have developed large libraries of computer-checked results using different computer proof assistants based on type theory.
% Many of their publications are accompanied with computer-checked proofs of the results.
% The group regularly organizes schools on topics in theoretical computer science: for many years, the Midlands Graduate School has been co-organized by the group; Ahrens has organized two one-week schools on the topic of univalent mathematics.
% 
% \ednote{give an overview over previous work and projects that add to the \pn project}

% \paragraph*{Specific expertise:}

% \begin{compactitem}
% \item Development of mathematics in univalent foundations
% \item Building and maintaining large libraries of computer-checked results
% \item Syntax of type theories
% \item Organization of schools on the topic of type theory
% \item \ednote{give three to five specific areas of expertise that pertain to the \pn project}
% \end{compactitem}

% \paragraph*{Staff members undertaking the work:}

\paragraph*{Persons primarily responsible for carrying out the proposed activities:}

\begin{compactitem}
 \item 
\textbf{Benedikt Ahrens} is an expert in the development of mathematics in univalent foundations, and has written several works on category theory in univalent mathematics.
In another strand of work, he develops syntax and categorical semantics for programming languages with features such as typing and operational semantics.
He has recently received an EPSRC New Investigator Award for the development of a theory of type theories.
Ahrens has designed and organized two schools on univalent foundations, with 60 participants and 10 mentors each. He has furthermore chaired the Workshop on Homotopy Type Theory and Univalent Foundations (HoTT/UF) in 2017, 2018, and 2020. He is currently a guest editor for Mathematical Structures in Computer Science, editing a special volume on HoTT/UF.
\end{compactitem}



% \ednote{describe the site leader and his expertise}

% \ednote{provide the key publications below}
% \keypubs{UniMath, rezk_completion}


\end{sitedescription}
%%% Local Variables: 
%%% mode: latex
%%% TeX-master: "../propB"
%%% End: 

% LocalWords:  site-jacu.tex clange sitedescription emph compactitem pn semmath
% LocalWords:  prosuming-flexiformal KohSuc asemf06 GinJucAnc alsaacl09 StaKoh
% LocalWords:  tlcspx10 KohDavGin psewads11 ednote Radboud Bia ystok CALCULEMUS
% LocalWords:  textbf keypubs OntoLangMathSemWeb uwb Deyan Ginev Stamerjohanns
% LocalWords:  searchability
