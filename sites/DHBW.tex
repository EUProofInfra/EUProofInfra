\begin{sitedescription}{Stu}

\logo{DHBW}

Baden-Wuerttemberg Cooperative State University (\emph{Duale
  Hochschule Baden-Württemberg/DHBW}) was founded in 2009, combining
nine previously independent cooperative teaching academies. The
university's official seat is in Stuttgart. Based on the US State
University System, the organizational structure of DHBW is unique in
Germany, comprising a central level and a local level. The central
level consists of the DHBW Headquarter and the Center for Advaced
Studies. At the local level are nine DHBW academies with three
additional subordinate campuses. Through its acedemies, the university
offers a broad range of undergraduate study programmes in the field of
business, engineering, and social work. Graduate courses are provided
centrally via the Center for Advances Studies.

% DHBW is the first institution of higher education in Germany to
% combine on-the-job training and academic studies, thus achieving a
% close integration of theoretical academic work and practical work
% place experience. All degree programmes are both nationally and
% internationally accredited, count as intensive study programmes and
% are awaeded 210 ECTS credits. In addition, DHBW offers postgraduate
% degree programmes with integrated on-the-job training through its
% Center for Advanced Studies.

With around 34,000 enrolled students, over 9,000 partner companies and
more than 145,000 graduates, DHBW is one of the largest institutions
of higher education in the German Federal State of Baden-Wuerttemberg,
and by number of yearly graduates, one of the largest in Germany.
Since its 2009 integration as a \emph{Hochschule} proper, DHBW has
been building its research portfolio, in particular performing
cooperative reseach with its partner companies.

% One of the research projects at DHBW Stuttgart is the development of
% the equational theorem prover E, one of the leading theorem provers
% for first-order logic.

\paragraph*{Main tasks:}

\begin{compactitem}
\item Efficient production of detailed proofs for first-order
  reasoners (\WPref{atpetc},\taskref{atpetc}{instrumenting})
\item Improved automation for logical tools (\WPref{atpetc})
\item Dissemination and exploitation (\WPref{dissemination})
\end{compactitem}

\paragraph*{Publications, products or services:}

%\keypubs{SCV:CADE-2019,SS:APPA-2015,SSCB:LPAR-2012,SSCG:IJCAR-2006,VBCS:TACAS-2019}

\begin{compactitem}
\item “Faster, Higher, Stronger: E 2.3”, by Schulz, S., S. Cruanes, and P. Vukmirović. In: Proc. of the 27th CADE,
Natal, Brasil. Ed. by P. Fontaine. LNAI 11716. Springer, 2019, pp. 495–507.
\item “Proof Generation for Saturating First-Order Theorem Provers”, by Schulz, S. and G. Sutcliffe. In: All about
Proofs, Proofs for All. Ed. by D. Delahaye and B. Woltzenlogel Paleo. Vol. 55. Mathematical Logic
and Foundations. London, UK: College Publications, Jan. 2015, pp. 45–61. isbn: 978-1-84890-166-7.
\item “The TPTP Typed First-order Form with
Arithmetic”, by Sutcliffe, G., S. Schulz, K. Claessen, and P. Baumgartner. In: Proc. of the 18th LPAR, Merida. Ed. by N. Bjørner and A. Voronkov. Vol. 7180.
LNAI. Springer, 2012, pp. 406–419.
\item “Using the TPTP Language for Writing Derivations
and Finite Interpretations”, by Sutcliffe, G., S. Schulz, K. Claessen, and A. V. Gelder. In: Proc. of the 3rd IJCAR, Seattle. Ed. by U. Fuhrbach and N. Shankar.
Vol. 4130. LNAI. Springer, 2006, pp. 67–81.
\item “Extending a brainiac prover to lambda-free
higher-order logic”, by Vukmirović, P., J. C. Blanchette, S. Cruanes, and S. Schulz. In: Proc. 25th Conference on Tools and Algorithms for the Construction and
Analysis of Systems (TACAS’19). Ed. by T. Vojnar and L. Zhang. LNCS 11427. Springer, 2019,
pp. 192–210.
\end{compactitem}

\paragraph*{Previous projects or activities:}

\begin{compactitem}
\item Stephan Schulz is the creator and a core developer of the theorem
prover E.  He has participated in a number of research projects at TU
Munich. He currently is a senior collaborator in the project
\emph{Matryoshka} (Principal investigator: Jasmin Blanchette, Vrije
Universiteit Amsterdam, ERC Starting Grant 2016 Grant agreement
No. 713999), which is working towards reducing the gap between
efficient automated reasoning systems and more expressive interactive
proof assistants.
\end{compactitem}

\paragraph*{Infrastructures or technical equipment:}

\begin{compactitem}
\item Schulz, Stephan (et al.): The high-performance automated theorem
  prover \textbf{E}, \url{https://www.eprover.org}
\item DHBW provides reasonable computig resources for medium-scale
  experiments.
\end{compactitem}


% \paragraph*{Specific expertise:}

% \begin{compactitem}
% \item High-performance automated theorem proving
% \item Low-overhead proof extraction from efficient reasoning processes
% \item Design and implementation of languages for proof specification
%   and proof object representation
% \end{compactitem}

\paragraph*{Persons primarily responsible for carrying out the proposed activities:}

\begin{compactitem}
\item \textbf{Stephan Schulz} is a tenured Professor at DHBW, with a
  focus on foundations of computer science. He received his
  Ph.D. (\emph{Dr. rer.nat}) in Computer Science from the Technical
  University of Munich in 2000, and has taught at the University of
  Miami, the University of the West Indies, and the University of
  Hildesheim. From 2005 to 2014 he worked as technical project leader
  and project manager in the field of air traffic control systems. He
  joined DHBW in Mai 2014. Dr.\ Schulz has co-created the biannual
  \emph{Workshop on Practical Aspects of Automated Reasoning} in 2008
  and the yearly \emph{Conference on Artificial Intelligence and
    Theorem Proving} in 2016. He has served in a number of
  organisational roles, including as programme chair of the
  \emph{International Joint Conference on Automated Reasoning} in 2018
  and a trustee of CADE Inc.\ (ex-officio since 2017, elected
  2018). His research contributions include work in high-performance
  deduction systems, AI-based search heuristics, and practical
  languages for proof problems and proof objects.
% \textbf{Dr.\ Great Leader}\ednote{describe the site leader and his expertise}
% \textbf{Joe Implementor}\ednote{and more of them. }
% \ednote{provide the key publications below}
\end{compactitem}

\end{sitedescription}
%%% Local Variables:
%%% mode: latex
%%% TeX-master: "../propB"
%%% End:

% LocalWords:  site-jacu.tex clange sitedescription emph compactitem pn semmath
% LocalWords:  prosuming-flexiformal KohSuc asemf06 GinJucAnc alsaacl09 StaKoh
% LocalWords:  tlcspx10 KohDavGin psewads11 ednote Radboud Bia ystok CALCULEMUS
% LocalWords:  textbf keypubs OntoLangMathSemWeb uwb Deyan Ginev Stamerjohanns
% LocalWords:  searchability
