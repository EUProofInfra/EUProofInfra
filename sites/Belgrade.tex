\begin{sitedescription}{Bel}

  \logo{Belgrade}
  
  Founded in 1808, the University of Belgrade is the oldest and the
  largest higher educational institution in Serbia and one of the
  leading educational institutions in Central and Eastern Europe. The
  Faculty of Mathematics has a history of almost 150 years, with a
  number of well recognised researchers during this period. The
  Department of Computer Science at the Faculty of Mathematics builds
  upon the tradition going back to 1961. Today, the Department has
  around 50 academic staff, working in a range of computer science
  subfields, primarily artificial intelligence, automated reasoning,
  data mining, machine learning, optimisation, etc.

  % \url{http://argo.matf.bg.ac.rs}
  Automated Reasoning Group (ARGO) at
  Faculty of Mathematics, University of Belgrade is interested in
  automated reasoning, especially in SAT and SMT (satisfiability
  modulo theories), interactive theorem proving, automated theorem
  proving in coherent logic, automated reasoning in geometry, software
  verfication and other applications of automated and interactive
  theorem proving. This research group is internationally visible and
  well recognized: over the last 10 years its eight members published
  around 50 articles at leading international conferences and in
  leading scientific journals, its members served on programme
  committees of several conferences, gave invited lectures, hosted
  more than 60 distinguished researchers from 20 countries, and
  organized a number of seminars and international workshops. The
  group members participated in several national and international
  projects and their results and tools are widely cited and used by
  users at academia, IT industry, and educational institutions.

  \paragraph*{Main tasks:}

  \begin{compactitem}
  \item Predrag Janičić participates at \WPref{atpetc}, task
    \taskref{atpetc}{instrumenting} and 
    \WPref{atpetc}, task \taskref{atpetc}{tracetodedukti}. 
    He coauthored three automated theorem provers for coherent 
    logic, and worked on their applications and their links 
    with other reasoning tools.
  \item Vesna Marinković participates in \WPref{atpetc}, task
    \taskref{atpetc}{deduktitoatp}. She is a coauthor of an automated
    theorem prover for coherent logic.
  \item Filip Marić and Danijela Simić participate at
    \WPref{alignment}, task
    \taskref{alignment}{aligncasestudies}. They are coauthors of
    several formalizations of geometry (Euclidean and hyperbolic) in
    Isabelle/HOL.
  \end{compactitem}
  
  \paragraph*{Publications, products or services:}

  \begin{compactitem}
  \item ``Computer-Assisted Theorem Proving in Synthetic Geometry'' by
    Julien Narboux, Predrag Janičić, Jacques Fleuriot. Handbook of
    Geometric Constraint Systems Principles (editors Meera Sitharam,
    Audrey St. John, Jessica Sidman), pp. 21--60, Chapman and
    Hall/CRC, Taylor \& Francis Group, 2018. ISBN-13: 978-1-4987-3891-0
  \item ``Proof Simplification in the Framework of Coherent Logic'' by
    Vesna Marinković. Computing and Informatics, vol. 34, no. 2,
    pp. 337--366, 2015.
  \item ``Formalizing Complex Plane Geometry'' by Filip Marić,
    Danijela Petrović. Annals of Mathematics and Artificial
    Intelligence, November 2014, ISSN: 1012-2443, DOI:
    10.1007/s10472-041-9436-4
  \item ``Formalization and Implementation of Algebraic Methods in
    Geometry'' by Filip Marić, Ivan Petrović, Danijela Petrović,
    Predrag Janičić. Electronic Proceedings in Theoretical Computer
    Science 79, pp. 63--81. ISSN: 2075-2180. DOI: 10.4204/EPTCS.79.4
  \item ``A Coherent Logic Based Geometry Theorem Prover Capable of
    Producing Formal and Readable Proofs'' by Sana Stojanović, Vesna
    Pavlović, Predrag Janičić. Automated Deduction in Geometry, ADG
    2010, volume 6877 of Lecture Notes in Artificial Inteligence,
    pp. 200--219. Springer, 2011. DOI 10.1007/978-3-642-25070-5\_12
  \end{compactitem}
    
  \paragraph*{Previous projects or activities:}
  
  \begin{compactitem}
  \item National grant (Ministry of science of Serbia) 174021
    ``Automated Reasoning and Data Mining'' (2011-2019). Janičić was
    the grant holder.
  \item COST (EU) projekat IC0901 ``Rich-Model Toolkit - An
    Infrastructure for Reliable Computer Systems'' (2009-2013)
  \item Swiss fund SNF's SCOPES grant IZ73Z0\_127979 ``Decision
    Procedures: from Formalizations to Applications''
    (2010-2013). Jani\v ci\'c was grant co-holder.
  \item Serbian-French Technology Co-Operation grant EGIDE/``Pavle
    Savic'' 680-00-132/2012-09/12 ``Formalization and automation of
    geometry'' (2012-2013). Jani\v ci\'c was grant co-holder.
  \item COST project EUTypes CA15123 ``The European research network on types for programming and verification'' (2016--2020) 
  \end{compactitem}

  \paragraph*{Infrastructures or technical equipments:}

  \begin{compactitem}
  \item Argo group has at its disposal a cluster computer with 32 dual core
  2GHz Intel Xeon processors with 2GB RAM per processor. This computer
  can be used for carrying out computationally more demanding
  experiments needed within a project.
  \end{compactitem}
  
  \paragraph*{Persons primarily responsible for carrying out the proposed activities:}
  
  \begin{compactitem} % in alphabetical order
    
  \item{\bf Predrag Janičić} received his PhD in computer science from
    the Faculty of Mathematics, University of Belgrade in 2001, where
    he now holds a position of a full professor. His main research
    interests are in the area of automated reasoning (automated
    theorem proving in coherent logic, automated and interactive
    proving of geometrical theorems, SAT/SMT) and mathematical
    software (dynamic geometry software, symbolic computation). He has
    published seven books, one book chapter and more than fifty
    research article in international journals and conferences. He
    gave six invited lectures, and served as a PC chair or a PC member
    at a number of international conferences, such as ADG, CICM, CADE,
    CADGME. He worked as a visiting researcher at the University of
    Edinburgh and visited a number of other universities. He is the
    author of the intelligent dynamic geometry tool GCLC and of the
    system URSA used for uniform reduction to SAT. He is also actively
    working on provers for coherent logic.

  \item{\bf Filip Marić} received his PhD in computer science from the
    Faculty of Mathematics, University of Belgrade in 2009, where he
    now holds a position of an associate professor. His main research
    interests are in the area of interactive and automated theorem
    proving, and applications in formalizing mathematics and verifying
    algorithm and software correctness. He authored and co-authored
    several university and high-school text-books in informatics and
    programming, and more than 20 research articles in international
    journals and conferences. He had an internship at Google Inc., and
    had research visits at a number of universities. He is the author
    of the library ArgoLib of decision procedures and of the formally
    verified SAT solver called ArgoSAT. He is also the author of a
    language Stereos used for formulating 3d geometry
    constructions. He is one of the founders of Petlja Foundation with
    the aim of promoting and improving algorithmic literacy in Serbia,
    and is actively involved in organizing programming competitions
    within Mathematical Society of Serbia.

  \item{\bf Vesna Marinković} is an assistant professor at the
    Department of Computer Science, Faculty of Mathematics, University
    of Belgrade. She finished her PhD studies at the same faculty in
    2015. Her main areas of research include automated reasoning in
    geometry and automated and formal theorem proving in coherent
    logic. She has more than 10 research articles in international
    journals and conferences. During 2009 she worked for three months
    as a visiting researcher at the Polytechnical University of
    Valencia, Spain. She is an author of a tool ArgoTriCS used for
    solving triangle construction problems in geometry and a co-author
    of a coherent logic prover ArgoCLP.

  \item{\bf Danijela Simić} is a teaching assistent at the Computer
    Science Department, Faculty of Mathematics, University of
    Belgrade. She finished her PhD studies at the same faculty in
    2017. Her main areas of research include interactive theorem
    proving and automated reasoning in geometry. Together with Filip
    Marić she is a co-author of many formalizations like formalization
    of Poincaré disc model
    %(available at \url{https://www.isa-afp.org/entries/Poincare_Disc.html})
    and formalization of complex geometry.
    %(available at \url{https://www.isa-afp.org/entries/Complex_Geometry.html})
    She received the reward for excellency for her PhD thesis -- Award
    from Mathematical Institute of Serbian Academy of Art and Science
    for PhD in computer science.
  
  \item{\bf Sana Stojanović Đurđević} is a teahing assitent at the
    Department of Computer Science, Faculty of Mathematics, University
    of Belgrade. She finished her PhD studies at the same faculty in
    2016.  Her main areas of research include axiomatic systems,
    automated reasoning in geometry and automated and formal theorem
    proving in coherent logic. She is one of the authors of the
    coherent logic prover ArgoCLP and an author of programme ArgoChecker
    used for automated verification of semi-formal proofs.
  \end{compactitem}
    
  % Vesna: ovaj tekst je sada visak?
%  \paragraph*{Specific expertise:}
  
%  \begin{compactitem}
%  \item Development of coherent-logic provers and their application in
%    automated theorem proving in geometry (Janičić, Stojanović
%    Đurđević, Marinković), and automated theorem proving in geometry
%    using algebraic methods (Janičić, Simić, Marić);
%  \item Automated solving of geometric construction problems and
%    verifying correctness of solutions (Marinković);
%  \item Isabelle/HOL verification of underlying SAT and SMT solving
%    procedures (Marić, Janičić);
%  \item Applications of SAT and SMT solvers integrated into
%    Isabelle/HOL in verifying algorithm correctness and solving
%    combinatorial conjectures in a formal, mechanical verified setting
%    (Marić, Janičić);
%  \item Interactive theorem proving in Euclidean an hyperbolic
%    geometry (Marić, Simić)
%  \end{compactitem}
  
\end{sitedescription}
%%% Local Variables: 
%%% mode: latex
%%% TeX-master: "../propB"
%%% End: 

% LocalWords:  site-jacu.tex clange sitedescription emph compactitem pn semmath
% LocalWords:  prosuming-flexiformal KohSuc asemf06 GinJucAnc alsaacl09 StaKoh
% LocalWords:  tlcspx10 KohDavGin psewads11 ednote Radboud Bia ystok CALCULEMUS
% LocalWords:  textbf keypubs OntoLangMathSemWeb uwb Deyan Ginev Stamerjohanns
% LocalWords:  searchability
