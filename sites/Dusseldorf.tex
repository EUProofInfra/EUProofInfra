% !TEX root = ../propB.tex
\begin{sitedescription}{Dus}

\paragraph*{Organization:}
\ednote{Give a one-paragraph run-down of the site and the team there. }

The University of D\"usseldorf currently has more than 20,000 students and its departments enjoy an excellent reputation due to an exceptionally high number of collaborative research centres.
The faculty of mathematics and sciences holds several ERC Advanced Grants and is involved in the Cluster of Excellence in Plant Sciences (CEPLAS).
The Institute of Informatics of the University of D\"usseldorf is also involved in CEPLAS and is a relatively new department, whose distinguishing aspect is the tight link with other scientific disciplines (in particular Biology and Medicine).

The contribution of the University of D\"usseldorf to Logipedia will be made by the STUPS group, 
which has internationally leading experts in formal methods and programming languages.
In the area of programming languages, STUPS has expertise in constraint programming,
 partial evaluation, just-in-time compilation and abstract interpretation.
In the area of formal methods, STUPS has considerable experience in methods and
 tools for verifying safety critical systems.
 %animation, model checking and test-case generation, in particular for B, Z and CSP.
STUPS has significant collaborative work with industry,
in particular Siemens, Alstom, Thales, Systerel, ClearSy, Praxis Critical Systems, Bosch  and AWE.
The STUPS group has hosted the iFM'09 conference as well as the AVoCS'10 workshop.



\paragraph*{Main tasks:}

\begin{compactitem}
\item\ednote{specify the main tasks and reference the respective work packages} 
\item T.1.4 (Instrument Atelier-B/Rodin) instrument ProB disprover, prover and possibly model finder
\end{compactitem}


\paragraph*{Relevant previous experience:}

\ednote{give an overview over previous work and projects that add to the \pn project}


The STUPS team and its members have been involved in several EU projects
 (Advance, Deploy, ASAP, PyPy, RODIN, POST) and the Eureka Eurostars project PyJIT.
Of particular interest here is the development of the Rodin platform in the EU projects RODIN, Deploy and Advance.
The STUPS team is maintaining the ProB validation tool, and has a broad experience
 in linking up various formal languages and technologies.
For example, the ProB tool, initially developed for the B and Event-B method using constraint logic programming,
 is now capable to read in TLA+, Alloy and Z models and
 can translate constraints also to SAT and SMT representations.
%The constraint-solving capabilities of ProB can also be used for various applications such as
%  disproving, proving or for data validation.
ProB is available as a prover within the Atelier-B and Rodin platforms.
The most prominent industrial application of ProB is
 data validation which started in the EU project Deploy, and led to
 ProB's use to validate the configuration of many railway systems worldwide
 (e.g.,
 Paris Line 1,  S\~{a}o Paulo line 4, Barcelona line 9 for Siemens
 or Alstom's URBALIS 400 CBTC system %for Mexico, Toronto
 or the Thales ETCS radio block centre
 ).

\paragraph*{Specific expertise:}

\begin{compactitem}
\item %\ednote{give three to five specific areas of expertise that pertain to the \pn project}
      developing and maintaining the ProB validation tool, along with
      its disprover based on constraint-logic programming
 \item experience with translating formal methods (B,Z,TLA+, ...)
       to SAT and SMT representations,
 \item experience in building and certifying tools for industrial usage (ProB has been certified T2
  for several applications according to the industrial norm EN50128)
 \item experience in the practical applications of formal methods and tools.
\end{compactitem}

\paragraph*{Staff members undertaking the work:}

\begin{itemize}
\item
\textbf{Michael Leuschel}\ednote{describe the site leader and his expertise}
Michael Leuschel is head of the STUPS group and was co-founder of the spin-off
company Formal Mind.
He has developed the ProB tool-set for the validation of formal specifications.
%ProB is currently in use by many companies, such as Thales, Alstom, ClearSy, Systerel and many more.
Outside of formal methods, his main research areas are automatic programme analysis and optimization (notably partial evaluation and abstract interpretation).
He was the chair of many formal methods conferences.
He was a member of the PEPM and LOPSTR steering committees and
 the editorial board of the Journal of Theory and Practice of Logic Programming.
He has over 5300 citations, published over 170 papers, 
and developed several tools such as the ECCE and LOGEN partial evaluation systems.
%He has been involved in several EU projects (Advance, Deploy, ASAP, PyPy, RODIN, POST) and the Eureka Eurostars project PyJIT.

\item
\textbf{Jens Bendisposto}
Has developed the Java-API for ProB and has experience in distributed validation technologies.
He has many years experience in developing in general and formal methods tools in particular.
\end{itemize}

\ednote{provide the key publications below}
\keypubs{providemore}

Publications:
\begin{itemize}
\item Michael Leuschel, Michael J. Butler: ProB: an automated analysis toolset for the B method. STTT 10(2): 185-203 (2008).
 (journal version of the FM'2003 conference paper, together over 1000 citations).
\item Michael Leuschel, J{\'e}r{\^o}me Falampin, Fabian Fritz, Daniel Plagge.
Automated property verification for large scale {B} models with {ProB},
Formal Asp. Comput. 23 (6): 83--709, 2011.
\item Dominik Hansen, Michael Leuschel: Translating B to TLA+ for validation with TLC. Sci. Comput. Program. 131: 109-125 (2016).
\item Sebastian Krings, Michael Leuschel:
Proof assisted bounded and unbounded symbolic model checking of software and system models. Sci. Comput. Program. 158: 41-63 (2018).
\item Fabrice Kordon, Michael Leuschel, Jaco van de Pol, Yann Thierry-Mieg:
Software Architecture of Modern Model Checkers. Computing and Software Science 2019: 393-419.
\end{itemize}


\end{sitedescription}
%%% Local Variables: 
%%% mode: latex
%%% TeX-master: "../propB"
%%% End: 

% LocalWords:  site-jacu.tex clange sitedescription emph compactitem pn semmath
% LocalWords:  prosuming-flexiformal KohSuc asemf06 GinJucAnc alsaacl09 StaKoh
% LocalWords:  tlcspx10 KohDavGin psewads11 ednote Radboud Bia ystok CALCULEMUS
% LocalWords:  textbf keypubs OntoLangMathSemWeb uwb Deyan Ginev Stamerjohanns
% LocalWords:  searchability
