\begin{sitedescription}{Cha}

\logo{Chalmers}

Chalmers University of Technology conducts research and education in
technology, science, shipping and architecture with an emphasis on
sustainability at a global scale. Chalmers has 10\,300 full-time
students and 3\,100 employees.  Its Computer Science and Engineering
Department is joint with Gothenburg University.

Chalmers' previous involvement in research and training programmes:
During the last two Framework Programmes for Research and
Technological Development of the EC (FP7 and Horizon2020), Chalmers
was and is involved in 146 EU research projects and has participated
in over 100 ERC grants in total.  Current involvement in Research and
Training Programmes: Currently, Chalmers is involved in more than 100
Horizon 2020 projects.  To date, researchers at Chalmers have been
granted 26 ERC Grants.

\paragraph*{Main tasks:}

\begin{compactitem}
\item Myreen leads task \taskref{instrumentation}{HOL4}.  Myreen is a
  developer and expert user of HOL4. At Chalmers, he leads
  a research group of six people where everyone uses HOL4 as the
  primary tool for their work.
\item Myreen leads task \taskref{libraries}{cakeml}. He is a founding
  member of the team behind the verified CakeML compiler and proof
  tools built around CakeML.
\end{compactitem}

\paragraph*{Publications, products or services:}

\begin{compactitem}
\item
  ``The verified CakeML compiler backend'',
  by Tan, Myreen, Kumar, Fox, Owens, Norrish.
  Journal of Functional Programming, Cambridge University Press, 2019.
\item
  ``Self-Formalisation of Higher-Order Logic;
  Semantics, Soundness, and a Verified Implementation'',
  by Kumar, Arthan, Myreen, Owens.
  Journal Automated Reasoning, Springer, 2016.
\item
  ``CakeML: A Verified Implementation of ML'',
  by Kumar, Myreen, Norrish, Owens.
  In Principles of Programming Languages, ACM, 2014
\end{compactitem}

\paragraph*{Previous projects or activities:}

%\ednote{a list of up to 5 relevant previous projects or activities, connected to the subject of this proposal}

\begin{compactitem}
\item Myreen is PI on ``Trustworthy software by programming and compiling in logic'' (2017-2021) from the Swedish Foundation for Strategic Research. The grant is worth 12 million SEK (approx.\ 1.12 million EUR)
\end{compactitem}

\paragraph*{Infrastructures or technical equipments:}

%\ednote{a description of any significant infrastructure and/or any major items of technical equipment, relevant to the proposed work}

\begin{compactitem}
\item The open-source verified \href{https://cakeml.org/}{CakeML}
  compiler jointly developed by a group including Myreen.
\end{compactitem}

\paragraph*{Persons primarily responsible for carrying out the proposed activities:}

\begin{compactitem}
\item \textbf{Magnus O. Myreen} is an associate professor at Chalmers
  University of Technology.  Myreen completed his PhD on programme
  verification at the University of Cambridge in 2009. His PhD
  dissertation was selected as the winner of the BCS Distinguished
  Dissertation Competition 2010. In 2012, Myreen became a Royal
  Society University Research Fellow. In 2014, Myreen moved to
  Chalmers where he became associate professor in 2015.  He is a
  member of the steering committee for the conference on Interactive
  Theorem Proving and the streering committee for Certified Programmes
  and Proofs.  Myreen regularly serves on programme committees for top
  conferences on the topic of mechanised reasoning and programming
  languages.
\end{compactitem}

\end{sitedescription}
%%% Local Variables:
%%% mode: latex
%%% TeX-master: "../propB"
%%% End:

% LocalWords:  site-jacu.tex clange sitedescription emph compactitem pn semmath
% LocalWords:  prosuming-flexiformal KohSuc asemf06 GinJucAnc alsaacl09 StaKoh
% LocalWords:  tlcspx10 KohDavGin psewads11 ednote Radboud Bia ystok CALCULEMUS
% LocalWords:  textbf keypubs OntoLangMathSemWeb uwb Deyan Ginev Stamerjohanns
% LocalWords:  searchability
