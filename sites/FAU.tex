\begin{sitedescription}{Fau}
  Friedrich Alexander Universit\"at Erlangen/N\"urnberg (FAU) is a public research
  university in the cities of Erlangen and Nuremberg, Germany. FAU is the second largest
  state university in the state Bavaria. It has 5 faculties, 23 departments/schools, 30
  clinical departments, 19 autonomous departments, 656 professors, and ca 40\,000
  students.

  The KWARC (KnoWledge Adaptation and Reasoning for Content~\cite{KWARC:online}) Group
  headed by {\emph{Prof.\ Dr.\ Michael Kohlhase}} specialises in knowledge management for
  STEM.  Formal logic, natural language semantics, and semantic web technology provide the
  foundations for the research of the group. Its group working on \pn will inclue: Michael
  Kohlhase, Florian Rabe, Tom Wiesing, and Jonas Betzendahl.

% \subsubsection*{Curriculum vitae}
% % Curriculum of the personnel at this institution
% \input{CVs/Michael.Kohlhase}
% \input{CVs/Christian.Maeder}
% \input{CVs/Mihnea.Iancu}

\subsubsection*{Relevant previous experience:}

The KWARC group is the lead implementor of the OMDoc (Open Mathematical Document) format
for representing mathematical knowledge \cite{Kohlhase:OMDoc1.2} and redeveloped its
formal core in the OMDoc/MMT format~\cite{RabKoh:WSMSML13}. The latter has been
implemented in the MMT system~\cite{MMTSVN:on,RabKoh:WSMSML13} which provides efficient
implementations of the computational primitives such as type checking, flattening, and
presentation at a logic/foundation-independent level.  The group has developed services
powered by such semantically rich representations, different paths to obtaining them, as
well as platforms that integrate both aspects.  \emph{Services} include the adaptive
context-sensitive presentation framework provided by the MMT API and the semantic search
engine MathWebSearch\cite{KohSuc:asemf06,ProKoh:mwssofse12}. 

Semantic services can be integrated into the documents generated from OMDoc/MMT
representations, making them into ``active documents'', i.e. documents that are
interactive and adaptive to the user and situation.  For \emph{obtaining} rich content,
the group investigates assisted manual editing \cite{JucKoh:sidesc10:biblatex} as well as
automatic annotation using linguistic techniques \cite{GinJucAnc:alsaacl09}.  Finally,
KWARC has developed the \textsf{MathHub.info} portal a community-based library and
knowledge management system for flexiformal libraries, which can be used for semantic
publishing and eLearning~ \cite{KohDavGin:psewads11,MathHub:on,IanJucKoh:sdm14}.

The \textsf{OMDoc/MMT} knowledge representation format and the \textsf{MathHub.info}
system will an important basis for the developments Work Packages 4 and 6.

Michael Kohlhase has initiated and led the CALCULEMUS! IHP-Research and Training Network
and participated in the FP6 IST MoWGLI (Mathematics on the Web: Get it by Logic and
Interfaces) project, the FP6 CSA Once-CS (Open Network of Centres of Excellence in Complex
Systems), The FP7 EDC project WebALT (Web Advanced Learning Technologies), and the H2020
Infrastructure project OpenDreanKit

\subsubsection*{Specific expertise:}
\begin{compactitem}
\item Modelling formal structures of mathematical knowledge in a web-scalable way.
\item Transforming large collections of legacy scientific publications to semantically
  structured markup.
\item Designing user interfaces for authoring and interacting with mathematical knowledge.
\end{compactitem}

\site{Fau} leads \WPtref{structuring} and tasks related to formal alignments and
alignment-based translation in \WPtref{alignment}.
\end{sitedescription}

%%% Local Variables: 
%%% mode: LaTeX
%%% TeX-master: "../propB"
%%% End: 
