\begin{sitedescription}{Fau}

  \logo{FAU}
  
FAU is the second largest state university in the state Bavaria.
It has 5 faculties, 23 departments/schools, 30 clinical departments, 19 autonomous departments, 656 professors, and about 40\,000 students.

FAU has a strong departments of Computer Science and Mathematics (both 25 Professors) with strong groups in scientific computing, mathematical modeling, and simulation, data management, data visualization, and Pattern recognition.
All of these deal with mathematical data in some way and thus constitute a conducive and supportive environment for \pn.
Importantly the collaborating departments constitute a reservoir of know-how and potential user expertise the \pn project can draw upon for evaluation and testing.

The site leader is Prof. Dr. Michael Kohlhase.
PD Dr. Florian Rabe will be co-PI for the site.

\paragraph*{Main tasks:}

The site co-leads \WPtref{structuring} and leads task \tasktref{alignment}{alignsearch}.

\paragraph*{Publications, products or services:}

% a list of up to 5 relevant publications, and/or products, services (including widely-used datasets or software), or other achievements relevant to the call content
The site developed 
\begin{compactitem}
 \item the OMDoc (Open Mathematical Document) format for representing mathematical knowledge \cite{Kohlhase:OMDoc1.2}, which anticipated much of the research proposed here,
 \item the MMT logical framework \cite{RabKoh:WSMSML13} (which uses OMDoc as its representation format), a close relative of Dedukti,
 \item the notion of alignments \cite{GKKMR:alignments:17}, which will be critical in \WPref{alignment},
 \item the comprehensive representation of many major proof assistant libraries (albeit often proof objects) in OMDoc/MMT \cite{KR:oafexp:20},
 \item the MathWebSearch search engine for symbolic formulas knowledge \cite{ProKoh:mwssofse12}.
\end{compactitem}

\paragraph*{Previous projects or activities:}

% a list of up to 5 relevant previous projects or activities, connected to the subject of this proposal

\begin{compactitem}
 \item Prof. Kohlhase and Dr. Rabe were co-PIs of the OpenDreamKit H2020 infrastructure project (2015--2019) on virtual research environments that integrate mathematical computation systems.
 \item Prof. Kohlhase and Dr. Rabe are the PIs of the German-funded OAF project (2014--2020) about representing proof assistant libraries in logical frameworks, specifically the \mmt framework.
 \item Prof. Kohlhase co-initiated and organized the three NTCIR community challenges for mathematics information retrieval in 2014/16/17.
 \item Prof. Kohlhase initiated and led the CALCULEMUS! IHP-Research and Training Network.
 \item Prof. Kohlhase participated in the FP6 IST MoWGLI (Mathematics on the Web: Get it by Logic and Interfaces) project.
\end{compactitem}

\paragraph*{Infrastructures or technical equipments:}
% description of any significant infrastructure and/or any major items of technical equipment, relevant to the proposed work

The site hosts \url{http://mathhub.info}, a portal for formalized mathematics, mathematical databases, and active mathematical documents.
It hosts about 10GB of data (Theorem Prover Libraries, OEIS, semantic course materials and a multilingual mathematical glossary), serves it via the MMT system and a lightweight browser-based front-end, and includes services like 2D/3D theory graph visualization of the modular structure.

\paragraph*{Persons primarily responsible for carrying out the proposed activities:}

\textbf{Prof. Kohlhase} holds the \emph{Professorship for Knowledge Representation and Management} in the Computer Science Department.
The KWARC (KnoWledge Adaptation and Reasoning for Content~\cite{KWARC:online}) Group headed by him specializes in knowledge management for Science, Technology, Engineering, and Mathematics (STEM), focusing on the last as a test subject.
Formal logic, natural language semantics, and semantic web technology provide the foundations for the research of the group.
Its group working on \pn will be composed of the following non-exhaustive list: Prof. Dr. Michael Kohlhase, Dr. Florian Rabe, Tom Wiesing, Dennis M\"uller, Jonas Betzendahl and Jan Frederik Schaefer.

\textbf{Dr. Rabe} obtained his PhD in 2008 and his habilitation in 2014 and is now Akad. Oberrat and Privatdozent in the KWARC group.
Within the group, Kohlhase and Rabe co-advise most students and have shared the administration of research projects for 10 years.
He is an expert in the design, implementation, and evaluation of representation languages for mathematical knowledge and meta-logical frameworks.
He currently chairs the steering committee of \emph{Logical Frameworks and Meta-Languages: Theory and Practice}.
He is the main developer of the \mmt language and system and the LATIN logic atlas and has designed and overseen the representation of proof assistant libraries in the OAF project.
\end{sitedescription}

%%% Local Variables: 
%%% mode: LaTeX
%%% TeX-master: "../propB"
%%% End: 
