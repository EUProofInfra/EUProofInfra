\begin{sitedescription}{Sou}

\paragraph*{Organization:}
%\ednote{Give a one-paragraph run-down of the site and the team there. }
Southampton University is a global Top-100 public university situated in Southampton, Hampshire (UK); it is a founding member of the Russell Group of public research universities in the United Kingdom. The University has a student population for around 25,000, over 6,500 of whom are international. The University has an outstanding reputation in Electronics and Computer Science and has a strong body of local expertise in formal methods.

\paragraph*{Main tasks:}
\begin{compactitem}
%\item\ednote{specify the main tasks and reference the respective work packages} 
\item Developiing tool support for interfacing Rodin with Logipedia.  
\item Creating a plugin that will allow users to export proof obligations to Dedukti.
\item Exploring development of Rodin theories (e.g. the theory of real numbers) in Dedukti.
\end{compactitem}


\paragraph*{Relevant previous experience:}
%\ednote{give an overview over previous work and projects that add to the \pn project}
The development of Rodin was supported by the EU ICT project, ADVANCE: \texttt{www.advance-ict.eu} (2011 -- 2014). Originally Rodin development was funded by the EU project Deploy (2008 -- 2012) and Rodin (2004 -- 2007).

\paragraph*{Specific expertise:}
Formal methods and formal modelling in Event-B, formal verification of physical systems, interactive theorem proving, software development and verification.

%\begin{compactitem}
%\item \ednote{give three to five specific areas of expertise that pertain to the \pn project}
%\end{compactitem}

\paragraph*{Staff members undertaking the work:}~\newline 

\textbf{Prof Michael Butler} has made key theoretical and methodological contributions to the Event-B formal method that enable it to scale to large complex systems.

\textbf{Dr Thai Son Hoang} has over a decade of experience with Event-B and was one of the original developers of the Rodin platform at ETH Zurich; he has maintained a strong track record of software contributions to the Rodin platform ever since its inception.

\textbf{Dr Andrew Sogokon} has worked on applying theorem proving to verify safety and liveness of continuous and hybrid dynamical systems; he has a track record of developing tools for automating proof discovery in the KeYmaera X theorem prover for hybrid systems.

%\ednote{provide the key publications below}
\paragraph*{Key Publications} 

\begin{compactitem}
\item Jean-Raymond Abrial, Michael Butler, Stefan Hallerstede, Thai Son Hoang, Farhad Mehta, Laurent Voisin,
	Rodin: an open toolset for modelling and reasoning in Event-B.
{\em International journal on software tools for technology transfer}, 12 (6), 2010, 447--466.
\item 
Thai Son Hoang, Andreas F\"urst, Jean-Raymond Abrial,
Event-B patterns and their tool support,
{\em Software \& Systems Modeling } 12 (2), 2013, pp. 229--244.


\item Jean-Raymond Abrial, Michael Butler, Stefan Hallerstede, Laurent Voisin,
	An open extensible tool environment for Event-B,
	{\em International Conference on Formal Engineering Methods}, 2006, pp. 588--605. 
\end{compactitem}


\end{sitedescription}
%%% Local Variables: 
%%% mode: latex
%%% TeX-master: "../propB"
%%% End: 

% LocalWords:  site-jacu.tex clange sitedescription emph compactitem pn semmath
% LocalWords:  prosuming-flexiformal KohSuc asemf06 GinJucAnc alsaacl09 StaKoh
% LocalWords:  tlcspx10 KohDavGin psewads11 ednote Radboud Bia ystok CALCULEMUS
% LocalWords:  textbf keypubs OntoLangMathSemWeb uwb Deyan Ginev Stamerjohanns
% LocalWords:  searchability
