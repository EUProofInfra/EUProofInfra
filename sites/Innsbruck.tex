\begin{sitedescription}{Inn}
\logo{Innsbruck}

%\paragraph{Organization:}
The University of Innsbruck is a global Top-200 university and the second
largest university in Austria. UIBK has been involved in a number of FWF
projects related to formal proof, and a large number of national and
international projects including dozens of projects as part of FP5,
FP6, FP7, and H2020.
%
The research of the Computational Logic group is concerned with the logical
foundations of computer science and their application to the analysis and
verification of complex systems. The group has developed the IsaFoR library,
the largest formalisation of rewriting with more than 5000 theorems. Various
hammer systems developed in the group are today strongest automation techniques
for various formalizations including the Flyspeck project.

\paragraph{Main tasks:}

\begin{compactitem}
\item Specification of the Mizar logical foundations and type system, \WPref{theories}, \taskref{theories}{mizar}.
\item Internal proof automation using direct proof term construction, % with heuristic rewriting
  \WPref{atpetc}, \taskref{atpetc}{readiness}.
\item Contributions to automatic search for alignments using neural methods, \WPref{alignment}, \taskref{alignment}{aligntheories}.
\end{compactitem}

\paragraph{Publications, products or services:}
\begin{itemize}
\item ``Semantics of Mizar as an Isabelle Object Logic'', C.~Kaliszyk and Karol Pąk,
journal paper, 2019.
  % {J.~Automated Reasoning}, 63(3): 557--595, 2019.
%\newblock \doi{10.1007/s10817-018-9479-z}.

\item ``Aligning Concepts across Proof Assistant Libraries'', T.~Gauthier and C.~Kaliszyk, journal paper, 2019.
% , {J. Symbolic Computation}, 90:89--123, 2019.
%\newblock \doi{10.1016/j.jsc.2018.04.005}.

\item ``Hammer for Coq: Automation for Dependent Type Theory'', Ł.~Czajka and C.~Kaliszyk, journal paper, 2018.
% , {J.~Automated Reasoning}, 2018.
%\newblock \doi{10.1007/s10817-018-9458-4}.
\end{itemize}

\paragraph{Previous projects or activities:}

\begin{compactitem}
\item 2017--2022, ERC starting grant, ``Strong Modular proof Assistance: Reasoning across Theories''.
\item 2013--2017, FWF grant, ``Interactive Proof: Proof Translation, Premise Selection, Rewriting''
\end{compactitem}

%\ednote{give an overview over previous work and projects that add to the \pn project}

% \paragraph{Specific expertise:}

% \begin{compactitem}
% \item Development of proof automation for several proof assistants.
% \item Automatic search for alignments using statistical methods.
% \item Specification of the Mizar object logic in the Isabelle logical framework.
% \end{compactitem}

\paragraph{Persons primarily responsible for carrying out the proposed activities:}

\begin{itemize}
\item \textbf{Cezary Kaliszyk} has been working on making proof assistants
more accessible by developing proof automation, proof advice, and other packages for formal
proofs. He has worked on the Isabelle/Mizar object logic, where features of Mizar were
expressed in a logical framework. Kaliszyk has also worked on machine learning for interactive
proofs and has co-organized the AITP conference on the topic in the last few years (AITP). He
has developed multiple hammer systems for higher-order logic and intuitionistic type theory.
Kaliszyk has also worked on alignments between formal systems and between informal and formal
mathematics.
\end{itemize}

%Postdoc and PhD students that unfortunately will end likely in 2021: Joshua Chen, Miroslav Olšák, Stanisław Purgał


\end{sitedescription}
%%% Local Variables: 
%%% mode: latex
%%% TeX-master: "../propB"
%%% End: 

% LocalWords:  site-jacu.tex clange sitedescription emph compactitem pn semmath
% LocalWords:  prosuming-flexiformal KohSuc asemf06 GinJucAnc alsaacl09 StaKoh
% LocalWords:  tlcspx10 KohDavGin psewads11 ednote Radboud Bia ystok CALCULEMUS
% LocalWords:  textbf keypubs OntoLangMathSemWeb uwb Deyan Ginev Stamerjohanns
% LocalWords:  searchability
