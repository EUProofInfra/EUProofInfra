\begin{sitedescription}{Got}

\logo{Gothenburg}

% \ednote{Give a one-paragraph run-down of the site and the team there. }
Founded in 1891, the University of Gothenburg is located in the city centre
of Gothenburg and is home to 47,500 students and 6,000 employees across 39
departments. Strong research and attractive study programmes attract
scientists and students from all around the world. The University of
Gothenburg is environmentally certified and works actively for sustainable
development.

The Logic and Types group at Gothenburg University has been a leading group
in the research on dependent type theory and interactive theorem proving
since the 1980's and has implemented several well known proof systems. Most
recently the widely used Agda system, designed and implemented by Dr. Ulf
Norell as part of his PhD thesis, and still actively developed by the
group. The group also has a strong presence in formalised mathematics, led
by professor Thierry Coquand and has received several European grants in
this area.

\paragraph*{Main tasks:}

\begin{compactitem}
% \item\ednote{specify the main tasks and reference the respective work packages}
\item Task \taskref{instrumentation}{agda}:
  Instrumenting the Agda system to produce Dedukti proofs.
\item Task \taskref{instrumentation}{agda}:
  Investigating possible designs for a core language for Agda, in
  order to facilitate the exporting of Agda developments to Logipedia.
\end{compactitem}
As an expert on the implementation of proof systems, and the Agda system in
particular, Dr. Ulf Norell is exceptionally qualified to carry out these
tasks.

\paragraph*{Publications, products or services:}
\begin{compactitem}
  \item ``Towards a practical programming language based on
  depedent types'', by Ulf Norell. PhD thesis, Chalmers University of Technology, 2007.
  \item ``Interactive programming with dependent types'', by Ulf Norell.
  Proceedings of the 18th ACM SIGPLAN international conference on
  Functional programming. 2013.
\end{compactitem}

\paragraph*{Previous projects or activities:}
\begin{compactitem}
\item Formalisation of Mathematics, an EU FP7 STREP FET-open project led by
Thierry Coquand at University of Gothenburg. March 2010--July 2013, Grant
nr. 243847.
\item Formalization of Constructive Mathematics, an EU FP7 ERC Advanced
Grant led by Thierry Coquand at University of Gothenburg. April
2010--March 2015, Grant nr. 247219.
\item Types for Proofs and Programs, a Swedish Research Council project led
by Thierry Coquand at University of Gothenburg. Jan 2013--Dec 2016. Grant
nr. 2012-05294.
\item Termination Certificates for Dependently-Typed Programs and Proofs
via Refinement Types, a Swedish Research Council project led by Andreas
Abel at University of Gotheburg. Jan 2015--Dec 2018. Grant nr. 2014-04864.
\end{compactitem}

\paragraph*{Infrastructures or technical equipments:}
\begin{compactitem}
\item The Agda proof system is developed by the Logic and Types group at
Gothenburg University. \\ {\tt https://github.com/agda/agda/}
\end{compactitem}

\paragraph*{Persons primarily responsible for carrying out the proposed activities:}

\begin{compactitem}
\item{\bf Ulf Norell} got his PhD from Chalmers University of Technology in
2007 on the design and implementation of dependently typed programming
languages. After his PhD he continued as a PostDoc at Chalmers, and since
2011 he has been working as a research engineer at Gothenburg University.
He is the main developer and maintainer of the Agda proof assistant, which
is widely used in both research and teaching. He gave the keynote at ICFP
2013, and is a member of IFIP Working Group 2.8 on functional programming.
\end{compactitem}

\end{sitedescription}
%%% Local Variables:
%%% mode: latex
%%% TeX-master: "../propB"
%%% End:

% LocalWords:  site-jacu.tex clange sitedescription emph compactitem pn semmath
% LocalWords:  prosuming-flexiformal KohSuc asemf06 GinJucAnc alsaacl09 StaKoh
% LocalWords:  tlcspx10 KohDavGin psewads11 ednote Radboud Bia ystok CALCULEMUS
% LocalWords:  textbf keypubs OntoLangMathSemWeb uwb Deyan Ginev Stamerjohanns
% LocalWords:  searchability
