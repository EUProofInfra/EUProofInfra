\begin{sitedescription}{Got}

% \ednote{Give a one-paragraph run-down of the site and the team there. }
The Logic and Types group at Gothenburg University has been a leading group
in the research on dependent type theory and interactive theorem proving
since the 1980's. The current version of the system was designed and
implemented by Dr. Ulf Norell as part of his PhD thesis and has been
actively developed by the group since then. Agda is widely used in both
research and teaching.

\paragraph*{Main tasks:}

\begin{compactitem}
% \item\ednote{specify the main tasks and reference the respective work packages}
\item Instrumenting the Agda system to produce Dedukti proofs
  (\WPtref{instrumentation}).
\item Investigating possible designs for a core language for Agda, in
  order to facilitate the exporting of Agda developments to Logipedia.
\end{compactitem}

\paragraph*{Publications, products or services:}
\begin{itemize}
  \item Jesper Cockx, Andreas Abel. ``Elaborating dependent
  (co)pattern matching.'' Proceedings of the ACM on Programming
  Languages, 2(ICFP), 2018.
  \item Andreas Abel, Joakim \"Ohman, and Andrea Vezzosi. ``Decidability of
  Conversion for type theory in type theory.'' Proceedings of the ACM on Programming
  Languages, 2(POPL), 2018.
  \item Ulf Norell. ``Towards a practical programming language based on
  depedent types.'' PhD thesis, Chalmers University of Technology, 2007.
\end{itemize}

\paragraph*{Previous projects or activities:}

\paragraph*{Infrastructures or technical equipments:}

\paragraph*{Persons primarily responsible for carrying out the proposed activities:}

\begin{itemize}
\item{\bf Ulf Norell} is the main developer and maintainer of the Agda
proof assistant. He got his PhD from Chalmers University of Technology in
2007 on the design and implementation of dependently typed programming
languages.
\item{\bf Andreas Abel} is an expert on the meta theory of dependent type
theory and one of the core developers of Agda. He got his PhD from
University of Munich in 2006.
of type
\end{itemize}

\end{sitedescription}
%%% Local Variables:
%%% mode: latex
%%% TeX-master: "../propB"
%%% End:

% LocalWords:  site-jacu.tex clange sitedescription emph compactitem pn semmath
% LocalWords:  prosuming-flexiformal KohSuc asemf06 GinJucAnc alsaacl09 StaKoh
% LocalWords:  tlcspx10 KohDavGin psewads11 ednote Radboud Bia ystok CALCULEMUS
% LocalWords:  textbf keypubs OntoLangMathSemWeb uwb Deyan Ginev Stamerjohanns
% LocalWords:  searchability
