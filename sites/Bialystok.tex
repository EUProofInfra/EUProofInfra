\begin{sitedescription}{Bia}

\paragraph{Organization:}

The University of Bialystok (UwB) was established in 1997 from a~branch of Warsaw University after 29 years of its existence.
Today UwB is one of the largest and strongest academic centers in North-Eastern Poland.
It consists of nine faculties (including one located in Vilnius, Lithuania) and five institutes.
Classes and lectures are delivered by approx. 850 academic teachers (nearly 200 are independent research scholars).
At present UwB educates over 8000 students in almost 30 fields of study.

The Mizar research group at UwB has several decades of experience in designing formal languages 
for efficient encoding of mathematical data and implementing formal proof-checking software.
The group coordinates the development of the Mizar Mathematical Library (MML) -- 
a~large centralized collection of formalized mathematical definitions, theorems and their proofs 
authored by over 260 contributors from 20 countries.
The library is maintained and distributed in a~variety of data formats, 
including interactive web-based documents and automatically generated natural language journal articles. 
The members of the group have participated in a~number of EU funded research
and collaboration projects,
as well as the EUTYPES Cost Action.
The Mizar group has also organized the MKM 2004 and CICM 2016 conferences.

\paragraph{Main tasks:}

\begin{compactitem}
\item Expressing the foundations of the Mizar logic in Dedukti \WPref{theories}
\item Extracting in-depth knowledge from the Mizar proofs \WPref{theories}
\item Developing Dedukti techniques corresponding to Mizar proof checking \WPref{theories}
\end{compactitem}

\paragraph{Relevant previous experience:}

The Mizar research group has carried out several grants within European Union Framework Projects 
and also funded by Polish National Science Center and Office of Naval Research, US.
The most related to the project are:

\begin{compactitem}
\item ``Isabelle Emulator for Mizar: Environment for Mizar Mathematical Library Re-verification'',
funded by Polish National Science Center, project manager: Karol Pąk, 7/2016--7/2019
\item ``Independent Verification of Mizar Logic'', funded by the OeAD Scientific \& Technological Cooperation with Poland,
project coordinator at the Polish side: Karol Pąk, 5/2016--4/2018
\item ``Algorithms Concerning the Legibility of Natural Deduction Proofs'', funded by Polish National Science Center,
project manager: Karol Pąk, 7/2013--1/2017
\item ``Management of a~Large Repository of Computer Verified Mathematical Knowledge'',
funded by Polish Ministry of Science and Higher Education, project manager: Andrzej Trybulec, 5/2009--5/2012
\item ``Types for Proofs and Programs'', TYPES II EU FP6 510996,
site of the project coordinated by Chalmers, 9/2004--8/2007
\end{compactitem}

\paragraph{Specific expertise:}

\begin{compactitem}
\item Expressing and translating formal semantics
\item Experience with Mizar kernel augmentation for proof object extraction
\item In-depth knowledge of the Mizar foundations
\item Managing large mathematical repositories
\end{compactitem}

\paragraph{Staff members undertaking the work:}

\begin{compactitem}

\item\textbf{Dr.\ Artur Korniłowicz} is the deputy director for science 
and head of the Department of Programming and Formal Methods
at the Institute of Informatics at the University of Bialystok.
Korniłowicz received his PhD in computer science from Shinshu University, Nagano, Japan in 2001.
In 2017 he received habilitation from the University of Warsaw, Poland.
Korniłowicz's main research interests are in formal verification of mathematics and verification of algorithms.
He is one of the key developers of the Mizar proof-assistant and the author of over 100 Mizar formalizations.
In 2005 he was awarded Śleszyński Prize for Formalization of the Jordan Curve Theorem.
In the period 7/2001--6/2002 Korniłowicz was a~CALCULEMUS postdoctoral fellow
at the Istituto per la Ricerca Scientifica e~Tecnologica, Trento, Italy under the CALCULEMUS project within EU FP5;
and in the period 7/2003--3/2005 he was a~Japan Society for the Promotion of Science 
postdoctoral fellow at the Shinshu University, Nagano, Japan.

\item\textbf{Dr.\ Czesław Byliński} is head of the Computer Networks Section at the University of Bialystok.
He received his PhD in computer science from Shinshu University, Japan in 1998.
Since 1978 he has been a~member of the Mizar Project.
He participates in the implementation of the Mizar language and the developing the Mizar system tools. 
Since 2014, he has been in charge of the Mizar implementation team.

\item\textbf{Dr.\ Adam Grabowski} is an adjunct at UwB since 2006, 
with a~focus on the formalization of mathematics and computer science.
He received his PhD in mathematics from the University of Silesia in Katowice, Poland in 2005 
and PhD in computer science from Shinshu University, Nagano, Japan in 2005.
Currently, he works on the application of automated proof assistants
in the modeling of the reasoning under uncertainty: fuzzy and rough sets.
He has authored over 120 papers in refereed journals and international conference
proceedings, including over 70 formalizations in Mizar.
He received twice the Śleszyński Prize (1998, 2000) granted by the Association of Mizar Users.
Since 1999 he has been the head of the Library Committee of Association of Mizar Users, taking care of
the management and development of the Mizar Mathematical Library.

\item\textbf{Dr.\ Adam Naumowicz} is a~member of the core Mizar development team. 
With his background in mathematics and linguistics, he received his PhD in computer science in 2005
for research on formalizing recent mathematical results. His recent works focus on extending Mizar checker's computational power, 
interacting with external tools and developing web-based services. 
He's been elected twice to serve as Mathematical Knowledge Management representative to the Steering Committee
of Conference on Intelligent Computer Mathematics (CICM).
He was also the main organizer of CICM 2016 held at the University of Bialystok. 
He acts as Poland's representative in the Management Committee of the European research network on types
 for programming and verification (Cost Action EUTypes).

\item\textbf{Dr.\ Karol Pąk} has developed the Isabelle/Mizar system where
he specified Mizar in the Isabelle logical framework
giving the complete semantics of the system, including
the underlying first-order logic variant, soft type system, and definitional mechanisms.
Additionally, he proposed a semi-automatic translation of several MML articles
to the resulting object logic to cross-verify them.
Furthermore he has been developing methods
that automatically improve readability of natural deduction proofs
by the step order manipulation as well as lemma extraction.

\end{compactitem}

\paragraph{Publications:}

\begin{compactitem}

\item Bancerek, G., Byliński, C., Grabowski, A., Korniłowicz, A., Matuszewski, R., Naumowicz, A., Pąk, K.:
``The role of the {M}izar {M}athematical {L}ibrary for interactive proof development in {M}izar''.
Journal of Automated Reasoning \textbf{61}(1), 9--32 (2018).
\\\url{https://doi.org/10.1007/s10817-017-9440-6}

\item Bancerek, G., Byliński, C., Grabowski, A., Korniłowicz, A., Matuszewski, R., Naumowicz, A., Pąk, K., Urban, J.:
``Mizar: State-of-the-art and Beyond''.
Intelligent Computer Mathematics, International Conference, CICM 2015, Washington, DC, USA, 
July 13--17, 2015, Proceedings., (M. Kerber, J. Carette, C. Kaliszyk, F. Rabe, V. Sorge Ed(s).), 
Lecture Notes in Comput. Sci. vol. 9150, Springer, Berlin, 2015, pp. 261--279.
\\\url{https://doi.org/10.1007/978-3-319-20615-8_17}

\item Kaliszyk, C. Pąk, K.: ``Semantics of Mizar as an Isabelle Object Logic''.
Journal of Automated Reasoning \textbf{63}(3), 557--595 (2019).
\\\url{https://doi.org/10.1007/s10817-018-9479-z}

\item Kaliszyk, C. Pąk, K.: ``Scalable Declarative Proof Translation'',
Tenth International Conference, Interactive Theorem Proving, ITP 2019, Portland,
OR, USA. Proceedings,  LIPIcs, Vol. 141, 35:1--35:7 (2019).
\\\url{https://doi.org/10.4230/LIPIcs.ITP.2019.35}

\item Brown, C.E., Kaliszyk, C., Pąk, K.: ``Higher-order Tarski Grothendieck as a~Foundation for Formal Proof'',
Tenth International Conference, Interactive Theorem Proving, ITP 2019, Portland,
OR, USA. Proceedings,  LIPIcs, Vol. 141, 9:1--9:16 (2019).
\\\url{https://doi.org/10.4230/LIPIcs.ITP.2019.9}

\end{compactitem}

\end{sitedescription}

%%% Local Variables:
%%% mode: latex
%%% TeX-master: "../propB"
%%% End:

% LocalWords:  site-jacu.tex clange sitedescription emph compactitem pn semmath
% LocalWords:  prosuming-flexiformal KohSuc asemf06 GinJucAnc alsaacl09 StaKoh
% LocalWords:  tlcspx10 KohDavGin psewads11 ednote Radboud Bia ystok CALCULEMUS
% LocalWords:  textbf keypubs OntoLangMathSemWeb uwb Deyan Ginev Stamerjohanns
% LocalWords:  searchability
