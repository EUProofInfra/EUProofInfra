\begin{sitedescription}{Min}

ARMINES (\url{http://www.armines.net}), created in 1967, is a
600-employee, non-profit research organization that operates under the
framework of the law of April, the 18th, 2006, on the reform of
research in France. With €42 millions revenues in 2016, ARMINES is the
largest industrial research institution in France and boasts a
computer science research center.

The CRI (Centre de Recherche en Informatique) is a joint Computer
Science Laboratory of ARMINES and MINES ParisTech, located in
Fontainebleau. The CRI has been involved in programming language
research for more than 20 years, mainly in relation with parallel
architecture, and has been developing expertise in formal methods for
analyse, modelization and validation.


\paragraph{Main tasks:}

\begin{compactitem}
\item\ednote{specify the main tasks and reference the respective work packages} 
\end{compactitem}


\paragraph{Relevant previous experience:}

The Computer Science Laboratory of ARMINES has participated to the
French Research Agency's projects BWare (2012–2016), on the
development of a platform for the verification of B proof obligation,
and FEEVER (2013–2017) on the development and formalization of Faust,
a DSL language for audio signal processing.

\ednote{give an overview over previous work and projects that add to the \pn project}

\paragraph{Specific expertise:}

\begin{compactitem}
\item \ednote{give three to five specific areas of expertise that pertain to the \pn project}
\end{compactitem}

\paragraph{Staff members undertaking the work:}

\textbf{Pr. Olivier Hermant} is an expert in Deduction Modulo
Theory. The study of the properties of its logics (semantics,
completeness, cut admissibility and proof normalization algorithms)
have led him to contribute to the development of Zenon modulo (PhD
thesis of P. Halmagrand) and to be involved in the development of
Dedukti (PhD thesis of R. Saillard). He has supervised numerous internships, 5 PhD
theses (1 ongoing) and 1 postdoctoral researcher.
\ednote{provide the key publications below} \keypubs{providemore}


\end{sitedescription}
%%% Local Variables: 
%%% mode: latex
%%% TeX-master: "../propB"
%%% End: 

% LocalWords:  site-jacu.tex clange sitedescription emph compactitem pn semmath
% LocalWords:  prosuming-flexiformal KohSuc asemf06 GinJucAnc alsaacl09 StaKoh
% LocalWords:  tlcspx10 KohDavGin psewads11 ednote Radboud Bia ystok CALCULEMUS
% LocalWords:  textbf keypubs OntoLangMathSemWeb uwb Deyan Ginev Stamerjohanns
% LocalWords:  searchability
