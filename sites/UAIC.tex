\begin{sitedescription}{Ias}

%\ednote{a description of the legal entity}

\logo{UAIC.png}

Alexandru Ioan Cuza University of Iaşi is the oldest higher education institution in Romania. 
Since 1860, the university has been carrying on a tradition of excellence and innovation in 
the fields of education and research. With over 24.000 students and 700 academic staff, 
the university enjoys high prestige at national and international level and cooperates with  
over 500 universities world-wide. 

Faculty of Computer Science (FII) was established in 1992 as a natural development of the computer 
science chair of the Faculty of Mathematics. FII has active research groups in the area of 
Data Engineering for Optimization, Evolutionary Computing and Machine Learning, Cryptography,
Natural Language Processing, Applied Distributed Systems, and Formal Methods in Software Engineering (FMSE).

The main goal of the FMSE research group is to develop methods and tools helping software
engineers in applying mathematical-based proof techniques during software development.
The use of the formal methods helps in revealing the inconsistencies, incompleteness,
ambiguities in a system's or language's design. The group’s ambitious is to apply the formal
methods in a mechanical way. The primary strength of the group is the use of the algebraic
specification theory, logics and rewriting techniques. The main achievements include the
development of the CIRC coinductive prover and the major contribution in the design and the
implementation of the K framework.

\paragraph*{Main tasks:}

UAIC will be fully involved in the task Instrument K Prover~\taskref{theories}{matching}. Its main responsibilities in this 
include the translation of matching logic proofs expressed in Kore into Dedukti, and the integration
of these proofs with those generated by the automated provers. Andrei Arusoaie and 
Dorel Lucanu have a rich experience in the development of the foundations and the implementation of the K Framework.
Main achievements related to the project topics include:
\begin{compactitem}
\item  a coinduction-based formalization of the symbolic execution, which is language independent, and its implementation in K Framework 3.4.
\item contributions to the development of the K Framework, an independent rewrite-based language framework in which programming languages, type systems and formal analysis tools can be defined and executed.
\item a proof system for the circular coinduction and its implementation in CIRC, a prover for equations behavioural equivalence. The method was extended to prove the equivalence of the non-deterministic coalgebras.
\end{compactitem}
Rodica Condurache has experience in temporal logic. All these match very well with the goals of the task,  which will be  accomplished by the following two main activities:

%\ednote{its main tasks, with an explanation of how its profile matches the tasks in the proposal}

%\ednote{specify the main tasks and reference the respective work packages}

\begin{compactitem}
\item Translating Kore into Dedukti
\item Assembling the Kore translation with those produced by automated provers in a single proof
\end{compactitem}

\paragraph*{Publications, products or services:}

%\ednote{a list of up to 5 relevant publications, and/or products, services (including widely-used datasets or software), or other achievements relevant to the  call content}

\begin{compactitem}
\item ``Unification in Matching Logic'', by Andrei Arusoaie, Dorel Lucanu. FM 2019: 502-518
\item ``A generic framework for symbolic execution: A coinductive approach'', by Dorel Lucanu, Vlad Rusu, Andrei Arusoaie. J. Symb. Comput. 80: 125-163 (2017)
\item ``A language-independent proof system for full program equivalence'', by Stefan Ciobaca, Dorel Lucanu, Vlad Rusu, Grigore Rosu. Formal Asp. Comput. 28(3): 469-497 (2016)
\item ``Language definitions as rewrite theories'', by Vlad Rusu, Dorel Lucanu, Traian-Florin Serbanuta, Andrei Arusoaie, Andrei Stefanescu, Grigore Rosu. J. Log. Algebraic Methods Program. 85(1): 98-120 (2016)
\item ``Circular Coinduction: A Proof Theoretical Foundation'', by Grigore Rosu, Dorel Lucanu. CALCO 2009: 127-144
\end{compactitem}

\paragraph*{Previous projects or activities:}

%\ednote{a list of up to 5 relevant previous projects or activities, connected to the subject of this proposal}

\begin{compactitem}
\item \href{https://fmse.info.uaic.ro/grant/dak/}{DAK}: An Executable Semantic Framework for Rigorous Design, Analysis and Testing of Systems
\item \href{https://fmse.info.uaic.ro/grant/circ/}{CIRC}: Automated Verification by Circularities
\end{compactitem}

\paragraph*{Infrastructures or technical equipments:}

%\ednote{a description of any significant infrastructure and/or any major items of technical equipment, relevant to the proposed work}

\begin{compactitem}
\item The needed research infrastructure is provided by the ICT Research Center of the Faculty of Computer Science (www.erris.gov.ro/FCSICTRC) and Formal Methods in Software Engineering (FMSE) research group. Since this infrastructure was used to develop the first versions of K, it can support to integrate the proofs generated by the K prover into the Logipedia infrastructure. We will use the FMSE server (fmse.info.uaic.ro) for the webpage of the project. 
\end{compactitem}

\paragraph*{Persons primarily responsible for carrying out the proposed activities:}

\begin{compactitem} % in alphabetical order
\item{\bf Andrei Arusoaie}
Andrei is currently an associate professor at Alexandru Ioan Cuza University. He has received his PhD in 2014 from Alexandru Ioan Cuza University with a thesis on symbolic execution, applied to program verification. He had a postdoctoral research stay at INRIA Lille, France. He has extensive experience with formal verification, including with formal proof tools such as the Coq proof assistant. Andrei also has experience in developing tools for formal methods. He started working on the K framework during his master studies and continued to develop K for 3.5 years. During this period he worked on the K compiler for language definitions, an engine for symbolic execution, and a prover based on symbolic execution. 
Andrei is now part of the FMSE group led by Dorel Lucanu, and he was one of the key members in the DAK research project. Recently, Andrei coordinates a university  research project where his goal is to generate proof certificates for programs that run on the blockchain.

Main achievements related to the project topics:
\begin{compactitem}
\item Co-author of a coinduction-based formalization of the symbolic execution, which is language independent, and its implementation in K Framework 3.4.
\item Contributions to the development of the K Framework, an independent rewrite-based language framework in which programming languages, type systems and formal analysis tools can be defined and executed.
\item Certification of a procedure for program verification based on symbolic execution.
\item Generation of proof certification for unification in Matching Logic.
\end{compactitem}

\item{\bf Rodica Condurache}
Rodica is currently a lecturer at Alexandru Ioan Cuza University. She has received her PhD in 2016 from Universite Paris-Est Paris and Universite Libre de Bruxelles with the thesis Synthesis of interactive reactive systems(Synthese des systemes reactifs interactifs) [C2016]. Both the work during the thesis and the further collaborations provided Rodica good experience in verification and synthesis of reactive systems. 
The main contribution of the thesis consists of procedures to solve (rational) synthesis problems from temporal logic specifications that may lead to efficient implementations. To illustrate the feasibility of some procedures, she also developed a prototype tool. Some of this contribution was part of the project EQUINOCS from the LACL laboratory in Paris Creteil University.
Rodica has also experience in working with temporal logic as ATL to express and verify voting protocols as Three-Ballot. Moreover, more recently she is involved in a project studying the verification of dynamic systems against specifications that may also be given as temporal logic formulas.

After Rodica moved to UAIC, she started to study Matching Logic with focus on deductive proving in of temporal formulas.

\item{\bf Dorel Lucanu}
Dorel Lucanu is currently professor at Alexandru Ioan Cuza University of Iași. His main research interests include rewriting logic, matching logic, coinductive techniques, and their application to supply formal semantics for programming languages, and to develop tools for program verification and analysis. He is the head of Formal Methods in Software Engineering (FMSE) group at the Faculty of Computer Science and he has experience in applying formal methods, with a focus on the formal semantics of programming languages and program verification and analysis. 
Dorel Lucanu coordinated two large related research projects: DAK (together with Grigore Roșu, from UIUC) and CIRC. Within the DAK project the first versions of the K Framework (1.0 - 3.4) were defined, and the main contribution of CIRC is the development of a prover for coinductive properties.
He was a Management Committee (MC) member of the ICT COST Action IC0701 - Formal Verification of Object-Oriented Software (2008-2012) and currently is a MC member of the CA COST Action CA15123 - The European research network on types for programming and verification (EUTYPES), in both actions representing Romania.


%\item{\bf Dorel lucanu} is a PhD Student supervised by Dorel Lucanu. His thesis consists of investigating and implementing (co)inductive proving in Applicative Matching Logic.
\end{compactitem}

\end{sitedescription}

%%% Local Variables:
%%% mode: latex
%%% TeX-master: "../propB"
%%% End:
