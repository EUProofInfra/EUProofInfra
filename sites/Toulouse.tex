\begin{sitedescription}{Tou}

\logo{IRIT}

Institut National Polytechnique de Toulouse (INPT) is a French federation of 7 Higher Schools ("Grandes Ecoles") based in Toulouse. About 7 000 students are present on the 12 INPT sites, and the Engineering Schools awards about 1 300 Engineers diploma and INPT 150 PhD per year. A thousand researchers and research students work within 18 research units, most of which are associated with the CNRS or with the INRA organisations.

Research conducted in Computer Science and Information Technology 
at INPT takes place at IRIT (Institut de Recherche en Informatique de Toulouse), the imposing mixed research unit UMR 5505 at the national level (CNRS, INPT, Toulouse universities 1, 2 and 3). IRIT is composed of 270 researchers and research professors, on a global workforce of 700 people. It holds ERC grants and is involved in the 3IA (French Artificial Intelligence programme) with the Aniti project. 
The 20 research teams, are structured in 7 scientific topics (Information Analysis and Synthesis, Indexing and Information Search, Interaction, Autonomy, Dialogue and Cooperation, Reasoning and Decision, Modelisation  Algorithms and High Performance Calculus,  Architecture, Systems and Networks, Reliability of Systems and Software). 

%\paragraph*{Relevant previous experience:} 

 IRIT Safe software and/or system development group, named ACADIE belonging to the Reliability of Systems and Software group,  aims at improving the costs and delays of software and more generally of system validation. This purpose is achieved by using,  formal approaches (type theory and proof assistants, design of proof certified development methods, distributed systems  modelling, refinement  and proof, domain specific languages, static analysis for distributed object-oriented technologies). It has recognised competencies in the area of formal methods and refinement with expertise in the Event-B and Coq methods. Among the application domains of the conducted work are: first stepwise formal modelling, validation using animation and model checking and its application to substantial case studies related to avionic, robotic and medical domains. Several research projects have supported this work either locally or at the national or European level both with academic and industrial partners. Moreover, the team has also been reinforced by competencies issued from domain modelling, particularly in the engineering area.  
 
 All the researchers involved in LOGIPEDIA proposal are members of the ACADIE research team. They bring their expertise in 1) Type theory based formalisation of development processes and decision procedures; 2) Proof and refinement based formal methods: Event-B; 3) Type theory, type systems as formal proof systems; 4) Formal development and verification of distributed, real-time, hybrid, interactive systems; 5)  Ontology based formal system engineering.


\paragraph*{Main tasks:}

\begin{compactitem}
\item \taskref{instrumentation}{atelier-b} Instrument Atelier-B/Rodin
\item \taskref{structuring}{strrefonto} Reference ontology
\end{compactitem}

\paragraph*{Publications, products or services:}

\begin{compactitem}
\item ``Using Event-B for Critical Device Software Systems'', by Neeraj Kumar Singh. 2013, Springer Book ISBN 978-1-4471-5260-6, published by Springer-Verlag GmbH.
\item ``Making explicit domain knowledge in formal system development'', by Yamine Ait-Ameur and Dominique M\'ery. Science of Computer Programming SCP. Elsevier. Vol 121.  2O16.
\item ``Event algebra for transition systems composition Application to timed automata'', by Elie Fares, Jean-Paul Bodeveix, Mamoun Filali. In : Acta Informatica, Springer, Vol. 1, p. 1-38,  2017. 
\item ``Proof-Based Approach to Hybrid Systems Development: Dynamic Logic and Event-B'', by Guillaume Dupont, Yamine Aït Ameur, Marc Pantel, Neeraj Kumar Singh. 6th International conference on state based formal method ABZ 2018. LNCS 10817. pp: 155-170
\item ``Mechanically Verifying the Fundamental Liveness Property of the Chord Protocol'', by Jean-Paul Bodeveix, Julien Brunel, David Chemouil, Mamoun Filali. In : Formal Methods (FM 2019), LNCS  11800, pp. 45-63, 2019.
\end{compactitem}

\paragraph*{Previous projects or activities:}

\begin{compactitem}
 \item  \textbf{EBRP [2019-2023]} is a French ANR funded project, coordinated by IRIT,  whose purpose  is to enhance Event-B and the corresponding RODIN tool set by engaging in some basic research dealing with various mathematical theories that are not currently available in Event-B and RODIN. In order to validate this research a number of new case studies, issued from various engineering domains, will be performed. 
\item  \textbf{DISCONT [2018-2022]}  is a French ANR funded project dealing  with the verification of the correctness of such hybrid systems formal models including both discrete and continuous behaviours.  Correctness should arise from a design process based on sound abstractions and models of the relevant physical laws. In this project, the IRIT partner has developed a generic Event-B model together with Event-B theories for continuity and control theory. It has been applied to several hybrid systems. 
\item \textbf{MOISE [2015-2019]} (MOdels and Information sharing for System engineering in Extended enterprise) is a French IRT St Exupéry funded project. It targets the upstream part of the design cycle: after capturing user needs from the product definition phase and before the detailed design of system components. In this project, the IRIT partner has
formalised architectural patterns with variability and their
application to component-based models. Event-B has been used as the underlying semantic framework.
\item \textbf{Comet IntegR [2015-2018]} In this Franco-Austrian project, the IRIT partner has developed a correct-by-construction Event-B model for characterising realisable choreographies. A formal Event-B development based on refinement and proof has been designed to synthesise asynchronous distributed systems communicating through FIFO channels whose behaviour is equivalent to a defined choreography. 
 \item  \textbf{FORMEDICIS [2017-2020]}  is a French ANR funded project dealing with the design of critical human-computer interactive systems. The targetd application domain is aerospace interactive systems. A formal modelling language, named Fluid, has been designed to fill the gap between high level requirements and concrete interactive systems. Many formal verification techniques, among them Event-B (contribution of IRIT), have been applied on Fluid Models. 
\item \textbf{IMPEX [2013-2016]} Implicit and Explicit Semantics Integration in Proof Based Developments of Discrete Systems. 
The objective of IMPEX is to build formal models that explicitly take into account the context of the system under development. The correct system behaviour is then represented as a ternary relation between the requirements, the system, and its context. The project started in December 2013 and is funded by the French ANR.  INPT-IRIT participates to this project coordinated by LORIA-VeriDis.
 \end{compactitem}

\paragraph*{Infrastructures or technical equipments:}

\begin{compactitem}
\item Neeraj SINGH  has  developed the EB2ALL toolchain (downloaded more than 1500 times since 2011)  for code generation  from Event-B formal specification to in multiple programming languages.
\end{compactitem}

\paragraph*{Persons primarily responsible for carrying out the proposed activities:}

\begin{compactitem} % in alphabetical order

\item{\bf   Yamine Ait-Ameur}    is full professor at INPT in Toulouse and member of the ACADIE research group at IRIT. 
 His research addresses model heterogeneity reduction. Two main important aspects characterise his work. On the one hand the fundamental aspects are studied through the use of formal modelling techniques based on refinement and proof (in particular using Event-B), explicit formalisation of semantics using formal ontology models. On the other hand, practical aspects are addressed through the development of operational applications, allowing validating the proposed approaches. Embedded systems in avionics, engineering, interactive systems, CO2 capture, cyber physical systems are some of the application domains targeted by this work.  
Yamine Ait-Ameur has participated to several national and European research and industrial projects.   He has published several research papers, edited special issues of international journals and he is the member the program committee of well established conferences in formal methods and ontology based modelling.

\item{\bf  Jean-Paul Bodeveix} is full professor of Computer Science at Universit\'e Toulouse III - Paul Sabatier  and member of the ACADIE research team. His research interests concern formal development
methods, semantics of programming or modelling languages and formal verification of real-time systems and combine the use of model checking, theorem proving and refinement-based development. His Coq-based developments are related to the meta-modelization of synchronous or real-time models for verifying semantic properties (determinism, timed-bisimulation). His work on refinement-based development are supported by the B method and are about the verified development of protocols such as Chord. Work around real-time model checking was based on model transformations.

\item{\bf  Mamoun Filali-Amine}  is a CNRS researcher at IRIT (Toulouse Institute for Research in Computer Science)
within the ACADIE team (Assistance to the Certification of Distributed and Embedded Applications).
His research focuses mainly on specification, development by refinement and validation of distributed and real-time algorithms using  assisted theorem provers and/or automatic provers. In the recent years, he has been more particularly interested in critical real-time embedded systems and more particularly in the study of architecture languages and the application of formal methods in this area.

\item{\bf   Neeraj Kumar Singh} is an Associate Professor at INPT and member of  ACADIE research since September 2015. He leads his research in the area of theory and practice of rigorous software engineering and formal methods to design and implementation of safe, secure and dependable critical systems related to automotive, medical, avionic and nuclear domains. He is an active participant to the Pacemaker Grand Challenge. He holds PhD in computer science from University of Lorraine, France (2011), entitled  \textit{"Using Event-B for Critical Device Software Systems"}. From  2012 to  2013, he was a research associate in the Computer Science Department of University of York, UK, working on the  EPSRC funded project: High-integrity Java Applications using Circus (HiJaC). From 2013 to August 2015, he was a research fellow and team leader in the Centre for Software Certification (McSCert) at McMaster University, Canada, working on Ontario Research Fund - Research Excellence (ORF-RE) funded project: Certification of Safety Critical Software-Intensive Systems, and Automotive Partnership Canada (APC) funded project: Centre is the Network for the Engineering of Complex Software-Intensive Systems (NECSIS) for Automotive Systems.  The results of his research works are published in more than 45 refereed articles in well known journals, books and international conferences.

\end{compactitem}

\end{sitedescription}

%%% Local Variables:
%%% mode: latex
%%% TeX-master: "../propB"
%%% End:
