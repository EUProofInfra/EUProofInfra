The construction of such a system-independent online encyclopedia of formal proofs requires a cooperation between teams that have experience in
1. the development of proof systems,
2. interoperability between proof systems (point-to-point and larger scale), including concept alignment, that is the identification of isomorphic structures and notions, such as Cauchy and Dedekind real numbers,
3. the development and use of logical frameworks,
4. the development of large libraries of proofs.

We have identified 19 groups in 10 European countries ready to
contribute to this effort~: Belgrade, Bialystok, Bologna, Cambridge, Clearsy, Erlangen, G\"oteborg, Innsbruck, Liège, TU München, LMU München, Nancy, Paris, Prague, Saclay, Sophia-Antipolis, Strasbourg and Toulouse.

These include researchers working on the development of proof systems
(such as Agda, Coq, Isabelle, and Mizar) researchers working on
interoperability (for instance, between HOL Light and Coq or between
proof systems and automated theorem provers: hammers), researchers
working on concept alignment (in particular using machine learning to
detect similarities), researchers working on databases of theorems
(such as the MMT library), researchers working on the development of
Dedukti and Logipedia, researchers working on large libraries (such as
the Isabelle Archive of Formal Proofs, MathComp, and GeoCoq).


We also are in contact with non-European colleagues at SRI, Nasa
Langley Research Center, Amazon WS, the University of Iowa, and Data61
working on HOL Light, PVS, LFSC, and seL4.

The proposal will focus on

1. Hiring doctoral students, post-doctoral researchers, and engineers to solve the above mentioned problems.

2. Organizing a yearly meeting to present the state of the art of the development of the encyclopedia.

3. Organizing smaller meetings where four to eight researchers work on a specific theory or library.

4. Building a permanent advisory board where industrial partners, and
international academic partners (including non-European ones) will
discuss the future of the encyclopedia. This board will include among
others 
\begin{itemize}
\item June Andronick (Data61, Kensington NSW), 
\item Denis Cousineau (Mitsubishi Electric), 
\item Thomas Letan (ANSSI), 
\item Jacques Fleuriot, 
\item Natarajan Shankar (SRI),
\item Aaron Stump (Iowa), 
\item Laurent Voisin (Systerel).
\end{itemize}


\subsection{Risks and opportunities technical, organizational, external, management}

1. Risks

- more difficult that expected (probability: low, severity medium). Mitigation: at least some systems

- too many specific proofs (probability: low (empirical evidence in informal maths), severity medium). Mitigation: do not translate these proofs and start understanding why they need strong axioms, but translate the rest (basic maths).

- too high complexity (time and memory), difficulty to scale up (probability medium, severity medium). Mitigation: lower the objectives (basic maths, use more powerful machines, wait for Moore's law to help you).

- one partner leaves (probability low) or does not deliver (probability medium). Mitigation: downsize the project.

- difficulty to find people (doctoral students, post-docs) in some countries. Mitigation use the size of the network to find more peoples in others.

- Logipedia splits into several libraries: face the risk  (we have avoided to have a classical and a constructive logipedia, a predicative one and a non-predicative one, the diversity of theories expressed in logipedia permits to make the probability very low). 

- a beautiful encyclopedia, but nobody cares (probability low). Mitigation: improve the interface, the communication, make it more completed

2.  Opportunities

- More people want to join: model checking, sat solvers… 

- math teachers want to use it for teaching


