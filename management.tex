The Logipedia consortium will gather 31 beneficiaries and partners
from 11 European countries during four years. The project management
structure will be tailored to the specificities and needs of this
large consortium and its ongoing network development.

\subsection{Organisational structure}

\subsubsection*{The project management team}

{\bf The Scientific and Technical Coordinator}: the scientific and
technical coordinator is responsible for the coordination of
scientific and technical activities in order to meet the objectives
set by the European Commission in the Grant Agreement. The scientific
and technical coordinator works closely with the work package leaders
within the steering committee, in order to monitor the progress of the
scientific and technical work and identify potential risks within each
work package. The scientific and technical coordinator will daily
collaborate with the European project manager in charge of the
day-to-day management of Logipedia. The project will be managed by Pr
Gilles Dowek, permanent senior researcher at Inria Saclay. He will
also chair the meetings of both the general assembly and steering
committee.

{\bf The European Project Manager}: The European project manager
member of the Technology Transfer and Partnership Office of Inria
Saclay, is in charge of all administrative, financial and legal
management tasks as listed in \WPref{management}. The
European project manager is the interface between the project and the
European Commission as it represents the point of contact for the
European Commission. The European project manager has the overall
administrative and financial responsibility for the organisation and
administrative and financial monitoring of the project.

{\bf The Chief Engineer}: The chief engineer is an experienced
research engineer from Inria Saclay and is responsible for ensuring
the development and maintenance of tools at Inria Saclay and
supervising the development tasks achieved at the other
beneficiaries. The chief engineer will ensure the coherence of the
Logipedia tools development, according to the defined schedule in the
Grant Agreement.

Innovation management and intellectual property rights issues will be
handled by Inria and the European project manager, supported by the
experienced Technology Transfer and Partnerships Office of Inria
Saclay. The project management team will establish appropriate
policies and rules for the management of intellectual property rights
for the knowledge developed within the project, as well as the
identification of the opportunities for the exploitation of the
project results in innovation activities. Issues related to innovation
and/or intellectual property rights management will be tackled at
every steering committee meeting.

\subsubsection*{The operational level}

{\bf The Steering Committee}: The steering committee is composed of
the scientific coordinator, the chief engineer, the European project
manager and the work package leaders. The steering committee is the
supervisory body for the implementation of the project. The steering
committee is responsible for monitoring the activities of the project
and the implementation of decisions taken by the general assembly. It
can formulate proposal for changes in the description of action and
the related consortium budget. Those changes will have to be agreed
by the general assembly first and then the European
commission. The steering committee is chaired by the scientific
coordinator.

{\bf The Work Package Leaders}: The work package leaders are
responsible for the monitoring and management of the activities and
results within their work packages. In particular, work package
leaders i) identify deviations from the project plan and report them
to the steering committee, ii) manage and supervise the preparation of
reports and their timely delivery, iii) control and monitor activities
of tasks and regularly meet once per month with task leaders, iv)
manage the information flow with other work packages via the steering
committee.

{\bf The Task Leaders}: The task leaders are responsible for
coordinating the scientific and technical work in their task and
making the day to day technical decisions that solely affect their
task. Inter-task decisions are coordinated with the work package
leaders.

{\bf The Club Leaders}: The club leaders are in charge of
disseminating of the tools developed by the Logipedia consortium in
various communities. They organize the activity of the club. They give
ongoing feedback to the consortium during the course of the project.


\subsubsection*{The strategic level}

{\bf The General Assembly}: The general assembly is composed by one
representative of each partner within the consortium, with each
representative having one vote. Every new partner will have a voting
right. The general assembly will gather at least once a year, and as
many virtual meetings as needed. The general assembly is the main
governance and ultimate decision-making body of the consortium. The
general assembly must review the project progress, decide on
contingency actions in case of deviations from the plan and take final
decisions on policy and contractual issues and conflicts as requested
by the steering committee.

{\bf The Advisory Board}: The advisory board is a consultation body to
the steering committee and general assembly. It will bring external
and non-legally binding perspective on the scientific and technical
development of the project, ecosystem building and the future of the
encyclopedia. The advisors of this board will attend the yearly
general assembly plenary meeting and will be consulted on the strategy
of the project. The advisory board should aim at representing the
stakeholders of the Logipedia ecosystem without including any
beneficiary or associate partner’s employees. It will be composed of,
among others, industrial and international academic partners
(including non-European ones) such as
\begin{compactitem}
\item June Andronick (Data61, Kensington NSW), 
\item Denis Cousineau (Mitsubishi Electric), 
\item Thomas Letan (ANSSI), 
\item Jacques Fleuriot, 
\item Natarajan Shankar (SRI),
\item Aaron Stump (Iowa), 
\item Laurent Voisin (Systerel).
\end{compactitem}

 
\subsubsection*{Internal communication and collaborative ecosystem}

The communication of the consortium including their internal tools is
managed in task 10.2.  The consortium will make use of a number of
project management tools, such as a visio conferencing tool, a project
repository to have an updated account of the project’s important
documents, the progress of the work packages work and deliverables,
all the advances in the project and all the meetings minutes, mailing
lists, etc. that facilitate the smooth execution of the project. This
collaboration environment will be provided by the coordinator of the
project.

Work packages, chaired by work package leaders, will have monthly
planned visio conferences and meetings as need by the work plan;
additional technical meetings may be set up by task leaders or
individual partners. The steering committee will have monthly visio
conferences and will meet every two months. Dedicated working groups will
be planned as needed according to the work plan.  All meetings will be
documented by minutes listing major decisions and action items.


The project management team will be in charge of all organisation
issues in the general assembly meetings, supported by the local
partner. The project will organise meetings of the general assembly at
least once a year. To equally share travel costs among partners,
physical meetings will be located by rotation at partners’
locations. Project review meetings will be done on a regular basis
according the Grant Agreement provisions.

\subsection{Decision-making Process}

Our approach for the decision-making process is to locate the decision
as close as possible to the level responsible for the execution (from
task level to general assembly level). Decisions are managed within
frequent project meetings, either on-site or via
teleconference. Decisions can be also managed by consultation. If
voting is needed, the agenda should clearly indicate this fact. Quorum
and voting rules will be defined in the Consortium
Agreement. Decisions are binding once the relevant part of the meeting
minutes has been accepted. Any changes to the project plan and scope
must be reviewed and approved by all levels of project management,
before proposing these changes to the steering committee and any
modifications will be considered rejected, after rejection on any of
these involved levels.

Another guiding principle is to avoid conflicts. Nevertheless, should
one arise, a conflict resolution will be ready to be put in place to
deal with it accordingly. The conflict resolution foresees that each
conflict will be mediated, solved or decided at the lowest level
possible. Attempts to solve issues within the consortium will be
carried out in increasing order of authority first at task level
(management of task leader), work package level (management of work
package leaders), and then following the management bodies till the
general assembly. Further rules related to conflict resolutions will
be laid out in the Consortium Agreement.

\subsection{Monitoring and reporting}

\subsubsection*{Internal reporting}

The project management team continuously monitors the project plan
with its milestones and critical paths. Each work package leader will be
responsible for the correct execution of the implementation plan for
the corresponding work package. In terms of reporting, this means the work package leaders
will be in charge of gathering the information related to their own
work packages.

Regular audio-conferences of the Steering Committee are foreseen,
which allows work package leaders to identify and raise risks and
discuss them together. This ensures that management (coordination,
European project manager) is aware of potential problems and
deviations and can initiate countermeasures long before a situation
becomes critical. This ensure to spot the blocking points in due time
and to find that the solutions will be available in time.

In case there is a deviation from the work plan, the scientific and
technical coordinator will initiate corrective actions through the
task leader and the work package leader. The work package leader will
be responsible to implement these actions in dialogue with the
different partners involved in their work packages.


\subsubsection*{Reporting to the European Commission}

The Logipedia consortium will follow the mandatory reporting period
required by the European Commission. The following reporting will be
achieved: Period 1 (M01-M18), Period 2 (M19-M36) and Period 3
(M36-M48).

The project management team will provide the necessary templates in
order to achieve the reporting in due time. Work package leaders will
be asked to gather the relevant information provided by the task
leader regarding their work package and to summarise in order to be
reviewed by the steering committee. It will then be treated by the
scientific coordinator and European project manager and sent to the
European Commission.

\subsection{Significant Risks and Associated Contingency Plans}\label{sec:risks}

\begin{todo}{from the proposal template}
  Describe any significant risks, and associated contingency plans
\end{todo}
\begin{oldpart}{need to integrate this somewhere. CL: I will check other proposals to see how they did it; the Guide does not really prescribe anything.}
\paragraph{Global Risk Management}
The crucial problem of \pn (and similar endeavors that offer a new basis for communication
and interaction) is that of community uptake: Unless we can convince scientists and
knowledge workers industry to use the new tools and interactions, we will
never be able to assemble the large repositories of flexiformal mathematical knowledge we
envision. We will consider uptake to be the main ongoing evaluation criterion for the network.
\end{oldpart}



1. Risks

- more difficult that expected (probability: low, severity medium). Mitigation: at least some systems

- too many specific proofs (probability: low (empirical evidence in informal maths), severity medium). Mitigation: do not translate these proofs and start understanding why they need strong axioms, but translate the rest (basic maths).

- too high complexity (time and memory), difficulty to scale up (probability medium, severity medium). Mitigation: lower the objectives (basic maths, use more powerful machines, wait for Moore's law to help you).

- one partner leaves (probability low) or does not deliver (probability medium). Mitigation: downsize the project.

- difficulty to find people (doctoral students, post-docs) in some countries. Mitigation use the size of the network to find more peoples in others.

- Logipedia splits into several libraries: face the risk  (we have avoided to have a classical and a constructive logipedia, a predicative one and a non-predicative one, the diversity of theories expressed in logipedia permits to make the probability very low). 

- a beautiful encyclopedia, but nobody cares (probability low). Mitigation: improve the interface, the communication, make it more completed

2.  Opportunities

- More people want to join: model checking, sat solvers… 

- math teachers want to use it for teaching


{\color{red} Draft a list of milestones}

\subsection{Milestones}\label{sec:milestones}

\begin{todo}{from the proposal template}
  Milestones are control points where decisions are needed with regard to the next stage
  of the project. For example, a milestone may occur when a major result has been
  achieved, if its successful attainment is a requirement for the next phase of
  work. Another example would be a point when the consortium must decide which of several
  technologies to adopt for further development.

  Means of verification: Show how you will confirm that the milestone has been
  attained. Refer to indicators if appropriate. For examples: a laboratory prototype
  completed and running flawlessly, software released and validated by a user group, field
  survey complete and data quality validated.
\end{todo}

\ednote{maybe automate the milestones}

\ednote{Rabe: I suggest having exactly 3 milestones, namely at months 18, 36, and 48 (corresponding to the EU's review schedule), possibly more milestones in the beginning e.g., at months 6 and 12}

\begin{milestones}
  \milestone[id=kickoff,verif=Inspection,month=1]
    {Organization setup}
    {Set up the organizational infrastructure of the project: mailing lists, web site, consortium agreement, activity tracking, \ldots}

  \milestone[id=logipedia-v1,verif=Inspection,month=12]
     {Logipedia v1}
     {Release of a first version of Logipedia with HOL Light standard library and parts of Matita standard library in 5 different systems: Coq, Matita, Lean, HOL and PVS}

  \milestone[id=coq-stdlib,verif=Inspection,month=24]
     {Coq in Logipedia}
     {Integration of most of Coq standard library in Logipedia}

  \milestone[id=isabelle-stdlib,verif=Inspection,month=24]
     {Isabelle/HOL in Logipedia}
     {Integration of most of Isabelle/HOL standard library in Logipedia}

  \milestone[id=compcert,verif=Inspection,month=36]
     {CompCert in Logipedia}
     {Integration of most of the CompCert library in Logipedia}

  \milestone[id=logipedia-v2,verif=Inspection,month=36]
     {Logipedia v2}
     {Release of a second version of Logipedia integrating important parts of the libraries of Isabelle, Coq, Matita and HOL4, and their translations in other systems}

  \milestone[id=atelierb,verif=Inspection,month=48]
     {Atelier B in Logipedia}
     {Release of a tool able to translate a complete development in Atelier B into a complete Dedukti proof}

\end{milestones}
