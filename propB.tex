% The main file for developing the proposal. 
% Variants with different class options are 
% - submit.tex (no draft stuff, no ednotes, no revision information)
% - public.tex (like submit.tex, but no financials either) 
\providecommand{\classoptions}{,notecites} % to be overwritten in variants
\documentclass[12pt,noworkareas,deliverables,report\classoptions]{euproposal}
\extrafloats{1000}
\usepackage{nimbusserif}
\usepackage{logipedia}
\usepackage{framed}
\usepackage{multicol}
\usepackage{longtable}
\usepackage{array}

\addbibresource{lib/kbibs/kwarcpubs.bib}
\addbibresource{lib/kbibs/extpubs.bib}
\addbibresource{lib/kbibs/kwarccrossrefs.bib}
\addbibresource{lib/kbibs/extcrossrefs.bib}
\addbibresource{lib/pub_rabe.bib}
\addbibresource{lib/rabe.bib}
\addbibresource{lib/proposal.bib}
\addbibresource{lib/interface.bib}

\WAperson[id=GiDo, 
           personaltitle=Prof. Dr.,
           academictitle=Professor of ????,
           affiliation=Inr,
           email=gilles.dowek@inria.fr,
           workaddress={????},
           worktel=?????]
           {Gilles Dowek}

% institutions in the same order as on the submission website

\WAinstitution[id=Inr,
        countryshort=FR,
        acronym=Inria]
        {Institut National de Recherche en Informatique et Automatique}

\WAinstitution[id=Str,
        countryshort=FR,
        acronym=Unistra]
        {Université de Strasbourg}

\WAinstitution[id=Tou,
        countryshort=FR,
        acronym=INPT]
        {Institut National Polytechnique de Toulouse}

\WAinstitution[id=Inn,
        countryshort=AT,
        acronym=UIBK]
        {Universität Innsbruck}

\WAinstitution[id=Lie,
        countryshort=BE,
        acronym=ULiege]
        {Université de Liège}

\WAinstitution[id=Bol,
        countryshort=IT,
        acronym=Unibo]
        {Alma Mater Studiorum – Università di Bologna}

\WAinstitution[id=Bel,
        countryshort=RS,
        acronym=UBel]
        {Matematički fakultet, Univerzitet u Beogradu}

\WAinstitution[id=Tum,
        countryshort=DE,
        acronym=TUM]
        {Technische Universität München}

\WAinstitution[id=Del,
        countryshort=NL,
        acronym=TUDelft]
        {Technische Universiteit Delft}

\WAinstitution[id=Sac,
        countryshort=FR,
        acronym=USaclay]
        {Université Paris-Saclay}

\WAinstitution[id=Fau,
        countryshort=DE,
        acronym=FAU]
        {Friedrich-Alexander Universit\"at Erlangen-N\"urnberg}

\WAinstitution[id=Lee,
        countryshort=UK,
        acronym=ULeeds]
        {University of Leeds}

\WAinstitution[id=Got,
        countryshort=SE,
        acronym=UGot]
        {G\"oteborgs Universitet}

\WAinstitution[id=Cha,
        countryshort=SE,
        acronym=Chalmers]
        {Chalmers Tekniska H\"ogskola}

\WAinstitution[id=Lmu,
        countryshort=DE,
        acronym=LMU]
        {Ludwig-Maximilians-Universit\"at M\"unchen}

\WAinstitution[id=Imt,
        countryshort=FR,
        acronym=IMT]
        {Institut Mines-Télécom}

\WAinstitution[id=Bia,
        countryshort=PL,
        acronym=UBia]
        {Uniwersytet w Białymstoku}

\WAinstitution[id=Cle,
        countryshort=FR,
        acronym=ClearSy]
        {ClearSy}

\WAinstitution[id=Oca,
        countryshort=FR,
        acronym=OcamlPro]
        {OCamlPro}

\WAinstitution[id=Bir,
        countryshort=UK,
        acronym=UoB]
        {University of Birmingham}

\WAinstitution[id=Cea,
        countryshort=FR,
        acronym=CEA]
        {Commissariat à l'Energie Atomique et aux Energies Alternatives}

\WAinstitution[id=Stu,
        countryshort=DE,
        acronym=DHBW]
        {Duale Hochschule Baden-Württemberg}

\WAinstitution[id=Irt,
        countryshort=FR,
        acronym=SystemX]
        {Institut de Recherche Technologique System X}

\WAinstitution[id=Edu,
        countryshort=FR,
        acronym=Edukera]
        {Edukera}

\WAinstitution[id=Med,
        countryshort=AT,
        acronym=MED-EL]
        {MED-EL Elektromedizinische Geraete GmbH}

\WAinstitution[id=Pro,
        countryshort=FR,
        acronym=P\&R]
        {Prove \& Run}

\WAinstitution[id=Zib,
        countryshort=DE,
        acronym=ZIB]
        {Konrad-Zuse-Zentrum für Informationstechnik Berlin}

\WAinstitution[id=Ias,
        countryshort=RO,
        acronym=UAIC]
        {Universitatea Alexandru Ioan Cuza din Iasi}

\WAinstitution[id=Run,
        countryshort=RO,
        acronym=RV]
        {Runtime Verification SRL}

%%% Local Variables: 
%%% TeX-master: "../propB"
%%% mode: latex
%%% End: 

% LocalWords:  WAperson miko personaltitle academictitle privaddress privtel
% LocalWords:  workaddress worktel workfax gc worktelfax pcg pcsa WAinstitution
% LocalWords:  shortname partof streetaddress townzip countryshort efo 3kd89
% LocalWords:  jacobs-logo.png Seefahrtstrasse Kruislann Montparnasse Universit
% LocalWords:  baz Westerfield
% Some sections of the included files depend on this.

\counterwithout{figure}{chapter}

\begin{document}
%{\color{red} version du 5 mars à 10:29}

\begin{proposal}[%
  coordinator=GiDo,
  coordinatorsite=Inr,
  acronym={Logipedia},
  acrolong={},
  title=\pn \protect\pnlong, % sites in the order of the submission site
  site=Inr,
  site=Str,
  site=Tou, 
  site=Inn, 
  site=Lie,
  site=Bol,
  site=Bel,
  site=Tum,
  site=Del, 
  site=Sac, 
  site=Fau,
  site=Lee,
  site=Got,
  site=Cha,
  site=Lmu,
  site=Imt,
  site=Bia,
  site=Oca,
  site=Cle,
  site=Bir,
  site=Cea,
  site=Stu,
  site=Irt,
  site=Edu,
  site=Med,
  site=Pro,
  site=Zib,
  site=Ias,
  site=Run,
  botupPM, % we want to work via bottom up PM distribution,
  % alternative: (can be combined)
  % topdownPM, % the coordinator distributes PM as follows:
  % FAURM=72,
  callname = Integrating and opening research infrastructures\\ of European interest,
  callid = H2020-INFRAIA-2018-2020,
  instrument=instrument,
  challengeid=challengeid,
  challenge=challenge,
  objectiveid=objectiveid,
  objective=objective,
  months=48,
  compactht]

  \newcommand\logo[1]{\includegraphics[width=3cm]{logos/#1}}

  \includegraphics[width=16cm]{map-countries}
  \vspace*{1cm}
  
  \begin{center}
    {\Large\bf Companies willing to join\footnote{Their letters of support are provided in the Appendix.} the club of industrial users:}
  \end{center}

  \vspace*{5mm}
  \begin{tabular}{ccccc}% in alphabetical order
    Alstom
    & \logo{ARM}
    & \logo{CEAList}
    & \logo{ClearSy}
    & \logo{Edukera}
    \\[8mm]
    \logo{Facebook}
    & \logo{IBM}
    & \logo{MED-EL}
    & \logo{MERCE}
    & \logo{NomadicLabs}
    \\[8mm]
    \logo{OCamlPro}
    & \logo{Onera}
    & \logo{OriginLabs}
    & \logo{ProveRun}
    & \logo{TrustInSoft}
    \\[8mm]
    RATP
    & \logo{RV}
    & \logo{SystemX}
    & \logo{Systerel}\\
  \end{tabular}
  
  \begin{abstract}
    max 2000 characters

    the objectives of the proposal

    how they will be achieved

    their relevance to the work programme

    will be used as the short description  of the proposal in the evaluation process and in communications  with the programme management committees and other interested parties

    do not include any confidential information

    use plain typed text, avoiding formulae and other special characters

    use e-infrastructure

    key-words: 200 characters max
  \end{abstract}

  \tableofcontents

\chapter{Excellence}
Bugs kill and, if testing may reveal some bugs, only
formal verification can guarantee their absence.  Thus, we should
never fly in an autonomous plane fully driven by a piece of software,
that has not been formally verified.

Today, the trust in critical systems relies on formal verification, in
particular formal proofs, that guaranty the safety of the people using
transportation systems---autonomous car, subways, trains, planes,
etc.---, health systems---robotic surgery, etc.---, energy provided by
nuclear plants, financial applications, e-governance, etc. The crucial
role of formal proof is highlighted by several successes, like the
correctness proof of the automatic Paris metro line 14 \cite{metro14},
the detect-and-avoid system for unmanned aircraft system developed
by NASA \cite{Munoz16}, the proved operating system seL4 \cite{Klein09},
or the proved C compiler CompCert \cite{Leroy06}. Because formal methods are crucial in the
development of the information society, because people can die and
companies bankrupt because of a bug, it is crucial for Europe to
master this technology and its evolution.

\thispagestyle{empty}

\begin{figure}
\begin{tabular}{ll}
  {\sc \underline{Abella}}~~~~~~~~~~~~~~~~~~~~~~~~~~~~~~&{\sc Acl2}\\
{\sc \underline{Agda}} &  {\sc \underline{HOL Light}}\\
{\sc \underline{Atelier B}} &  {\sc IMPS}\\
{\sc \underline{Coq}}  &  {\sc \underline{Lean}}\\
{\sc \underline{FoCaliZe}}  &  {\sc Nuprl}\\
{\sc \underline{HOL4}}  &  {\sc \underline{PVS}}\\
{\sc \underline{Isabelle}}  &  {\sc \underline{TLA+}}\\
{\sc \underline{Matita}}\\
{\sc \underline{Minlog}}\\
{\sc \underline{Mizar}}\\
{\sc \underline{Rodin}}\\
\end{tabular}
\caption{Some major proof systems. The European ones are in the first column.
  Those addressed in the project are underlined\label{systems}}
\end{figure}

A lot of formal proofs developed for one application could be used in
another.  Unfortunately, the development of formal methods is slowed
down by the large number of proof systems and the lack of a common
theory used by these systems.  Because each small community is
centered around one theory and one system, and each system has its own
library of proofs, interoperability, sustainability, and
crossverification are restricted, and often same work has to be done
again elsewhere.  For instance, the Paris metro line 14 has been proved
correct in {\sc Atelier B}, while the Nasa detect-and-avoid system for
unmanned aircraft system has been proved correct in {\sc PVS}. Some
projects, such as the Flyspeck project, have been started in different
systems and required some effort to be completed in just one system.

There are around twenty major proof systems in the world (Figure
\ref{systems}) and making these systems interoperable would avoid work
duplication, reduce the development time, and allow independent
verification.  After three decades dedicated to the development of
these systems, allowing such a cooperation between these systems is
the next step in the development of the formal proof technology.  A
cloud of formal proofs could bring to the applications of formal proof
technology the same boost that the cloud has brought to computing.

In January 2019, we put online a first prototype of {\sc Logipedia},
an encyclopedia of formal proofs, expressed in the language of five
different systems. At that time, this encyclopedia contained only a
few hundreds proofs. We organized a meeting to discuss the future
of this project {\tt
  http://deducteam.gforge.inria.fr/seminars/190121.html}.  This
meeting brought together 38 researchers from Austria, Czech Republic,
France, Italy, the Netherlands, and Poland. Since then, colleagues
from Belgium, Germany, Serbia, Sweden, and the United Kingdom have
manifested interest. These researchers are ready to contribute to
develop this encyclopedia aiming at having in twenty years all the
formal proofs then developed, in a single encyclopedia.

%%% Local Variables:
%%%   mode: latex
%%%   mode: flyspell
%%%   ispell-local-dictionary: "english"
%%% End:


\section{Objectives}
To foster the interoperability of proof systems and the sustainability
and the cross-verification of formal proofs, we propose to collect
them in an online encyclopedia, called {\sc Logipedia}.  For each
proof, {\sc Logipedia} will indicate in which systems it can be used
and, when this is the case, it will provide a version of this proof in
the theory of these systems.

Such a project will not only foster the use of
formal proofs in research in mathematics and computer science, but
also in industry, by allowing cross-verification, sustainability, and
interoperability of formal proofs, and in education, freeing the
teaching of formal proof technology from being bound to one system. 

Building this encyclopedia is {\em per se} a networking activity
between the academic and industrial partners involved in the project.
This encyclopedia will be freely accessible through a web browser,
from any country in Europe and beyond, and an important effort will be
made to increase its accessibility, with interfaces, search engines,
etc.  This project will also trigger joint research activities between
its users and between its developers.

Such a project can only have a worldwide ambition. However, as a
majority of proof systems are developed in Europe, there is a unique
opportunity for Europe to take the lead on such a project and prepare
the grounds for the economic spinoffs from the project benefitting
European industry. That is why the consortium gathers most of the
European actors active on formal proof systems, while also developing
links with non-European partners.

Currently, we know how to express the theories of {\sc HOL Light}
\cite{Assaf12}, {\sc Matita} \cite{Assaf15}, {\sc Coq} and {\sc FoCaliZe}
\cite{Cauderlier16} in {\sc
  Dedukti} and recheck proofs developed in these systems.  We aim at
including, in twenty years, all formal proofs developed at that time,
and in the next four years, we plan to address the theories of {\sc
  Abella}, {\sc Agda}, {\sc Atelier B}, {\sc HOL4}, {\sc Isabelle},
{\sc Minlog}, {\sc Mizar}, {\sc SMT-Lib}, \tlaplus, {\sc Why3}, {\sc
  LFSC}, {\sc PVS}, and {\sc TSTP}.  Other systems, such as {\sc
  ACL2}, {\sc IMPS}, {\sc Lean}, {\sc Nuprl}, and {\sc Rodin}, are
kept for later, except if some other partners join the project (Figure
\ref{systems}).

Beyond our main focus on interactive systems, we also plan to
integrate some proofs coming from automated theorem provers, SMT
solvers, and model checkers, when these proofs have a manageable
size. We already have experience with Archsat \cite{Bury19}, iProver
\cite{Burel10}, and Zenon \cite{CauderlierHalmagrand15}. We plan to go
further in this direction, in cooperation with our colleagues working
on LFSC \cite{Stump09}.

To measure the level of integration of an existing proof system and
associated proof library in {\sc Logipedia}, we introduce a metric:
{\em the {\sc Logipedia} integration levels} (Figure \ref{lil}) that
counts six levels.  Among the 17 systems we shall focus on we plan to
increase the {\sc Logipedia} integration levels of 10 of them to level
5 or 6.

\begin{figure}
\begin{framed}
  \begin{itemize}
\item[LIL 1:]
The theory implemented in the system has been defined in
the lambda-Pi-calculus modulo theory and in Dedukti.

\item[LIL 2:]
The system has been instrumented so some of its proofs can be exported
and checked in Dedukti.

\item[LIL 3:]
A significant part of the library of the system has been exported and checked in
Dedukti.

\item[LIL 4:]
A tool has been defined to analyze the Dedukti proofs for the system,
detect those that can be expressed in a theory weaker than that of the
system, and translate those proofs into a weaker logic.

\item[LIL 5:]
Certain proofs of the system have been made available in Logipedia.

\item[LIL 6:]
All proofs of the system have been exported, translated,
and made available in Logipedia.
\end{itemize}
\caption{The Logipedia integration levels (LIL)\label{lil}}
\end{framed}
\end{figure}

The various systems addressed in the project are currently at those levels:

\begin{tabular}{ll}
Matita:& level 5\\
FoCaliZe:& level 3\\
HOL Light:& level 5\\
Coq:& level 2\\
Agda:& level 2\\
Atelier B:& level 2\\
Isabelle:& level 2\\
HOL4:& level 1\\
Minlog:& level 0\\
PVS:& level 0\\
Abella:& level 0\\
Mizar:& level 0\\
TLA+:& level 0\\
SMT-Lib:& level 0\\
Why3:& level 0\\
LFSC:& level 0\\
TSTP:& level 0\\
\end{tabular}

and we plan to increase these levels to 

\begin{tabular}{ll}
Matita:& from level 5 to level 6\\
HOL Light:& from level 3 to level 5\\
FoCaliZe:& from level 3 to level 5\\
Coq:& from level 2 to level 5\\
Agda:& from level 2 to level 3\\
Atelier B:& from level 2 to level 5\\
Isabelle:& from level 2 to level 5\\
HOL4:& from level 1 to level 5\\
Minlog:& from level 0 to level 3\\
PVS:& from level 0 to level 2\\
Abella:& from level 0 to level 2\\
Mizar:& from level 0 to level 3\\
TLA+:& from level 0 to level 2
SMT-Lib:& from level to level ???\\
Why3:& from level 0 to level ???\\
LFSC:& from level 0 to level ???\\
TSTP:& from level 0 to level ???\\
\end{tabular}

Finally, we must also structure this encyclopedia: some of the
libraries we start with already have a structure (modules, qualified
names, etc.) that it is important to preserve.

But, in addition, each library contains definitions of natural
numbers, real numbers, etc.\ and, most importantly, logical connectors,
that must be aligned.  All these objectives contribute to building a
new formal proof community, focused on the values of knowledge
exchange and sustainability.

%%% Local Variables:
%%%   mode: latex
%%%   mode: flyspell
%%%   ispell-local-dictionary: "english"
%%% End:



\section{Relation to the work program}
\begin{todo}{from the proposal template}\color{red}
  This section must explain how the proposal addresses the specific challenge and scope of
  that topic, as set out in the work programme. Page 53-55 of the call.
\end{todo}
\subsection{Research e-infrastructures for formal proofs}

Proof systems and automated theorem provers are research
infrastructures, as they allow engineers, computer scientists,
logicians, and mathematicians to build and study formal proofs, just
like particle accelerators allow physicists to build and study
particles. Each proof system comes with its own library, and these
libraries are also part of research infrastructure.

Currently these infrastructures are small, distributed and
disconnected.  Our project aims at building a large, worldwide
infrastructure from the small ones by integrating them in an
encyclopedia of formal proofs.

The integration effort is substantial because the proofs in the
various libraries are expressed in different theories, and because key
definitions are formalized in a different way.  But this integration
effort is doable and contributes to the challenge to bring European
researchers and engineers an effective and convenient access to the
best research infrastructures in order to conduct research for the
advancement of knowledge and technology.

{\color{red}  red : e-infra : not much equipment}

\subsection{A starting community}

This project has never been supported under FP7 or Horizon 2020 calls.
It mobilizes twenty research groups in eleven European Countries, that is
almost all groups working on formal proof technology in Europe.

Some countries have a large number of participants, some others fewer,
reflecting the diversity of maturity of the research on formal methods
in Europe. This project will contribute to develop the formal proof
community in the European countries where it is still inceptive.

This community is also complemented with a club of industrial users and
a club of teachers. 

{\color{red} Not the 42th telescope : a project that is quite unique}

\subsection{Compliance policy with EU regulations}

A data management plan will be provided according to Inria’s
compliance policy with EU regulations. We also engage ourselves to
make the evolution of the {\sc Logipedia} e-infrastructure compliant
with the European charter for access to research infrastructures.

\subsection{Networking activities, transnational access, joint
  research activities}

This project fosters networking activities and joint research
activities. First, between the members of the consortium, as
expressing in {\sc Dedukti} the theories implemented in the various
systems raises research questions, that for most of the systems have
been solved, but for some of them, still need to be addressed.  Then,
between the members of the consortium and other academic partners, in
particular those developing automated theorem proving systems as we
want to include also proofs coming from such systems (see the research
oriented work package 4). Then, between the consortium and the
industrial users of proof systems, because using such a system will
require to improve the user interface, develop a search engine,
concept alignment algorithms, etc.  Finally between the industrial
users themselves, as using proofs developed by others will foster joint
developments.

Such networking activities currently exist in Europe to some extent. {\sc
  Logipedia} is a unique opportunity to strengthen the existing ones and create new ones, through a
joint project.

The encyclopedia in the cloud that is the main goal of this project
will be publicly and freely accessible online through a web browser or some package management tool, so transnational
access is provided directly. Its administration will be decentralized
in various places in Europe.

\subsection{Towards standardization}
This project will be a stepping stone for a possible standardization
of proof languages. Instead of choosing one system as a standard {\em
de facto}, which is always partial, such a cooperative effort will
allow to eventually develop a standard in a cooperative way, bringing
a better and more accepted standard.

\subsection{Inter-disciplinarity}
This project includes mathematicians, logicians, and computer
scientists and therefore is clearly inter-disciplinary.

But the project is inter-disciplinary in a more fundamental
way. Proving properties of a piece of software driving a car or
piloting an aircraft require to formalize part of the physical world
in which this piece of software evolves. In the same way proving
properties of simulation software, requires to formalize some
properties of the simulated object. As mathematics and software are in
any part of modern science, formal proof will spread, over time, in
all areas of science, once it will be standardized and made more
accessible.



%%% Local Variables:
%%%   mode: latex
%%%   mode: flyspell
%%%   ispell-local-dictionary: "english"
%%% End:


\section{Concept and methodology}
{\bf \Large (a) Concept}

\begin{framed}
\begin{center}
{\bf \Large How do proofs contribute to safety and security of software?}
\end{center}
  
Imagine the following casino game. At the beginning, a player is given
eleven euros. At each round, she throws a dice. If the results is a
six, then game is over.  If it is a five, a four, a three, or a two,
she is given twice the amount of money she already has. If it is a one
she is taken two euros, if she has at least two.  When the game is
over, the player wins the money she got, except if she has zero, in
which case she looses one million.

This game can be modeled by program $p$
%import random 
\begin{verbatim}
n = 11
stay = True;
while stay: 
    roll = random.randint(1,6)
    if roll == 6:
        stay = False
    else:
        if roll >= 2:
            n = n + 2 * n
        else:
            if (n >= 2):
                n = n - 2
print(n)
\end{verbatim}

To be on the safe side, the player wants to be sure, before starting
playing, that she will never finish with zero.  And indeed, in all runs,
the content of the variable {\tt n} is always an odd number, thus it
cannot be zero. This property is a consequence of two simple
theorems of arithmetic
$$\forall x~(\mbox{\it odd}(x) \Rightarrow \mbox{\it odd}(x + 2 * x))$$
$$\forall x~(\mbox{\it odd}(x) \Rightarrow \mbox{\it odd}(x - 2))$$
%$$\forall x~(\mbox{\it odd}(x) \Rightarrow \mbox{\it odd}(x))$$
Hence, verifying the safety of this program, that is that the
proposition 
${\mbox{\it safe}}(p)$ holds, 
amounts to prove these three theorems of arithmetic.

A tiny bug in the program, for instance replacing the {\tt 2} by a
{\tt 3} in the the line {\tt n = n + 2 * n} makes the program unsafe
as shown by the sequence $11, 9, 7, 5, 3, 1, 4, 2, 0$. Yet, testing
this program will, most likely, not reveal this bug, that manifests
very rarely.  In contrast, attempting to prove the correctness of this
program will
reveal the bug as it is impossible to prove the proposition
$$\forall x~(\mbox{\it odd}(x) \Rightarrow \mbox{\it odd}(x + 3 * x))$$
\end{framed}
\pagebreak
\begin{framed}
  \begin{center}
    {\bf \Large What is a formal proof? What is a proof system?}
    \end{center}
        
Since Antiquity, we have known that
proofs, both purely mathematical ones, as in Euclid's elements or the
recent proof of the Kepler conjecture by Thomas Hales, and proofs used
to establish the safety and security of software, can be built with a
limited number of rules, for example
\begin{itemize}
\item From $A \Rightarrow B$ and $A$, deduce $B$.
\item From $A$, deduce $A~\mbox{\it or}~B$.
\item ...
\end{itemize}
Yet, through history, most mathematical proofs have been written in
pidgin of natural language and mathematical formulas. When proofs are
very long (as it is often of the proofs used in safety and security,
but also of some proofs in pure mathematics), mistakes in proofs are
very difficult to detect. For instance dozen of wrong proofs of
parallel postulate have been given through history, sometimes by the
best mathematicians such as Ptolémée, Proclus, al-Haytam, Tacket,
Clairaut, Legendre...

In the 1960s, Robin Milner and Nicolaas De Bruijn noticed that the
correctness of a mathematical proof could be checked by a
computer. This led to the development of the two first proof systems
in history: Milner's LCF or De Bruijn's Automath.  From
the axioms
$$\forall x~(\mbox{\em philosopher}(x) \Rightarrow \mbox{\em human}(x))$$
$$\forall x~(\mbox{\em human}(x) \Rightarrow \mbox{\em mortal}(x))$$
we can deduce
$$\forall x~(\mbox{\em philosopher}(x) \Rightarrow \mbox{\em mortal}(x))$$
In the language implemented in Automath, this proof is written
$$\lambda x \lambda h~(g~x~(f~x~h))$$
\end{framed}

\begin{framed}
\begin{center}
{\bf \Large What is a theory?}
\end{center}

Deduction rules such as ``From $A \Rightarrow B$ and $A$, conclude
$B$'' are universal, but building proofs require more rules, that are
often specific to a domain of knowledge and are called
``axiom''. Examples are the axioms of geometry, the axioms or
arithmetic. These axioms constitute a theory.

At the beginning of the 20th century an axiomatic theory, {\em set
  theory}, has been proposed to express all mathematical proofs. In
the first half of the 20th century a few variants of set theory, and a
few alternatives have been proposed (such as Simple type theory).
But, because these theories had not been designed for being
implemented on a computer, the rise of computer checked formal proofs
has led to a multiplication of such alternative theories. Each proof system,
such as Coq, Isabelle/HOL, Mizar, Atelier B,
etc. implements its own theory.

This is the major obstacle to interoperability.
\end{framed}

\bigskip

\noindent
{\bf \Large Logical Frameworks}

We had, in the past, an (informal) proof of Pythagoras’ theorem
or Fermat’s little theorem. The same proof now has different
formalizations in PVS, Isabelle/HOL, Coq, etc., jeoprardizing the
universality of logical truth.

In the 19th century, the universality of logical truth has been
jeopardized in a similar way, by non-Euclidean geometries, as some
statements could be true in one geometry, but false in others.  But,
at the beginning of the 20th century, a solution was found.  The
definition of the various geometries in predicate logic
\cite{HilbertAckermann} restored the universality of logical truth,
allowing to determine which axiom was used in which proof and which
theorem held in which geometry.

Predicate logic is not a theory in itself, but it is a framework in
which one can express theories, as sets of axioms: {\em a logical
  framework}.  Expressing the various geometries in predicate logic
allowed to analyze which proof contained which axiom, and hence which
proof could be expressed in which theory.

In 1928, predicate logic was a huge success, since three important
theories used at that time (geometry, arithmetic and set theory) could
be expressed in it. But it also has limitations, which explains that
another of the major theories used at that time (Russell's type
theory, from The Principia Mathematica) has not been expressed in
it. Since then, several other theories, such as Church's type theory
\cite{Church40}, Martin-L\"of's type Theory \cite{Martin-Lof84}, and
the Calculus of constructions \cite{CoquandHuet88}, have also been
defined as autonomous theories. These theories are those implemented
in most of the current proof systems, yet they cannot be expressed in
predicate logic.

This failure has led, in the field of proof systems, to the
abandonment of predicate logic, and even of the concept of logical
framework: the theories implemented in Coq, Matita, HOL Light,
etc. are often defined as autonomous systems, and not in a logical
framework.

However, a different line of research has attempted to understand the
limitations of predicate logic and to propose other logical frameworks
repairing them. The most prominent limitations of predicate logic are
the lack of function symbols binding variables, the lack of a syntax
for proof terms, the lack of a notion of computation, the lack of a
notion of cut for axiomatic theories, and the impossibility to express
constructive proofs. These limitations have led to the development of
logical frameworks such as $\lambda$-Prolog \cite{NadathurMiller88,
  MillerNadathur12}, Isabelle \cite{Paulson90}, the $\lambda
\Pi$-calculus (also called the ``Edinburgh logical framework'')
\cite{HarperHonsellPlotkin91}, Deduction modulo theory
\cite{DowekHardinKirchner03, DowekWerner03}, Pure Type Systems
\cite{Berardi88,Terlouw89}, and ecumenical logics
\cite{Prawitz15,Dowek15,PereiraRodriguez17}.

All these logical frameworks have been unified in the $\lambda
\Pi$-calculus modulo theory \cite{CousineauDowek07}, implemented in
the system Dedukti \cite{Assaf16}
on which Logipedia is based. Geometry, arithmetic and set
theory, but also Russell's type theory, Church's type theory,
Martin-L\"of's type theory, and the Calculus of constructions can be
expressed in this framework.

For instance, although the details are not important, Church's type
theory can be expressed in Dedukti , with 8 déclarations and 3
rewrite rules.
\begin{framed}
$\begin{array}{rcl}
type&:&Type\\
\eta&:&type \rightarrow Type\\
o&:&type\\
\mbox{\it nat}&:&type\\
\mbox{\it arrow}&:&type \rightarrow type \rightarrow type\\
\varepsilon&:&(\eta~o) \rightarrow Type\\
\Rightarrow&:&(\eta~o) \rightarrow (\eta~o) \rightarrow (\eta~o)\\
\forall&:&\Pi a:type~(((\eta~a) \rightarrow (\eta~o)) \rightarrow (\eta~o))\\
\\
(\eta~(\mbox{\it arrow}~x~y)) &\longrightarrow& (\eta~x) \rightarrow (\eta~y)\\
(\varepsilon~(\Rightarrow~x~y)) &\longrightarrow& (\varepsilon~x) \rightarrow (\
\varepsilon~y)\\
(\varepsilon~(\forall~x~y)) &\longrightarrow& \Pi z:(\eta~x)~(\varepsilon~(y~z))
\end{array}$
\end{framed}

So the theories implemented in various systems can all be expressed in
Dedukti and the proofs developed in these systems can be translated to
Dedukti. Just like in the case of non Euclidean geometry, this allows
to analyse the symbols and rewrite rules used in each proof
\cite{Thire18,Dowek17} (a domain traditionally called ``reverse
mathematics'' \cite{Friedman76,Simpson09}) and to deduce in which
systems each proof can be used.  This analysis is the basis of the
interoperability between proof systems.


\bigskip

\noindent
{\bf \Large The Logipedia integration levels}

\medskip

Thus, to make a formal proof, developed in some system $X$, accessible to
the users of other systems, the first step is to express the theory
$D[X]$, implemented in the system $X$, in the logical framework
  Dedukti.  Then, we must instrument the system $X$ so that the proof
can be exported from it, as a piece of data, expressed as a proof in
$D[X]$ and included in Logipedia. Next, we need to analyze this
proof in order to determine which symbols, axioms and rewrite rules of
$D[X]$ it actually uses and, thus, in which alternative theories it
can be expressed.  Finally, we must align its concepts with the
definitions already present in Logipedia and decide where it
fits in the general structure of the encyclopedia.

To measure the level of integration of an existing proof system and
associated proof library in Logipedia, we introduce a metric:
{\em the Logipedia integration levels} (Figure \ref{lil}) that
counts six levels.  

\begin{framed}
\begin{center}
{\bf The Logipedia integration levels (LIL)\label{lil}}
\end{center}

\begin{itemize}
\item[LIL 1:]
The theory implemented in the system has been defined in
the $\lambda\Pi$-calculus modulo theory and in Dedukti.

\item[LIL 2:]
The system has been instrumented so some of its proofs can be exported
and checked in Dedukti.

\item[LIL 3:] A significant part of the library of the system has been
  exported and checked in Dedukti.

\item[LIL 4:] A significant part of the library of the system have
  been made available in Logipedia.

\item[LIL 5:]
A tool has been defined to analyze the Dedukti proofs for the system,
detect those that can be expressed in a theory weaker than that of the
system, and translate those proofs into a weaker logic.

\item[LIL 6:]
All proofs of the system have been exported, translated,
and made available in Logipedia.
\end{itemize}
\end{framed}

The sixteen systems addressed in this project currently have different
integration levels, and we have different targeted levels for them.. 

\begin{center}
\begin{tabular}{|l|c|c|}
\hline
System & current level & targeted level\\
\hline
Matita & 5 & 6\\
\hline
HOL Light & 3 & 5\\
\hline
FoCaLiZe & 3 & 5\\
\hline
Coq & 3 & 5\\
\hline
Agda & 2 & 4\\
\hline
Atelier B & 1 & 5\\
\hline
ProB & 1 & 5\\
\hline
Isabelle & 2 & 5\\
\hline
HOL4 & 1 & 5\\
\hline
TSTP & 1 & {\color{red} ???}\\
\hline
Minlog & 0 & 4\\
\hline
PVS & 0 & 2\\
\hline
Abella & 0 & 2\\
\hline
Mizar & 0 & 4\\
\hline
Smart & 0 & 3\\
\hline
TLA+ & 0 & 2\\
\hline
 Why3 & 0 & {\color{red} ???}\\
\hline
LFSC & 0 & {\color{red} ???}\\
\hline
\end{tabular}
\end{center}

These systems can be roughly divided into two groups. Those in the
first half of the array (Matita, HOL Light, 
  FoCaLize, Coq, Agda, Atelier B, ProB, Isabelle,
HOL4, and TSTP) for which we have preliminary results and
that we plan to bring to a very high level of integration. Those in
the second half (Minlog, PVS, Abella, Mizar,
Smart, TLA+, SMT-Lib, Why3, and LFSC) with which we
are starting and for which our goal are less ambitious.

These two groups of systems will be addressed in different
work packages, but both are key to the project. The first ones will
constitute Logipedia in 2024 and the second ones prepare the
long term future of the infrastructure.

\bigskip

\noindent
{\bf \Large Instrumentation}

\medskip

Consider a system $X$ at LIL 1, so the basic theory $D[X]$ has
already been expressed in Dedukti. The next step is to implement a
method to automatically translate proofs from system $X$ to its
expression in Dedukti, and make these proofs available in Logipedia so
they can be exported to other systems. This step is called the
\emph{instrumentation} of system $X$.

The systems considered in this project fall into three broad classes:
\begin{description}

  \item[Systems based on dependent type theory.] Agda, Coq, and Matita
  are theorem provers based on Martin-L\"of's intuitionistic type
  theory.

  \begin{itemize}
    \item Coq is an interactive theorem prover developed at Inria
    since the 1984.  It is based on Type Theory and was used to
    formally verify the correctness of both industrially relevant
    software such as the CompCert C compiler and complex mathematical
    proofs such as the one of the Four Color theorem and the one of
    the Odd Order theorem. In 2013 Coq received the ACM system award.

    \item Matita is an interactive theorem prover developed at the
    University of Bologna and used for teaching logic courses and to
    verify software and mathematical proofs, with special attention to
    predicative foundations. The first generation of the system (up to
    version 0.5.9) was born as a by-product of the MoWGLI FET-Open
    Project, it was compatible with the logic of Coq and it could
    re-use its libraries. It was an important test-bench for the
    integration of Mathematical Knowledge Management techniques with
    Interactive Theorem Proving, featuring for example a library of
    theorems distributed over multiple servers, innovative indexing
    and search techniques and automatic translation of proofs between
    declarative and procedural styles. The second generation of the
    system (up to the current version 0.99.3) was a re-implementation
    from scratch that departed from the logic of Coq and that
    experimented with the most concise ways to implement an efficient
    theorem prover. Several ideas later migrated into Coq. The
    currently available largest library is the formal certification of
    a complexity-preserving and cost-model-inducing compiler from C to
    MCS-51 machine code, developed in the FET project CerCo (Certified
    Complexity).

    \item Agda is a dependently typed programming language and
    interactive proof assistant developed at Chalmers University of
    Technology as well as other places in Europe. The theory of Agda
    is similar to Coq and Matita, but is more focused on interactive
    development and direct manipulation of proof terms (in contrast to
    using a tactic language to generate the proof terms). To support
    the construction of proof terms, Agda provides powerful features
    such as dependent pattern and copattern matching, eta equality for
    functions and record types, first-class universe polymorphism, and
    definitional proof irrelevance. In addition, Agda provides an
    experimental option for extending the language with user-defined
    rewrite rules, which are very similar to the rewrite rules
    provided by Dedukti.
  \end{itemize}

  \item[Systems based on simple type theory.] (HOL4, Isabelle)
  \ednote{We still need some text to go here!}

  The HOL4 proof assistant is home to a few medium to large scale
  specifications and associated proof developments that have value
  outside of HOL4. These specifications include the formal semantics
  of the CakeML language (and its verified compiler) and an extensive
  specification of the ARM instruction set architecture (ISA) as
  formalised by Anthony Fox at the University of Cambridge.

  Isabelle as a logical framework \cite{paulson700} is an intermediate
  between Type-Theory provers (like Coq or Agda) and classic LCF-style
  systems (like HOL Light or HOL4).

\item[Systems based on set theory and first-order logic.] Atelier B,
  Rodin and ProB are platforms or tools to develop models written
  using the B method, a version of set theory expressed in predicate
  logic.  A model in these systems encodes a state machine constrained
  by invariant properties. Verification of the model correctness
  implies to verify some proof obligations produced by a weakest
  precondition calculus. For example, spanning trees algorithms,
  distributed algorithms, access control policies have been formalized
  respectivement in EventB and B method.

  The development process for the B method is based on formal proof:
  proof obligations are automatically generated and must be proven by
  automatic or interactive provers. This can include external solvers
  such as SMT solvers, and (in the case of ProB) constraint solvers
  such as Sicstus Prolog. The difference between Atelier B and Rodin
  lies mainly in the refinement process they implement: in the B
  method used by Atelier B refinement is done by deriving a program,
  while in Event-B used by Rodin refinement is done by defining a
  model of the system and iteratively introducing details. Finally,
  ProB is an animator and model checker: it helps users to gain
  confidence in their specifications, and it can be used as a
  disprover to discover counter-examples to proof obaligations.

  Smart is based on a classical first-order logic extended with rank-1
  polymorphism, algebraic datatypes, recursive definitions and
  inductive predicates. ProvenTools will turn models implemented in
  Smart to proof obligations which can be proved in the system using a
  mix of interactive and automated proving. The resulting proofs are
  reviewable objects, akin to proof traces built of the various atomic
  proof rules supported by ProvenTools' kernel, such as definition
  unfolding, case analysis or equality propagation.
\end{description}
Despite the large differences between these three classes of provers,
there are several common points between them where solutions that work
for one system can also be applied to other systems.
%
Concretely, we plan to tackle the following
research challenges:
\begin{description}

\item[Improving encodings.] Since Dedukti is designed to be a small
  logical framework with a minimal number of features, there are many
  features in the systems we consider that are not supported in
  Dedukti. Instead, such features need to be encoded in some way. In
  this work package, we consider systems for which it is known how to
  encode all features \emph{in theory}. However, depending on the
  choice of encoding, the instrumentation may be very hard or even
  unfeasible in practice. For example:
  \begin{itemize}

    \item Many Agda libraries (as well as some Coq libraries) rely
    heavily on type-directed conversion rules such as eta-equality for
    functions and record types, or definitional proof
    irrelevance. Dedukti does not provide type-directed conversion
    directly, so instead these rules have to be encoded by adding
    additional type information to the proof terms. This can hence
    lead to a large blow-up in the size of those proof terms, and thus
    greatly increase the cost of typechecking.

    \item Coq, Agda, and Matita all provide support for coinductive
    (infinite) structures, which are not supported natively in
    Dedukti. These structures can be encoded by rewrite rules in
    Dedukti, but this may cause the typechecker to go into an infinite
    loop.

    \item A few of the systems rely on AC rewriting (i.e.~rewriting
    modulo associativity and commutativity of certain operations), for
    example to encode universe polymorphism in dependent type
    theories. While Dedukti has support for AC rewriting, the current
    implementation is very slow, which makes typechecking the proofs
    in Dedukti unfeasible.

  \end{itemize}
  Finding better ways to encode such features could have a great
  impact on the quality and speed of the translation process.  We plan
  to investigate two possible approaches to improve the encodings of
  common features of proof assistants in Dedukti. First, we will
  investigate whether some information in the current encodings is
  redundant and can thus be omitted. Second, if this is not possible
  we will investigate what minimal changes need to be made to Dedukti
  itself in order to overcome these limitations.

  \item[Reconstructing transient proof components.] In all theorem
  provers, there is some information that is created during the
  process of checking a proof that is not stored in the final proof
  term. In systems such as Coq, Agda, and Matita, this concerns for
  example the types of each subexpression. In other systems such as
  HOL4, Isabelle, Atelier B, and Rodin, even more information about
  the proof is discarded and only the high-level proof steps remain.
  However, the translation from the system to Dedukti might rely on
  this transient information. Developing a method to reconstruct this
  transient information is thus crucial to the goals of this work
  package.

  In order to instrument systems with transient proof information, the
  proof terms need to be complemented with this transient information
  by either logging or re-synthesizing it on demand. Both approaches
  may be used, depending on the the tradeoff between computation time
  and space for storage.

  However, this approach is insufficent for systems that rely on
  external provers such as Isabelle or Event B. In order to handle
  these cases, we will instrument these external provers so they can
  produce the missing parts of the proof terms. 

  \item[Producing more compact proofs.] When translating a proof from
  some system $X$ to Dedukti, we want to produce proof terms that are
  as small as possible, so it is easy to store them, recheck them, or
  export them to a different system. However, there are at least two
  reasons why proof terms might become large. First, a typical proof
  in system $X$ might be very large (because big parts of it are
  constructed automatically). Second, translating a proof from $X$ to
  its encoding in Dedukti might greatly increase the size of the
  proof, for example by normalizing it or by adding type annotations.
  This means translating the proofs can take a very long time, and
  doing anything with the translated proof will be difficult. This
  also has an obvious impact on the exporting of complete libraries to
  Dedukti, which is the goal of WP3.

  To reduce the size of proof terms produced by the translation to
  Dedukti, we plan to investigate how to avoid unneccessary
  normalization or duplication of (parts of) proofs. We will also
  investigate what parts of each proof can be safely omitted because
  they can be inferred from the rest of the proof.
\end{description}

For each of the systems considered in this work package, we can
identify some preliminary work on translating proofs to Dedukti. Some
of these tools are in a preliminary stage and can only handle toy
examples, while others have been used to export whole libraries but
have not been updated to work with the latest version of the
system. Concretely, we plan to build upon the previous work done on
the following tools:
\begin{itemize}
\item The standard and arithmetic libraries of Matita has been the
  first libraries to be exported to Logipedia using Krajono, a fork of
  Matita. The forked system is also actually the only one able to
  import Logipedia proofs. The choice of Matita as a test-bench for
  Logipedia is easily understood considering that the implementation
  of the 0.99.x series was aimed at obtaining a well-documented,
  minimal but fast implementation of a theorem prover, two order of
  magnitudes smaller than Coq.  We plan to make this translation a
  part of the code of Matita itself so that it is maintained with the
  rest of the system.
\item CoqInE is a prototype tool that can translate Coq proofs to
  Dedukti. Recent work to include advanced features of Coq, such as
  universe polymorphism, has dramaticaly increased the coverage of
  this translation. We plan to make this translation a part of the
  code of Coq itself so that it is maintained with the rest of the
  system.
  \item In the summer of 2019, Guillaume Genestier worked together
  with Jesper Cockx on the implementation of an experimental
  translator from Agda to Dedukti during a research visit at Chalmers
  University in Sweden. This translator is still work in progress, but
  it is already able to translate 142 modules of the Agda standard
  library (about 25\%) to a form that can be checked in Dedukti. This
  exploratory work uncovered several challenges and opportunities for
  further work (see research challenges above).
  \item The inference kernel of Isabelle has already been instrumented
  to output proofs as $\lambda$-terms that can be understood by
  Dedukti. However, this has so far been only used for small examples
  \cite{Berghofer-Nipkow:2000:TPHOL}. The challenge is to make
  Isabelle proof terms work robustly for the basic libraries and
  reasonably big applications.  Preliminary work by Wenzel (2019) has
  demonstrated the feasibility for relatively small parts of
  Isabelle/HOL, but this requires scaling up.
  \item HOL4 has support for exporting proofs to the OpenTheory proof
  exchange format, and there has been some work on importing
  OpenTheory proofs into Dedukti.
  \item In the context of the BWare project, an encoding of the set
  theory of the B method has been provided as a theory modulo, i.e. a
  rewrite system rather than a set of axioms. This encoding is used by
  the automatic prover Zenon modulo which features a backend to
  Dedukti. Thus, as a first step through instrumentation of Atelier B
  and Rodin, proof obligations coming from Atelier B can be proved by
  Zenon modulo producing Dedukti proofs, hence providing a better
  confidence in the proofs produced by the native proof tools of
  Atelier B \cite{Bware}.
\end{itemize}

\bigskip
\noindent
{\bf \Large Automatic theorem provers: SAT/SMT solvers, first-order provers, etc.}

\medskip

We have discussed, in the last section, the instrumentation of proof
systems, that is interactive systems, where the users build formal
proofs with the help of the machine. Automatic Theorem Proving are
another class of systems, where the machine builds proofs without any
human intervention. The proofs built by these systems are often
expressed in simpler theories than those developed using proof
systems, but they are not of a different nature. We will also include
in Logipedia proofs built by such automatic systems, whether they are
called automatic theorem provers, SAT solvers, SMT solvers, etc.

Several reasons explain the importance of these proofs for Logipedia.
{\em Per se}, many formal proofs nowadays rely on the use of automatic
provers. They cover various domains and various kinds of application,
{\em e.g.} combinatorial
mathematics~\cite{DBLP:journals/ai/KonevL15,DBLP:conf/sat/HeuleKM16},
where they are expected to solve one large propositional problem, or
proof of
programs~\cite{DBLP:conf/esop/FilliatreP13,DBLP:journals/pacmpl/ProtzenkoZRRWBD17},
where they are given thousands of small problems in a combination of
quantified theories.

In a complementary way, automatic theorem provers will be used to automatically make a
coherent whole out of Logipedia. A fruitful interaction between
formal proofs requires low-level glue which falls in the scope of automatic theorem provers.
For instance, they will be employed to fill the holes that appear when
considering provers with various granularity, to reduce the gaps between
proof systems, and to discharge proofs of concept alignment (in close
interaction with the work package 6).

Finally, automatic theorem provers will also benefit from
Logipedia. Obviously, Logipedia will be an extensive library of formal
statements that can be used and combined by automatic theorem provers
in their proof search. This is not trivial though: lemma selection is
crucial to avoid an overhead. It can also be a source of benchmarks to
evaluate their expressivity and automation.  Lastly, Logipedia and
Dedukti will form a framework to make automatic theorem provers
cooperate with each other and with other tools, in a safe way.

PF: TODO: Talk somewhere above about the various automatic theorem
proving tools, their specificity and why they are each of them useful
or would benefit from Logipedia

\medskip

\noindent
{\bf \large From automatic theorem provers to Dedukti}

Similarly to Interactive Theorem Provers, connecting ATPs to the Logipedia
infrastructure strongly relies on the ability for ATPs to import statements from
Dedukti and export proofs in some theory in Dedukti.

\begin{description}
\item[Instrumenting ATPs to produce proof traces.] The first step for connecting ATPs to the Logipedia infrastructure, and the library of proofs, is that ATPs should actually output some kind of proof, without enforcing strong requirement on the format or even the level of granularity of those proofs.  SAT (Satisfiability) solvers for propositional logic do have a perfectly well specified format for proof traces~\cite{TODO} and most SAT solvers are actually able to produce proof traces in that format.  So, considering SAT solvers, the current status is actually meeting the Logipedia needs on this aspect.

  Considering other automated reasoners however, it is mandatory to significantly improve the picture.  Some SMT (Satisifiability Modulo Theories) solvers do produce some kind of proof trace, but the format is specific to the solver, and the proofs are sometimes difficult to replay due to their granularity.  Most mainstream first-order theorem provers output proofs in a standardized language~\cite{TODO}, but this language does not clearly specify the semantics of the proof steps.  Model-checkers do not provide proof traces per se, although such traces would be useful for various usages and notably certification.

  Our goals are to improve reasoning tools to demonstrate the feasibility to
  produce sufficiently detailed proofs for connecting to Logipedia, and to
  design a set of theoretical methods and practical tools that can be used to
  further connect Logipedia to the other existing automated reasoners of the
  same kind.  We will work on three SMT solvers (alt-ergo, CVC4, veriT), two
  first-order provers (the E-prover, Zipperposition), and one model-checker
  (cubicle).  We will also consider more specific reasoning tools, with the aim
  to demonstrate that this approach also applies to more specific reasoners.  We
  envision to experiment the approach on a Coherent Logic reasoner.  These
  reasoners are particularly useful for geometry reasoning, and as such, they
  are quite complementary to the other considered automatic reasoners.  They are
  also very suitable as as a first experiment, since their proofs are fairly
  straightforward to interpret.  They thus constitute a very interesting low
  hanging fruit.
  
  It is not possible, and not even desirable, to require all tools to directly
  talk in the language of Logipedia.  Indeed, proof trace languages that are
  specific to one kind of reasoning tool are more appropriate than Dedukti for
  instrumenting already large pieces of software, enabling quick output, and
  allowing proof post-processing at the right level of abstraction on the
  produced proof traces.  Furthermore, provided that those proofs are detailed
  enough, translation of traces to Dedukti will not be a difficult task, and a
  the work necessary to translate proof traces for a myriad of very different
  reasoners will be implemented in a unique tool (Ekstrakto, see below) to take
  advantage of the fact that reasoning techniques, and thus proof methods, are
  themselves pretty shared among the solvers.

\item[Translate ATP traces into Dedukti] As pointed out above, it is easier to
  instrument provers to make them output traces instead of directly provide
  Dedukti proofs. The second step to connect ATPs to the Logipedia
  infrastructure is to reconstruct the proof traces in order to build Dedukti
  proofs from them. The proposed process is the following: each step of the
  trace is transformed into an independent subproblem; each of these subproblems
  is given to a prover that can output Dedukti proofs; proofs of the subproblems
  are then combined to produce a global proof of the original problem.  Since
  subproblems correspond to atomic steps of the proof trace, they are relatively
  simple, so that we are confident that the prover producing Dedukti proofs will
  not struggle to find a proof. This process is quite similar to what is done by
  the hammer tools of interactive theorem provers (Sledgehammer in Isabelle/HOL,
  HOLyHammer for HOL4, etc.) which reconstruct proofs from traces produced by
  automated theorem provers.

  This scheme has already been prototyped in a tool called
  \href{https://github.com/Deducteam/ekstrakto}{Ekstrakto}. Ekstrakto takes a
  TSTP file, as can be produced by e.g.\ the provers E and Zipperposition, and
  it uses Zenon Modulo and ArchSAT to prove the subproblems. Ekstrakto was
  designed to be agnostic w.r.t.\ the prover producing the trace; in particular
  it does not depend on the specific set of inference rules of the prover. It
  was also designed to be agnostic w.r.t.\ the prover used to prove the
  subproblems; it is only required that the prover can output a Dedukti proof in
  the correct encoding of first-order logic.

  Although Ekstrakto has already shown that it is a valuable approach, it is a
  work in progress. In particular, the following issues will be addressed in the
  project:

\begin{compactenum}
  % extension to other proof trace formats
\item Up to now, Ekstrakto can only understand traces in TSTP format as
  input. We plan to make it accept traces in other formats, notably traces
  from SMT solvers, as well as all formats that will appear when instrumenting
  other ATPs.

  % unprovable steps

\item  Some steps in the proof traces are not provable: their conclusion is
  not a logical consequence of their premises. However, they preserve
  provability: the original problem has a proof if and only if the
  problem with the conclusion of the step also has a proof. This is the
  case for instance of the Skolemization step in first-order automated
  theorem provers, of the introduction of new definitions, as well as
  the RAT property in traces produced by SAT solvers. The approach of
  Ekstrakto cannot be used here, because the subproblem corresponding to
  the step cannot be proved. However, since provability is preserved, it
  should be possible to transform a
  proof using the conclusion of the step into a proof using its
  premises. Such a transformation depends on the nature of the step that
  has been used. We plan to include in Ekstrakto a way to handle
  Skolemization and definition introduction, which are the two step
  families that are missing to be able to manage all traces from the
  major first-order theorem provers.

  % specialization for theories

\item  Dedukti-producing provers used by Ekstrakto, namely Zenon Modulo and
  ArchSAT, are meant for pure first-order logic. However, we would like
  to deal with proof traces that use some specialized theory,
  e.g.\ arithmetic or bit-vectors, as could be output by SMT
  solvers. Although such theories could be presented as a set of axioms
  in first-order logic, it is almost certain that neither Zenon Modulo
  nor ArchSAT could be able to find non-trivial proofs using these
  axioms. Here, the idea would be to develop small provers dedicated to
  a particular theory, and outputting Dedukti proofs. Such provers would
  be called when a step in the trace relies on said theory. These
  provers need not be very optimized, since trace steps are relatively
  small; this should help producing Dedukti traces. A way to achieve
  this could be to extend Zenon Modulo: indeed, Zenon modulo can find
  proofs modulo arithmetic, but it is not able to produce a Dedukti
  proof yet.

\end{compactenum}
  
\end{description}

\noindent
{\bf \large From Dedukti to automatic theorem provers}
%\label{concept:wp4:deduktitoatp}

In the other direction, Logipedia will constitute a source of
knowledge for automatic theorem provers. For this to be affordable, this workpackage will
study the following challenges.

\begin{description}
\item[Translate Dedukti statements into automatic theorem provers
  inputs] Automatic theorem provers are mostly based on (parts of)
  first-order logic.  Logipedia theorems, which mostly come from
  interactive provers, will be expressed in the Dedukti encodings of
  much more expressive logics, such as dependent type theory or simple
  type theory.  Theorem statements thus need to be encoded again to be
  manipulable by automatic theorem provers.

  Encodings from expressive logics to first order logic already exist,
  and are used for instance in {\em hammers} for using automatic theorem provers into
  interactive theorem
  provers~\cite{DBLP:conf/lpar/PaulsonB10,DBLP:journals/jar/CzajkaK18}.
  In these works, the encodings are specific to one system to be
  affordable. In Logipedia, we have to combine and take benefit
  from statements coming from the encodings of different systems based
  on different logics. The key challenge here are thus:
  \begin{itemize}
  \item to avoid the loss of meaning coming from a succession of
    encodings; and
  \item to encompass statements coming from different systems.
  \end{itemize}
  We plan to investigate a new approach where, instead of hammers, the
  encoding is a succession of fine-grained encodings dedicated to one
  aspect. These ``small'' encodings will offer the possibility to be
  activated independently depending on the origin of the statement. They
  give the other advantages of being modular, easily extensible, and
  more reliable: each encoding is simple, and may output proofs, {\em
    e.g.} using Ekstrakto. It will also crucially rely on the concept
  alignment provided by the workpackage 6.

  It is common in the automatic theorem proving community to evaluate the performance on
  sets of benchmarks. As a side effect of this task, a new set of
  benchmarks will be extracted from Logipedia to measure the
  performance of automatic theorem provers on problems coming from different logic, and from
  a combination of these logics. In our case, it will not only allow to
  compare automatic theorem provers but also to measure the success of this task by evaluated
  the quality of encodings, and comparing our ``small'' encodings by
  activating them or not.

\item[Logipedia as a source of knowledge for automatic theorem provers]
  Pascal

Some input from Innsbruck and Prague:
\begin{itemize}
\item the amount of knowledge available in the whole of Logipedia is
  significantly larger than in any of the individual proof assistant
  libraries, and as such selecting the knowledge relevant for a
  particular goal might require more complex techniques than
  before~\cite{Irving-deepmath}.
\item we will experiment with machine learning techniques for formal
  proofs. This includes both techniques for selection of relevant
  knowledge for a goal and for the selection of most promising
  constructors in the direct proof term
  construction~\cite{ZielenkiewiczSchubert2016}.
\end{itemize}


\end{description}


\noindent
{\bf \large Large scale application: formal verification of C code}

The use of formal methods in the industry requires to have approaches
that apply to a wide variety of cases. For the verification of C code,
the Frama-C platform features numerous techniques. One of them relies on
automatic solvers: Frama-C-WP. Since automatic theorem provers are built by making many
choices such as which heuristics or which algorithms to use, they
display blind spots where they are less efficient to solve some cases.
In order to overcome that, it is necessary to use the wide variety of
solvers which exist in a portfolio manner. The Why3 tool features the
ability to send problems to lots of provers in a uniform way. However
with so many tools written by different teams involved, the meaning of
the same concept in the different tools could be different. It would
lead to errors in the overall verification results. Moreover in order to
be applied more widely, formal methods must handle more concepts such as
floating points where their exact meaning is less clear that
mathematical integers. That leads to more opportunities for two tools to
interpret differently the same concept. Finally, an industrial user
sometimes needs to add some reasoning or simplifications for their
particular concept so that it is better handled by automatic solvers. It
is very easy to make an error in those simplifications.

In order to overcome these problems, we propose to gain insurances in
the interaction of these different tools and to speed the addition of
new features to handle new cases by using proof objects:
\begin{itemize}
\item Where the industrial user needs to define the concepts he wants to
  verify in its C code, we will add the possibility to import concepts
  from Logipedia. The user gain time by not having to define them
  himself and it ensures that we have proofs for the accompanying
  lemmas.
\item Where the simplification rules written for the industrial case are
  executed in Frama-C-WP, we will instrument it to generate Ekstrakto
  input in order to produce a proof of the correction of these
  simplifications.
\item Where the problems are transformed to fit the different provers in
  Why3, we will inspire from the fine-grained encodings (previous
  paragraph) to also generate Ekstrakto proof.
\item In the end, we will gather the Dedukti or Ekstrakto proof of the
  provers, the encodings, and the simplification rules, in order to
  assemble them in a coherent whole.
\end{itemize}
This part thus combines and validates most of the previous aspects of
this workpackage, but also constitutes the challenge to instrument a
large-scale formal tool.

\medskip

\noindent
{\bf \large Automatic theorem provers to increase Logipedia readiness}

Providing automation on the level of an encyclopedia of formal proofs is
both interesting and challenging. Indeed, making so much formal systems
cooperate requires a lot of proof manipulation, transformations, gluing,
that can only be reasonably done with automation. On a higher level,
being able to more efficiently write proofs directly in Dedukti would be
of interest to the community in itself. Making this automation
practicable in Logipedia raises the question of scalability.


\begin{description}
\item[Automatic theorem provers for Dedukti]

  Automation has been a key to the success of proof assistants, since it
  allows much more efficient formalization~\cite{Hales-Developments}.
  Many different techniques to provide proof assistant automation have
  been developed over the years: techniques based on rewriting,
  tableaux~\cite{Paulson-blast}, or even the integration of efficient
  superposition-based first-order~\cite{hurd-metis} and higher-order
  provers~\cite{asperti-matita-paramodulation}. Most recently, various
  machine learning techniques have been developed for proof assistants
  allowing for more precise selection of relevant
  knowledge~\cite{blanchette-h4qed-jfr} or even the prediction of useful
  proof techniques~\cite{gauthier-tactictoe}.

  The automation techniques offered by different proof assistants differ
  significantly. For systems mostly based on classical first-order logic
  it is often possible to provide sound and complete translations to the
  most efficient first-order ATPs by only expressing the intricacies of
  their type systems~\cite{kaliszyk-miz40}. This allows for
  straightforward native proof reconstruction. On the other end of the
  spectrum, providing efficient automation for systems based on involved
  foundations, such as the intuitionistic type theory behind Coq and its
  variants, providing even a slightly useful automation is a significant
  challenge. Most efficient automation still relies on translations to
  external tools, however the useful translations from such logics to
  the logic of SMT solvers~\cite{DBLP:conf/cpp/ArmandFGKTW11} or
  first-order logic~\cite{DBLP:journals/jar/CzajkaK18} are incomplete
  and often (partially) unsound. This means that to reconstruct such
  proofs in the logic of the proof assistant separate intricate
  components need to be developed.

  As part of the project we will experiment with various kinds of proof
  automation on the level of Dedukti and verify their applicability to
  the different libraries imported as part of the Logipedia project.
  Furthermore, we will experiment with machine learning techniques for
  formal proofs. This includes both techniques for selection of relevant
  knowledge for a goal and for the selection of most promising
  constructors in the direct proof term
  construction~\cite{ZielenkiewiczSchubert2016}. Finally we plan to look
  at computational proof reconstruction, to complement the link to
  automated theorem provers developed in the other parts of the
  workpackage.

\item[Automation for full proof verification in Logipedia] The
  aforementioned automation will be crucial to enable full proof
  verification. First, considering systems one at a time, a number of
  proof exports defined in workpackages 1 and 2 are not complete, that
  is the actual systems do not provide all the proof step details and
  internal Logipedia automation would be necessary.

  Under progress
\end{description}


\subsubsection{Description of the work package 3}

\subsubsection{Description of the work package 7}

\subsubsection{Description of the work package 9}

\subsubsection{Description of the work package 2}

\subsubsection{Description of the work package 5 and 6}

{\color{red} Goals}

The development of formal reasoning tools has led to an ever-growing
corpus of mechanized mathematical theories. As these libraries span a
wide spectrum of proof assistants, the underlying logical foundations
of these systems and libraries can different greatly. For example,
some proof assistants use set theory as a foundation while others use
first-order logic, higher-order logic, or different variants of type
theory.  In order to combine the power of different proof assistants
and their respective communities, it is critically important that
these different logical foundations can be related to each other in
ways that ensure their interoperability.  Even when one has
successfully aligned these different logical foundations, one must
also align the many different mathematical concepts and theorems that
are in the libraries produced by the various proof assistants. Thus,
successful interoperability of both proof assistants and formalized
mathematical libraries starts with identifying the correspondences
between the concepts underpinning these formalizations, i.e., their
alignment.

In the simplest case, an alignment is a pair of symbol identifiers
from two different libraries such that both symbols are ``the same
informal mathematical concept''. For example, although real numbers
can be defined by Cauchy sequences, Dedekind cuts, and also stream
representations due to corecursion, all such different formalizations
introduce essentially the same mathematical concept. More precisely,
following the work of \site{Fau} and \site{Inn}
\cite{GKKMR:alignments:17} and the alignment-based mediator developed
by \site{Fau} in the OpenDreamKit project \cite{ODK:mitm:18}, we
consider an alignment to be a pair of
\begin{compactitem}
  \item $2$ identifiers $c,c’$ (usually from two different libraries
    $L,L’$), which are considered to represent the same mathematical
    concept
  \item additional data that governs how $L$-expressions using $c$ and
    $L’$-expressions using $c’$ correspond to each other.
\end{compactitem}
The source of alignments can vary and include manual collections
\cite{MRLR:alignments:17} and automatic, machine-learned collections
\cite{align_kaliszyk}.

Even when the various different proof assistants are all exporting their
proofs in Dedukti, the proper alignment of the theories underpinning
those systems is an essential task: for examples, simply taking the
union of all the different theories can create an inconsistent and,
therefore, useless combination.
%
Alignments can also be useful for joining formalizations done within a
single prover, since different formal libraries developed within a
single system can introduce different definitions of the same
mathematical concept (each more suitable to the library where it is
introduced).

To explain the need for concept alignment, we can compare this aspect
of the Logipedia project to the virtualization of computer operating
systems. In such virtualized systems, it is not sufficient for the
user to be able to run system A inside system B, she also needs to
have access to some device of the host system as device of the guest
system. For example, all operating systems have the concept of file
systems, keyboards, mice, etc, but they are all using technically
different definitions of these concepts.  For virtualization to
succeed, significant work is required to build bridges that formally
connect the different versions of these concepts.

The goals of this work package are first to understand the concept of
alignment, then to identify alignments across the various proof
assistants, and then to build tools to bridge these alignments. We
will develop standards, tools, and techniques that will allow concepts
to be aligned so that formal proof developments from one proof
assistant can be meaningfully used in other systems.

{\color{red} Content}

Concept alignment can happen at different levels:
\begin{itemize}
\item alignments of logical foundations,
\item alignments of theorem proving objects deals with aligning types,
  constants and theorems,
\item alignment of proofs deals with aligning tactics and
  inference rules that have similar effects on the state of proof
  development.
\end{itemize}
Within this work package, we will primarily focus on the alignment of
logics and theorem proving objects (the first two of these levels).

% The following is used instead of \paragraph in the alignment WP.
\newcommand{\parag}[1]{\medskip \noindent {\bf #1}} 
% To revert, uncomment the following line
%\newcommand{\parag}[1]{\paragraph{#1}}

\parag{Aligning logical foundations}
In order to align logical foundations we will develop novel ways of
relating theories represented in Dedukti, building on extensive
experience of the team in type theory, computer-assisted
formalisation, proof theory and reverse mathematics. We will
articulate our work according to the following key parameters for
distinguishing between theories: classical vs intuitionistic logic,
predicative vs impredicative definitions, treatments of equality, and
presence and absence of certain additional axioms, e.g., the axiom of
choice.

One of the major goals of this task will be the formulation and
representation in Dedukti of a core logical system that is both common
to all logical systems and supportive of modular analysis of its
extensions. We believe that such a logical system can be obtained as a
type-theoretic version of geometric logic, a certain well-behaved
fragment of first-order logic.  Such a treatment of geometric logic
will be of interest because we
believe that, over theories in such a logic, we can transform
classical proofs, using the law of excluded middle and the axiom of
choice, into constructive proofs.  This transformation will not also be
a theoretical result but also have practical value since we 
will seek to obtain feasible algorithms to
transform proofs based on classical principles to constructive proofs.
This effort will add a new dimension to a long-standing development of
proof theory that is still very much unexplored.

A further reason for the interest in this type theory is that it could
provide, with a sequence of extensions, a new setting for the
development of reverse mathematics in a type-theoretic setting. This
setting is in contrast to the standard reverse mathematics developed
in the context of second-order arithmetic. Again, the presence of
proof terms (encoded using Dedukti) adds a new dimension to a
well-established topic.

One consequence of this work is that it can make features of one
system available in other systems. For example, Minlog implements the
refined Dragalin-Friedman $A$-translation, so that classical proofs in
a certain class can be constructivised, which means that programs can
be extracted from such transformed proofs.  We would like to make this
feature of Minlog available to other proof assistants through Dedukti.
In order to carry it out, we will study both-directed encodings
between a subsystem of Dedukti and Minlog.

\parag{Aligning theorem proving objects (case studies)}
Since sets, functions, relations and numbers are ubiquitous in all
formalized mathematics, we will pay special attention to their
alignment. Additionally, geometry is a very interesting case study,
since it is traditionally introduced in many quite different ways
(both analytically, using various number fields, and synthetically,
using different axiom systems), and we shall pay special attention
to aligning different existing formalizations of geometry.

There are several ways to define numbers. For the natural numbers,
one can, for example, choose a definition based on Peano’s axioms which
relies on a unary representation of numbers which is suitable for
proving theorems. But for computing, a definition based on a binary
representation is better. For real numbers and functions, their
definitions often differ even between different libraries of the same
proof assistant.  Real numbers, for example, can be defined using 
Bishop style modulated Cauchy sequence, regular Cauchy sequence,
Dedekind cut, and stream representations using corecursion.
Similarly, geometry can be defined synthetically using Hilbert’s,
Euclid’s or Tarski’s axioms or analytically using coordinate systems and
algebra.  Therefore, our first step will be to import into Dedukti the
equivalence theorems between these different axiomatizations. In
practice we want to use a theorem $\Gamma \vdash_S T$ expressed in
system $S$ based on some axioms $\Gamma$ in a system $S’$ based on
different axioms $\Gamma'$ and different definitions.

\parag{Automated theory alignment}
Although alignments can be manually discovered and described, the
sheer size of existing formal libraries forces us to develop automated
support for finding and organizing alignments.


We propose a workflow based on database and semantic web
methodologies. Specifically, we aim to build an automatic alignment
engine on top of an ontology framework that defines mappings between
base concepts belonging to different theories. Relying on a graph of
dependencies that indicates how the theories are structured within a
given library, we set to propagate the alignments with the help of a
certified matching-based engine. The engine will provide a way of
inferring new alignments, based on the alignment of the primitive
concepts and on a set of rules that will describe further
derivations. In order to enable flexible alignments, we are prepared to allow
not only exact matchings, but also matchings with respect to a given
similarity measure, as in \cite{???}.

An alternative method for discovering ontologies of alignments is
to use unsupervised learning methods to find alignments between proof
assistant repositories. For this, heuristic algorithms and dynamic
processes have been tried in the past \cite{???}.  As part of the
project we intend to also try to use unsupervised machine translation
algorithms \cite{???} to directly find correspondences between
statements and their constituent constants and types in proof
assistant libraries imported in Logipedia.

\parag{Alignment based services}
This task is concerned with the implementation of useful services
enabled by the discovered alignments. Each service implemented on top
of Dedukti---such as accessing a library via browsing or searching
or interoperability and reuse provided by translations across
libraries---will benefit from working up-to-alignments. For example, a
user of system $L$ can go to Logipedia to find a theorem about $t’$
about concept $c’$, defined in $L’$ that is missing in the library of
$L$. Even if $L$ contains the corresponding concept $c$, because $t’$
is stated and proved in the logic of $L’$, it will use a
different definition of $c’$ in $L’$ than the one of $c$ in $L$.
Therefore, $t’$ cannot be directly used in $L$. By translating $t’$
along the alignment $(c,c’)$, it induces a theorem $t$ that can be
used in $L$. The details are subtle both in theory and in practice,
and this task will explore how to scale up the alignments found in
\taskref{alignment}{alignlogic}-\taskref{alignment}{aligntheories} to
provide strong services.

This task will leverage this simple idea in multiple different
ways. Firstly, we will use alignments to search across libraries. In
its simplest form, this involves a simple search interface into which
the user enters the term $t$ and the system finds $t’$ because it uses
the alignment $(c,c’)$. Secondly, we will develop translation services
that use alignments to port proofs across libraries. This translation
will allow porting the proof of $t’$ in $L’$ to a (potential) proof of
$t$ in $L$.

% DM continue here

{\color{red} PROBLEMS}

Often there are many notions that are tightly connected, but not
equivalent. For example, some definitions are only special cases of
the others, sometimes there are definitions valid in any dimension and
some that are valid only in some dimension, sometimes definitions
differ only in the treatment of some corner cases (e.g., Hilbert's
axioms of geometry use strict betweenness, but Tarski's axioms use
non-strict betweenness relation). Although such alignments are also
needed, they are very hard to detect and use. An automatic prover
would fail because the two concepts are not equivalent, but they are
related in the sense that equivalent up to some special cases. Some
equivalences are highly non-trivial but for some others we can expect
to prove them automatically.

{\color{red} CONTRIBUTION TO NETWORKING, ACCESS AND RESEARCH}

Concept alignment between such a large body of formal libraries and
tools has never been attempted before, so it requires a joint research
activity of a large number of researchers all around Europe. Alignment
based services developed within this work package will significantly
facilitate access to the available body of formalized mathematical and
computer science knowledge.

{\color{red} RELATION TO THE OTHER WORK PACKAGES}

For geometry, we will rely on the \taskref{libraries}{geocoq}
(GeoCoq), and for constructive analysis, we will rely on
\taskref{theories}{minlog}. We envisage a collaboration with
\WPref{structuring}, aimed at reusing the ontology framework
integrated in the core of Dedukti. We also plan to collaborate with
\WPref{atpetc} on developing a case study regarding ATP proof
obligation discharge modulo alignment. It will be crucial to make
automatic theorem provers benefit from Logipedia as a library.

{\color{red} THE WAY TO KNOW YOU HAVE SUCCEEDED }

The final product of this workpackage are alignment aware services
that enable searching and browsing modulo alignment, and also enable
automated translation of theorem proving objects and proofs. However,
before such an ambitious goal could be achieved, many lessons must be
learned and several milestones must be achieved.
\begin{itemize}
\item A very precise definition of alignments should be given and a
  format (and an ontology) for describing and organizing alignments
  must be described.
\item Proof theoretical results relating different logical foundations
  must be obtained and an efficient algorithm for translating
  statements from one logical system to another must be implemented.
\item Alignment of basic mathematical objects (sets, functions,
  relations, numbers) must be ensured.
\item A method for automated detection of alignments must be devised,
  implemented and applied on a large corpus of formal libraries.
\item A case study of geometry must show that it is possible to align
  large formal theories developed in different theorem provers, based
  on different approaches (synthetic vs analytic).
\end{itemize}

{\bf (b) Methodology}

The methodology is the same for integrating any library to Logipedia,
but due to a difference of readiness of the various systems we focus on,
our priorities are different.

\begin{enumerate}
\item We already know how to express most of the theories of Atelier B,
Coq, FoCaLiZe, HOL Light, HOL4, 
Isabelle, Matita, and Rodin in  Dedukti. We propose
to instrument these systems so that they can produce Dedukti
proofs that we can include in Logipedia.

\item For other theories, such as those of Abella, Agda, Lean, Minlog,
  Mizar, PVS, TLA+, the work is in progress, or not yet started.  So
  we must first understand how they can be expressed in Dedukti.

\item Besides the standard libraries of these systems, large libraries
  have been developed: the Isabelle Archive of formal Proofs \cite{AFP},
   Flyspeck \cite{Flyspeck}, MathComp\cite{Mathcomp}, 
  CompCert \cite{Compcert},  CakeML \cite{CakeML}, ...  We aim to include
  some of these libraries in Logipedia.
  
\item
Besides proof systems, we also want to include proofs coming from
automated theorem provers, SAT solvers, SMT solvers, and model
checkers.  So we must instrument some of these systems so that they
can produce Dedukti proofs that we can include in Logipedia.

\item
We want to develop algorithms to analyze which symbol, axiom, rewrite
rule is used in each proof, and consequently in which system each proof
can be used. We also want to develop algorithms to eliminate some of the 
symbols, axioms, and rewrite rules used in a proof in order to be able to 
use it in more systems.


\item
Each library imported in Logipedia will come with its own
definition of natural numbers, real numbers, etc. We want to develop
``concept alignments algorithms'' to transport theorems from one
structure to another isomorphic one.

\item 
Besides data, we propose to include in Logipedia, metadata and
an inner structure.
\end{enumerate}

\subsubsection{Methodology of work package 1: instrumentation}

We know how to express in Dedukti the theories implemented in Matita,
HOL4, Coq, Agda, Isabelle and Atelier B, and some of these systems
have been partially instrumented to export proofs that can be checked
in Dedukti. Our first work package is to complete this instrumentation
to be able to export most of the proofs of these systems. As a
consequence, the work includes a strong practical component.

Three methods have to be used here: some of the systems (Automath
style), such as Coq and Agda already have proof-terms that can be
output, thus the main task is to translate these proofs into the
Dedukti format. Others (LCF style), such as Isabelle and HOL4, have an
inference kernel that can be instrumented, the main task here is to
transform the internal proof-object into an external
proof-term. Others, such as Atelier B are slightly more difficult to
address. For those, we need to use the water ford method: extract an
incomplete trace (a sequence of lemmas) and fill the gap using
automated theorem proving, as experimented with Atelier B and Zenon.

% A technological hurdle in the instrumentation of the systems under
% consideration is that all of them are actively developed systems
% that are constantly evolving.

\subsubsection{Methodology of the work package 4}

\subsubsection{Methodology of the work package 3}

\subsubsection{Methodology of the work package 7}

\subsubsection{Methodology of the work package 9}

\subsubsection{Methodology of the work package 2}

\subsubsection{Methodology of the work package 5 + 6}

\paragraph{Aligning logical foundations}

Our first step will be to divide logical theories in clusters,
according to the key parameters mentioned above, and then to build a
web of syntactic translations between systems. We do not expect full
back-and-forth translations, as some systems are well-known to be
proof-theoretically stronger than others, but we will seek to
establish suitable equiprovability results for fragments of the
relevant languages.

We shall also deal with specific case studies to test our work.  For
example, to test our methods for translating classical proofs into
constructive ones, one could verify Michael Beeson's "wholesale
importation" (he uses the double negation translation to import all
the negative results from \cite{} to intuitionistic logic) using the
library of GeoCoq proofs.

\paragraph{Aligning theorem proving objects (case studies)}

We call big scale concept alignment the equivalence between different
axiom systems, and small scale concept alignment the equivalence (or
relationship) between different definitions.  Big scale concept
alignment is the alignment of different theories, usually expressed
using different axiom systems (eg., Tarski, Hilbert, Euclid) for
different geometries: euclidean and non euclidean. It also includes
alignment of different kind of analytic definitions of geometry i.e.,
between different analytic models (e.g., real closed fields, reals,
complex numbers), different definitions of projective
geometry. Porting GeoCoq to other proof assistant will bring some of
these big scale concept alignments.  Small scale concept alignment
requires proof of equivalence between different definitions of the
same concept in the same or different language.  To some extent this
task could be automated, but the difficulty is that some equivalences
are valid only in some contexts.

\paragraph{Automated theory alignment}

In one line of research, we set to use logic-deduction approaches that
have been developed for the automatic alignment of database schemas
and instances, such as \cite{}.  In order to align theories across
libraries, we will propagate alignment information, based on a
certified probabilistic inference engine, which will extend the work
in \cite{}. We envisage a collaboration with WP7, aimed at reusing the
ontology framework integrated in the core of Dedukti. Concretely, each
of the domain-specific meta-data, essentially the theory schemas, will
be specified in the ontology framework. Their validity with respect to
the underlying instances will be mechanically checked and the
alignment of the schemas will inform that of the underlying theories.

In the second line of research, we will employ machine learning
techniques, based on neural networks, to design heuristics for finding
new alignments.

\paragraph{Alignment based services}

\textbf{Expression Translation.}

\textbf{Search Service.}

\textbf{Proof Rewriting.} In the recent literature there are two main
approaches to alignment based proof rewriting. The first one, based on
logical relations, has been proposed to translate a proof about $X$
into a proof about $X'$ remaining in the same logic. A relation is
established between elements of $X$ and elements of $X'$ where, for
example, $X$ can be the type of sorted lists of numbers, $X'$ the type
of balanced search trees and the relation holds between data
structures that contain the same set of integers. Then,
oversimplifying, for each pair of corresponding functions acting
respectively on $X$, $X'$, it is shown that they map related elements
to related elements. Continuing the example, the function that inserts
a new integer into the sorted list and the one that does the same into
the balanced search tree map data structures that contain the same
elements to data structures with the same property. Such proofs can be
obtained fully automatically when the functions are obtained
compositionally, without operating directly on the data structures. In
the remaining cases a human needs to provide the proof.

The first methodology is very accurate, but it requires human
intervention and it can be applied only when the alignments can be
simply expressed as relations between types and when everything is in
just one logic. The second methodology is less accurate, but somehow
more practical: it converts a proof into a sequence of intermediate
statements that need to hold, it translates each statement from one
system to the other ignoring the justification for the statement and
it fires an automated prover to fill the gap in the target system. By
varying the level of granularity the automated provers are allowed to
find alternative proofs, for example when some low-level properties of
data structure $X'$ are not available on data structure $X$, but a
different proof can still establish a statement that does not involve
them. Even when the provers fail to fill the gap the proof sketch
obtained can still be useful to the user that can try to manually fill
the gaps instead of restarting from scratch.

The task requires implementing multiple transformations and
translations of expressions containing binders (statements, proof
terms, etc.), which is well known to be a delicate task. ELPI,
developed by a join Ubo-Inr team, is a very high level programming
language of the logic programming family that allows to concisely
manipulate expressions with binders eliminating the most frequent
sources of mistakes (name capture, for example). ELPI comes with an
interpreter implemented in Ocaml that has been designed to be easily
integrated in other Ocaml based tools, like Coq. We plan to first
integrate ELPI in Dedukti so that Dedukti/Logipedia expressions can be
directly manipulated into ELPI. We also plan to implement means to
call external provers from ELPI via Logipedia translations. Then we
will use a mix of ELPI code and Dedukti rewrite rules to implement
alignment based proof and statement rewriting. Statement rewriting
will find direct application to alignment based search and browsing as
well.


\subsection{Readiness of the project}

This idea of building such a standard for proofs has already been
investigated in the past, such as in the Qed manifesto \cite{Qed94}, but
has produced limited results.

Our thesis is that, since the
Qed project, the situation has radically changed. After
thirty years of research, we have an empirical evidence that most of
the formal proofs developed in one of these systems can also be
developed in another. We understand the relationship between the
theories implemented in these systems much better. We have developed
several logical frameworks, extending predicate logic, in which these
theories can be expressed. And we have developed reverse mathematics
algorithms to analyze which axioms and rules are used in each proofs
and algorithms, such as constructivization algorithms, to translate
proofs from one theory to another.


%%% Local Variables:
%%%   mode: latex
%%%   ispell-local-dictionary: "english"
%%% TeX-master: "propB"
%%% End:


\section{Ambition}
\begin{todo}{from the proposal template}\color{red}
  This section must explain how the proposal addresses the specific ambitions (i), (ii),
  and (iii) on page 54 of the call, that are described in more details in appendix D. page
  107 of the call.
\end{todo}
\subsection{Progress beyond current achievements}


\paragraph{Networking activities}
This project will include networking activities, to foster a culture
of cooperation between scientific communities, that are today often
too centered around one system and one theory. Instead, such a project
will incentivize them to build this encyclopedia together and to always
wonder if the proofs they develop are specific to one system or are
universal. This will also foster a culture of cooperation between
scientific communities and two communities of users: teachers and
industrial partners. This is why the project includes two ``clubs of
users'', one of teachers and one of companies using formal methods.

Besides these two clubs, the research directions will be discussed every
year by the advisory board that will gather {\sc Logipedia} contributors
and users.

The development and maintenance of {\sc Logipedia} will eventually lead
to the discussion of standards for proof languages, even if such an
effort is premature today.

We plan to organize a yearly workshop of {\sc Logipedia} developers
and users, continuing the effort started on the January 2019 meeting.

As explained before, the development of a system-independent and
theory-independent encyclopedia of proofs will trigger new
ways to teach formal proofs to new a audience.


\paragraph{Transnational access}
Our encyclopedia being on line, it will of course be accessible from
every country in Europe and beyond.


\paragraph{Joint research activities}
The project includes two types of joint research activities.  First,
as any infrastructure, it will allow joint research projects
between the users of this infrastructure that will be able to develop
new proofs together using different systems.

Second, as any infrastructure, {\sc Logipedia} raises new research
problems. Some of them have already been solved in the past and
require to be implemented jointly in a first version of the
encyclopedia. Others are newer and will trigger new cooperation
between the teams of the project.


\subsection{Innovation potential}

Formal methods are at a turning point. Several academic and
industrial successes have proved the readiness of the technology, but
this technology takes too much time to be generalized, except in the
transportation industry.

Analyzing this phenomenon, it is clear that the redundancy of the
efforts to develop proof systems and the lack of common theories,
benchmarks, and standards for these systems is a limiting factor.

This effort to integrate the scientific and technological effort
around formal proofs in Europe is a way to address this issue so that 
the economic spinoffs from the project benefits the European industry.

%%% Local Variables:
%%%   mode: latex
%%%   mode: flyspell
%%%   ispell-local-dictionary: "english"
%%% End:
\newpage 

{\color{red} Compare with OpenDreamKit, Software Heritage, ProofPeer,
  Wolfram alfa}






%%% Local Variables:
%%% mode: latex
%%% TeX-master: "propB"
%%% End:

\newpage
\chapter{Impact}\label{chap:impact}

\section{Expected impacts}\label{sec:expected-impact}

As discussed above, the shift from informal, pencil and paper, proofs
to formal computerized proof is a major improvement on the never
ending quest for logical rigor, with a strong impact both on
mathematics, where much more complex proofs can be built, and computer
science, where safety and security can be dramatically improved with
the use of formal methods. But this major step forward also has a
negative side effect: we have moved from a time where we had
(informal) proofs of Pythagoras' theorem or Fermat's little theorem,
to a time where we have (formal) proofs in Coq, in Matita, in HOL
Light, in PVS, etc. of these theorems, jeopardizing the universality
of mathematical truth.

We see this loss of universality of mathematical truth as the main
obstacle to the diffusion of the notion of formal proof, in the
communities of mathematicians and computer scientists, but also
engineers and students. Our long-term goal is to resurrect the
universality of mathematical truth in order to build a strong formal
proof community including specialists and non-specialists such as
working mathematicians, engineers and students.

This requires to express the theories implemented in these systems in
a common logical framework, each with a finite number of axioms and
reduction rules, in order to be able to say, not that a proof is
expressed in one system or in another, but to say which axioms and
reduction rules it uses, as we have been used to since the development
of non-Euclidean geometries.

Having a standard for expressing theories and proofs and resurrecting
this way the universality of mathematical truth will also make proof
systems interoperable and will allow the construction of an on-line
system-independent encyclopedia. More importantly, this will suppress
one of the main obstacles to the diffusion of formal proofs in
mathematics, computer science, industry, and education, just like the
development of the html standard induced a renewal of document sharing
in general and the definition of predicate logic induced a renewal of
logic in the 1930's.

Innovation Formal methods are now an important part of some advanced
industrial projects. For instance, mastering formal methods is key to
give Europe a competitive advantage in conquering the market of
autonomous cars, trains, planes, and drones. But this penetration of
formal methods in industry hits the same obstacle that researchers
often promote one method, theory or system, while their industrial
partners are in search of universality. We expect to make formal
proofs more accessible to industry by avoiding each project to
redevelop elementary proofs, but instead benefit of the formalization
work shared with other communities.


\paragraph{Key exploitable results.}

- Logipedia in itself from TRL 3 to TLR 4

- Representation of theories implemented in various systems

- Rechecking formal proofs for higher Evaluation Assurance Level



\section{Measures to maximise impact}

A detailed {\em Dissemination and communication plan} will be
delivered at month 4 (to be updated in the course of the project as
necessary). That plan will outline all of the activities to be
undertaken, including the identification of partners' participation in
international, European and national conferences and events, relevant
publication channels and opportunities for collaboration and
cooperation with other projects and initiatives. Special attention
will be given to the exploitation strategy to align the dissemination
objectives with the exploitation goals. The Plan will also:
\begin{compactitem}
\item increase the awareness of Logipedia in academia, 
\item use Logipedia as a way to increase the cooperation between academia and
industry,
\item prepare the sustainability and exploitation of Logipedia before the
  end of the project, illustrating the philosophy of 
the 
\href{http://roadmap2018.esfri.eu/media/1048/rm2018-part1-20.pdf}{2018
  roadmap} of the European Strategy Forum on Research Infrastructures
(ESFRI): 
``A robust long-term vision is essential to successfully and
sustainably develop, construct and operate Research Infrastructures.''
even if Logipedia is still in its ``incubation phase''.
\end{compactitem}



%%%%%%%%%%%%%%%%%%%%%%%%%%%%%%%%%%%%%%%%%%%%%%%%%%%%%%%%%%%%%%%%%%%%%%%%%%%%%%
\subsection*{(a) Plan for dissemination and exploitation of results}
\label{sec:dissemination}


\subsubsection*{Dissemination}

Logipedia will strictly follow the open access policy of Horizon 2020
by providing online access to scientific information that is free of
charge to the end-user and that is reusable. In the context of this
project, scientific information refers to peer-reviewed scientific
research articles (published in scholarly journals), articles,
conference paper, etc. As such, the project will combine different
measures to foster open access to knowledge. With
reference to Open Data, Logipedia will comply with the requirements of
the Open Research Data Pilot fully.




\begin{longtable*}{|p{0.55\textwidth}|p{0.12\textwidth}|p{0.15\textwidth}|}
\hline
{\bf Action}
&
{\bf Target}
&
{\bf Indicator and schedule}
\\
%%%%%%%%%%%%%%%%%%%%%%%%%%%%%%%%%%%%%%%%%%%%%%%%%%%%%%%%%%%%%%%%%%%%%%%%%%%%%%
\hline
{\bf 1. Participation to conferences.}
In computer science, publishing in conference
proceedings are often favoured over journal publication. We have targeted
more than ten conferences (see below).
&
Researchers.
&
10 papers per year.
\\
\hline
{\bf 2. Organisation of conferences.}
We shall organise our own yearly event, with a conference, parallel
specialised workshops, and the general assembly, and a final
conference to draw the conclusions of the past four years and discuss the
road map for the exploitation of the infrastructure after the end of the
project.
&
Researchers, industrials.
&
100 participants each year. Four conferences total.
\\
\hline
{\bf 3. Organising summer schools}
open to anyone and not only the partners.
We shall organise several training sessions targeting the
different communities of users: master and PhD students to teach them
the foundations of Logipedia, teachers to help them use interactive
theorem provers at school and university, and to engineers to help
them use formal methods tools in their work. Such training sessions
are key dissemination events that will accompany the growing of the
Logipedia community and contribute to educate a new generation of researchers,
teachers and engineers.
&
Master and PhD students, researchers, engineers.
&
2 Summer schools. 100 participants total.
\\
\hline
    {\bf 4.
Supervising PhD students} 
to educate a new generation of researchers and engineers, 
Some will have academic and industrial co-advisors.
&
PhD students.
&
3 PhD students start each year.
\\
\hline
{\bf 5. Co-building the Logipedia strategy}
by participating to joint meetings, such as
the clubs and advisory board.
&
Researchers, industrials.
&
Participation to the meetings
every year.
\\
\hline
{\bf 6. Disseminate Logipedia in relevant communities.}
The fours clubs contribute to the dissemination of
Logipedia in their own ecosystem, by organising talks,
courses, meetings.
In particular, we will organise once a year an event, 
with the workshops of all the clubs to discuss our main achievements,
future developments, and exploitation.
&
Research, industry,
education, publishing.
&
At least one event organised by each club every year.
\\
\hline
{\bf 7. Textbooks}
co-produced by 
the partners of the project and the members of
the club of users in education. 
&
Under- graduate and secondary education students
&
At least one textbook during the project.
\\
\hline
{\bf 8. Using Logipedia to increase reproducibility in science}
by referencing formal proofs in a single place. 
&
Publishers and researchers.
&
Researchers outside the consortium use Logipedia as a reference.
\\
\hline
{\bf 9. Initiate a discussion with certification authorities}
about the use of a common language across the European Union.
&
Certification agencies and the industry. 
&
Two meetings are organised with several European certification agencies.
\\
\hline
{\bf 10. Use a free licence for data and software} to make 
the data is findable, accessible, interoperable and reusable
and develop Open data / Open science / Open innovation.
&
Scientists and innovators.
&
Delivery of Logipedia at month 14.
\\
\hline
{\bf 11. Publish in Open access venues.}
&
Scientists and students.
&
All the publications of the partners are open.
\\
\hline
\end{longtable*}

Targeted conferences:
\begin{compactitem}
\item
  CADE (Conference on Automated Deduction)
\item
  CICM (Intelligent Computer Mathematics)
\item
  CPP (Certified Programs and Proofs)
\item
  CSL (Computer Science Logic)
\item
  FSCD (Formal Structures for Computation and Deduction)
\item
  FROCOS (Frontiers of Combining Systems)
\item
  ICALP (International Colloquium on Automata, Languages, and Programming)
\item
  IJCAR (International Joint Conference on Automated Reasoning)
\item
  ITP (Interactive Theorem Proving)
\item
LFMTP (Logical Frameworks and Meta-Languages: Theory
and Practice)
\item
  LICS (Logic in Computer Science)
\item
  LPAR (Logic Programming and Automated Reasoning)
\item
  PxTP (Proof eXchange for Theorem Proving)
\end{compactitem}
  
\subsubsection*{Exploitation}

The objective of the exploitation activities is to
to make sure Logipedia remains functional with a
healthy community and development roadmap beyond the project.
The partners will conduct the following actions.

\begin{longtable*}{|p{0.30\textwidth}|p{0.30\textwidth}|p{0.30\textwidth}|}
\hline
{\bf Action}
&
{\bf Stakeholders}
&
{\bf Indicator and schedule}
\\
\hline
%{\bf Monitoring the innovation during the course of the project.}
%&
%The steering commitee, the European project manager, the 
%transfer, innovation, and partnership department of Inria Saclay and
%any relevant member of our partners' institution.
%&
%Meetings organized with the transfer and innovation department
%upon request and at least once a year.
%\\
%\hline
{\bf 1. Build an organisation in charge of managing Logipedia
after the end of the project}.
&
The project management team, volunteer members of the consortium,
after consulting the advisory board.
&
The structure is created at month 48 at the latest.
\\
\hline
{\bf 2. Raise funds to manage Logipedia}.
&
The project management team, after consulting the advisory board.
&
Sponsors and fundings have been identified at month 36 at the latest.
\\
\hline   
{\bf 3. Find a server to permanently host Logipedia}.
&
The project management team, volunteer members of the consortium,
after consulting the advisory board.
&
Logipedia is kept functional.
\\
\hline
{\bf 4. Continue developing Logipedia}
&
New developers taking over.
&
New libraries (for instance that of ACL2 or Nuprl) are integrated.
New features are added to Logipedia.
\\
\hline
{\bf 5. Generate new projects, such as ``Logipedia, security, and
certification'', ``Logipedia and automated theorem proving'',
``Logipedia and the B method''...}
&
Special interest groups within Logipedia. 
&
New communities adopt Logipedia.
\\
\hline
\end{longtable*}

%%%%%%%%%%%%%%%%%%%%%%%%%%%%%%%%%%%%%%%%%%%%%%%%%%%%%%%%%%%%%%%%%%%%%%%%%%%%%%
\subsection*{(b) Communication activities}

Communication activities aim at raising the awareness about Logipedia
to potential stakeholders that would not be concerned by our
dissemination actions. It will ensure the growth of the Logipedia
ecosystem and be a way to advertise the work achieved by the partners
and the clubs of users. It also serves the purpose of informing the
European citizen of the research findings she has been financially
contributing to.

Communication activities will be closely linked to dissemination
objectives.  Six person-months of an experienced communication
officer, from the Inria Saclay communication team, are dedicate to the
sole task of communication.  She will work in close cooperation with
the dissemination, communication, and exploitation work package
leader and the project management team.

The preparation of a Dissemination and communication plan in line with
the project dissemination and communication strategy will be one of
the first tasks to be addressed in work package 8. The Plan will identify
fundamental elements of the dissemination and communication strategy,
including:
\begin{compactitem}
\item key messages to be conveyed (what),
\item tools and channels used (how),
\item timing of the planned activities (when),
\item geographical level (local, national, European) (where) providing
  a guide for the project's and partners' dissemination activities to
  maximise the impact of Logipedia.
\end{compactitem}

In this project, we also have the ambition to communicate to the
general public, even if, in the past, these communication activities
have been considered as less important than the dissemination towards
the research, industry, and education communities.

\begin{longtable*}{|p{0.30\textwidth}|p{0.30\textwidth}|p{0.30\textwidth}|}
\hline {\bf Action} & {\bf Target audience} & {\bf Indicator and
  schedule} \\
\hline {\bf Promoting the existence of the project}
including defining the project visual identity,
creating a web site,
designing flyers, posters, and videos, and
publishing press releases.
&
Research, industry, and education.
& Website at month 3.
\\

\hline
{\bf Promoting the results and the values of the project} through
outreach actions: publication of articles in popular science
magazines, online videos\footnote{such as {\tt
https://www.facebook.com/TheatreLaReineBlanche/videos/518698965681202},
a one-minute video hosted by le Th\'e\^atre de la Reine Blanche,
presenting in
French the example given at the beginning of this document.}, and live
events such that the European Researchers Night or {\em La Fête de la
  Science}, focusing, as it is our habit.  & General public.  & At
least five partners organise an outreach activity in their country.
\\ \hline
\end{longtable*}

%%% Local Variables:
%%%   mode: latex
%%%   mode: flyspell
%%%   ispell-local-dictionary: "british"
%%% End:


%%% Local Variables: 
%%% mode: LaTeX
%%% TeX-master: "propB"
%%% End: 

% LocalWords:  ednote

\newpage
\chapter{Implementation}\label{chap:implementation}

\section{Work plan --- Work packages, deliverables}

\subsection{Overall structure of the work plan}

Our work plan is divided into seven scientific work packages.
The first group of work packages is dedicated to the networking
activities that are needed to gather the proofs today located in
different libraries.
\begin{longtable}{|p{0.02\textwidth}|p{0.2\textwidth}|p{0.7\textwidth}|}
\hline
1
&
Integration
&
Instrument the systems for which we already know how to encode the
proofs in Dedukti, and make available these proofs in Logipedia.
\\
\hline
2
&
Automatic theorem proving
& 
Develop automatic theorem provers to populate,
help, and benefit from Logipedia.
\\
\hline
3
&
Large libraries
&
Export large dedicated libraries in curated form 
to Logipedia for end-user applications.
\\
\hline
\end{longtable}
The second is dedicated to making these proofs accessible, beyond
trans-national and virtual access.
\begin{longtable}{|p{0.02\textwidth}|p{0.2\textwidth}|p{0.7\textwidth}|}
\hline
4
&
Access
&
Define and build the Logipedia hardware and software infrastructure in
which the proofs will be integrated.
\\
\hline
5
&
Structure of the encyclopedia
&
Provide infrastructure for the structured ontological representation
of libraries and use it to enrich the information about formal
libraries in Logipedia.
\\
\hline
\end{longtable}
The third is dedicated to joint research activities that prepare
the future of Logipedia. 
\begin{longtable}{|p{0.02\textwidth}|p{0.2\textwidth}|p{0.7\textwidth}|}
\hline
6
&
Theories
&
Bringing proof systems implementing a theory 
that has not yet been expressed in Dedukti LIL 2 or better.
\\
\hline
7&Proof engineering &
Investigate methods for detecting concept alignments and apply
them to build a library of alignments present across the Logipedia database.
\\
\hline
\end{longtable}
Together with these seven scientific work packages, 
two more work packages are dedicated to dissemination, communication and
exploitation and to madagement.
\begin{longtable}{|p{0.02\textwidth}|p{0.2\textwidth}|p{0.7\textwidth}|}
\hline
8
&
Dissemination, communication, and exploitation
&
Expand the use of Logipedia in research, industry, education, and publishing.
\\
\hline
9
&
Management
&
Coordinate this large community, in a benevolent atmosphere, for optimal
efficiency.
\\
\hline
\end{longtable}

\subsection{Timing of the different work packages and their components}

Gant diagaram : a line for each task
and a column for each month.

Finishes with a miletones, deliverable and project meeting.

Put a lot of deliverables at Month 18 and 36 and 48

\ganttchart[draft,xscale=.45] 

\begin{tabular}{|c|c|c|c|c|c|c|c|c|c|c|c|c|c|c|c|c|c|c|c|c|c|c|c|c|c|c|c|c|c|c|c|c|c|c|c|c|c|c|c|c|c|c|c|c|c|c|c|c|}
\hline
& 1 & 2 & 3 & 4 & 5 & 6 & 7 & 8 & 9 & 10 & 11 & 12 & 13 & 14 & 15 & 16 & 17 & 18 & 19 & 20 & 21 & 22 & 23 & 24 & 25 & 26 & 27 & 28 & 29 & 30 & 31 & 32 & 33 & 34 & 35 & 36 & 37 & 38 & 39 & 40 & 41 & 42 & 43 & 44 & 45 & 46 & 47 & 48\\
\hline
\textsl{aaa} \\
\hline
\end{tabular}

\subsection{Detailed work description}

\wpfigstyle{\footnotesize}
\wpfig[pages,type,start,end]

\begin{workplan}
\newpage
\begin{workpackage}[id=instrumentation,type=RTD,
  short=Integration,% for Figure 5.
  title=Integration,
  lead=Del,
  DelRM=14,
  GotRM=4,
  TumRM=5,
  ChaRM=20,
  CleRM=0,  % TODO
  ImtRM=0,  % TODO
  TouRM=0,  % TODO
  BolRM=16, % 12 on Coq, 4 on Matita
  InrRM=0]  % TODO

%\ednote{MK: We need one coordinating site. original coordinators: Frédéric Blanqui and
%  Jesper Cockx}
%\ednote{MK: interested parties (add their sites and RM here): David Deharbe,
%Tobias Nipkow, Guillaume Genestier, Jesper Cockx, Guillaume Burel, Filip Marić, Makarius
%Wenzel, Nicolas Magaud, Gaspard Férey, Ulf Norell, Claudio Sacerdoti Coen}

\begin{wpobjectives}
  The objective of this work package is to bring systems already at least at
  LIL 1 to higher LIL levels, that is, to instrument the systems
  for which we already know how to encode the proofs in Dedukti, and
  make available these proofs in Logipedia so that
  they can be exported to other systems.
\end{wpobjectives}

\begin{wpdescription}
  Concretely, this work package includes the following tasks:
\end{wpdescription}

\begin{tasklist}
\begin{task}[id=agda,
  title=Instrument Agda,
  lead=Del,
  DelRM=14,
  GotRM=4]
%[G\"oteborg, Delft]

%\textbf{Budget requirements:} One research engineer at Chalmers, and one PhD student or postdoc at TU Delft.

%Question: does this task belong to WP1 or WP2?

Agda is a popular dependently typed programming language / proof
assistant based on Martin-L\"of’s intuitionistic type theory. Its theory
is similar to Coq and Lean, but is more focused on interactive
development and direct manipulation of proof terms (in contrast to
using a tactic language to generate the proof terms). Agda has a
sizable standard library (available at
https://github.com/agda/agda-stdlib) that consists of both utilities
for programming and mathematical proofs.


In the summer of 2019, Guillaume Genestier worked together with Jesper
Cockx on the implementation of an experimental translator from Agda to
Dedukti during a research visit at Chalmers University in Sweden. This
translator is still work in progress, but it is already able to
translate 142 modules of the Agda standard library to a form that can
be checked in Dedukti. This exploratory work uncovered several
challenges and opportunities for further work, which are outlined
below.

\begin{enumerate}
\item To support the construction of proof terms, Agda provides powerful
features such as dependent pattern and copattern matching, eta
equality for functions and record types, and definitional proof
irrelevance. The first one – dependent pattern matching – can be
translated directly to rewrite rules in Dedukti. However, the two
latter features – eta equality and irrelevance – rely on Agda’s
type-directed conversion algorithm, while Dedukti’s conversion is
untyped. Hence in order to translate Agda proofs to Dedukti these
features need to be encoded.

One particular concern with the encoding of eta-equality is that in
general it requires storing of additional type information in the
proof terms. It can hence lead to a large blow-up in the size of those
proof terms, and thus greatly increase the cost of typechecking. The
same problem also occurs in other parts of Agda; for example
constructors of parametrized datatypes do not store the values of the
parameters, but they need to be reconstructed in the translation to
Dedukti. We plan to investigate two possible approaches to this
problem: either we can try to find a better encoding which reduces the
size of the type annotation, or alternatively we can extend the
Dedukti language with type-directed conversion rules to render the
type annotations unneccessary.

\item Another unique feature of Agda is the support for first-class
universe level polymorphism. In particular, Agda has a built-in type
of levels that has complex structure of (in)equality between
levels. Compared to universe polymorphism in Coq, an additional
challenge is that levels in Agda can contain arbitrary terms as
subexpressions. Our plan is to define a sound and complete embedding
of Agda’s level type in Dedukti, based on the existing work on
encoding AC (associative-commutative) theories. This would both serve
as a stress test of how well Dedukti can handle complex equational
theories, and improve our understanding of type theories with
first-class universe level polymorphism, which would be useful for the
implementation of Agda.

\item In contrast to Coq and Lean, Agda does not have a well-defined
core language to which proofs are elaborated. Instead, definitions are
translated to an internal representation that is relatively close to
the user input. This provides a challenge when translating Agda proofs
to Dedukti: each feature in Agda’s internal syntax needs to have its
own translation. As part of this project, we will hence investigate
possible designs for a core language for Agda. Having such a core
language would have several benefits: it would deepen our
understanding of the Agda language, it would increase the
trustworthiness of Agda proofs, and it would make it much easier to
export Agda terms to other languages (such as Dedukti in the context
of this project).

\item Agda provides an experimental option for extending the language
with user-defined rewrite rules, which are very similar to the rewrite
rules provided by Dedukti. Because of this similarity, we expect it to
be straightforward to translate rewrite rules from Agda to
Dedukti. However, by comparing the two implementations we hope to gain
new insights and find opportunities for improvement on both sides. The
interest of some of these features goes beyond just the Agda
language. In particular, Lean also supports definitional proof
irrelevance, as does Coq with the recent addition of the SProp
universe. Hence we plan to collaborate with the teams working on those
languages to improve the support for these features where there is
overlap.
\end{enumerate}

%%% Local Variables:
%%% mode: latex
%%% TeX-master: "../propB"
%%% End:

\end{task}

\begin{task}[id=isabelle,
  title=Instrument Isabelle,
  lead=Tum,
  TumRM=5]
% task leader: Tobias

% participants: Makarius (Isabelle), David Matthews (Poly/ML)

% Moved to concept & methodology
% Isabelle as a logical framework \cite{paulson700} is an intermediate
% between Type-Theory provers (like Coq or Agda) and classic LCF-style
% systems (like HOL Light or HOL4). The inference kernel can already
% output proofs as $\lambda$-terms on request, but this has so far been
% only used for small examples \cite{Berghofer-Nipkow:2000:TPHOL}. The
% challenge is to make Isabelle proof terms work robustly for the basic
% libraries and reasonably big applications.  Preliminary work by Wenzel
% (2019) has demonstrated the feasibility for relatively small parts of
% Isabelle/HOL, but this requires scaling up.

\begin{enumerate}
  \item Improve the efficiency of important aspects of the
  Isabelle/HOL logic implementation, such as normalization of proofs,
  type-class reasoning, and special representation of derived rules
  and definition principles.
  \item Reduce the volume of proof terms in the Dedukti encoding.
  \item Improve memory usage of the Isabelle/ML implementation
  platform.
\end{enumerate}

\end{task}

\begin{task}[id=HOL4,
  title=Instrument HOL4,
  lead=Cha,
  ChaRM=20]
[G\"oteborg]

The HOL4 proof assistant is home to a few medium to large scale
specifications and associated proof developments that have value
outside of HOL4. These specifications include the formal semantics of
the CakeML language (and its verified compiler) and an extensive
specification of the ARM instruction set architecture (ISA) as
formalised by Anthony Fox at the University of Cambridge.

HOL4 has support for exporting proofs to the OpenTheory proof exchange
format, and there has been some work on importing OpenTheory proofs
into Dedukti. However, the current state of these techniques and their
implementations does not scale to real examples such as those
mentioned above.

This part of the project will be about re-thinking and re-designing
the tools HOL4-to-OpenTheory and OpenTheory-to-Dedukti tools such that
they scale to the point where real examples of interest, such as those
mentioned above, can be exported.

2 Person Years at Chalmers


\end{task}

\begin{task}[id=atelier-b,
  title=Instrument Atelier-B/Rodin,
  lead=Cle,
  CleRM=0, % TODO
  ImtRM=0, % TODO
  TouRM=0] % TODO
% task leader: Catherine

% other participants:

%\ednote{Southhampton, Toulouse, Clearsy writes this}

% Moved to concept & methodology
% Atelier B, Rodin and ProB are platforms or tools to develop models
% written in B method, Event-B or B system. The development process is
% based on formal proof: proof obligations are automatically  generated
% and must be proven by automatic or interactive provers. ProB is an
% animator and model checker, it helps users to gain confidence in their
% specifications. It is also a disprover aiming at  discovering
% counter-examples for proof obligations. Atelier B and Rodin use native
% B proof tool, they also enable the use of external provers such as SMT
% solvers. ProB calls SMT and SAT solvers, it also uses contraints
% solvers such as Sicstus Prolog. All of them relies on the B logics,
% mainly a first order language with set theory. Regarding B/Event-B/B
% system, there are some variants, mainly regarding the refinement
% process they all implement. Refinement means that models are developed
% by successive steps, from an abstract model to a more  concrete
% model. Refinement in B method mainly means deriving a program while
% EventB and B System refinement aim at defining a model of a system by
% introducing details.


% Moved to concept & methodology
% In the context of the BWare project, an encoding of the set theory of
% the B method has been provided as a theory modulo, i.e. a rewrite
% system rather than a set of axioms. This encoding is used by the
% automatic prover Zenon modulo which features a backend to
% Dedukti. Thus, as a first step through instrumentation of Atelier B
% and Rodin, proof obligations coming from Atelier B can be proved by
% Zenon modulo producing Dedukti proofs, hence providing a better
% confidence in the proofs produced by the native proof tools of Atelier
% B \cite{Bware}.

\begin{enumerate}

  \item Continue the encoding of the B set theory in Dedukti to be
  able to handle all kind of proof obligations and rules.

 %\item Instrument the native provers to produce proofs (in WP4 ?).

  \item Instrument and Exporting B models to Dedukti.

  \item Importing logipedia lemmas in B models

\end{enumerate}

\end{task}

\begin{task}[id=matita,
  title=Integrate the Matita translator in Matita itself,
  lead=Bol,
  BolRM=4]
%[Bologna]

%\textbf{Budget requirements:} One one PhD student or postdoc at UBo.

Matita is an interactive theorem prover developed at the University of Bologna and used for teaching logic courses and to verify software and mathematical proofs, with special attention to predicative foundations. The first generation of the system (up to version 0.5.9) was born as a by-product of the MoWGLI FET-Open Project, it was compatible with the logic of Coq and it could re-use its libraries. It was an important test-bench for the integration of Mathematical Knowledge Management techniques with Interactive Theorem Proving, featuring for example a library of theorems distributed over multiple servers, innovative indexing and search techniques and automatic translation of proofs between declarative and procedural styles. The second generation of the system (up to the current version 0.99.3) was a re-implementation from scratch that departed from the logic of Coq and that experimented with the most concise ways to implement an efficient theorem prover. Several ideas later migrated into Coq. The currently available largest library is the formal certification of a complexity-preserving and cost-model-inducing compiler from C to MCS-51 machine code, developed in the FET project CerCo (Certified Complexity).

The standard and arithmetic libraries of Matita has been the first libraries to be exported to Logipedia using Krajono, a fork of Matita. The forked system is also actually the only one able to import Logipedia proofs. The choice of Matita as a test-bench for Logipedia is easily understood considering that the implementation of the 0.99.x series was aimed at obtaining a well-documented, minimal but fast implementation of a theorem prover, two order of magnitudes smaller than Coq.

The task will achieve the following results
\begin{enumerate}
\item Merge Krajono and Matita, update the code to the latest version and transfer the maintenance effort to the Matita team.
\item Export all the remaining Matita libraries. In particular:
\begin{itemize}
 \item The libraries developed in CerCo contain several gigantic proof terms (nested proofs by cases on the 256 opcodes of the MCS-51 processor) that will stress the encoding and the tools developed around the Logipedia library.
 \item The proofs in the arithmetic libraries of Matita, now converted to HOL proofs inside Logipedia, do not exploit dependent types. Other libraries rely heavily on dependent types, triggering more interesting translations between theories encoded in Logipedia.
\end{itemize}
\item The logics of Matita and Coq remain quite similar, sharing a common core. However no complete automatic translation from Coq to Matita or vice versa is possible any more and only partial translations with high coverage are known, but not implemented, due to the intricacies of having to make the two code bases interact. We will study how to implement the partial translations directly in Logipedia, without knowledge of the internals of the two systems, and we will rely on automatically generated alignments to augment coverage of the translation.\ednote{CSC: this point probably does not belong to this WP}
\end{enumerate}

%%% Local Variables:
%%% mode: latex
%%% TeX-master: "../propB"
%%% End:

\end{task}

\begin{task}[id=coq,
  title=Instrument Coq,
  lead=Bol,
  BolRM=12,
  InrRM=0]  %TODO
% task leader: Claudio Sacerdoti Coen
% participants: Enrico Tassi, Claudio Sacerdoti Coen
%\textbf{Budget requirements:} One one PhD student or postdoc at UBo.

\ednote{Bol: 12 MM = 44,630 euros (39,372 euros salary + travels etc.); Inr: 12 MM}

% Moved to concept & methodology
% Coq is an interactive theorem prover developed at Inria since the 1984.
% It is based on Type Theory and was used to formally verify the correctness
% of both industrially relevant software such as the CompCert C compiler and
% complex mathematical proofs such as the one of the Four Color theorem and the
% one of the Odd Order theorem. In 2013 Coq received the ACM system award.

This task about instrumenting Coq includes the following steps:
\begin{enumerate}
% deliverable 1
\item Access Coq internal data structures to gather logical data, such as
statements and proof terms (for tasks in WP3)
% deliverable 1
\item Implement the translation of Coq terms to Dedukti terms
% deliverable 2
\item Access Coq internal data structures to gather extra-logical data,
such as the role played by a constant in the library like begin an implicit
cast from one algebraic structure to another (for tasks in WP3)
% deliverable 2
\item Make extra-logical data available in a structured and
  extensible format (for tasks in WP5)
\end{enumerate}

% Moved to concept & methodology
%A technological hurdle steps (a) and (c) have to overcome is that Coq is an actively
%developed system that is constantly evolving. Previous attempts at extracting
%data from Coq without a direct interaction with the developers of Coq resulted
%in prototypes like CoqinE that quickly became outdated.
To overcome the problems associated with external tools such as
CoqInE, we plan to have the required instrumentation merged in Coq
proper and reuse it for other projects that could benefit from it so
to amortize its development cost. More precisely E. Tassi is a core
Coq developer acquainted with its development process and he will take
care of the integration of the infrastructure in Coq and foster its
reuse in third party projects with similar needs such as SerAPI,
CoqHammer and Coq-Elpi.

% Moved to concept & methodology
% Step (b) is also problematic for two reasons. The first one is that the encoding
% in Dedukti requires some information, typically types of sub-expressions, that
% are not stored in Coq, it is transient. So the instrumentation for steps (a) and
% (c) needs to be complemented by providing not only access to existing data but
% also to log transient data or re-synthesize it on demand. Both approaches may
% be used, depending on the the tradeoff between computation time and space for
% storage. The second reason, which is more critical, is ...

\ednote{Should this part also be moved to the methodology section?}
Since the type theory of Coq is extremely large, with features that
have no corresponding representation in Dedukti, the type theory of
Coq needs to be translated to a core one representable in Dedukti.
This translation to a core calculus is not implemented in Coq and the
amount and complexity of code necessary for it is very significant and
indeed the CoqinE prototype only covers a small subset of
Coq. Feedback from WP2 will be of guidance\ednote{Tassi: there is no
task in WP2 about Coq's TT, maybe it is included in HoTT?} to extend
the translation currently available CoqinE to cover a larger subset of
Coq.

In step (d) we plan to take advantage of the work done by Sacerdoti Coen (UBo)
in 2019 in exporting non trivial logical and extra-logical data from Coq to
an XML format. Data in that format was then translated by Kohlhase et al
(UBo + FAU) to the MMT system, another logical framework to encode different
logics and their libraries. We plan to extend that format to include even
more extra-logical data as well as the data gathered in step (a) and (c). We shall
evaluate if all the data needed for step (b) can be saved in this format, and
give us the freedom to implement step (b) in a standalone tool making no
requests to Coq in order to re-synthesize missing data.

%%% Local Variables:
%%% mode: latex
%%% TeX-master: "../propB"
%%% End:

\end{task}
\end{tasklist}

\begin{wpdelivs}
  \begin{wpdeliv}[due=3,miles=startup,id=requirements,dissem=PU,nature=DEM,lead=Inr]
      {Requirements Analysis and Synchronization}
  \end{wpdeliv}
  \begin{wpdeliv}[due=12,miles=logipedia-v1,id=isabelle1,dissem=PU,nature=DEM,lead=Tum]
      {Robust export of proof terms for Isabelle}
  \end{wpdeliv}
  \begin{wpdeliv}[due=12,miles=logipedia-v1,id=isabelle1,dissem=PU,nature=DEM,lead=Tum]
      {Improved memory management and monitoring for Poly/ML}
  \end{wpdeliv}
  \begin{wpdeliv}[due=18,miles=agda-stdlib,id=agda,dissem=PU,nature=DEM,lead=Del]
      {Export Agda's standard library to Dedukti}
  \end{wpdeliv}
  \begin{wpdeliv}[due=8,miles=logipedia-v1,id=coq1,dissem=PU,nature=DEM,lead=Inr]
    {Export of proof terms from Coq, no meta data}
  \end{wpdeliv}
  \begin{wpdeliv}[due=24,miles=logipedia-v2,id=coq2,dissem=PU,nature=DEM,lead=Bol]
    {More scalable export of proof terms and meta data from Coq}
  \end{wpdeliv}
  \begin{wpdeliv}[due=12,miles=logipedia-v1,id=matita1,dissem=PU,nature=DEM,lead=Bol]
    {Export of proof terms and meta data from Matita}
  \end{wpdeliv}
\end{wpdelivs}
\end{workpackage}

%%% Local Variables:
%%% mode: latex
%%% TeX-master: "../propB"
%%% End:

\newpage  
\begin{workpackage}[id=atpetc,wphases=0-48,type=RTD,
  short=ATPs etc.,% for Figure 5.
  title={ATP, SAT, SMT, Model checkers},
  lead=ULi,
  ULiRM=10]
  
\ednote{MK: We need one coordinating site. original coordinators: Pascal Fontaine and Chantal Keller}
\ednote{MK: interested parties (add their sites and RM here): David Deharbe,
Cezary Kaliszyk, Pascal Fontaine, Dale Miller, Stephan Merz, Josef Urban, Martin Suda,
Guillaume Burel, Filip Marić, Chantal Keller, Julien Narboux, Thibault Gauthier}

\begin{wpobjectives}
  The objective of this work package is to \ldots

This includes notably:
  \begin{compactitem}
  \item \ldots
  \end{compactitem}
  A key aspect will be to foster \ldots
\end{wpobjectives}


\begin{wpdescription}

The importance of proofs in automated theorem provers, satisfiability
modulo theories solvers, propositional satisfiability solvers and
model checkers is increasingly recognized.  While for the
propositional case, the community agrees on a well defined proof
format, the situation is not clear for the other kind of automated
reasoners.  There is no clear format for SMT, and the TSTP format for
automated theorem provers fixes a syntactic template for proofs rather
than providing an unambiguous framework to express proofs
semantically.

Some preliminary works predating this proposal clearly establish that
Dedukti can accommodate proofs in Satisfiability Modulo Theories,
automated theorem provers, and SMT.  In this work package, we will
build on those preliminary work and provide a set of conduits from the
established formats used in automated tools. For the tools that do not
have yet an established format, we will make a selection of tools
(Zipperposition and E for automated theorem provers, CVC4 and veriT
for SMT, ??? for model checking) and provide a conduits for those
tools.  These conduits and the techniques used in the embedded
translation will be properly documented, to ease integration of
further tools of the kind.  If a standardized proof format appears for
some kind of tools, the conduits will be updated to adopt the new
standard.

In this work package, we also plan to integrate in Logipedia some
well-chosen proofs coming from automated tools.  Well-chosen proofs
will have to be representative of typical applications of the tools,
and be of reasonable size.  They will serve as examples to the
community, to illustrate the potentials of Dedukti and Logipedia.


Create the infrastructure to enable the long term goal: be able to split a large proof
obligation into smaller parts and distribute to the appropriate automatic engines, that
would all produce proofs, glued together in a single large proof for the original proof
obligation.
\ednote{Nancy, Liège}
\end{wpdescription}

\begin{tasklist}
  \begin{task}[id=tools,title=Automatic Tools Exporting Proofs]
  \end{task}

  \begin{task}[id=challenges,title=Logipedia as a Source of Challenges for Automatic Reasoners]
    --> Translation to TPTP, SMT-LIB, DIMACS
  \end{task}
  \begin{task}[id=commang,title=A language for Communication between Automatic Reasoners]
  \end{task}
\end{tasklist}

\begin{wpdelivs}
  \begin{wpdeliv}[due=3,miles=startup,id=requirements,dissem=PU,nature=DEM,lead=ISa]
      {Requirements Analysis and Synchronization}
\end{wpdeliv}
\end{wpdelivs}
\end{workpackage}

%%% Local Variables:
%%% mode: latex
%%% TeX-master: "../propB"
%%% End:

\newpage  
\begin{workpackage}[id=libraries,type=RTD,wphases=1-48,
  short={Large libraries},% for Figure 5.
  title={Large libraries},
  lead=Tum,
  StrRM=18,
  ChaRM=12,
  TumRM=27]
%TUM: 3 for AFP/Makarius (25k EUR) - the latter do not generate overheads!

\begin{wpobjectives}
The objective of this WP is to export large dedicated libraries in
curated form to Dedukti and thus to Logipedia for end-user applications.
The focus is \emph{Access} and \emph{Scalability}.
\begin{compactitem}
\item This WP is responsible for supplying the lion's share of proofs in
Logipedia.  As a result it will be a stress test for the results of \WPref{instrumentation}.

\item The target libraries are dedicated to particular application
areas. They provide a substantial coverage of that application area
and do so in a structured manner. This may require reworking the
libraries for better access.

\item The libraries are curated for end-user application. That is, they
are structured according to application specific ontologies that
support browsing and search. The structuring leverages the
infrastructures of \WPref{structuring} and will be a
stress test for the results of that WP.
\end{compactitem}
\end{wpobjectives}


\begin{wpdescription}
Translating the standard libraries of the systems is part of the \WPref{instrumentation}.
This WP focusses on advanced libraries selected according to the following criteria:
relevance, coverage and maturity.
As a result we selected the following libraries: MathComp, Coq's revised
Analysis library, the Archive of Formal Proofs, Isabelle's revised Analysis and Probability library,
GeoCoq, Flyspeck and CakeML. In the future we plan to incorporate
CompCert, seL4, and selected Mizar and PVS libraries (once Mizar and
PVS have reached LIL 2).
\end{wpdescription}


\begin{tasklist}
%\begin{task}[id=mathcomp,title=MathComp]
%\ednote{Sophia, Saclay (Gonthier), Paris}
%\end{task}

%\begin{task}[id=milc,title=Revised Coq Analysis Library]
%\ednote{Saclay (Boldo), Paris, Sophia}
%\end{task}

%\begin{task}[id=mizar,title=The Mizar library]
%\ednote{Innsbruck, Bialystok}
%\end{task}

\begin{task}[
  id=afp,
  title=Isabelle's Archive of Formal Proofs,
  lead=Tum,
  TumRM=3,
  wphases=13-18]
%\ednote{Wenzel}
Isabelle's Archive of Formal Proofs (AFP) \cite{isabelle-afp} is a
growing user-contributed online library for Isabelle. In Feb-2020, the
AFP consisted of more than 500 entries (articles of formalized
mathematics) by 340 authors, and required approx.\ 60h CPU time for
checking (using many gigabytes of memory).  The purpose of this task
is to scale up the Isabelle instrumentation for Dedukti further, to
cover major parts of this library. The ultimate aim is to export the main
substance of the AFP without promising full coverage: some entries
with prohibitive resource requirements will be omitted.
\end{task}

\begin{task}[
  id=isaAnalysisProb,
  title=The Isabelle Analysis \& Probability Theory library,
  lead=Tum,
  TumRM=24,
  wphases=1-24]
%
  This library consists of more than 200.000 lines
  of definitions and proofs, corresponding to almost 4000 printed
  pages. It is fair to say that it is the most advanced
  machine-checked library in the area of analysis and probability
  theory. Because analysis and probability theory are key to many
  applications in enginnering and science, this library will be a key
  exploitable result of the project: it is a fundamental enabling
  resource for almost any formal verification activity in these
  application areas. The purpose of this task is to structure,
  document and develop this library for optimal accessibility, ease of
  use and comprehensiveness.

For better access, the library needs to be modularized, which requires
a significant refactoring effort.  At the same time we need to add
metadata (as provided by \WPref{structuring}) to the source material
to turn this structured collection of theorems and proofs into a
curated library at the Logipedia level.

The following areas of the library need to be developed further. The
library support for integrals is extensive but suffers from the
coexistence of different kinds of integrals. This requires unification
and refactoring. Further essential material for mathematics, physics
and engineering needs to be added: Fourier
analysis and esp.\ the Fourier transform; stability theory for
differential equations and dynamical systems, in particular Lyapunov
functions; stochastic differential equations.
\end{task}

\begin{task}[
  id=geocoq,
  title=The GeoCoq library,
  lead=Str,
  StrRM=18,
  wphases=1-18]
%
The GeoCoq library consists of more than 100.000 lines of definitions and proofs. It is mostly based on synthetic approaches, where the axiom system is based on some geometric objects and axioms about them, but, following Descartes and Tarski, the analytic approach can be derived, where a field F is assumed (usually R) and the space is defined as $F^n$. Moreover, it contains a model of Tarski's axioms, based on the analytic approach, thus establishing the connection between these two approaches in the opposite direction. The main axiom system in this library is the one of Tarski, but Hilbert's axiom system and a version of Euclid's axioms sufficient to prove the propositions in Book 1 of Euclid's Elements are also defined. In the library, the focus is not only on axiom systems but also on axioms themselves. Eleven continuity axioms are available and are hierarchically organised. Finally, it contains a new refinement of Pejas’ classification of parallel postulates together with proofs of the classification of 34 versions of the parallel postulate.

One of the remaining obstacles is the frequent use of computational steps in Coq proofs. The issue is that proofs containing "proof by reflection" reach a level of complexity that makes verification by Dedukti impractical. An approach is to isolate these proofs by reflection so that they are not perceived as simple conversion steps in the type theory proofs, but marked as proofs to be treated by an automatic tool. Another challenge is that CoqInE, a tool developed to translate Coq proofs into Dedukti type-checkable terms, produces terms in a expressing of the Calculus of Inductive Constructions in Dedukti. Currently, it is not possible to export these Dedukti terms to other proof assistant. However, another tool, Universo, has been developed and paves the way for the export of these terms.
\end{task}

\begin{task}[
  id=flyspeck,
  title=The Flyspeck library,
  lead=Inr,
  wphases=1-36]
%\ednote{Saclay (Grienenberger)}
The {HOL Light} library is large and varied. One of its key libraries is the
multivariate analysis library
%\footnote{\url{https://github.com/jrh13/hol-light/tree/master/Multivariate}},
which spans the fields of metric spaces, topology, homology, linear algebra,
convexity, real and complex analysis and transcendentals, derivatives, and
integration. The {Flyspeck} project gives a formal proof of the {Kepler}
conjecture, based on an original proof of Thomas {Hales}
\cite{DBLP:journals/corr/HalesABDHHKMMNNNOPRSTTTUVZ15}, and formalized
largely in {HOL Light} \url{https://github.com/flyspeck/flyspeck}.
Some of these results are not formalized in any other system, motivating the
project of importing the {HOL Light} library and {Flyspeck} project in the
{Dedukti} system, in view of its integration into {Logipedia}.

\textbf{Challenges:}
Proofs coming from the HOL systems, including {HOL Light}, are known to be very
large, adding to the issue of the scalability of exporting software for large
libraries \cite{DBLP:conf/tphol/Wong95,DBLP:conf/cade/ObuaS06,
DBLP:conf/itp/KellerW10,DBLP:conf/cade/Kumar13}. Scalable export techniques
{HOL Light} proofs have been investigated \cite{KaliszykK13} and can provide a
solid base to this project.

The main milestones of this task are the further automation of the export from
{HOL Light} to {Dedukti}, the import of the multivariate analysis library, of
the whole {HOL Light} library, and of the {Flyspeck} in {Dedukti}.
\end{task}

\begin{task}[
  id=cakeml,
  title=The CakeML compiler library,
  lead=Cha,
  ChaRM=12,
  wphases=12-23]
%
The CakeML
compiler \cite{KumarMNO14} (verified with the HOL4 prover) is one of only two verified compilers for real
languages, the other being CompCert. Its export to Dedukti is one of
the KERs of this project.
% Because of the size and importance of this
%library, we will approach the export from two angles.
%
%The first approach utilizes OpenTheory-based technology.
HOL4 can export proofs in the OpenTheory format, which can in turn be
translated into Dedukti. Currently this link from HOL4 via OpenTheory
to Dedukti does not scale to something as sizeable as the CakeML
compiler proof. This part of this task will rework the route via
OpenTheory to scale better, possibly taking inspiration from an
OpenTheory-like approach that scaled well for the HOL light
prover~\cite{KaliszykK13}.

%The second approach establishes a connection from HOL4 via Isabelle to
%Dedukti. The basis is a promising new approach of virtualizing HOL4
%inside Isabelle \cite{ImmlerRW19}. That is, the inference kernel of
%HOL4 is replaced by that of Isabelle and the resulting system produces
%Isabelle theorems instead of HOL4 theorems. As a benchmark of the
%viability of this approach we plan to export CakeML via Isabelle to
%Dedukti.
\end{task}

%\begin{task}[id=unimath,title=The UniMath library]
%\ednote{Birmingham (Ahrens)}
%\end{task}

%\begin{task}[id=pvs,title=The NASA PVS library]
%\end{task}
%\begin{task}[id=sel4,title=The seL4 library]
%\end{task}

%\begin{task}[id=compcert,title=The CompCert library]
%\end{task}
\end{tasklist}

\begin{wpdelivs}
  \begin{wpdeliv}[due=3,id=requirements,dissem=PU,nature=DEM,lead=Inr]
      {Requirements Analysis and Synchronization}
  \end{wpdeliv}
  \begin{wpdeliv}[due=36,id=requirements,dissem=PU,nature=DEM,lead=Tum]
      {Scalable export of proof terms for major parts of Isabelle/AFP}
  \end{wpdeliv}
  \begin{wpdeliv}[due=36,id=requirements,dissem=PU,nature=DEM,lead=Tum]
      {Export of Isabelle's extended analysis and probability theory library}
  \end{wpdeliv}
  \begin{wpdeliv}[due=18,id=geocoq-import,dissem=PU,nature=OTH,lead=Str]
      {Export of most of GeoCoq library}
  \end{wpdeliv}
  \begin{wpdeliv}[due=26,id=requirements,dissem=PU,nature=DEM,lead=Cha]
      {Export of CakeML library}
  \end{wpdeliv}
\end{wpdelivs}
\end{workpackage}

%%% Local Variables:
%%% mode: latex
%%% TeX-master: "../propB"
%%% End:

\newpage  
\begin{workpackage}[id=access,wphases=0-48,type=MGT,
  short=Access,% for Figure 5.
  title={Access to the infrastructure},
  lead=Inr,
  InrRM=28,
  OcaRM=6]

\begin{wpobjectives}
  The objective of this work package is to \ldots

This includes notably:
  \begin{compactitem}
  \item \ldots
  \end{compactitem}
  A key aspect will be to foster \ldots
\end{wpobjectives}

\begin{wpdescription}

\end{wpdescription}

\begin{tasklist}

  \begin{task}[id=basic,title=Defining the architecture of the infrastructure]
    We will define the architecture of the infrastructure and install
    it on some server at Inria. The server will be duplicated in
    Münich for security. The architecture includes how the proof files
    for the different proof systems will be organized and stored, how
    they will be generated, etc.
  \end{task}

  \begin{task}[id=web,title=Giving access to the infrastructure on the world-wide web]
    We will develop a web interface to access the infrastructure,
    navigate into the available proofs and downaload them.
  \end{task}

  \begin{task}[id=opam,title=Giving access to the infrastructure in proof tools]
    Users need to have an easy access to the proofs in logipedia, to integrate/use
    them in their ongoing work; this access should be guaranteed universal, without
    lock-in, web standards-compliant, through an open source tool. While WP7
    will give a structure to the proof database, and T2 of this WP will give access
    to that structured database through web browsing, this task aims at
    providing a proof manager for users of Logipedia. This proof manager
    will enable users to automatically download and install proofs as well as their
    dependencies in order to ease the integration of proofs from logipedia in
    developments.

    opam \cite{opam} is an open-source source-based package manager, which has
    been successfully used by the OCaml community since 2012, where it manages
    2585 versioned packages for a total of 13196 combinations of package and
    version, guaranteeing its ability to connect people across large communities.
    Furthermore, opam is meant to provide management capabilities not only to
    OCaml, but to any language, which is why it is already used as a proof
    manager by the Coq community where it has been proven to be reliable and
    suited to managing formal proofs. This makes it a prime candidate to be the
    proof manager for logipedia.

    This task would thus use the opam management tool to develop a repository
    containing all the proofs in logipedia, allowing users across Europe to
    automatically and transparently download and install proofs and their
    dependencies via opam. This would primarily entail the creation of a new
    tool able to read the proof database of logipedia and create a corresponding
    opam repository, as well as the necessary work to automate this work so that
    it can run automatically on the infrastructure built in T1.

  \end{task}

  % Search task, importing some content that was previously in WP7
  \begin{task}[id=search,title=Providing search
    tools,lead=Inr,InrRM=28,FauRM=24,SacRM=6,BolRM=4]
    % task leader: Pierre Senellart, Inria
    We will provide users with search tools enabling them to perform
    queries on Logipedia in order to find specific theorems or proofs.
    First, users will be able to search libraries theorems by their
    names and other metadata (see task~\taskref{structuring}{strdofimpl}), including complex semantic
    queries expressed in the SPARQL language for semantic annotations
    produced in task~\taskref{structuring}{strrefonto}. Second, it will be possible to
    search theorems and proofs based on their structure and mathematical
    content (types, operators, used axioms and rules, etc.), using exact
    matching, regular expressions over fomulas, and deeper content
    matching, such as the one done in the
    \hyperlink{https://kwarc.info/systems/mws/}{MathWebSearch} system. Users can
    use this both to find a specific theorem that
    could be useful in their current development and to analyze the
    proofs themselves, e.g., to find all proofs using a given set of
    axioms.    
    Finally, users will be able to
    search in the full text of theorems and proofs that have been
    extracted from natural-language research articles in
    task~\taskref{structuring}{strtext}.
  \end{task}

\end{tasklist}

\begin{wpdelivs}
  \begin{wpdeliv}[due=3,miles=startup,id=requirements,dissem=PU,nature=DEM,lead=Inr]
      {Requirements Analysis and Synchronization}
  \end{wpdeliv}
  \begin{wpdeliv}[due=2,miles=???,id=acessopamtool,dissem=PU,nature=DEM,lead=Oca]
      {'Proof to opam' tool : Tool to translate the logipedia database format into an opam repository}
  \end{wpdeliv}
  \begin{wpdeliv}[due=1,miles=???,id=acessopamrepo,dissem=PU,nature=DEM,lead=Oca]
      {'Proof opam repository': opam repository populated with the generated proof packages }
  \end{wpdeliv}
  \begin{wpdeliv}[due=1,miles=???,id=accessopamconfig,dissem=PU,nature=DEM,lead=Oca]
    {'opam for logipedia': opam configuration to use it (only or also) for logipedia}
  \end{wpdeliv}
  \begin{wpdeliv}[due=1,miles=???,id=accessopam,dissem=PU,nature=DEM,lead=Oca]
    {'Provide logipedia opam' : installation of the repository in the infrastructure of WP9T1 }
  \end{wpdeliv}
\end{wpdelivs}
\end{workpackage}


%%% Local Variables:
%%% mode: latex
%%% TeX-master: "../propB"
%%% End:

\newpage
\begin{workpackage}[id=structuring,type=RTD,wphases=1-48,
  short={Structure of the encyclopedia},% for Figure 5.
  title={Structure of the encyclopedia},
  activity=tna,
  lead=Fau,
  SacRM=40,
  FauRM=22,
  BolRM=4
%  TouRM=12,
%  InrRM=14
]

%\ednote{Which sites are interested?
%David Deharbe and Etienne Prun (Clearsy); Nicola Gambino, Michael Rathjen, Claudio Sacerdoti Coen, Dale Miller, Emilio J. Gallego Arias, Michael Butler, Pierre Senellart}

% David Deharbe and Etienne Prun (Clearsy): use B Method, would like to doublecheck B proofs, integrate B with other proof assistant at high-level

\begin{wpobjectives}
Providing infrastructure for the structured ontological representation
of libraries and use it to enrich the information about formal
libraries in Logipedia.  Enabling the exchange and reuse the knowledge
between prover systems.
\end{wpobjectives}


\begin{wpdescription}
We proceed in three steps.
Firstly, Tasks~\localtaskref{strlibstructure} and~\localtaskref{strdofimpl} extend the Dedukti language with features for high-level representations that are critical for accessing parts of and searching libraries.
This includes a framework to \emph{define} typed meta-data in form of ontologies, and to \emph{enforce} them in 
the Dedukti libraries.
Secondly, Tasks~\localtaskref{strrefonto} builds ontologies serving as technical exchange format as well as domain-specific descriptions of libraries.
Thirdly, Tasks~\localtaskref{strontorepml} fills the ontology with data from both formal libraries and natural language articles and use the ontology to relate to each other.

This work package will be jointly led by Burkhart Wolff at \site{Sac} and Florian Rabe at \site{Fau}.
(Where a single leader is needed for formal purposes, the latter site will be the primary leader.)
Burkhart Wolff implemented a document ontology framework in Isabelle and developed several applications
in the field of formal software engineering.
% there should be no references here, move them to chapter 1
%\cite{brucker.ea:ontologies-certification:2019,brucker.ea:isabelle-ontologies:2018,brucker.ea:ontologies-certification:2019}
Florian Rabe has extensive experience in designing and implementing knowledge representation languages
%\cite{RK:mmt:10,rabe:recon:17}
as well as in exporting theorem prover libraries.
%\cite{KR:oafexp:20,CKMRSW:ulo:19}
\end{wpdescription}

\begin{tasklist}
\begin{task}[id=strlibstructure,title=Library Structure,shorttitle=Struct.,lead=Fau,FauRM=8, SacRM=6, wphases=1-28!.5]
This task extends the Dedukti language with primitives for representing library, document, informal annotations, and theory structure.
This includes in particular the definition of unique identifiers for all declarations, which is critical for alignments.
%We extend the Dedukti language with features for high-level representations.
%This will include
%\begin{compactitem}
%\item theories: a general term we use to unify a variety of module system constructs such as type classes or locales,
%\item derived declarations: high-level declarations such as inductive type definitions, whose semantics is given by elaboration into more primitive constructs,
%\item metadata annotations: a general framework for attaching information about semantics, document structure, and tool interaction.
%\end{compactitem}
%
%The low- and high-level representations will be tightly integrated: any declaration or object may be given alternatively through either or both of these.
%The prover exports from instrumentation will be such that they produce both representations whenever possible.
%
%Then we leverage this design in several applications including automated prover interaction and a Logipedia-wide search service.
\end{task} 

\begin{task}[id=strdofimpl,title=Ontological Framework for Meta-Data,shorttitle=F/W,lead=Sac,SacRM=24,wphases=1-24!1.0]
This tasks extends the Dedukti language with a framework for meta-data annotations.
This will cover all levels of the structure introduced in \localtaskref{strlibstructure} as well as the 
level of subexpressions of Dedukti expressions. It will also provide a mechanism to validate meta-data
according to assertions.
\end{task} 

% suggested for removal in budget arbitration meeting; now mentioned in task on reference ontology
%\begin{task}[id=strdomonto,title= Domain Ontologies for Formal Methods in SE,shorttitle= Domain Ontologies for Formal Methods in SE,lead=Tou,TouRM=12, SacRM=0]
%Y. Aitameur at \site{Tou}
%\begin{compactitem}
%\item domain ontologies as descriptive models for engineering domains 
%\item links/imports with/from standards and certification
%\item engineering models annotations
%\item strengthening engineering models by references to domain ontologies
%\item Case studies could be certification, safety, security.
%\end{compactitem}
%\end{task} 

\begin{task}[id=strrefonto,title=Reference Ontology,shorttitle=Ref. Ont.,lead=Sac,FauRM=6,SacRM=6,wphases=12-36!.5]
This tasks compiles, integrates, and curates the various ontologies used for describing libraries in the project.
These come from several sources:
\begin{compactitem}
 \item The ontology induced by the structuring features built in task \localtaskref{strlibstructure}.
 \item The ontologies built by users using the ontology framework built in task \localtaskref{strrefonto}.
 \item Manually written ontologies or imports of existing ontologies for knowledge formalised in prover libraries, such as the Upper Library Ontology
and domain-specific ontologies.
 The latter may include for example the ontologies for engineering and their relation to descriptive models and certification standards that are planned to be developed by \site{Tou}.
\end{compactitem}
\end{task}

\begin{task}[id=strontorepml,title=Ontological Representation of Formal Libraries,shorttitle=Ont. Repr.,lead=Fau,FauRM=6,BolRM=4,SacRM=5,wphases=12-48!.5]
  This task extends the exports from Isabelle and Coq developed in
  \WPref{libraries} with structural and ontological data that conforms to the language features introduced in Tasks~\localtaskref{strlibstructure} and \localtaskref{strdofimpl}.
  It will also build on the ontological export of RDF triples relative to
the Upper Library Ontology developed for Isabelle and Coq.

The task leader will collaborate with M. Wenzel for Isabelle and C. Sacerdoti Coen at \site{Bol} for Coq, with whom long-standing collaborations on these library exports exist.
%\cite{MRS:coq:19,CKMRSW:ulo:19,KRW:isabelle:19}
Wenzel's involvement will take the form of a sub-contract of
\site{Fau} corresponding to roughly 4 person-months, an arrangement
that has already been used twice in other projects.  The resources for
this subcontract are not included in the person-months listed here.
\end{task}

% Moved to WP9
%\begin{task}[id=strontosearch,title=Ontological Search,shorttitle=Ont. Search,lead=Fau,FauRM=12,SacRM=6]
%Search based on ontological data (RDF triples) using systems like SPARQL
%\ednote{possible participation of S. Dumbrava; is there a site for this?}
%\end{task} 

%\begin{task}[id=strformsearch,title=Formula-based Search,shorttitle=Form. Search,lead=Fau,BolRM=4,FauRM=12]
%Search based on formula structure using systems like MathWebSearch
%\end{task} 

% suggested for removal in budget arbitration meeting
%\begin{task}[id=strtext,title=Ontological Representation of Natural Language Articles,shorttitle=Articles,lead=Inr,FauRM=6,InrRM=14]
%    % task leader: Pierre Senellart, Inria
%This task extracts ontological information from natural language research articles and link them with the formal representations in Isabelle and Coq.
%P. Senellart at \site{Inr} and M. Kohlhase at \site{Fau} will work on the automatic extraction and annotation of natural-language theorem statements and proofs from published articles, as well as building libraries of such theorems and proofs.
%  % Search capabilities have been moved to WP9
%  %and search and querying capabilities
%\end{task} 

\end{tasklist}


\begin{wpdelivs}
  \begin{wpdeliv}[due=28,id=deliv-str-framework,dissem=PU,nature=R,lead=Sac]
        {This deliverable describes the language developed in Tasks 1 and 2.}
  \end{wpdeliv}
  \begin{wpdeliv}[due=36,id=deliv-str-ontology,dissem=PU,nature=R,lead=Sac]
        {This deliverable describes the reference ontology developed in Task 3.}
  \end{wpdeliv}
  \begin{wpdeliv}[due=48,id=deliv-str-libraries,dissem=PU,nature=R,lead=Fau]
        {This deliverable describes the representation of major formal libraries developed in Task 4.}
  \end{wpdeliv}
\end{wpdelivs}


%\begin{enumerate}
%\item concrete/surface syntaxes 
%\item Central Library Backend Systems 
%\item Cross-System Front-Ends/Portals (Logipedia, ...)
%\item Semantic Middleware-based System Interoperability
%\end{enumerate} 
%
%Since proof-objects for substantial theory developments tend to be
%very large (the representation of current POs for the Isabelle/AFP can
%easily reach several TB although using techniques for compression), A
%technical pre-requisite for interchangeability, connectivity and
%advanced search consists in a structured, typed format for meta-data
%together with a flexible mechanism of their validation. Technically,
%this kind of meta-data has the form of a function annoconst : arg1 ->
%... -> argn -> proof-term -> proof-term where annoconst is a constant
%symbol which represents an identity in the proof-term (so, any import
%function of a specific system can actually ignore it), and where the
%argi represent terms with meta-information such as, eg., “this
%proof-term represents a free data-type construction of the form ...”,
%or “this part of the proof is a derivation of a free data-type of the
%following form ...”, “this lifting over assumptions represents in
%Isabelle a Locale-instantiation”, “this part of a theory
%development is connected to ... ”, “this theorem belongs to the
%sub-class of XXX ... theorems”, etcpp. For arguments of annotations,
%validation-functions can be defined that may check that the argument
%terms satisfy a certain property wrt. to the proof-term and the
%current logical context. Dedukti will provide a framework that allows
%for each proof-system (Coq, HOL4, Isabelle...) to declare meta-data
%together with validations and thus communicate tool-specific knowledge
%to other systems. This framework can be seen as a particular form of
%an ontology definition language.
% 
%WP8: Indexing and browsing [?]  Construct tools to index and browse
%this encyclopedia, that is find the theorem one needs, either by
%looking for it with its name, with its statement, or with symbols
%occurring in it.


\end{workpackage}

%%% Local Variables:
%%% mode: latex
%%% TeX-master: "../propB"
%%% End:

\newpage
\begin{workpackage}[id=theories,wphases=0-48,type=RTD,
  short=Theories in Dedukti,% for Figure 5.
  title= Defining theories in Dedukti,
  lead=Inn,
  InnRM=10]

\ednote{MK: We need one coordinating site. original coordinators: Cezary Kaliszyk and Stephan Merz}

\ednote{MK: interested parties (add their sites and RM here): David Deharbe, Stephan Merz, Guillaume Genestier, Guillaume
Burel, Yamine Ait Ameur, Jean-Paul Bodeveix, Mamoun Filali, Arthur
Chargueraud, Gaspard Férey
}

\begin{wpobjectives}
  The objective of this work package is to \ldots

This includes notably:
  \begin{compactitem}
  \item \ldots
  \end{compactitem}
  A key aspect will be to foster \ldots
\end{wpobjectives}

\begin{wpdescription}
  For other theories, such as Abella, PVS, Mizar and TLA+, we have not yet investigated
  the possibility to express them in Dedukti.
\end{wpdescription}

\begin{tasklist}
\begin{task}[id=pvs,title=Express the theory of PVS in Dedukti and instrument the system]
\ednote{Saclay}
\end{task}

\begin{task}[id=mizar,title=Express the theory of Mizar in Dedukti and instrument the system]
\ednote{IInnsbruck, Bialystok}
\end{task}

\begin{task}[id=tla,title=Express the theory of TLA+ in Dedukti and instrument the system]
\ednote{Nancy, Liège}
\end{task}

\begin{task}[id=abella,title=Express the theory of Abella in Dedukti and instrument the system]
\ednote{Saclay}

The usual approach to capturing either Peano and Heyting arithmetics
is to use various axioms (and an axiom scheme for induction) on top of
classical and intuitionistic first-order logic.  Indeed, this is the
approach used in the Dedukti proof checker.


A different approach to encoding arithmetic has been developed over
the past 20-30 years, starting with papers by Schroeder-Heister and
Girard in the early 1990s and extended in a series of papers by
Baelde, Gacek, McDowell, M, Momigliano, Nadathur, and Tiu.  In this
new setting, first-order logic is extended by considering both
equality and the least fixed point operator as \emph{logical
  connectives}: these logical connectives are not available directly
in Dedukti.

This new foundations for arithmetic has been implemented in two
systems: the automated Bedwyr prover and the interactive Abella
prover.  While neither Bedwyr nor Abella are as popular as many of the
theorem provers that are covered by this proposal, there are two
important reasons to consider incorporating them into the Logipedia
effort.

First, the Bedwyr prover is capable of constructing proofs for the
kind of queries that are part of emph{model checkers}.  This class of
provers has not yet been incorporated into Dedukti.  The
proof-theoretic work behind model checking in Bedwyr should provide
some of the insights needed for allowing Dedukti to proof check the
results of model checkers.

Second, Bedwyr and Abella provide for direct and elegant support of
meta-level reasoning.  Given that the foundations for Bedwyr and
Abella have been given using Gentzen's sequent calculus, it was
possible to enrich their foundations to allow for the treatment of
binding structures within terms.  As a result, it is possible to
reason directly on terms representing $\lambda$-terms and
$\pi$-calculus expressions.  In particular, the Abella prover has
probably the most natural and compact formal treatment of the
$\pi$-calculus and its meta-theory when compared to all other attempts
in any other theorem provers.  More generally, the Abella prover
should be able to treat the meta-theory of programming and
specification languages as well as various logics and their
proofs. While these tasks are not the typical tasks considered by the
majority of theorem provers within the scope of this proposal,
meta-theory results do play an important role at times: in fact, the
ultimate questions as to whether or not a proof checkers (such as that
used by Dedukti) is correct or not will involve meta-theoretic
questions.

We propose to work on the general problem of exporting proofs from
Abella to Dedukti.  (Since all proofs that are constructed
automatically via Bedwyr can also be constructed manually within
Abella, we shall limit our discussion below to Abella only.)  The
proposed work will serve not only to answer the question of how to
relate these two different foundations for arithmetic but also to
allow Abella's particular style of proofs to find applications in the
wider world of formalized proofs.

The general problem described above has the following constituent parts.

(1) Proofs involving searching finite structures. Proofs built for
model checking problems over finite structures have two different
kinds of phases.  To illustrate, consider trying to find a specific
node within a binary tree.  If such a node exists, then the proof
essentially encodes the path to the node in the tree.  If, however, no
such node exists, then the proof of that negative fact is essentially
a computation that exhaustively explores the tree.  Using the Dedukti
terminology: in the first case, the proof involves several deduction
steps, while in the second case, the proof involves a pure
computation. When dealing with model checking problems such as
simulation (in concurrency theory) and winning strategies (in game
theory), proofs will involve alternating phases involving either
deduction or computation.  Since the notion of computation in
Abella-style proofs involves backtracking search, that style
computation will be quite different from Dedukti's notion of
computation as confluent rewriting.

(2) Extending model checking problems to the general case of infinite
structures and the associated inductive reasoning methods. Although
the formal basis of Abella uses least and greatest fixed-point
combinators and explicit (co-)invariants, the Abella implementation of
(co-)induction is based on cyclic reasoning using size-annotated
relations. It is known, in principle, how to convert cyclic proofs
using annotations to proofs with explicit invariants, but an invariant
extraction procedure that works in all cases is still missing. Once
such invariants are available, incorporating them into Dedukti should
be straightforward in association with part (1).

(3) Binding structures. Abella, as well as several other computational
logic systems ($\lambda$Prolog, Isabelle/Pure, Twelf, Beluga, etc)
make use of the so-called \emph{$\lambda$-tree syntax} (a form of
\emph{higher-order abstract syntax}, HOAS) approach to represent
bindings. This approach is further enriched in Abella with the
$\nabla$-quantifier that allows inductive and co-inductive properties
to be defined based on the \emph{structure} of $\lambda$-terms. We
propose to examine encodings of $lambda$-tree syntax in Dedukti. The
best approach probably involves extending the underlying theory of
Dedukti with a quantifier similar to Abella's $\nabla$-quantifier.

(4) Reflective treatment of unification. One of the features of
Abella's style of proofs is the use of left-introduction rules for
equality that exhaustively examine complete sets of unifiers for
$\lambda$-terms. This is implemented in terms of a unification engine
that is currently a trusted black box, which complicates any proposal
for exporting proofs to different implementations of unification or
equality. In Dedukti the unification procedure can be recast as a
rewrite system, but it is unclear how to derive reflective properties
based on the unifiability of terms.
\end{task}

\begin{task}[title=id=hott,title=expressing HoTT]
\ednote{Saclay, Leeds}
\end{task}
\end{tasklist}

\begin{wpdelivs}
  \begin{wpdeliv}[due=3,miles=startup,id=requirements,dissem=PU,nature=DEM,lead=INR]
      {Requirements Analysis and Synchronization}
\end{wpdeliv}
\end{wpdelivs}
\end{workpackage}


%%% Local Variables:
%%% mode: latex
%%% TeX-master: "../propB"
%%% End:

\newpage
\begin{workpackage}[id=alignment,wphases=0-48,type=RTD,
  short=Concept Alignment,% for Figure 5.
  title=Concept Alignment,
  lead=Pra,
  PraRM=10]
  
\ednote{We need one coordinating site. original coordinators: Filip Marić and Dale Miller}

\ednote{Parties initially expressing interest (add their sites and RM
  here): Florian Rabe, Cezary Kaliszyk, Dale Miller, Josef Urban,
  Yamine Ait Ameur, Jean-Paul Bodeveix, Mamoun Filali, Chantal Keller,
  Julien Narboux, Nicola Magaud, Arthur Charguéraud, François Thiré}

\begin{wpobjectives}
The various proof assistants have different treatments of fundamental
concepts used in logic and arithmetic.  This WP will develop
standards, tools, and techniques that will allow these concepts to be
aligned so that proofs in one proof assistant can be meaningfully used
in other systems.

The following are the three main points on which this workpackage will
focus. 
\begin{compactitem}
\item Alignments at the levels of logic.  The alignments between
  classical and intuitionistic proofs is the main challenge here.
  There are different kinds of embeddings of classical proofs into
  intuitionistic logic.  A secondary challenge is to align the various
  treatments of induction and co-induction in proof assistants: these
  treatments include explicit presentations of invariants,
  applications of invertible inference rules, and cyclic proof
  structures.

\item Alignments of theorem proving objects such as constants,
  theorems, and types.
  \begin{compactitem}
     \item Similar concepts in different libraries can have many
       significant differences once one examines the concepts in detail.

     \item approximate matching constant, where some properties that
          holds for c1 also holds for c2.
  \end{compactitem}

\item Alignments of proofs
  \begin{compactitem}
     \item Often it is not enough that we simply trust a proof to have been
          checked.  We occasionally need to work with proofs in order
          to extract an explanation or its constructive content.  
     \item Identify tactics and inference rules that have the same
       effect on proof state. 
     \item Recognizing proofs that have the same structure. (proof porting)
  \end{compactitem}
\end{compactitem}
\end{wpobjectives}

\begin{wpdescription}
Construct tools and proofs to analyze these proofs and align concepts, that is unify
concepts such as connectives and quantifiers, the concept of natural number, etc. and
theorems that occur in several libraries.  [Paris, Saclay, Innsbruck, Prague,
Strasbourg, Belgrade]

Discovery/finding objects in different theories that refers to the
same informal concepts.
There are logic inspired techniques (check that one interface formally
entails another interface): ATPs might be able to automatically handle
such checks. (Some articulation needed with WP4.)
There can be other, statistical or linguistic clues that might help
narrow down on discovering possibly useful related concepts.

\end{wpdescription}

Task: Should we attempt to ``make classical logic proofs constructive'' when possible?

\begin{tasklist}
\begin{task}[id=aligndef,title=Tracking classical and intuitionistic proof steps in logic and arithmetic] 
\end{task}

\begin{task}[id=aligndef,title=Alignment in particular domains]
  Arithmetic (FAU Erlangen-Nürnberg), Geometry (University of
  Strasbourg), and real analysis (Inria Saclay).
\end{task}

\begin{task}[id=aligndef,title=Definition of an Alignment Language]
(FAU Erlangen-Nürnberg)
\end{task}

\begin{task}[id=translate,title=Libraries as intermediates for translations]
(FAU Erlangen-Nürnberg)
\end{task}

\begin{task}[id=translate,title=Alignment of proof structures]
Inria Saclay
\end{task}

\begin{task}[id=aligntranslate,title=FAIR Services: Reuse across Libraries,lead=Fau,FauRM=12]
\ednote{Florian Rabe: At the workshop I was asked to lead this task, but the text in this file looks outdated.
So I'm putting some text here to store my notes. The coordinators should get back to me to discuss details.}
This task leverages the large library of alignments built in the previous tasks by building a major FAIR service focusing on \textbf{reuse}.
It uses alignment to translate formulas and theorems from one library to another.
\end{task}

\begin{task}[id=alignsearch,title=FAIR Services: Search across Libraries,lead=Fau,FauRM=12]
\ednote{Florian Rabe: At the workshop I was asked to add this task, but the text in this file looks outdated.
So I'm putting some text here to store my notes. The coordinators should get back to me to discuss details.}
This task leverages the large library of alignments built in the previous tasks by building a major FAIR service focusing on \textbf{search}.
Here users enter a search query relative to the central library built above and chooses which libraries to search in.
Using the functionality developed in \localtaskref{aligntranslate}, the query is then translated into the requested libraries and searched in each one.
Results from all libraries are aggregated and returned.
\end{task}

\end{tasklist}

\begin{wpdelivs}
  \begin{wpdeliv}[due=3,miles=startup,id=requirements,dissem=PU,nature=DEM,lead=Inr]
      {Requirements Analysis and Synchronization}
\end{wpdeliv}
\end{wpdelivs}
\end{workpackage}

%%% Local Variables:
%%% mode: latex
%%% TeX-master: "../propB"
%%% End:

\newpage
\begin{workpackage}[id=dissemination,wphases=0-48,
  short=Dissemination,% for Figure 5.
  title={Dissemination, communication, and exploitation},
  lead=ISa,
  ISaRM=10]
  
\begin{wpobjectives}
  The objective of this work package is to \ldots

This includes notably:
  \begin{compactitem}
  \item \ldots
  \end{compactitem}
  A key aspect will be to foster \ldots
\end{wpobjectives}

\begin{wpdescription}
  \ednote{Gilles will write} \ednote{MK: it is probably a good idea to copy from
    OpenDreamKit: see
    \url{https://github.com/OpenDreamKit/OpenDreamKit/blob/master/Proposal/WorkPackages/DisseminationCommunityBuilding.tex}}
\end{wpdescription}

\begin{tasklist}
  \begin{task}[id=industry,title=Club on industrials]
  \end{task}
  \begin{task}[id=teachers,title=Club on Teachers]
  \end{task}
\end{tasklist}

\begin{wpdelivs}
  \begin{wpdeliv}[due=3,miles=startup,id=requirements,dissem=PU,nature=DEM,lead=ISa]
      {Requirements Analysis and Synchronization}
\end{wpdeliv}
\end{wpdelivs}
\end{workpackage}


%%% Local Variables:
%%% mode: latex
%%% TeX-master: "../propB"
%%% End:

\newpage
\begin{workpackage}[id=management,type=MGT,
  short=Management,
  title=Management,
  lead=Inr,InrRM=36,BirRM=1,InnRM=1,SacRM=1,TumRM=1,IrtRM=1,LeeRM=1]
  
  \begin{wpobjectives}
    The scope of this work package is the overall management of the project activities led by the consortium. The management of the Logipedia project will ensure the necessary conditions to enable the project to achieve its objective(s) while meeting its cost, time and quality requirements. This includes the scientific, administrative, financial and legal management.

    Gilles Dowek, Inria senior researcher and professor at ENS Paris-Saclay, will be the coordinator and leader of this work package. Gilles is PI of many projects, including under the H2020 framework programme and international projects. He will also be supported by a deputy coordinator, an European project manager from the Innovation, Partnership and Transfer Office of Inria Saclay and a Chief engineer.

This therefore includes:
\begin{compactitem}
\item A scientific and technical coordination to create a vibrant scientific and technical environment within the project.
\item The overall management of the project and consortium according to the governance structure and procedures explained in section 3.2.
\item An efficient project management, as specified in 3.2, including:
  \begin{compactitem}
  \item Overall administrative and financial project management, including reporting to the European Commission.
  \item Quality management.
  \item Assessment and risk management, including conflict or dispute management.
  \end{compactitem}
\end{compactitem}
\end{wpobjectives}

\begin{tasklist}
  \begin{task}[id=coordination,title=Scientific and technical coordination,lead=Inr,InrRM=12,wphases=1-48]
    The scientific and technical coordination will be led by Inria senior researcher Gilles Dowek. He will be in charge of ensuring the implementation of the scientific strategy of Logipedia and thereby ensuring the growth of the Logipedia community. This task foresees a key role of scientific animation and to impulse the organisation of scientific activities, together with the WP leader in charge of dissemination. Gilles Dowek will supervise the ongoing scientific and technical coordination and help with the innovation management, together with the technical manager and the steering committee. The scientific coordinator will also chair the steering committee and the general assembly. The scientific coordinator will be the scientific point of contact within the consortium and for the consortium when the project needs to be represented.
  \end{task}

  \begin{task}[id=admin,title=Administrative and Financial Management,lead=Inr,InrRM=24,wphases=1-48]
    A European Project Manager (EPM) from the Transfer and Innovation team of Inria Saclay which has an extensive experience in handling innovation from research projects such as Logipedia. The consortium will therefore benefit from tailored and on-demand advice regarding the use and potential transfer of the research results during the course of the project.
The EPM will ensure the day-to-day management as it will be the administrative and financial point of contact for the consortium and a dedicated contact point for the European Commission. The meeting preparation and follow-up will be another task of the EPM and will include organising plenary meeting with the partner organisation, general assembly, minutes, review meeting with the consortium. Financial aspects will be a crucial task of the EPM and include: payment to partners; ensuring financial monitoring within the consortium and leading the financial reporting to the European Commission.
The EPM will also be responsible, together with the scientific coordinator, for ensuring the technical work and deliverables meet the technological objectives of the project according to the defined schedule. The EPM will work very closely with the work package leaders in order to monitor the progress of the technical work and to identify potential risks within each WP. The EPM also acts as a quality manager to ensure that the content of the deliverables meets the quality standards defined for the project. The EPM reports to the Scientific Coordinator.
The development and maintenance of collaborative tools will be ensured by Inria and monitored by the EPM. A common teleconference tool, storage space, reporting and intranet will be set up and detailed in the collaborative tools deliverable of M2.
  \end{task}

  \begin{task}[id=legal,title={Legal Management (data, ethics, GDPR)},wphases=1-48]
    The preparation of the Consortium and Grant agreement will be led by the European Project Manager at Inria Saclay. If a change arises during the course of the project, amendment will be prepared by the project management team, in close collaboration with Inria legal team. 
Data Protection, ethics and GDPR Compliance Management will also be ensured by INRIA. The main objective here is to provide guidance on data protection for the research activities of the project in the context of the European General Data Protection Regulation (GDPR). If at some point during the course of the project, the consortium or any scientist is unsure about how to handle a particular situation or requires advice on ethical issues, the partners or the individuals, supported by the EPM, will refer to the operational ethical committee of Inria (the COERLE) before proceeding.
  \end{task}

\end{tasklist}

\begin{wpdelivs}
  
  \begin{wpdeliv}[due=2,miles=???,id=collab-tools,dissem=PU,nature=DEC,lead=Inr]{Collaborative Tools} Document or Notice introducing the collaborative tools of the consortium
  \end{wpdeliv}

  \begin{wpdeliv}[due=3,miles=???,id=guide,dissem=PU,nature=R,lead=Inr]{Logipedia Partner Guide} In order to present the processes and governance within the consortium
  \end{wpdeliv}

  \begin{wpdeliv}[due=6,miles=???,id=data-plan,dissem=PU,nature=R,lead=Inr]{Data Management Plan}
  \end{wpdeliv}
  
\end{wpdelivs}

\end{workpackage}


%%% Local Variables:
%%% mode: latex
%%% TeX-master: "../propB"
%%% End:

\end{workplan}

\newpage

\subsubsection*{List of all deliverables}\label{sec:deliverables}

{\footnotesize\inputdelivs{8cm}}

%%% Local Variables: 
%%% mode: latex
%%% TeX-master: "propB"
%%% End: 


\subsection{Relation between the components}


\section{Management structure, milestones and procedures}

\begin{todo}{}\color{red}
  * Describe the organisational structure and the decision-making ( including a list of milestones (table 3.2a))

  * Explain why the organisational structure and decision-making mechanisms are appropriate to the complexity and scale of the project.

  * Describe, where relevant, how effective innovation management will be addressed in the management structure and work plan.

  Innovation management is a process which requires an understanding of both market and technical problems, with a goal of successfully implementing appropriate creative ideas. A new or improved product, service or process is its typical output. It also allows a consortium to respond to an external or internal opportunity.

  * Describe any critical risks, relating to project implementation, that the stated project's objectives may not be achieved. Detail any risk mitigation measures. Please provide a table with critical risks identified and mitigating actions (table 3.2b)

  * Give a summary of the trans-national and/or virtual access to be provided (table 3.2c).


  {\color{red} A table for risks}

    Risk name / WP / Impact if occurs / Probability / LEvels / Prevetive and
    Contengency action.

  {\bf Definition:}
  
  \underline{Milestones}: means control points in the project that help to chart progress. Milestones may correspond to the completion of a key deliverable, allowing the next phase of the work to begin. They may also be needed at intermediary points so that, if problems have arisen, corrective measures can be taken. A milestone may be a critical decision point in the project where, for example, the consortium must decide which of several technologies to adopt for further development.
\end{todo}


{\color{red} A table with a list of milestones}

{\color{red} Inria Saclay transfer, innovation, and partnetship
  department will contribute to the innovation management}
The Logipedia consortium will gather twenty nine beneficiaries and partners
from eleven European countries during four years. The project management
structure will be tailored to the specificities and needs of this
large consortium and its ongoing network development.

\subsection{Organisational structure}

\subsubsection*{The project management team}

{\bf The Coordinator}: the 
coordinator is responsible for the coordination of
scientific and technical activities in order to meet the objectives
set by the European Commission in the Grant Agreement. The 
coordinator works closely with the work package leaders
within the steering committee, in order to monitor the progress of the
scientific and technical work and identify potential risks within each
work package. The coordinator will daily
collaborate with the European project manager in charge of the
day-to-day management of Logipedia. The project will be managed by Pr
Gilles Dowek, permanent senior researcher at Inria Saclay. He will
also chair the meetings of both the general assembly and steering
committee.

{\bf The Deputy Coordinator}: The Deputy coordinator seconds and
replaces the coordinator.

{\bf The European Project Manager}: The European project manager
member of the Technology Transfer and Partnership Office of Inria
Saclay, is in charge of all administrative, financial and legal
management tasks as listed in \WPref{management}. The
European project manager is the interface between the project and the
European Commission as it represents the point of contact for the
European Commission. The European project manager has the overall
administrative and financial responsibility for the organisation and
administrative and financial monitoring of the project.

{\bf The Chief Engineer}: The chief engineer is an experienced
research engineer from Inria Saclay and is responsible for ensuring
the development and maintenance of tools at Inria Saclay and
supervising the development tasks achieved at the other
beneficiaries. The chief engineer will ensure the coherence of the
Logipedia tools development, according to the defined schedule in the
Grant Agreement.

Innovation management and intellectual property rights issues will be
handled by Inria and the European project manager, supported by the
experienced Technology Transfer and Partnerships Office of Inria
Saclay. The project management team will establish appropriate
policies and rules for the management of intellectual property rights
for the knowledge developed within the project, as well as the
identification of the opportunities for the exploitation of the
project results in innovation activities. Issues related to innovation
and/or intellectual property rights management will be tackled at
every steering committee meeting.

\subsubsection*{The operational level}

{\bf The Steering Committee}: The steering committee is composed of
the coordinator, the chief engineer, the European project
manager and the work package leaders. The steering committee is the
supervisory body for the implementation of the project. The steering
committee is responsible for monitoring the activities of the project
and the implementation of decisions taken by the general assembly. It
can formulate proposal for changes in the description of action and
the related consortium budget. Those changes will have to be agreed
by the general assembly first and then the European
commission. The steering committee is chaired by the 
coordinator.

{\bf The Work Package Leaders}: The work package leaders are
responsible for the monitoring and management of the activities and
results within their work packages. In particular, work package
leaders i) identify deviations from the project plan and report them
to the steering committee, ii) manage and supervise the preparation of
reports and their timely delivery, iii) control and monitor activities
of tasks and regularly meet once per month with task leaders, iv)
manage the information flow with other work packages via the steering
committee.

{\bf The Task Leaders}: The task leaders are responsible for
coordinating the scientific and technical work in their task and
making the day to day technical decisions that solely affect their
task. Inter-task decisions are coordinated with the work package
leaders.

{\bf The Club Leaders}: The club leaders are in charge of
disseminating of the tools developed by the Logipedia consortium in
various communities. They organize the activity of the club. They give
ongoing feedback to the consortium during the course of the project.


\subsubsection*{The strategic level}

{\bf The General Assembly}: The general assembly is composed by all
the members of the consortium, with each representative having one
vote. Every new partner will have a voting right. The general assembly
will gather at least once a year, and as many virtual meetings as
needed. The general assembly is the main governance and ultimate
decision-making body of the consortium. The general assembly must
review the project progress, decide on contingency actions in case of
deviations from the plan and take final decisions on policy and
contractual issues and conflicts as requested by the steering
committee.

{\bf The Advisory Board}: The advisory board is a consultation body to
the steering committee and general assembly. It will bring external
and non-legally binding perspective on the scientific and technical
development of the project, ecosystem building and the future of the
encyclopedia. The advisors of this board will attend the yearly
general assembly plenary meeting and will be consulted on the strategy
of the project. The advisory board should aim at representing the
stakeholders of the Logipedia ecosystem without including any
beneficiary or associate partner’s employees. It will be composed of,
among others, industrial and international academic partners
(including non-European ones) apointed by the coordinator after
consulting the steering committee. To start with, we suggest to include
\begin{compactitem}
\item June Andronick (Data61, Kensington NSW), 
\item Denis Cousineau (Mitsubishi Electric), 
\item Thomas Letan (ANSSI), 
\item Jacques Fleuriot, 
\item Natarajan Shankar (SRI),
\item Aaron Stump (Iowa), 
\item Laurent Voisin (Systerel).
\end{compactitem}

 \subsubsection*{Internal communication and collaborative ecosystem}

The communication of the consortium including their internal tools is
managed in task 10.2.  The consortium will make use of a number of
project management tools, such as a visio conferencing tool, a project
repository to have an updated account of the project’s important
documents, the progress of the work packages work and deliverables,
all the advances in the project and all the meetings minutes, mailing
lists, etc. that facilitate the smooth execution of the project. This
collaboration environment will be provided by the coordinator of the
project.

Work packages, chaired by work package leaders, will have monthly
planned visio conferences and meetings as need by the work plan;
additional technical meetings may be set up by task leaders or
individual partners. The steering committee will have monthly visio
conferences and will meet twice a year. Dedicated working groups will
be planned as needed according to the work plan.  All meetings will be
documented by minutes listing major decisions and action items.


The project management team will be in charge of all organisation
issues in the general assembly meetings, supported by the local
partner. The project will organise meetings of the general assembly at
least once a year. To equally share travel costs among partners,
physical meetings will be located by rotation at partners’
locations. Project review meetings will be done on a regular basis
according the Grant Agreement provisions.

\subsection{Decision-making Process}

Our approach for the decision-making process is to locate the decision
as close as possible to the level responsible for the execution (from
task level to general assembly level). Decisions are managed within
frequent project meetings, either on-site or via
teleconference. Decisions can be also managed by consultation. If
voting is needed, the agenda should clearly indicate this fact. Quorum
and voting rules will be defined in the Consortium
Agreement. Decisions are binding once the relevant part of the meeting
minutes has been accepted. Any changes to the project plan and scope
must be reviewed and approved by all levels of project management,
before proposing these changes to the steering committee and any
modifications will be considered rejected, after rejection on any of
these involved levels.

Another guiding principle is to avoid conflicts. Nevertheless, should
one arise, a conflict resolution will be ready to be put in place to
deal with it accordingly. The conflict resolution foresees that each
conflict will be mediated, solved or decided at the lowest level
possible. Attempts to solve issues within the consortium will be
carried out in increasing order of authority first at task level
(management of task leader), work package level (management of work
package leaders), and then following the management bodies till the
general assembly. Further rules related to conflict resolutions will
be laid out in the Consortium Agreement.

\subsection{Monitoring and reporting}

\subsubsection*{Internal reporting}

The project management team continuously monitors the project plan
with its milestones and critical paths. Each work package leader will be
responsible for the correct execution of the implementation plan for
the corresponding work package. In terms of reporting, this means the work package leaders
will be in charge of gathering the information related to their own
work packages.

Regular audio-conferences of the Steering Committee are foreseen,
which allows work package leaders to identify and raise risks and
discuss them together. This ensures that management (coordination,
European project manager) is aware of potential problems and
deviations and can initiate countermeasures long before a situation
becomes critical. This ensure to spot the blocking points in due time
and to find that the solutions will be available in time.

In case there is a deviation from the work plan, the 
coordinator will initiate corrective actions through the
task leader and the work package leader. The work package leader will
be responsible to implement these actions in dialogue with the
different partners involved in their work packages.


\subsubsection*{Reporting to the European Commission}

The Logipedia consortium will follow the mandatory reporting period
required by the European Commission. The following reporting will be
achieved: Period 1 (M01-M18), Period 2 (M19-M36) and Period 3
(M36-M48).

The project management team will provide the necessary templates in
order to achieve the reporting in due time. Work package leaders will
be asked to gather the relevant information provided by the task
leader regarding their work package and to summarise in order to be
reviewed by the steering committee. It will then be treated by the
coordinator and European project manager and sent to the
European Commission.

\subsection{Significant Risks and Associated Contingency Plans}\label{sec:risks}

\begin{todo}{from the proposal template}
  Describe any significant risks, and associated contingency plans
\end{todo}
\begin{oldpart}{need to integrate this somewhere. CL: I will check other proposals to see how they did it; the Guide does not really prescribe anything.}
\paragraph{Global Risk Management}
The crucial problem of \pn (and similar endeavors that offer a new basis for communication
and interaction) is that of community uptake: Unless we can convince scientists and
knowledge workers industry to use the new tools and interactions, we will
never be able to assemble the large repositories of flexiformal mathematical knowledge we
envision. We will consider uptake to be the main ongoing evaluation criterion for the network.
\end{oldpart}



1. Risks

- more difficult that expected (probability: low, severity medium). Mitigation: at least some systems

- too many specific proofs (probability: low (empirical evidence in informal maths), severity medium). Mitigation: do not translate these proofs and start understanding why they need strong axioms, but translate the rest (basic maths).

- too high complexity (time and memory), difficulty to scale up (probability medium, severity medium). Mitigation: lower the objectives (basic maths, use more powerful machines, wait for Moore's law to help you).

- one partner leaves (probability low) or does not deliver (probability medium). Mitigation: downsize the project.

- difficulty to find people (doctoral students, post-docs) in some countries. Mitigation use the size of the network to find more peoples in others.

- Logipedia splits into several libraries: face the risk  (we have avoided to have a classical and a constructive logipedia, a predicative one and a non-predicative one, the diversity of theories expressed in logipedia permits to make the probability very low). 

- a beautiful encyclopedia, but nobody cares (probability low). Mitigation: improve the interface, the communication, make it more completed

2.  Opportunities

- More people want to join: model checking, sat solvers… 

- math teachers want to use it for teaching


{\color{red} Draft a list of milestones}

\subsection{Milestones}\label{sec:milestones}

\begin{todo}{from the proposal template}
  Milestones are control points where decisions are needed with regard to the next stage
  of the project. For example, a milestone may occur when a major result has been
  achieved, if its successful attainment is a requirement for the next phase of
  work. Another example would be a point when the consortium must decide which of several
  technologies to adopt for further development.

  Means of verification: Show how you will confirm that the milestone has been
  attained. Refer to indicators if appropriate. For examples: a laboratory prototype
  completed and running flawlessly, software released and validated by a user group, field
  survey complete and data quality validated.
\end{todo}

\ednote{maybe automate the milestones}

\ednote{Rabe: I suggest having exactly 3 milestones, namely at months 18, 36, and 48 (corresponding to the EU's review schedule), possibly more milestones in the beginning e.g., at months 6 and 12}

\begin{milestones}
  \milestone[id=kickoff,verif=Inspection,month=1]
    {Organization setup}
    {Set up the organizational infrastructure of the project: mailing lists, web site, consortium agreement, activity tracking, \ldots}

  \milestone[id=logipedia-v1,verif=Inspection,month=12]
     {Logipedia v1}
     {Release of a first version of Logipedia with HOL Light standard library and parts of Matita standard library in 5 different systems: Coq, Matita, Lean, HOL and PVS}

  \milestone[id=coq-stdlib,verif=Inspection,month=24]
     {Coq in Logipedia}
     {Integration of most of Coq standard library in Logipedia}

  \milestone[id=isabelle-stdlib,verif=Inspection,month=24]
     {Isabelle/HOL in Logipedia}
     {Integration of most of Isabelle/HOL standard library in Logipedia}

  \milestone[id=compcert,verif=Inspection,month=36]
     {CompCert in Logipedia}
     {Integration of most of the CompCert library in Logipedia}

  \milestone[id=logipedia-v2,verif=Inspection,month=36]
     {Logipedia v2}
     {Release of a second version of Logipedia integrating important parts of the libraries of Isabelle, Coq, Matita and HOL4, and their translations in other systems}

  \milestone[id=atelierb,verif=Inspection,month=48]
     {Atelier B in Logipedia}
     {Release of a tool able to translate a complete development in Atelier B into a complete Dedukti proof}

\end{milestones}


\section{Consortium as a whole}\label{sec:consortium}


\begin{todo}{}\color{red}

  The individual members of the consortium are described in a separate
  section 4. There is no need to repeat that information here.

  Describe the consortium. How will it match the project’s objectives, and bring together the necessary expertise? How do the members complement one another (and cover the value chain, where appropriate)? 

  In what way does each of them contribute to the project? Show that each has a valid role, and adequate resources in the project to fulfil that role. 

  If applicable, describe the industrial/commercial involvement in the project to ensure exploitation of the results and explain why this is consistent with and will help to achieve the specific measures which are proposed for exploitation of the results of the project (see section 2.2). 

  Other countries and international organisations: If one or more of the participants requesting EU funding is based in a country or is an international organisation that is not automatically eligible for such funding (entities from Member States of the EU, from Associated Countries, from one of the countries in the exhaustive list included in General Annex A of the work programme, and, under specific conditions, from further countries identified in the work programme1,are automatically eligible for EU funding), explain why the participation of the entity in question is essential to carrying out the project .
\end{todo}

\begin{todo}{from the proposal template}
  Describe how the participants collectively constitute a consortium capable of achieving
  the project objectives, and how they are suited and are committed to the tasks assigned
  to them. Show the complementarity between participants. Explain how the composition of
  the consortium is well-balanced in relation to the objectives of the project.

  If appropriate describe the industrial/commercial involvement to ensure exploitation of
  the results. Show how the opportunity of involving SMEs has been addressed
\end{todo}

The project partners of the \pn project have a long history of successful collaboration;
Figure~\ref{tab:collaboration} gives an overview over joint projects (including proposals) and
joint publications (only international, peer reviewed ones).

\jointorga{Fau,Bol}% CICM
\jointorga{Inn,Bol}% CICM
\jointpub{Fau,Bol}% CICM paper
\jointpub{Tum,Bol}% CICM paper
\jointpub{Fau,TUM}%
%\jointsup{Fau,}
\jointsoft{Fau,Tum}% Isabelle Extension
\jointsoft{Fau,Bol}% Coq exporter
\jointpub{Inr,Bol}% ELPI
\jointproj{Inr,Bol}% MoWGLI


\jointpub{Pra,Stu}% CADE 2015 paper
\jointOrga{Inr,Stu}% 3rd PAAR, 5th PAAR

\coherencetable

\subsection{Subcontracting}\label{sec:subcontracting}

\begin{todo}{from the proposal template}
  If any part of the work is to be sub-contracted by the participant responsible for it,
  describe the work involved and explain why a sub-contract approach has been chosen for
  it.
\end{todo}

The tasks \taskref{instrumentation}{isabelle},
\taskref{libraries}{afp} (both handled by \site{Tum}) and part of
task~\taskref{structuring}{strontorepml} (handled by \site{Fau}) will
be carried out by subcontracting Dr.\ M.\ Wenzel.  Each subcontract
will cover roughly the equivalent of $3$ person-months.  Concretely,
it concerns the export of proof terms and other data from the Isabelle
system.  Wenzel is the main Isabelle developer and has spent the
last $10$ years building the technological prerequisites for the
required work and is the natural person to carry it out.  However, he
has left academia and started his own company that specializes on
Isabelle kernel development and routinely carries out subcontracts for
Isabelle-related research projects.  Therefore, a subcontract is the
best option as developing the necessary expertise in-house would take
an additional 6-12 person-months per task.  \site{Fau} has already worked with
Wenzel in similar subcontracts twice before (including the
OpenDreamKit EU infrastructure project), and the collaborations have
been very effective and efficient. Wenzel obtained his Ph.D. at
\site{Tum} advised by Nipkow.

The task \taskref{instrumentation}{isabelle} (handled by \site{Tum})
requires special assistance by Dr.\ David Matthews (PROLINGUA LTD,
Edinburgh): As provider of the underlying Poly/ML infrastructure,
Matthews is in a unique position to provide extra scalability of ML
heap management, and thus allow Isabelle to export more library
material.


\subsection{Other Countries}\label{sec:other-countries}
\begin{todo}{from the proposal template}
  If a one or more of the participants requesting EU funding is based outside of the EU
  Member states, Associated countries and the list of International Cooperation Partner
  Countries\footnote{See CORDIS web-site, and annex 1 of the work programme.}, explain in
  terms of the project’s objectives why such funding would be essential.
\end{todo}

\subsection{Additional Partners}\label{sec:assoc-partner}
\begin{todo}{from the proposal template}
  If there are as-yet-unidentified participants in the project, the expected competences,
  the role of the potential participants and their integration into the running project
  should be described
\end{todo}

%%% Local Variables:
%%% mode: latex
%%% TeX-master: "propB"
%%% End:


{\color{red} A table with all partners in lines and Key expertise in colum
  and explain who is good at what}


\section{Resources to be Committed}\label{sec:resources}

\begin{todo}{}\color{red}
Please make sure the information in this section matches the costs as stated in the budget table in section 3 of the administrative proposal forms, and the number of person months, shown in the detailed work package descriptions.

Please provide the following:

- a table showing number of person months required (table 3.4a)

- a table showing ‘other direct costs’ (table 3.4b) for participants where those  costs exceed 15\% of the personnel costs (according to the budget  table in section 3 of the administrative proposal forms) and participants providing trans-national access under this project and incurring travels and subsistence costs for supporting users' access.

Please note that the distribution of resources between access (TA/VA), JRA, and NA components must be duly justified.
\end{todo}


{\color{red} A table with WP in columns and parners in line and
  ressources in pm}

\subsection{Travel Costs and Consumables}\label{sec:travel-costs}

\subsection{Subcontracting Costs}\label{sec:subcontracting-costs}

As explained in Section~\ref{sec:subcontracting}, \site{Tum} asks for
EUR~50.000 and \site{Fau} for EUR~20.000 to subcontract Dr.\ M.\ Wenzel to
carry out (part of) the work in task
\taskref{instrumentation}{isabelle}, \taskref{libraries}{afp} and
\taskref{structuring}{strontorepml}. Further EUR~15.000 are required
by \site{Tum} to subcontract Dr.\ D.\ Matthews to participate in
critical parts of task \taskref{instrumentation}{isabelle}.
\ednote{Rabe@Dowek: This is part
  of the WP7 budget. But it is not included in the PMs in WP7 because
  money for subcontracts must be declared in a different column in the
  official EU tables. Instead, it is listed here.}




\subsection{Other Costs}

%%% Local Variables: 
%%% mode: LaTeX
%%% TeX-master: "propB"
%%% mode: flyspell
%%% ispell-local-dictionary: "english"
%%% End: 

% LocalWords:  pn newpage site-FAU site-efo site-baz jointpub efo baz
% LocalWords:  jointproj coherencetable assoc-partner

\newpage
\chapter{Members of the consortium}

\section{Individual Participants}\label{sec:partners}
\begin{todo}{from the proposal template}
For each participant in the proposed project, provide a brief description of the legal entity, the main
tasks they have been attributed, and the previous experience relevant to those tasks. Provide also a
short profile of the individuals who will be undertaking the work.\\
Maximum length for Section 2.2: one page per participant. However, where two or more departments within
an organisation have quite distinct roles within the proposal, one page per department is acceptable.\\
The maximum length applying to a legal entity composed of several members, each of which is a separate
legal entity (for example an EEIG1), is one page per member, provided that the members have quite distinct
roles within the proposal.
\end{todo}
\newpage
\begin{sitedescription}{UBel}

\paragraph{Organization:}
Automated Reasoning Group (ARGO, \url{http://argo.matf.bg.ac.rs}) at
Faculty of Mathematics, University of Belgrade is interested in
automated reasoning, especially in SAT and SMT (satisfiability modulo
theories), interactive theorem proving, automated theorem proving in
coherent logic, automated reasoning in geometry, software verfication
and other applications of automated and interactive theorem proving.

\paragraph{Main tasks:}

\begin{compactitem}
\item\ednote{specify the main tasks and reference the respective work packages} 
\end{compactitem}


\paragraph{Relevant previous experience:}

Members of the group have expertise in the field of automated theorem
proving (especially in coherent logic and geometry, SAT/SMT solving,
and interactive theorem proving). They have been involved in several
national, bilateral and international projects and grants in the field
of automated reasoning (a COST project, Swiss SCOPES grant,
Serbian-French Technlology Co-Operation grant, etc).

\paragraph{Specific expertise:}

\begin{compactitem}
\item Coherent-logic provers and their application in automated
  theorem proving in geometry (Janičić, Stojanović-Đurđević,
  Marinković), and automated theorem proving in geometry using
  algebraic methods (Janičić, Simić, Marić);
\item Automated solving of geometric construction problems and
  verifying correctness of solutions (Marinković);
\item Isabelle/HOL verification of underlying SAT and SMT solving
 procedures (Marić, Janičić);
\item Applications of SAT and SMT solvers integrated into Isabelle/HOL
  in verifying algorithm correctness and solving combinatorial
  conjectures in a formal, mechanical verified setting (Marić,
  Janičić);
\item Interactive theorem proving in Euclidean an Hyperbolic geometry
  (Marić, Simić)
\end{compactitem}

\paragraph{Staff members undertaking the work:}

\textbf{Filip Marić}\ednote{describe the site leader and his expertise}
\textbf{Predrag Janičić}
\textbf{Vesna Marinković}
\textbf{Danijela Simić}
\textbf{Sana Stojanović-Đurđević}
\ednote{provide the key publications below}
\keypubs{providemore}


\end{sitedescription}
%%% Local Variables: 
%%% mode: latex
%%% TeX-master: "../propB"
%%% End: 

% LocalWords:  site-jacu.tex clange sitedescription emph compactitem pn semmath
% LocalWords:  prosuming-flexiformal KohSuc asemf06 GinJucAnc alsaacl09 StaKoh
% LocalWords:  tlcspx10 KohDavGin psewads11 ednote Radboud Bia ystok CALCULEMUS
% LocalWords:  textbf keypubs OntoLangMathSemWeb uwb Deyan Ginev Stamerjohanns
% LocalWords:  searchability
\newpage
\begin{sitedescription}{Bia}

\logo{Bialystok}

%\paragraph*{Organization:}

The University of Bialystok (UwB) was established in 1997 from a~branch of Warsaw University after 29 years of its existence.
Today UwB is one of the largest and strongest academic centres in North-Eastern Poland.
It consists of nine faculties (including one located in Vilnius, Lithuania) and five institutes.
Classes and lectures are delivered by approx. 850 academic teachers (nearly 200 are independent research scholars).
At present UwB educates over 8000 students in almost 30 fields of study.

The Mizar research group at UwB has several decades of experience in designing formal languages 
for efficient encoding of mathematical data and implementing formal proof-checking software.
The group coordinates the development of the Mizar Mathematical Library (MML) -- 
a~large centralised collection of formalised mathematical definitions, theorems and their proofs 
authored by over 260 contributors from 20 countries.
The library is maintained and distributed in a~variety of data formats, 
including interactive web-based documents and automatically generated natural language journal articles. 
The members of the group have participated in a~number of EU funded research
and collaboration projects,
as well as the EUTYPES Cost Action.
The Mizar group has also organised the MKM 2004 and CICM 2016 conferences.

\paragraph*{Main tasks:}

\begin{compactitem}
\item Expressing the foundations of the Mizar logic in Dedukti. \WPtref{theories} \taskref{theories}{mizar}
\item Extracting in-depth knowledge from the Mizar proofs. \WPtref{theories} \taskref{theories}{mizar}
\item Developing Dedukti techniques corresponding to Mizar proof checking. \WPtref{theories} \taskref{theories}{mizar}
\end{compactitem}

\paragraph*{Publications, products or services:}

\begin{compactitem}

\item ``The role of the {M}izar {M}athematical {L}ibrary for interactive proof development in {M}izar'',
by G.~Bancerek and C.~Byliński and A.~Grabowski and A.~Korniłowicz and R.~Matuszewski and A.~Naumowicz and K.~Pąk,
Journal of Automated Reasoning \textbf{61}(1), pp.~9--32, 2018.
%\url{https://doi.org/10.1007/s10817-017-9440-6}

\item ``Mizar: State-of-the-art and Beyond'',
by G.~Bancerek and C.~Byliński and A.~Grabowski and A.~Korniłowicz and R.~Matuszewski and A.~Naumowicz and K.~Pąk and J.~Urban,
Intelligent Computer Mathematics, International Conference, CICM 2015, Washington, DC, USA, 
July 13--17, 2015, Proceedings., (M. Kerber, J. Carette, C. Kaliszyk, F. Rabe, V. Sorge Ed(s).), 
Lecture Notes in Comput. Sci. vol. 9150, pp.~261--279, Springer, Berlin, 2015.
%\url{https://doi.org/10.1007/978-3-319-20615-8_17}

\item ``Semantics of Mizar as an Isabelle Object Logic'',
by C.~Kaliszyk and K.~Pąk,
Journal of Automated Reasoning \textbf{63}(3), pp.~557--595, 2019.
%\url{https://doi.org/10.1007/s10817-018-9479-z}

\item ``Scalable Declarative Proof Translation'',
by C.~Kaliszyk and K.~Pąk,
Tenth International Conference, Interactive Theorem Proving, ITP 2019, Portland, OR, USA. 
Proceedings,  LIPIcs, Vol. 141, 35:1--35:7, 2019.
%\url{https://doi.org/10.4230/LIPIcs.ITP.2019.35}

\item ``Higher-order Tarski Grothendieck as a~Foundation for Formal Proof'',
by C.E.~Brown and C.~Kaliszyk and K.~Pąk,
Tenth International Conference, Interactive Theorem Proving, ITP 2019, Portland, OR, USA. 
Proceedings,  LIPIcs, Vol. 141, 9:1--9:16, 2019.
%\url{https://doi.org/10.4230/LIPIcs.ITP.2019.9}

\end{compactitem}

\paragraph*{Previous projects or activities:}

The Mizar research group has carried out several grants within European Union Framework Projects 
and also funded by Polish National Science Center and Office of Naval Research, US.
The most related to the project are:

\begin{compactitem}
\item ``Isabelle Emulator for Mizar: Environment for Mizar Mathematical Library Re-verification'',
funded by Polish National Science Center, project manager: Karol Pąk, 7/2016--7/2019
\item ``Independent Verification of Mizar Logic'', funded by the OeAD Scientific \& Technological Cooperation with Poland,
project coordinator at the Polish side: Karol Pąk, 5/2016--4/2018
\item ``Algorithms Concerning the Legibility of Natural Deduction Proofs'', funded by Polish National Science Center,
project manager: Karol Pąk, 7/2013--1/2017
\item ``Management of a~Large Repository of Computer Verified Mathematical Knowledge'',
funded by Polish Ministry of Science and Higher Education, project manager: Andrzej Trybulec, 5/2009--5/2012
\item ``Types for Proofs and Programs'', TYPES II EU FP6 510996,
site of the project coordinated by Chalmers, 9/2004--8/2007
\end{compactitem}

%\paragraph*{Specific expertise:}

%\begin{compactitem}
%\item Expressing and translating formal semantics
%\item Experience with Mizar kernel augmentation for proof object extraction
%\item In-depth knowledge of the Mizar foundations
%\item Managing large mathematical repositories
%\end{compactitem}

\paragraph*{Infrastructures or technical equipments:}

\begin{compactitem} 
\item Mizar proof-assistant -- one of the pioneering systems for mathematics formalisation (since 1973).
\item Mizar Mathematical Library -- a centrally-managed mathematical knowledge base (established in 1989).
\end{compactitem}

\paragraph*{Persons primarily responsible for carrying out the proposed activities:}

\begin{compactitem}

\item\textbf{Czesław Byliński} is head of the Computer Networks Section at the University of Bialystok.
He received his PhD in computer science from Shinshu University, Japan in 1998.
Since 1978 he has been a~member of the Mizar Project.
He participates in the implementation of the Mizar language and the developing the Mizar system tools. 
Since 2014, he has been in charge of the Mizar implementation team.

\item\textbf{Adam Grabowski} is an adjunct at UwB since 2006, 
with a~focus on the formalisation of mathematics and computer science.
He received his PhD in mathematics from the University of Silesia in Katowice, Poland in 2005 
and PhD in computer science from Shinshu University, Nagano, Japan in 2005.
Currently, he works on the application of automated proof assistants
in the modelling of the reasoning under uncertainty: fuzzy and rough sets.
He has authored over 120 papers in refereed journals and international conference
proceedings, including over 70 formalisations in Mizar.
He received twice the Śleszyński Prize (1998, 2000) granted by the Association of Mizar Users.
Since 1999 he has been the head of the Library Committee of Association of Mizar Users, taking care of
the management and development of the Mizar Mathematical Library.

\item\textbf{Artur Korniłowicz} is the deputy director for science 
and head of the Department of Programming and Formal Methods
at the Institute of Informatics at the University of Bialystok.
Korniłowicz received his PhD in computer science from Shinshu University, Nagano, Japan in 2001.
In 2017 he received habilitation from the University of Warsaw, Poland.
Korniłowicz's main research interests are in formal verification of mathematics and verification of algorithms.
He is one of the key developers of the Mizar proof-assistant and the author of over 100 Mizar formalisations.
In 2005 he was awarded Śleszyński Prize for Formalisation of the Jordan Curve Theorem.
In the period 7/2001--6/2002 Korniłowicz was a~CALCULEMUS postdoctoral fellow
at the Istituto per la Ricerca Scientifica e~Tecnologica, Trento, Italy under the CALCULEMUS project within EU FP5;
and in the period 7/2003--3/2005 he was a~Japan Society for the Promotion of Science 
postdoctoral fellow at the Shinshu University, Nagano, Japan.

\item\textbf{Adam Naumowicz} is a~member of the core Mizar development team. 
With his background in mathematics and linguistics, he received his PhD in computer science in 2005
for research on formalising recent mathematical results. His recent works focus on extending Mizar checker's computational power, 
interacting with external tools and developing web-based services. 
He's been elected twice to serve as Mathematical Knowledge Management representative to the Steering Committee
of Conference on Intelligent Computer Mathematics (CICM).
He was also the main organiser of CICM 2016 held at the University of Bialystok. 
He acts as Poland's representative in the Management Committee of the European research network on types
 for programming and verification (Cost Action EUTypes).

\item\textbf{Karol Pąk} has developed the Isabelle/Mizar system where
he specified Mizar in the Isabelle logical framework
giving the complete semantics of the system, including
the underlying first-order logic variant, soft type system, and definitional mechanisms.
Additionally, he proposed a semi-automatic translation of several MML articles
to the resulting object logic to cross-verify them.
Furthermore he has been developing methods
that automatically improve readability of natural deduction proofs
by the step order manipulation as well as lemma extraction.

\end{compactitem}

\end{sitedescription}

%%% Local Variables:
%%% mode: latex
%%% TeX-master: "../propB"
%%% End:

% LocalWords:  site-jacu.tex clange sitedescription emph compactitem pn semmath
% LocalWords:  prosuming-flexiformal KohSuc asemf06 GinJucAnc alsaacl09 StaKoh
% LocalWords:  tlcspx10 KohDavGin psewads11 ednote Radboud Bia ystok CALCULEMUS
% LocalWords:  textbf keypubs OntoLangMathSemWeb uwb Deyan Ginev Stamerjohanns
% LocalWords:  searchability
\newpage
\begin{sitedescription}{Bol}

\paragraph{Organization:}
Founded in 1088, the Alma Mater Studiorum – Università di Bologna (UNIBO) is known as the oldest University of the western world. Nowadays, UNIBO still remains one of the most important institutions of higher education across Europe and the second largest university in Italy. UNIBO is organized in a multicampus structure with 5 operating sites and, since 1998, also a permanent headquarters in Buenos Aires: 11 Schools, 33 Departments, 12 Research and Innovation Centers and more than 84.000 students.

The activity of the University of Bologna are conducted within the Department of Computer Science and Engineering (DISI), which is one of the top Computer Science and Engineering departments in Italy, offering a broad spectrum of expertise ranging from theoretical computer science to software, hardware and application design and development.

The research group that will be in charge of Logipedia at UNIBO is leaded by Dr. Claudio Sacerdoti Coen. The group is active in the areas of formal methods, interactive theorem proving and mathematical knowledge management, which are all relevant to the project.

\paragraph{Main tasks:}

\begin{compactitem}
\item\ednote{specify the main tasks and reference the respective work packages} 
\end{compactitem}


\paragraph{Relevant previous experience:}

Under the former supervision of Prof. Andrea Asperti, the research group coorinated the FET-Open EU Project MoWGLI (Math on the Web: Get it by Logic and Interfaces). The project was focused on making the library of the Coq prover easily accessible outside the system and on the Web. MoWGLI explored independent verification, indexing, search and retrieval and transformation of proofs coming from Coq. All the previous services were implemented as web services using W3C technologies. Logipedia is more ambitious in aiming at providing the same services but for every system at once.

The code implemented in MoWGLI became later the core of the interactive theorem prover Matita, which has now been developed in Bologna for about 15 years and that constituted the first important testbench for the technology at the base of Logipedia. Matita was also central to the FET Open EU Project CerCo (Certified Complexity), coordinated by Dr. Sacerdoti Coen and focused on formal proofs applied to formal methods in the domain of real time systems and complexity preserving compilation.

An on-going collaboration with INRIA is focused on the development of ELPI, a very high level higher order constraint logic programming language that is an excellent domain specific language for writing interactive provers and programs that explicitly manipulate formulae and proofs. ELPI will be integrated with Dedukti in Logipedia to implement proof transformations in some work packages.

\paragraph{Specific expertise:}

\begin{compactitem}
\item Mathematical Knowledge Management
\item Implementation of Interactive Theorem Provers
\item Developers of Matita
\item Co-developers of the ELPI language
\end{compactitem}

\paragraph{Staff members undertaking the work:}~

\textbf{Dr.\ Claudio Sacerdoti Coen Leader} is associate professor of computer science since 2015. He published more than 15 journal papers and 50 conference papers on Mathematical Knowledge Management, Interactive Theorem Proving and the theory of lambda-calculus. The most recent project he coordinated was the EU FET Open Project CerCo (Certified Complexity). He was also work-package leader for the EU FET Open Project MoWGLI (Math on the Web, Get it by Logic and Interfaces).

\textbf{Prof.\ Andrea Asperti} is full professor of computer science at the University of Bologna. Before becoming an expert in interactive theorem proving he worked on category theory, lambda-calculus and linear logic. His current research interests also cover machine learning.

\textbf{One post-doc} to be hired using the project fundings.

\ednote{provide the key publications below}
\keypubs{providemore}

\end{sitedescription}
%%% Local Variables: 
%%% mode: latex
%%% TeX-master: "../propB"
%%% End: 

% LocalWords:  site-jacu.tex clange sitedescription emph compactitem pn semmath
% LocalWords:  prosuming-flexiformal KohSuc asemf06 GinJucAnc alsaacl09 StaKoh
% LocalWords:  tlcspx10 KohDavGin psewads11 ednote Radboud Bia ystok CALCULEMUS
% LocalWords:  textbf keypubs OntoLangMathSemWeb uwb Deyan Ginev Stamerjohanns
% LocalWords:  searchability
\newpage
\begin{sitedescription}{Cle}

\paragraph{Organization:}

CLEARSY is an SME specialised in the development of safety critical software and systems in the fields
of railways (main focus), microelectronics, information systems, defence, and automotive. CLEARSY
has developed or contributed to a number of CASE and engineering formal tools (including Atelier
B, B Automatic Refinement Tool, Brama model animator, Rodin platform, theorem provers, static
analysers, compilers) and also provides dedicated tools like supervision (SCADA), simulation, and
diagnosis software.

Engineering activities include:
\begin{itemize}
\item The realisation of worldwide projects committed to achieving results in the design and/or validation of systems and software.
\item A technical support activity in the fields of formal methods and operational safety..
\end{itemize}

CLEARSY engineers are skilled in various engineering domains (systems, mechanics, electronics, software,
operational safety) and apply IT tools and an electronic laboratory to create prototypes and
conduct trials. Collaborations with laboratories and industrial partnerships ensure the production of
the various systems components (sensors and interfaces).

\paragraph{Main tasks:}

\begin{compactitem}
\item WP1: task B-method.
\item WP4: task \emph{pp} theorem prover and connections to Zenon, ProB, SMT-Lib.
\item WP8: dissemination to industrial and certification actors.
\end{compactitem}

\paragraph{Relevant previous experience:}

CLEARSY has been involved in several collaborative research projects:
\begin{itemize}
\item EU R\&D projects: Reaims (1994-1995), FMERail (1998-2001), Matisse (2000-2003), Pussee (2001-2004), Rodin (2004-2007), and Deploy (2008-2012).
\item French R\&D projects: Forcoment (2001-2006), Equast (2002-2004), Verbatim (2003-2007), Rimel (2007-2010), Cercles-2 (2011-2014), DEPARTS (2012-2016)and BWare (2012-2015).
\end{itemize}

These projects are dedicated to the introduction of a formal method (B or Event-B) in the industry and through the development
of dedicated tools and methods are addressing software and electronic based system development.

CLEARSY developed and has been maintaining for the past 20 years proof tools addressing the logic of B and Event-B. These tools are 
packaged in the Atelier B and Rodin platforms. They are used routinely by large European players in the railways domain to assist the
development of safety-critical software such as automatic train control. CLEARSY also provides these actors technical assistance for the formal development of software and system, including proof-centric activities. 

\paragraph{Specific expertise:}

\begin{compactitem}
\item Formal methods for the development of software and systems.
\item Automatic theorem proving.
\item Development of proof rule libraries and their validation.
\end{compactitem}

\paragraph{Staff members undertaking the work:}

Dr.\ \textbf{David Déharbe} will be the site leader for CLEARSY. He obtained his PhD degree in Computer Science from Université Grenoble Alpes 
(France). He has held a software engineer position at CLEARSY since 2015, following an 18 year long academic career in UFRN (Brazil), 
where he was a key actor in the creation of the graduate studies in Computer Science, and a 2-year visiting research position at CMU 
(USA), where he developed a model checker for VHDL. He has published 40 conference papers and 18 journal papers, and has been involved 
in several national- and international-level research projects. He has been in the program committees of many scientific events. His 
research interests include formal methods and automatic proof techniques, and their application in industrial contexts. Gender: male.

Dr.\ \textbf{Guillaume Babin} is a formal methods engineer at CLEARSY. After obtaining a PhD in Computer Science from Université de 
Toulouse (France), Guillaume joined CLEARSY to apply formal methods to safety-critical software systems in the transportation industry. 
He is interested in tooling, automation and the application of formal methods in industrial systems.

Dr.\ \textbf{Lilian Burdy} is an expert in safety critical software. He has been participating to several safety critical software 
development since 1996, mainly in railway domain, but also in smart card domain. He has notably been working for Siemens, Gemplus, 
Alstom, Thales, RATP as employee or sub-contractor, being architect or developing safety critical parts of automatic train controllers, 
side-way equipments, etc. He has participated to several formal tools development, notably AtelierB for Clearsy or Jack for INRIA. He 
has published 12 conference papers and 3 journal papers, and has been involved in several national- and international-level research 
projects.

Dr.\ \textbf{Maximilien Colange} holds a PhD in Computer Science from Université Pierre et Marie Curie (France). He followed an 
academic career in Switzerland and France during 5 years, during which he published a dozen conference papers. His research interests
include formal methods, especially model-checking of both discrete and timed systems, and synthesis of reactive programs. He now holds 
a software engineer position at CLEARSY, with a focus on formal methods tools.

\textbf{Thierry Lecomte} is R\&D Project Director. He has been involved in several formal methods oriented, R\&D projects at European 
and French levels. His current subjects of interest include formal methods with proof, safety critical applications, safety computers. 
Gender: male.

\textbf{Etienne Prun} is Activity Manager for CLEARSY. He was project manager in several industrial projects in property-driven 
software analysis and property-Driven systems analysis. He has managed several European and French R\&D projects. He has been AtelierB 
development coordinator for 8 years. He was involved in teaching B methods in engineering school and for corporate training. His current 
research interests include safety system, safety software, with use of formal method with proof (automatic or not) in industrial 
context.

Dr.\ \textbf{Ronan Saillard} holds a PhD in Computer Science from Mines ParisTech (France) where he worked on both theoretical and praticable 
aspects of the implementation of Dedukti, a typechecker for the lambda-Pi calculus modulo. He has held a software engineer position at 
CLEARSY since 2015. His research interests include programming languages, formal methods and their application in industrial contexts.

Key publications:

\begin{itemize}
\item Thierry Lecomte, David Déharbe, Étienne Prun, Erwan Mottin:
Applying a Formal Method in Industry: A 25-Year Trajectory. SBMF 2017: 70-87.
\item Guillaume Babin, Yamine Aït Ameur, Marc Pantel:
Web Service Compensation at Runtime: Formal Modeling and Verification Using the Event-B Refinement and Proof Based Formal Method. IEEE Trans. Services Computing 10(1): 107-120 (2017).
\item Lilian Burdy, David Déharbe:
Teaching an Old Dog New Tricks - The Drudges of the Interactive Prover in Atelier B. ABZ 2018: 415-419.
\item Lilian Burdy, David Déharbe, Étienne Prun:
Interfacing Automatic Proof Agents in Atelier B: Introducing "iapa". F-IDE@FM 2016: 82-90.
\item Hakan Metin, Souheib Baarir, Maximilien Colange, Fabrice Kordon:
CDCLSym: Introducing Effective Symmetry Breaking in SAT Solving. TACAS (1) 2018: 99-114.
\item Typechecking in the lambda-Pi-Calculus Modulo : Theory and Practice. (Vérification de typage pour le lambda-Pi-Calcul Modulo : théorie et pratique). Mines ParisTech, France, 2015.
\end{itemize}

\end{sitedescription}
%%% Local Variables: 
%%% mode: latex
%%% TeX-master: "../propB"
%%% End: 

% LocalWords:  site-jacu.tex clange sitedescription emph compactitem pn semmath
% LocalWords:  prosuming-flexiformal KohSuc asemf06 GinJucAnc alsaacl09 StaKoh
% LocalWords:  tlcspx10 KohDavGin psewads11 ednote Radboud Bia ystok CALCULEMUS
% LocalWords:  textbf keypubs OntoLangMathSemWeb uwb Deyan Ginev Stamerjohanns
% LocalWords:  searchability
\newpage
\input{sites/Facebook}\newpage
\begin{sitedescription}{Del}

\paragraph{Organization:}
The Programming Languages Research Group at TU Delft is an
internationally leading research group in programming languages, and
active in areas such as language engineering, language design,
domain-specific languages, software verification, and program logics.
Specifically, Dr. Jesper Cockx is an expert on the Agda system.

\paragraph{Main tasks:}

\begin{compactitem}
\item Encoding of features that rely on type-directed conversion --
  such as eta-equality and definitional irrelevance -- in Dedukti
  (which does not have type-directed conversion).
\item Investigating possible designs for a core language for Agda, in
  order to facilitate the exporting of Agda developments to Logipedia.
\item Improving the current state-of-the-art on dependently typed
  languages with user-defined rewrite rules on areas such as
  confluence and termination checking.
\end{compactitem}


\paragraph{Relevant previous experience:}

Dr. Jesper Cockx is an expert on the theory and implementation of
Agda, a dependently typed programming language and proof assistant
that is widely used within the programming languages community and
beyond. He has worked on both foundational parts of Agda such as
elaboration of dependent (co)pattern matching and new extensions such
as rewrite rules and definitional proof irrelevance. He is also one of
the main contributors to the implementation of Agda.
  
\paragraph{Specific expertise:}

\begin{compactitem}
\item Elaboration of high-level programming techniques such as
  dependent pattern matching to a low-level representation that can be
  more easily transferred to other languages.
\item Extending dependently typed languages with user-defined rewrite
  rules that can be used to encode the theories of other languages.
\item Automatic translation of Agda developments to Dedukti
  (collaboration with Guillaume Genestier from Paris-Saclay).
\end{compactitem}

\paragraph{Staff members undertaking the work:}

\textbf{Dr.\ Jesper Cockx} (see above)
\keypubs{DBLP:conf/icfp/CockxDP14,DBLP:conf/icfp/CockxDP16,DBLP:journals/pacmpl/CockxA18}

Jesper Cockx


\end{sitedescription}
%%% Local Variables: 
%%% mode: latex
%%% TeX-master: "../propB"
%%% End: 

% LocalWords:  site-jacu.tex clange sitedescription emph compactitem pn semmath
% LocalWords:  prosuming-flexiformal KohSuc asemf06 GinJucAnc alsaacl09 StaKoh
% LocalWords:  tlcspx10 KohDavGin psewads11 ednote Radboud Bia ystok CALCULEMUS
% LocalWords:  textbf keypubs OntoLangMathSemWeb uwb Deyan Ginev Stamerjohanns
% LocalWords:  searchability
\newpage
\begin{sitedescription}{Fau}
  Friedrich Alexander Universit\"at Erlangen/N\"urnberg (FAU) is a public research
  university in the cities of Erlangen and Nuremberg, Germany. FAU is the second largest
  state university in the state Bavaria. It has 5 faculties, 23 departments/schools, 30
  clinical departments, 19 autonomous departments, 656 professors, and ca 40\,000
  students.

  The KWARC (KnoWledge Adaptation and Reasoning for Content~\cite{KWARC:on}) Group
  headed by {\emph{Prof.\ Dr.\ Michael Kohlhase}} specialises in knowledge management for
  STEM.  Formal logic, natural language semantics, and semantic web technology provide the
  foundations for the research of the group. Its group working on \pn will include: Michael
  Kohlhase, Florian Rabe, Tom Wiesing, and Jonas Betzendahl.

% \subsubsection*{Curriculum vitae}
% % Curriculum of the personnel at this institution
% \input{CVs/Michael.Kohlhase}
% \input{CVs/Christian.Maeder}
% \input{CVs/Mihnea.Iancu}

\subsubsection*{Relevant previous experience:}

The KWARC group is the lead implementor of the OMDoc (Open Mathematical Document) format
for representing mathematical knowledge \cite{Kohlhase:OMDoc1.2} and redeveloped its
formal core in the OMDoc/MMT format~\cite{RabKoh:WSMSML13}. The latter has been
implemented in the MMT system~\cite{MMTSVN:on,RabKoh:WSMSML13} which provides efficient
implementations of the computational primitives such as type checking, flattening, and
presentation at a logic/foundation-independent level.  The group has developed services
powered by such semantically rich representations, different paths to obtaining them, as
well as platforms that integrate both aspects.  \emph{Services} include the adaptive
context-sensitive presentation framework provided by the MMT API and the semantic search
engine MathWebSearch\cite{KohSuc:asemf06,ProKoh:mwssofse12}. 

Semantic services can be integrated into the documents generated from OMDoc/MMT
representations, making them into ``active documents'', i.e. documents that are
interactive and adaptive to the user and situation.  For \emph{obtaining} rich content,
the group investigates assisted manual editing \cite{JucKoh:sidesc10:biblatex} as well as
automatic annotation using linguistic techniques \cite{GinJucAnc:alsaacl09}.  Finally,
KWARC has developed the \textsf{MathHub.info} portal a community-based library and
knowledge management system for flexiformal libraries, which can be used for semantic
publishing and eLearning~ \cite{KohDavGin:psewads11,MathHub:on,IanJucKoh:sdm14}.

The \textsf{OMDoc/MMT} knowledge representation format and the \textsf{MathHub.info}
system will an important basis for the developments Work Packages 4 and 6.

Michael Kohlhase has initiated and led the CALCULEMUS! IHP-Research and Training Network
and participated in the FP6 IST MoWGLI (Mathematics on the Web: Get it by Logic and
Interfaces) project, the FP6 CSA Once-CS (Open Network of Centres of Excellence in Complex
Systems), The FP7 EDC project WebALT (Web Advanced Learning Technologies), and the H2020
Infrastructure project OpenDreanKit.

\subsubsection*{Specific expertise:}
\begin{compactitem}
\item Modelling formal structures of mathematical knowledge in a web-scalable way.
\item Transforming large collections of legacy scientific publications to semantically
  structured markup.
\item Designing user interfaces for authoring and interacting with mathematical knowledge.
\end{compactitem}

\site{Fau} leads \WPtref{structuring} and tasks related to formal alignments and
alignment-based translation in \WPtref{alignment}.

\ednote{@Rabe: add 15 line CV}
\end{sitedescription}

%%% Local Variables: 
%%% mode: LaTeX
%%% TeX-master: "../propB"
%%% End: 
\newpage
\begin{sitedescription}{Got}

% \ednote{Give a one-paragraph run-down of the site and the team there. }
The Logic and Types group at Gothenburg University has been a leading group
in the research on dependent type theory and interactive theorem proving
since the 1980's. The current version of the system was designed and
implemented by Dr. Ulf Norell as part of his PhD thesis and has been
actively developed by the group since then. Agda is widely used in both
research and teaching.

\paragraph*{Main tasks:}

\begin{compactitem}
% \item\ednote{specify the main tasks and reference the respective work packages}
\item Instrumenting the Agda system to produce Dedukti proofs
  (\WPtref{instrumentation}).
\item Investigating possible designs for a core language for Agda, in
  order to facilitate the exporting of Agda developments to Logipedia.
\end{compactitem}

\paragraph*{Publications, products or services:}
\begin{compactitem}
  \item Jesper Cockx, Andreas Abel. ``Elaborating dependent
  (co)pattern matching.'' Proceedings of the ACM on Programming
  Languages, 2(ICFP), 2018.
  \item Andreas Abel, Joakim \"Ohman, and Andrea Vezzosi. ``Decidability of
  Conversion for type theory in type theory.'' Proceedings of the ACM on Programming
  Languages, 2(POPL), 2018.
  \item Ulf Norell. ``Towards a practical programming language based on
  depedent types.'' PhD thesis, Chalmers University of Technology, 2007.
\end{compactitem}

\paragraph*{Previous projects or activities:}

\paragraph*{Infrastructures or technical equipments:}

\paragraph*{Persons primarily responsible for carrying out the proposed activities:}

\begin{compactitem}
\item{\bf Ulf Norell} is the main developer and maintainer of the Agda
proof assistant. He got his PhD from Chalmers University of Technology in
2007 on the design and implementation of dependently typed programming
languages.
\item{\bf Andreas Abel} is an expert on the meta theory of dependent type
theory and one of the core developers of Agda. He got his PhD from
University of Munich in 2006.
of type
\end{compactitem}

\end{sitedescription}
%%% Local Variables:
%%% mode: latex
%%% TeX-master: "../propB"
%%% End:

% LocalWords:  site-jacu.tex clange sitedescription emph compactitem pn semmath
% LocalWords:  prosuming-flexiformal KohSuc asemf06 GinJucAnc alsaacl09 StaKoh
% LocalWords:  tlcspx10 KohDavGin psewads11 ednote Radboud Bia ystok CALCULEMUS
% LocalWords:  textbf keypubs OntoLangMathSemWeb uwb Deyan Ginev Stamerjohanns
% LocalWords:  searchability
\newpage
\begin{sitedescription}{Inn}

\paragraph{Organization:}
The University of Innsbruck is a global Top-200 university and the second
largest university in Austria. UIBK has been involved in a number of FWF
projects related to formal proof, and a large number of national and
international projects including dozens of projects as part of FP5,
FP6, FP7, and H2020.

The research of the Computational Logic group is concerned with the logical
foundations of computer science and their application to the analysis and
verification of complex systems. The group has developed the IsaFoR library,
the largest formalisation of rewriting with more than 5000 theorems. Various
hammer systems developed in the group are today strongest automation techniques
for various formalizations including the Flyspeck project.

\paragraph{Main tasks:}

\begin{compactitem}
\item Collaboration on task ..., the specification of Mizar in Dedukti
\item Dedukti-internal proof automation
\item Alignments
\end{compactitem}

\paragraph{Relevant previous experience:}

\ednote{give an overview over previous work and projects that add to the \pn project}

\paragraph{Specific expertise:}

\begin{compactitem}
\item \ednote{give three to five specific areas of expertise that pertain to the \pn project}
\end{compactitem}

\paragraph{Staff members undertaking the work:}

\textbf{Assoc. Prof. Dr.\ Cezary Kaliszyk} has been working on making proof assistants
more accessible by developing proof automation, proof advice, and other packages for formal
proofs. He has worked on the Isabelle/Mizar object logic, where features of Mizar were
expressed in a logical framework. Kaliszyk has also worked on machine learning for interactive
proofs and has co-organized the AITP conference on the topic in the last few years (AITP). He
has developed multiple hammer systems for higher-order logic and intuitionistic type theory.
Kaliszyk has also worked on alignments between formal systems and between informal and formal
mathematics.

% Postdoc and PhD students that unfortunately will end likely in 2021: Joshua Chen, Miroslav Olšák, Stanisław Purgał

\end{sitedescription}
%%% Local Variables: 
%%% mode: latex
%%% TeX-master: "../propB"
%%% End: 

% LocalWords:  site-jacu.tex clange sitedescription emph compactitem pn semmath
% LocalWords:  prosuming-flexiformal KohSuc asemf06 GinJucAnc alsaacl09 StaKoh
% LocalWords:  tlcspx10 KohDavGin psewads11 ednote Radboud Bia ystok CALCULEMUS
% LocalWords:  textbf keypubs OntoLangMathSemWeb uwb Deyan Ginev Stamerjohanns
% LocalWords:  searchability
\newpage
\begin{sitedescription}{Lmu}

\newcommand\inquotes[1]{``#1''}

\logo{LMU}

Ludwig-Maximilians-Universit\"at (LMU) is a public research university
located in Munich, Germany.  LMU consists of 18 faculties which
accommodate various departments and institutes including the
Mathematisches Institut, where the research group of mathematical
logic is headed by Prof.\ Dr.\ Helmut Schwichtenberg.  The research
areas of the group are such as proof theory, realizability
interpretation, programme extraction, constructive analysis,
constructive algebra, and proof assistant.  In particular in the
research area of proof assistant, the logic group has been actively
developing the Minlog system since early 1990's.

\paragraph*{Main tasks:}

\begin{compactitem}
  \item Encoding the underlying theory of Minlog in Dedukti. \WPtref{theories} \taskref{theories}{minlog}
  \item Implementing the encoder, so that Minlog's libraries and formal proofs of constructive analysis is available in Dedukti with proof checking.
The Logipedia integration level of Minlog is increased to 3 from 0. \WPtref{theories} \taskref{theories}{minlog}
  \item Contributing to \WPtref{alignment}.  The main targets are concepts in constructive analysis and (co)induction/(co)recursion.

\end{compactitem}

\paragraph*{Publications, products or services:}

\begin{compactitem}
\item ``Refined program extraction from classical proofs'',
  by U.~Berger, W.~Buchholz, and H.~Schwichtenberg.
  Annals of Pure and Applied Logic, 114:3--25, 2002.

\item ``Dialectica interpretation of well-founded induction'',
  by H.~Schwichtenberg.
Math. Logic. Quarterly, 54(3):229--239, 2008.

\item ``Realizability interpretation of proofs in constructive analysis'',
  by H.~Schwichtenberg.
 Theory of Computing Systems, 43(3):583--602, 2008.

\item ``Basic Proof Theory'',
  by A.~S. Troelstra and H.~Schwichtenberg.
 Cambridge University Press, second edition, 2000.

\item ``Proofs and Computations'',
  by H.~Schwichtenberg and S.~S. Wainer.
Perspectives in Logic. Association for Symbolic Logic and Cambridge
  University Press, 2012.
\end{compactitem}

\paragraph*{Previous projects or activities:}

\begin{compactitem}
  \item 1997-2006, Speaker of the DFG-Graduiertenkolleg 301
    \inquotes{Logik in der Informatik}

\item 2004-2008, LMU Coordinator of the  EST (Early Stage Traning) Programme
  \inquotes{MathLogAps} (MEST-CT-2004-504029) of the EU, together with the
universities of Leeds, Manchester, Lyon and ENS Lyon

\item 2009-2013, LMU Coordinator of the ITN (Network for Initial
    Training) Programme PITN-GA-2009-238381 \inquotes{MALOA} of the
    EU, together with the universities of Leeds, Manchester, Oxford,
    CNRS, Paris 7, M\"unster, Prague

\item 2017-2021, LMU Coordinator of the 731143-CID project of LMU

\item 05/2018-08/2018, Co-organizer (with D.~Bridges, M.~Rathjen and
  P.~Schuster) of a Trimester on \inquotes{Types, Sets, Constructions}
  at the Hausdorff Institute for Mathematics, Bonn
\end{compactitem}

%% \paragraph*{Specific expertise:}
%% \begin{compactitem}
%% \item Implementation of the proof assistant Minlog.
%% \item Foundation for constructive mathematics accommodating partial functionals and realizability.
%% \item Constructive mathematics and programme extraction.
%% %% \item \ednote{give three to five specific areas of expertise that pertain to the \pn project}
%% \end{compactitem}

\paragraph*{Infrastructures or technical equipments:}

\begin{compactitem}
\item The logic group at LMU has developed the Minlog proof assistant since 1990.
\item The Minlog library for constructive analysis has been developed since 2004
and corecursion and coinduction have been involved since 2010.
\item The Minlog feature for classical extraction has been developed since 2002.
\end{compactitem}

\paragraph*{Persons primarily responsible for carrying out the proposed activities:}

\begin{compactitem}

\item \textbf{Josef Berger} is a Privatdozent at LMU.  He earned his
  Doctoral degree in 2002 from LMU, in Nonstandard stochastics,
  supervised by Horst Osswald, and his Habilitation in 2014 at LMU
  with a thesis on "Perspectives in Constructive Reverse Mathematics".

\item \textbf{Nils K\"opp} is a teaching assistant and a PhD student
  at LMU.  Master thesis 2017 on "Automatically verified programme
  extraction from proofs with applications to constructive analysis".

\item \textbf{Franz Merkl} is a professor and the chair of stochastics
  at LMU.  He has supervised some Diploma theses on subjects in
  probability theory, which were formalized in Mizar.  He himself has
  also worked with Mizar and published in the "Journal of Automated
  Reasoning", where only papers checked by Mizar are accepted.

\item \textbf{Kenji Miyamoto} is a teaching assistant and a postdoc
  researcher at LMU.  Doctorate 2013 at LMU with a thesis "Programme
  extraction from coinductive proofs and its application to exact real
  arithmetic".  Worked as Postdoc and teaching assistant at LMU and in
  Innsbruck (with Georg Moser).

\item \textbf{Iosif Petrakis} is a lecturer and a postdoc researcher
  at LMU.  Doctorate 2015 at LMU with a thesis "Constructive Topology
  of Bishop Spaces".  Presently preparing his Habilitation in
  Mathematics.

\item \textbf{Helmut Schwichtenberg} is a professor (emeritus) of
  Mathematics at LMU.  Book (with Stanley Wainer) on Proofs and
  Computations, Cambridge University Press, 2012.  Book (with Anne
  Troelstra) "Basic Proof Theory", Cambridge University Press, 2nd
  ed. 2000.  Coorganizer (with Douglas Bridges, Michael Rathjen and
  Peter Schuster) of the Hausdorff Trimester on Sets, Types and
  Constructions at the Hausdorff Institute, Universit\"at Bonn,
  May-August 2018.  Coorganizer (with Klaus Mainzer and Peter
  Schuster) of the annual Autumn School on Proofs and Computations.

\item \textbf{Franziskus Wiesnet} is a PhD student co-supervised by
  Peter Schuster (Verona) and Helmut Schwichtenberg (LMU).  Master
  thesis "Konstruktive Analysis mit exakten reellen Zahlen" 2017 at
  LMU.  He is supported by a Marie Sk{\l}odowska-Curie fellowship of
  the Istituto Nazionale di Alta Matematica

\item \textbf{Chuangjie Xu} is a postdoc researcher at LMU, holding a
  Humboldt grant.  PhD 2015 in Birmingham under the supervision of
  Martin Escardo.  Half of the theses consisted of an Agda
  implementation of the theoretical results achieved.

\end{compactitem}
%% \textbf{Dr.\ Great Leader}\ednote{describe the site leader and his expertise}
%% \textbf{Joe Implementor}\ednote{and more of them. }

%%\keypubs{providemore}

%%Helmut Schwichtenberg, Kenji Miyamoto

\end{sitedescription}
%%% Local Variables: 
%%% mode: latex
%%% TeX-master: "../propB"
%%% End: 

% LocalWords:  site-jacu.tex clange sitedescription emph compactitem pn semmath
% LocalWords:  prosuming-flexiformal KohSuc asemf06 GinJucAnc alsaacl09 StaKoh
% LocalWords:  tlcspx10 KohDavGin psewads11 ednote Radboud Bia ystok CALCULEMUS
% LocalWords:  textbf keypubs OntoLangMathSemWeb uwb Deyan Ginev Stamerjohanns
% LocalWords:  searchability
\newpage
\begin{sitedescription}{Lee}

\logo{Leeds}

The University of Leeds (UNIVLEEDS) is acclaimed world-wide for the quality of its teaching and research, and is ranked 93rd in the QS World University Rankings 2019. Leeds is in the top 10 universities in the UK (Times/Sunday Times, 2018). The results of the most recent Research Excellence Framework exercise (REF) identified that 82.76\% 
of its research activity has a top quality rating of either `world leading' or `internationally excellent' which makes it a constant member of the UK's prestigious Russell Group of research intensive universities. 

In 2017/18 it had an annual income of \pounds 715m and its annual research income exceeded \pounds 175m, of which 15.2\% was derived from EU awards. The University includes the School of Mathematics,
which hosts the Logic group, one of the strongest internationally, with expertise across the whole
spectrum of logic and good links with the School of Computing.




\paragraph*{Main tasks:}

\begin{compactitem}
\item Nicola Gambino leads  the \WPref{alignment} and participates to \taskref{theories}{hott}. He is an expert on type theory,
including Homotopy Type Theory, and categorical logic, with experience in computer-assisted proof-checking. 
\item Michael Rathjen leads \taskref{alignment}{alignlogic}. He is a leading figure at international level
on proof theory. 
\item Paul Shafer participates to  \taskref{alignment}{alignlogic}. He is an expert in reverse mathematics
and computability theory.
\end{compactitem}
The combination of expertise available at Leeds makes the team uniquely placed to develop  \taskref{alignment}{alignlogic},
as the task will require relating type theories, investigating their proof-theoretic properties and analyse the strength of some
statements via reverse mathematics. Dr Gambino's experience in HoTT makes him ideally suited to help in~\taskref{theories}{hott}.


\paragraph*{Publications, products or services:} 

\begin{compactitem}
\item ``Homotopy-initial algebras in type theory'', by S. Awodey, N. Gambino and K. Sojakova, 
{\em Journal of the Association for Computing Machinery}, 63 (6), 2017, 45pp.
\item ``The identity type weak factorisation system'' by N. Gambino and R. Garner, 
{\em Theoretical Computer Science} 409 (1), 2008, pp. 94-109.
\item ``Relativized ordinal analysis: The case of Power Kripke-Platek set theory'', by M. Rathjen, 
{\em Ann. Pure Appl. Logic}, 165(1), 2014, pp.~316-339 
\item ``Constructive Zermelo-Fraenkel Set Theory, Power Set, and the Calculus of Constructions'',
by M. Rathjen, {\em Epistemology versus Ontology}, 2012, pp.~313-349. 
\item ``The reverse mathematics of the Tietze extension theorem'' by P. Shafer,  
{\em Proceedings of the American Mathematical Society}, 144, 2016, pp.~5359-5370.
\end{compactitem}

\paragraph*{Previous projects or activities:}
 
\begin{compactitem}
\item From Mathematical Logic To Applications (MALOA), EU ITN Network (FP7-PEOPLE), October 2009 -- September 2013, Value: EUR 4.3M.
\item EPSRC Standard Grant, ``Homotopy Type Theory: Programming and Verification'', joint project with the University of Nottingham and the University of Strathclyde, March 2015 -- September 2019, Value: GBP 1.2M
\item EPSRC Standard Grant, ``Homotopical inductive types'', May 2013 -- June 2016, Value: GBP 283K.
\end{compactitem}

%\paragraph*{Specific expertise:}
%
%\begin{compactitem}
%\item Type theory,
%\item Proof theory,
%\item Reverse mathematics.
%\end{compactitem}

\paragraph*{Persons primarily responsible for carrying out the proposed activities:}

\begin{compactitem}
\item \textbf{Nicola Gambino} is Associate Professor in Pure Mathematics at the University of Leeds. His publication record includes papers leading journals in both mathematics (e.g. Memoirs of the AMS,  Journal of the LMS) and computer science (e.g. Journal of the ACM, Theoretical Computer Science). He was a plenary invited speaker at the International Conference in Category Theory in 2016 and Logic Colloquium in 2000. His research has been consistently funded by EPSRC and the US Air Force for Scientific Research. He successfully supervised 4 PhD students and 1 PDRA. He serves on the editorial boards of Mathematical Structure in Computer Science and Applied Categorical Structures.

Nicola Gambino's research focuses on type theory, categorical logic and category theory. He is one of the leading experts in Homotopy Type Theory, a subject to which he made fundamental contributions.

\item \textbf{Michael Rathjen} is Professor of Pure Mathematics at the University of Leeds. His publication record includes about 100 papers. He has been an invited speaker at the International Congress of Mathematicians in 2006 and Logic Colloquium (6 times, most recently in 2019), as well as many
other conferences in mathematical logic. His research has been consistently funded by the German Science Foundation, NSF, EPSRC,
Leverhulme Trust and the Templeton Foundation. He successfully supervised 11 PhD students and 5 PDRAs. He serves on the editorial boards of  Notre Dame Journal of Formal Logic, Oxford University Press Logic Guides, and Documenta Mathematica.
\item \textbf{Paul Shafer} is Lecturer in Mathematical Logic at the University of Leeds.  His publication record includes papers some of the top journals in mathematics (e.g., Transactions of the AMS, Proceedings of the AMS, Transactions of the LMS) and mathematical logic (e.g., Journal of Symbolic Logic, Annals of Pure and Applied Logic).  He is frequently invited to speak at major meetings in mathematical logic (e.g., Logic Colloquium, ASL North American Annual Meeting).  He has received prestigious fellowships from the Fondation Sciences Mathématiques de Paris (France) and the Fonds Wetenschappelijk Onderzoek (Belgium) as well as travel and exchange grants from EPSRC and the Royal Society.  He has successfully supervised 1 PhD student.
\end{compactitem}

% \keypubs{providemore}{1}



\end{sitedescription}
%%% Local Variables: 
%%% mode: latex
%%% TeX-master: "../propB"
%%% End: 

% LocalWords:  site-jacu.tex clange sitedescription emph compactitem pn semmath
% LocalWords:  prosuming-flexiformal KohSuc asemf06 GinJucAnc alsaacl09 StaKoh
% LocalWords:  tlcspx10 KohDavGin psewads11 ednote Radboud Bia ystok CALCULEMUS
% LocalWords:  textbf keypubs OntoLangMathSemWeb uwb Deyan Ginev Stamerjohanns
% LocalWords:  searchability
\newpage
\begin{sitedescription}{Lie}

\ednote{a description of the legal entity}

The University of Liège (ULiege, http://www.uliege.be) is located in the Fédération Wallonie-Bruxelles of Belgium in the Euregio region. ULiege is the only public and complete university institution of the French-speaking region of Belgium. The ULiege counts 2977 lecturers-researchers and 24688 students (incl. 2095 PhD students). 23\% of the students at ULiege are foreign students from 127 different countries. A wide variety of fundamental and applied research projects have emerged from about 43 Faculty and 11 interfaculty Research Units. On the international level, the University of Liege is actively involved in research projects with more than seventy countries worldwide. ULiege has been involved in 191 European FP7 and H2020 projects and is active in 8 H2020 INFRA projects. At the end of 2018, 2093 research agreements were in progress, of which 1458 involved an international partner. In parallel, ULiege has developed an active policy in terms of technology transfer, resulting in the creation of more than 144 spin-off companies and in the ownership of 834 patents.

The Montefiore Institute is the electricity, electronics and computer science
department of the Faculty of Applied Sciences of the University of Liège.  It
was founded in 1883.  Research in the Software Reliability and Security group of
the Montefiore Institute focuses on symbolic techniques for verification of
systems.  One objective is to study the theoretical properties of symbolic data
structures based on finite-state automata and logical formulas.  Another line of
research, connected to the first, relates to automated reasoning, and more
specifically, the satisfiability checking problem for large logical formulas, in
particular those expressed in a combination of theories.  Its main goal consists
in engineering tools known as Satisfiability Modulo Theories (SMT) solvers,
whose application field spans several areas of computer science, including
verification.  Automated reasoning is strongly linked to this project.

\paragraph{Main tasks:}

\begin{compactitem}
\item Connecting SMT solvers to Logipedia: \WPref{atpetc}, task \taskref{***}{***}
\item Connecting FOL solvers to Logipedia: \WPref{atpetc}, task \taskref{***}{***}
\end{compactitem}

\paragraph{Publications, products or services:}

\ednote{a list of up to 5 relevant publications, and/or products, services (including widely-used datasets or software), or other achievements relevant to the  call content}

\paragraph{Previous projects or activities:}

% SC-SQUARE
Members of the group have expertise in the field of automated theorem proving, notably SMT solving, and automata based symbolic techniques for arithmetic.  They have been part of several national and international projects, including
\begin{itemize}
\item ANR SMArT
\item SC-SQUARE
\item ERC Matryoshka
\end{itemize}

\paragraph{Infrastructures or technical equipments:}

% veriT
% LASH
% SMT-LIB

\begin{itemize}
\item David Déharbe, Pascal Fontaine, Haniel Barbosa.  The SMT solver veriT.
\item Clark Barrett, Pascal Fontaine, Cesare Tinelli. The SMT-LIB language reference and library.
%\item Bernard Boigelot.  LASH
\end{itemize}

\paragraph{Persons primarily responsible for carrying out the proposed activities:}

\begin{itemize}
\item{\bf Bernard Boigelot} is professor at the University of Liège
  since 1999.  His research interests mainly focus on computer-aided
  verification, particularly reachability analysis of infinite-state
  systems, and symbolic data structures and automata-based procedures
  for mixed integer and real arithmetic reasoning.  He has designed
  the LASH toolset for representing infinite sets and exploring
  infinite state spaces. He has been PC member of international
  conferences such as TACAS, ATVA, IJCAR and RP, and workshops such as
  SPIN and INFINITY. He is a regular co-organizer of the annual VTSA
  Summer School on Verification Technology, Systems \& Applications.

\item{\bf Pascal Fontaine} (co-leader of work package \WPref{atpetc}) is a
  professor at the University of Liège since 2019.  He obtained his PhD in 2004
  in Liège and was maître de conférence at the University of Loraine in the
  Inria team VeriDis between 2004 and 2019.  He obtained is habilitation (2019)
  at the University of Lorraine.  His research interests focus on automated
  reasoning, and particularly on satisfiability modulo theories.  He was PC
  member of international conferences such as CADE, FroCoS, IJCAI, IJCAR, SAT
  and Tableaux.  He has been PC chair of the international conferences CADE and
  FroCoS, and the workshops PAAR, SC-square and SMT.  Fontaine was co-founder of
  the PxTP (Proof eXchange for Theorem Proving) series of workshops.  He is a
  member of the steering committees of CADE and SMT.  He was co-organizer of the
  international Summer School on SAT and SMT, in Vienna 2014.  He is one of the
  main developers of the veriT SMT solver, which, among its strong features,
  provides detailed unsatisfiability proofs.
\end{itemize}

\end{sitedescription}
%%% Local Variables: 
%%% mode: latex
%%% TeX-master: "../propB"
%%% End: 

% LocalWords:  site-jacu.tex clange sitedescription emph compactitem pn semmath
% LocalWords:  prosuming-flexiformal KohSuc asemf06 GinJucAnc alsaacl09 StaKoh
% LocalWords:  tlcspx10 KohDavGin psewads11 ednote Radboud Bia ystok CALCULEMUS
% LocalWords:  textbf keypubs OntoLangMathSemWeb uwb Deyan Ginev Stamerjohanns
% LocalWords:  searchability
\newpage
\input{sites/NPIT}\newpage
\begin{sitedescription}{Oca}

\logo{OCamlPro}

The software company OCamlPro was created in 2011. They harness their OCaml expertise and formal methods know-how to design, prototype, and build high quality software in demanding projects. Their team of PhD-level engineers also contributes open source development tools for the programming language OCaml, helping to improve the efficiency and usability of the OCaml compiler and tools (the free and open-source OCaml package manager OPAM, the optimizing compiler flambda, the SMT Solver Alt-Ergo, etc.).

\paragraph*{Main tasks:}

\begin{compactitem}
\item \taskref{atpetc}{instrumenting}: Implement a proof trace output for
  the SMT solver Alt-Ergo. As the maintainers and maind eveloppers of Alt-Ergo,
  OCamlPro is uniquely competent in modifying the source code of Alt-Ergo
  for such a purpose. Additionally, Guillaume Bury already has experience
  generating formal proofs from an SMT solver as described in his PhD
  thesis\cite{BURY19}.
\item \taskref{atpetc}{deduktitoatp}: Task leader on the translation of
  dedukti statements into input format for automatic tools. As the developper
  of the dolmen\cite{dolmen} library for parsing input format for automatic
  tools, Guillaume Bury already has experience manipulating such formats,
  and is thus suited to leading this task aimed at translating dedukti
  statements into such formats.
\item \taskref{access}{opam}: provide access to proofs using the opam
  package manager. As the author, developper and maintainer of opam,
  OCamlPro is again uniquely suited to leading this task given its
  unparraleled expertise on the opam package manager.
\end{compactitem}

\paragraph*{Publications, products or services:}

\begin{compactitem}
\item Guillaume Bury's PhD thesis\cite{BURY19} presented a new automated theorem
  prover, named Archsat, capable of generating formal dedukti proofs. To date,
  Archsat and the tableaux-based theorem prover Zenon are the only two automated
  theorem provers able to produce dedukti proofs.
\item Dolmen\cite{dolmen} is an OCaml library developped by Guillaume Bury,
  that deals with parsing and type-checking most input languages used in
  the automated theorem prover community.
\item Opam\cite{OPAM} is a source-based package manager developped by OCamlPro,
  which has been successfully used by the OCaml community since 2012, where
  it manages 2585 versioned packages for a total of 13196 combinations of package
  and version, guaranteeing its ability to connect people across large communities.
  Furthermore, opam is meant to provide management capabilities not only to
  OCaml, but to any language, which is why it is already used as a proof manager
  by the Coq community where it has been proven to be reliable and suited to
  managing formal proofs.
\end{compactitem}

\paragraph*{Previous projects or activities:}

French R\&D projects:
\begin{compactitem}
\item FUI LCHIP (2017-2020)
\item ANR Vocal (2015-2020)
\item ANR BWare (2013-2016)
\item FUI HILITE (2010-2013)
\end{compactitem}

\paragraph*{Infrastructures or technical equipments:}

\begin{compactitem}
\item The Alt-Ergo solver\cite{ae2.2} is an SMT solver developped and
  maintained by OCamlPro. It is used behind software verification tools
  such as Frama-C, SPARK, Why3, Atelier-B and Caveat.
\end{compactitem}

\paragraph*{Persons primarily responsible for carrying out the proposed activities:}

\begin{compactitem} % in alphabetical order
  \item{\bf Raja Boujbel} Raja holds a PhD in software deployment and
    multi-agent systems from University of Toulouse. Previously, she had studied
    functional programming and compiler design at Université Pierre et Marie
    Curie, then worked on the Opa language among MLstate’s distribution team.
    She joined OCamlPro in March 2018 as a lead maintainer for opam, an
    open-source package manager for OCaml.
\item{\bf Guillaume Bury} Guillaume holds a research Master in computer
    science from Ecole Normale Supérieure in Paris, France, and has studied the
    integration of rewriting techniques inside SMT solvers during his PhD
    obtained under the direction of Gilles Dowek and David Delahaye in
    Deducteam at ENS Cachan. He joined OCamlPro in October 2018 and works in
    the Flambda team, on optimizations passes for the OCaml compiler.
\item{\bf Albin Coquereau} Albin has a PhD in computer science,
    which he obtained for his work on improving the performance of the SMT
    solver Alt-Ergo. He also helped adding a support for the SMT-LIB standard
    in Alt-Ergo allowing it to participate to the SMTCOMP 2018.
\item{\bf Mattias Roux} Mattias Roux holds a PhD in computer science
    for his work on the model checker Cubicle, with an extension of the backward
    reachability algorithm. He now works at OCamlPro on the Alt-ergo theorem
    prover.
\end{compactitem}

\end{sitedescription}

%%% Local Variables: 
%%% mode: latex
%%% TeX-master: "../propB"
%%% End: 
\newpage
\begin{sitedescription}{Inr}

\ednote{a description of the legal entity}

Established in 1967, Inria is the only French public research body fully dedicated to computational sciences. It is a national operator in research in digital sciences and is a primary contact point for the French Government on digital matters. Under its founding decree as a public science and technology institution, jointly supervised by the French ministries for research and industry, Inria's missions are to produce outstanding research in the computing and mathematical fields of digital sciences and to ensure the impact of this research on the economy and society in particular. Inria covers the entire spectrum of research at the heart of these activity fields and works on digitally-related issues raised by other sciences and by actors in the economy and society at large. Beyond its structures, Inria's identity and strength are forged by its ability to develop a culture of scientific innovation, to stimulate creativity in digital research. 
Throughout its 8 research centres and its 220 project teams, Inria has a workforce of 2 400 employees (including 1600 researchers) with an annual budget of 231 million euros, 25\% of which coming from its own resources.
Inria’s mission is to pursue excellent research in computer science and applied mathematics in order to play a major role in resolving scientific, societal and industrial challenges. Therefore, Inria actively collaborates with public and private bodies including strategic partnerships with large firms, SME’s technology platforms and industrial clusters. Technology transfer is further enhanced by helping to launch new companies (since 1984, about 160 companies have stemmed from Inria) and by forming partnerships with innovative SMEs.

The institute is strongly involved in European programmes aimed at fostering scientific excellence, such as the European Research Council (58 Grants) or the Marie S. Curie Actions (24 projects in Horizon 2020).

Inria makes a firm commitment to Horizon 2020, with which the institute’s strategic plan is aligned. The objective is to combine scientific excellence with a more focused consideration of major European and global societal challenges to which Inria can bring a key contribution. Inria is currently involved in more than 140 H2020 funded projects.

Inria is also playing a lead role in the development of the Knowledge and Innovation Community (KIC) EIT Digital as host of the French node. EIT Digital’s ambition is to create for Europe a structure dedicated to technology transfer and innovation in the digital field. Besides EIT Digital, Inria is also a core partner of the KIC EIT Health.

\paragraph{Main tasks:}

\ednote{its main tasks, with an explanation of how its profile matches the tasks in the proposal}

\begin{compactitem}
\item\ednote{specify the main tasks and reference the respective work packages} 
\end{compactitem}

\paragraph{Publications, products or services:}

\ednote{a list of up to 5 relevant publications, and/or products, services (including widely-used datasets or software), or other achievements relevant to the  call content}

\paragraph{Previous projects or activities:}

\ednote{a list of up to 5 relevant previous projects or activities, connected to the subject of this proposal}

\paragraph{Infrastructures or technical equipments:}

\ednote{a description of any significant infrastructure and/or any major items of technical equipment, relevant to the proposed work}

\paragraph{Persons primarily responsible for carrying out the proposed activities:}

\begin{itemize}
%%%%%%%%%%%%%%%%%%%%%%%%%%%%%%%%%%%%%%%%%%%%%%%%%%%%%%%%%%%%%%%%%%%%%%%%%%%%%%
\item{\bf Stephan Merz} (co-leader of work package \WPref{theories}) is a senior research
scientist and head of the \href{https://team.inria.fr/veridis/}{VeriDis}
research group, as well as the deputy for science, at Inria Nancy\,--\,Grand Est.
His research interests are centered on the formal specification and verification
of distributed algorithms, in particular proofs of safety and liveness
properties, as well as refinement relations between specifications expressed at
different levels of abstraction. He is a main contributor to TLAPS, the \tlaplus
Proof System. He obtained his PhD (1992) and habilitation (2002) degrees at the
University of Munich and joined Inria in 2002. He published more than 100 papers
in peer-reviewed international conferences and journals and has been a PC chair
of conferences such as IFM, ICFEM, and ITP. He co-founded the FRiDA (Formal
Reasoning in Distributed Algorithms) series of workshops as well as the VTSA
(Verification Technology, Systems, and Applications) summer school. He has been
a member of the scientific directorate of Schloss Dagstuhl, as well as a member
of the Inria Evaluation Committee and of the National Committee of Scientific
Research in France.

%%%%%%%%%%%%%%%%%%%%%%%%%%%%%%%%%%%%%%%%%%%%%%%%%%%%%%%%%%%%%%%%%%%%%%%%%%%%%%
\item {\bf Dale Miller} (co-leader of work package \WPref{reversemath}) is a senior research scientist
and former head of the Parsifal at Inria Saclay\,--\,\^Ile-de-France.  Miller
received his Ph.D. in Mathematics from Carnegie Mellon University in 1983.  He
has been on the faculty of the University of Pennsylvania (1983-1997), Penn
State University (1997-2002), and the Ecole Polytechnique (2002-2006). Miller
has been a two-term Editor-in-Chief of the ACM Transactions on Computational
Logic (ToCL) and is currently serving on the editorial board of the Journal of
Automated Reasoning. He is presently the General Chair of the ACM/IEEE Symposium
on Logic in Computer Science (LICS) and has served as a program committee chair
on a number of conferences in the general area of computational logic.  Miller
has twice received the LICS Test-of-Time Award for papers published in 1991 and
in 1994.  Miller received an ERC Advanced Grant titled “ProofCert: Broad
Spectrum Proof Certificates” for the years 2012-2016.  Miller's main research
interests are in computational logic, in particular, with the design of
automated and interactive theorem provers, the design and semantics of logic
programming and functional programming languages, and the design and
applications of proof certificates.

\end{itemize}

\end{sitedescription}

%%% Local Variables: 
%%% mode: latex
%%% TeX-master: "../propB"
%%% End: 

% LocalWords:  site-jacu.tex clange sitedescription emph compactitem pn semmath
% LocalWords:  prosuming-flexiformal KohSuc asemf06 GinJucAnc alsaacl09 StaKoh
% LocalWords:  tlcspx10 KohDavGin psewads11 ednote Radboud Bia ystok CALCULEMUS
% LocalWords:  textbf keypubs OntoLangMathSemWeb uwb Deyan Ginev Stamerjohanns
% LocalWords:  searchability
\newpage
\input{sites/USaclay}\newpage
\input{sites/IPP}\newpage
\input{sites/Mines}\newpage
\begin{sitedescription}{Pra}

\paragraph{Organization:}
\ednote{Give a one-paragraph run-down of the site and the team there. }

\textbf{Czech Institute of Informatics, Robotics, and Cybernetics
  (CIIRC)} is part of the Czech Technical University (CTU), which was
founded in 1707 and is one of the oldest technical universities in
Europe and currently the major technical university in the Czech
Republic. Offering high quality education and long tradition of
cutting edge science and engineering, CTU counts approx. 1700 members
of academic staff, 18500 students, 8 faculties, and 5 institutes. One
of the youngest of them, the Czech Institute of Informatics, Robotics,
and Cybernetics (CIIRC)\footnote{\url{http://www.ciirc.cvut.cz/?lang=en}} that
will participate in the project, was founded in 2013, starting its
operation in newly built facility in 2017.  Transfer of technology
from academia to industry is an important commitment for CIIRC. It
aims to concentrate an excellent research in the fields of AI,
robotics, automated reasoning, intelligent, distributed and complex
systems, automatic control, computer-aided manufacturing,
bioinformatics, biomedicine and assistive technologies. CIIRC supports
horizontal collaboration among all parts (faculties and institutes) of
CTU and opens the space for mutually beneficial cooperation with other
universities, with the Academy of Sciences of the Czech Republic, with
industrial companies and international institutions.

% CIIRC is a successful holder of national and international
% projects. In terms of success in European grant competitions (FP7,
% Horizon 2020), informatics, robotics and cybernetics teams at CTU
% belong to the strong national average. CIIRC managed to obtain grants
% from EU resources (mainly FP7, ERC, and H2020) worth more than \euro
% 15 M since its creation. Currently, CIIRC is proud to be the
% coordinator of H2020 project, RICAIP no. 857306. CIIRC hosts the H2020
% ERC Consolidator grant AI4REASON no. 649043 receiving funding from the
% European Research Council. The institute has also gained a broad
% industrial experience from providing research, development, and
% training services, and customized solutions to both local and
% international industrial partners, including Airbus, Rockwell
% Automation, Škoda Auto, Volkswagen, FORD, RWE GasNet, Eaton, and
% others. Moreover, CIIRC is the coordinator of the National Centre for
% Industry 4.0 enabling transfer of knowledge to industry and
% vice-versa. Recently, CIIRC became a full member of EIT Manufacturing
% and is also involved in a project of EIT Digital.  The Institute
% coordinates the national initiative on AI (AICZECHIA) and is involved
% in European networks on AI (CLAIRE\footnote{\url{https://claire-ai.org/}}, ELLIS\footnote{\url{https://ellis.eu/}}).

The research will be done within the
Automated Reasoning
Group
% \footnote{\url{http://arg.ciirc.cvut.cz/}} !!! CAREFUL - links are forbidden!!!
(ARG) of the 
Czech Institute of Informatics, Robotics, and Cybernetics (CIIRC)
%\footnote{\url{http://www.ciirc.cvut.cz/?lang=en}} 
at the Czech Technical University (CTU) in Prague. The group carries
out research in Automated Reasoning and its combinations with Machine
Learning and other AI methods. The group is largely responsible for
developing the area of combining Machine Learning and Automated
Reasoning, and for a number of major results in this area.  Its systems
regularly place high in the annual world championship for Automated
Theorem Proving (CASC), and it is the home of the ERC Consolidator
grant AI4REASON, running from September 1 2015 to August 31 2020. The
group currently comprises of three senior researchers, three postdocs
and five PhD students working with a number of ATP systems (E,
Vampire, Prover9, leanCoP, Satallax), ITP systems (Mizar, HOL, Coq,
Isabelle), machine learning systems and also SAT, QBF and SMT
solvers. The group has also close local connections with interested
mathematicians, in particular Dr. Stanovsk\'y from the Department of
Algebra at the Charles University in Prague. 

% A number of international
% collaborations with groups around the world are in place -- about 40
% international visitors were hosted in the past four years. The group has co-established and co-organizes the annual conference on AI and Theorem Proving (AITP) and a number of smaller events.
% The group members have access to large computational resources at CIIRC, comprising in total about 1000 CPUs and about 50
% high-performance GPU cards.  Other research groups at CIIRC (with some of them
% we already collaborate) cover a wide array of AI topics, such as
% pattern recognition and machine learning, computer vision, distributed
% artificial intelligence, mobile robotics, intelligent industrial
% systems, biomedical engineering and informatics, etc.
% The proposed project matches very well the research profile
% of the group. Implementation of the project will strengthen the
% group's position at the top of the AI-based research for 
% Automated Reasoning, and strengthen its international collaborations
% and status.


\paragraph{Main tasks:}

\begin{compactitem}
\item\ednote{specify the main tasks and reference the respective work packages} 
\end{compactitem}


\paragraph{Relevant previous experience:}

\ednote{give an overview over previous work and projects that add to the \pn project}

\begin{itemize}
    \item \textbf{AI4REASON} -- Artificial Intelligence for Large-Scale Computer-Assisted Reasoning (ERC - Consolidator Grant 649043, 2015 -- 2020, \url{http://ai4reason.org/}). PI J. Urban.
    \item \textbf{AI\&Reasoning} -- Artificial Intelligence and Reasoning. (Excellent Research Teams within the Operational Programme Research, Development, and Education and Ministry of Education, Youth and Sport of the Czech Republic \begin{math}CZ.02.1.01/0.0/0.0/15_003/0000466\end{math}. PI J. Urban.
    \item \textbf{Knowledge-based Automated Reasoning}, NWO Free Competition, 9/12 - 8/15. EUR 205000, PI J. Urban.
    \item \textbf{POSTMAN} -- Powering SMT Solvers by Machine Learning. ERC CZ grant by the Czech Ministry of Education.  PI Mikolas Janota. (2020 -- 2024).
    \item \textbf{Powering Automatic Theorem Provers by Machine Learning}. Junior grant of the Czech Science Foundation (GACR). PI M. Suda (2020 -- 2022). 
    \end{itemize}


\paragraph{Specific expertise:}

\begin{compactitem}
\item \ednote{give three to five specific areas of expertise that pertain to the \pn project}
\end{compactitem}

\paragraph{Staff members undertaking the work:}

% \textbf{Dr.\ Great Leader}\ednote{describe the site leader and his expertise}
% \textbf{Joe Implementor}\ednote{and more of them. }
\ednote{provide the key publications below}
\keypubs{providemore}

% Martin Suda, Josef Urban

\begin{itemize}
\item
\textbf{Dr. Josef Urban} (male) is a Principal Researcher at the Czech Institute of Informatics, Robotics, and Cybernetics (CIIRC) of the Czech Technical University (CTU) in Prague where he is heading the ERC-funded project
AI4REASON. His interests include Automated Reasoning, Formal Verification and Machine Learning. In particular, he is interested in development of combined inductive/learning and deductive/reasoning
``strong AI'' methods and systems over large formal (fully semantically specified) knowledge bases. Examples are large corpora of formally stated definitions, theorems and proofs in mathematics, software
verification and related fields.  He has made such corpora available to the AI methods, created the first benchmarks, and developed first approaches and systems combining learning and reasoning over such
corpora in various feedback loops.  The systems developed by him and his colleagues have won several competitions and the methods today assist formal verification in proof assistants. He has also
co-developed first learning/reasoning systems for automated formalisation of informal mathematics, and co-founded the conference on Artificial Intelligence and Theorem Proving (AITP).
He received his MSc in Mathematics and PhD in Computers Science from
the Charles University in Prague, worked as an assistant professor in
Prague, and as a researcher at the University of Miami and Radboud
University Nijmegen.

% \item \textbf{Dr. Thibault Gauthier} (male) is a researcher at CIIRC. He is the author
% of the first successful tactical theorem prover using Monte-Carlo search methods -- TacticToe for HOL4~\cite{Gauthier18}.
% He has also implemented strong \emph{hammer-style} methods for HOL4~\cite{hh4h4}. Both these systems have been integrated into the main HOL4 branch and serve HOL4 users in development of proofs in HOL4.
% He is also the author of methods for drawing analogies between heterogeneous proof libraries~\cite{DBLP:journals/jsc/GauthierK19}
% and deep reinforcement learning methods for interactive provers~\cite{DBLP:journals/corr/abs-1910-11797}. Dr. Gauthier received his PhD in Computer Science from the University of Innsbruck.


\item \textbf{Dr. Martin Suda}
Dr.~Martin Suda, has worked since 2008 in the field of
Automated Theorem Proving and related fields.  He is an author of a number of research
results in these fields, many of them published in top conferences such
as IJCAR (CORE A*), LPAR (CORE A), CADE (CORE A), SAT (CORE A) and TACAS (CORE A).  His paper
with Benjamin Kiesl \emph{A Unifying Principle for Clause Elimination
  in First-Order Logic}~\cite{DBLP:conf/cade/Kiesl017} won the best
paper award in CADE 2017.
Dr.~Suda has worked with a number of ATP systems, including
SPASS~\cite{WeidenbachDFKSW09}, Vampire and E~\cite{Schulz13}, and has recently implemented the first practically convincing neural machine learning guidance in the saturation-style ATP setting~\cite{abs-1903-03182}. He has
been an active developer of Vampire since 2014, and has been since
then a part of the team that won a number of first places in automated
theorem proving competitions with Vampire.
He has been an active member of the Automated Reasoning and AI
community, being a program committee member of conferences such as
CADE'19, SAT'19, IJCAI-ECAI'18. He has been a program co-chair of IWIL'18 and he is a
conference co-chair of CICM'19.
\end{itemize}


\begin{itemize}
 
%    \item Thibault Gauthier, Cezary Kaliszyk, Josef Urban, Ramana Kumar, Michael Norrish:
%Learning to Prove with Tactics. CoRR abs/1804.00596 (2018)
\item K. Chvalovsky, J. Jakubuv, M. Suda, J. Urban: ENIGMA-NG: Efficient Neural and Gradient-Boosted Inference Guidance for E. CADE 2019: 197-215
  \item Giles Reger, Martin Riener, Martin Suda:
    Symmetry Avoidance in MACE-Style Finite Model Finding. FroCos 2019: 3-21
\item 	Jan Jakubuv, Josef Urban:
  Hammering Mizar by Learning Clause Guidance. ITP 2019: 34:1-34:8
  \item Mikolas Janota, Martin Suda:
Towards Smarter MACE-style Model Finders. LPAR 2018: 454-470
\item Giles Reger, Martin Suda, Andrei Voronkov:
Unification with Abstraction and Theory Instantiation in Saturation-Based Reasoning. TACAS (1) 2018: 3-22
\item C. Kaliszyk, J. Urban, H. Michalewski, M. Olsak: Reinforcement Learning of Theorem Proving. NeurIPS 2018: 8836-8847
   \item Thibault Gauthier, Cezary Kaliszyk, Josef Urban:
     TacticToe: Learning to Reason with HOL4 Tactics. LPAR 2017: 125-143
   \item Benjamin Kiesl, Martin Suda:
A Unifying Principle for Clause Elimination in First-Order Logic. CADE 2017: 274-290
    \item J. Harrison, J. Urban, F. Wiedijk: History of Interactive Theorem Proving. Computational Logic 2014: 135-214
    \item J. C. Blanchette, C. Kaliszyk, L. C. Paulson, J. Urban:
      Hammering towards QED. J. Formalized Reasoning 9(1): 101-148 (2016)
          \item 	Krystof Hoder, Giles Reger, Martin Suda, Andrei Voronkov:
Selecting the Selection. IJCAR 2016: 313-329
%\item Thibault Gauthier, Cezary Kaliszyk:
%Premise Selection and External Provers for HOL4. CPP 2015: 49-57
    \item C. Kaliszyk, J. Urban:
Learning-Assisted Automated Reasoning with Flyspeck. J. Autom. Reasoning 53(2): 173-213 (2014)
\item Jasmin Christian Blanchette, David Greenaway, Cezary Kaliszyk, Daniel Kühlwein, Josef Urban:
A Learning-Based Fact Selector for Isabelle/HOL. J. Autom. Reasoning 57(3): 219-244 (2016)
\item Alexander A. Alemi, François Chollet, Niklas Eén, Geoffrey Irving, Christian Szegedy, Josef Urban:
DeepMath - Deep Sequence Models for Premise Selection. NIPS 2016: 2235-2243
\item Cezary Kaliszyk, Josef Urban:
MizAR 40 for Mizar 40. J. Autom. Reasoning 55(3): 245-256 (2015)
\item Josef Urban, Piotr Rudnicki, Geoff Sutcliffe:
ATP and Presentation Service for Mizar Formalizations. J. Autom. Reasoning 50(2): 229-241 (2013)
\item Josef Urban, Geoff Sutcliffe, Petr Pudlák, Jirí Vyskocil:
MaLARea SG1- Machine Learner for Automated Reasoning with Semantic Guidance. IJCAR 2008: 441-456
\item Josef Urban:
MPTP 0.2: Design, Implementation, and Initial Experiments. J. Autom. Reasoning 37(1-2): 21-43 (2006)
\end{itemize}



\end{sitedescription}
%%% Local Variables: 
%%% mode: latex
%%% TeX-master: "../propB"
%%% End: 

% LocalWords:  site-jacu.tex clange sitedescription emph compactitem pn semmath
% LocalWords:  prosuming-flexiformal KohSuc asemf06 GinJucAnc alsaacl09 StaKoh
% LocalWords:  tlcspx10 KohDavGin psewads11 ednote Radboud Bia ystok CALCULEMUS
% LocalWords:  textbf keypubs OntoLangMathSemWeb uwb Deyan Ginev Stamerjohanns
% LocalWords:  searchability
\newpage
\input{sites/Southhampton}\newpage
\begin{sitedescription}{Str}

\ednote{a description of the legal entity}

\logo{Strasbourg}
  
Located in the heart of Europe, the University of Strasbourg is heir to a great tradition born of
the humanism of the 16 th century.

On 1 January 2009 the University of Strasbourg was born - a unique and pioneering
example of merging universities in France: Louis Pasteur, Marc Bloch and Robert Schuman.
European by nature and international by design, the University’s fundamental training and
research goals include forging partnerships with European and international universities.
Located on 4 campuses spread all over the city, the University of Strasbourg is one of the
largest universities in France, with nearly 51 000 students (including 20 \% of international
students).

Certified Excellence Initiative (IdEx) - obtained in 2012 and definitively confirmed in 2016
by the national programme “Investissements d’Avenir - the University of Strasbourg
strengthens its position as an internationally attractive university. Implementing innovative
projects that foster excellence, the University of Strasbourg is involved in supporting its
researchers and students.
As a leading European centre for training and research, the University of Strasbourg has
developed a strong French-German cooperation and is now a privileged partner among the
Upper-Rhine universities.

The University is involved in national and European research projects within various
programmes. Since 2009, the University of Strasbourg obtained 74 FP7 projects, 76 H2020
projects, 30 INTERREG IV projects, 29 INTERREG V projects and 375 projects under the
French National Research Programme (ANR). Presently 107 ANR projects, 50 H2020
projects, 23 INTERREG V projects and 19 Erasmus+ projects (1 European University, 2
Erasmus Mundus master degrees, 5 Erasmus + strategic partnerships, 1 knowledge alliance, 1
capacity building project, 1 Erasmus + Sport project and 6 Jean Monnet actions - including 1
Centre of excellence) are active. It currently coordinates 42 EU projects and is preparing and
awaiting the evaluation of approximately 40 proposals.

ICube: Created in 2013, the laboratory brings together researchers of the University of Strasbourg, the CNRS(Centre National de la Recherche Scientifique), in the fields of engineering science and computer science.
In this context the IGG team focuses on geometric modeling, visualization, constraint solving and formalization of geometry. The member of the project focus on the formal definition of the geometric universe, proof of properties, automatic generation of geometric objects defined by a specification and deriving certified geometric algorithms. We work on computer science methods allowing to assist proofs, guarantee the correctness and the feasibility and, when possible, to insure automatically some task using Coq tactics or, geometric constraint solving. The results of these researches can be exploited in geometric modeling, computational geometry, pure geometry, mathematics teaching.

\paragraph{Main tasks:}

\ednote{its main tasks, with an explanation of how its profile matches the tasks in the proposal}

\ednote{specify the main tasks and reference the respective work packages}

\begin{compactitem}
\item Integration of the GeoCoq library in Logipedia: \WPref{libraries}, task \taskref{libraries}{geocoq}
\item Concept alignement for geometry: \WPref{alignment}, \taskref{}{aligncasestudies}
\item Animation of the club of users of Logipedia in Education: \WPref{dissemination}, \taskref{dissemination}{teachersclub} 
\item User interface for interactive theorem proving: \WPref{access} \taskref{access}{web}
\end{compactitem}

\paragraph{Publications, products or services:}

\ednote{a list of up to 5 relevant publications, and/or products, services (including widely-used datasets or software), or other achievements relevant to the  call content}

\begin{compactitem}
\item Nicolas Magaud. \emph{Changing Data Representation within the Coq System.} In TPHOLs'2003, volume 2758 of LNCS. Springer-Verlag, 2003
\item Pierre Boutry, Gabriel Braun, Julien Narboux. \emph{Formalization of the Arithmetization of Euclidean Plane Geometry and Applications.} Journal of Symbolic Computation, Elsevier, 2019, Special Issue on Symbolic Computation in Software Science, 90, pp.149-168.
\item Michael Beeson, Julien Narboux, Freek Wiedijk. \emph{Proof-checking Euclid.} Annals of Mathematics and Artificial Intelligence, Springer Verlag, 2019, pp.53.
\item Julien Narboux, David Braun. \emph{Towards A Certified Version of the Encyclopedia of Triangle Centers.} Mathematics in Computer Science, Springer, 2016
\item David Braun, Nicolas Magaud, Pascal Schreck, \emph{Two Cryptomorphic Formalizations of Projective Incidence Geometry}, Annals of Mathematics and Artificial Intelligence, Springer Verlag
\end{compactitem}

\paragraph{Previous projects or activities:}

\ednote{a list of up to 5 relevant previous projects or activities, connected to the subject of this proposal}

Members of the group have expertise in the field of interactive and automated theorem proving in geometry.
They have been involved in several national, bilateral and international projects (the French ANR project Galapagos, Serbian-French  Co-Operation grant EGIDE/Pavle Savic 680-00-132).

%\paragraph{Specific expertise:}

%\begin{compactitem}
%\item Interactive theorem proving in geometry (Magaud, Narboux, Schreck).
%\item Automatic theorem proving in geometry (Magaud, Narboux, Schreck).
%\item Axiomatization of geometry (Schreck, Narboux).
%\item Change of data representation in proofs (Magaud). 
%\end{compactitem}

\paragraph{Infrastructures or technical equipments:}

\ednote{a description of any significant infrastructure and/or any major items of technical equipment, relevant to the proposed work}

\begin{compactitem}
\item Michael Beeson, Pierre Boutry, Gabriel Braun, Charly Gries, Julien Narboux. GeoCoq. 2018,\\swh:1:dir:97ce53176b7d5e89d069bc60f49c3fa186831307
\end{compactitem}

\paragraph{Persons primarily responsible for carrying out the proposed activities:}

\begin{itemize}
\item{\bf Julien Narboux}\ednote{describe the site leader and his expertise}
Julien Narboux is an associate professor at the Department of Computer Science, University of Strasbourg, France since 2007. He received a doctorate from University of Orsay in 2006 about “Formalization and automation of geometric reasoning”. After that he held a postdoc positions at TUM.
He published about 30 papers in peer-rewieved international conferences and journals, and has been PC member of international conferences and workshops such as ADG, AISC, SCSS, FVPS, ThEdu. He is the head of the steering committee of the Automatic Deduction in Geometry conference. Julien Narboux is the leader of the GeoCoq project.

\item{\bf Nicolas Magaud} is an associate professor at the Department of
Computer Science, University of Strasbourg, France since 2005. He
received a PhD from the University of Nice Sophia-Antipolis, France in
2003. His thesis subject was ``changing data representation in the
calculus of constructions''. Before being hired by University of
Strasbourg, he was a senior research associate at the University of
New South Wales, Sydney, Australia. In Strasbourg, Nicolas Magaud
has been working on formalizing various aspects of geometry using Coq, spanning from
computational geometry algorithms to exact real computations applied to
discrete geometry. He published about 15 papers in peer-rewieved
international conferences and journals.  

\item{\bf Pascal Schreck} is full professor in computer science since 2002. He is interested in the formalization of various geometries from the rule and compass constructions to finite incidence geometry including geometric algebras, Tarski and Wu's geometries \emph{etc.} He studied some applications of theses formal geometries mainly in mechanical CAD and computer aided education.
\end{itemize}

\end{sitedescription}

%%% Local Variables: 
%%% mode: latex
%%% TeX-master: "../propB"
%%% End: 
\newpage
\begin{sitedescription}{Tum}

\paragraph{Organization:}

The Technical University of Munich (TUM) is characterized by a unique profile with its core domains natural sciences, engineering, life sciences and medicine. The institutional strategy is focused on strengthening the excellence of disciplinary core competences in research, teaching and learning, but is also targeted towards the promotion of ground-breaking, interdisciplinary research. TUM is committed toward the major challenges facing society in the 21st century in areas such as energy, climate, and environment, natural resources, health and nutrition, communication and information, mobility and infrastructure.
The student body of TUM is currently more than 41 000 students and is constantly rising. TUM is regularly among the best national performers in international rankings. For the fifth time in a row, TUM took the first place among the German universities in the renowned QS World University Ranking (rank 55 worldwide). Looking at the contributions published in the particularly renowned academic journals of the "Nature" Group and the "Science" Group, TUM is positioning itself as number 42 and 1st in Germany. TUM was ranked 6th in the Global University Employability Ranking in which companies worldwide evaluate the quality of university graduates. THE World University Ranking has rated the Technical University of Munich (TUM) as one of the four best technical universities in Europe. TUM placed second in comparison to all other universities in Germany and was ranked number 43 worldwide. 
TUM has been successful in all three funding lines of the German Excellence Initiative. In 2012 and 2019, TUM has again secured the title ``University of Excellence''.

Research and Training Programmes
Previous Involvement in Research and Training Programmes:
During the last two Framework Programmes for Research and Technological Development of the EC (FP7 and Horizon2020), TUM was and is involved in more than 500 EU research projects and has participated in over 100 ERC grants in total.
Current involvement in Research and Training Programmes: 
Currently, TUM is involved in more than 200 Horizon 2020 projects, for more than 75 of which TUM has a coordinating role. That includes 28 ERC Starting Grants, 28 ERC Consolidator Grants, 14 ERC Advanced Grants, 5 ERC Proof of Concept Grants and 1 ERC Synergy Grant.


\paragraph{Main tasks:}

\begin{compactitem}
\item\ednote{specify the main tasks and reference the respective work packages} 
\end{compactitem}


\paragraph{Previous projects or activities:}

A string of nationally funded projects to develop and use the Isabelle
system. Most recently the EUR 1.25 million DFG Koselleck grant Verified Algorithm Analysis.

\paragraph{Specific expertise:}

\begin{compactitem}
\item \ednote{give three to five specific areas of expertise that pertain to the \pn project}
\end{compactitem}

\paragraph{Persons primarily responsible for carrying out the proposed activities:}

\begin{itemize}
\item \textbf{Tobias Nipkow} (co-leader of work package
  \WPref{libraries}) is a full professor for Logic and
  Verificatiuon at TUM. He received his Ph.D. in Computer Science
  from the University of Manchester in 1987.  He has been a
  lecturer at the University of Manchester (1984--1987),
  post-doctoral associate at MIT (1988--1989) and at Cambridge
  University (1989-1992). He was appointed associate professor for Theory of Programming at
  TUM in 1992 and promoted to his current position in 2011. Since 2008
  he has been Editor-in-Chief of the  Journal of Automated Reasoning.
 and is currently serving on the editorial board of Logical Methods in
 Computer Science. He founded the steering committee for the
 conference Interactive Theorem Proving in 2007 and served as its
 chair until 2017.
 He has served as a program committee chair
on a number of conferences in the general area of computational logic.  Nipkow's main research
interests are in computational logic, in particular, with the design of
interactive theorem provers (he is one of the designers of the
Isabelle theorem prover), the design and semantics of programming
languages and in partticular the verification of functional and
imperative programs.
\item \textbf{Makarius Wenzel}
\end{itemize}
\ednote{provide the key publications below}
\keypubs{providemore}

\end{sitedescription}
%%% Local Variables: 
%%% mode: latex
%%% TeX-master: "../propB"
%%% End: 

% LocalWords:  site-jacu.tex clange sitedescription emph compactitem pn semmath
% LocalWords:  prosuming-flexiformal KohSuc asemf06 GinJucAnc alsaacl09 StaKoh
% LocalWords:  tlcspx10 KohDavGin psewads11 ednote Radboud Bia ystok CALCULEMUS
% LocalWords:  textbf keypubs OntoLangMathSemWeb uwb Deyan Ginev Stamerjohanns
% LocalWords:  searchability
\newpage

\section{Third parties involved in the project (including use of third party resources)}

%%% Local Variables:
%%% mode: latex
%%% TeX-master: "propB"
%%% End:

\newpage
\chapter{Ethics and Security}\label{chap:ethical}


\section{Ethics}

Regarding the section 4 ``Ethics'' of the proposal submission form,
Logipedia is not concerned by any ethical issues mentioned in this
form.  The Logipedia consortium will pay attention to any ethical
issue that might araise during the project. If at some point during
the course of the project, the consortium or any scientist is unsure
about how to handle a particular situation or requires advice on
ethical issues, the partners or the individuals, supported by the EPM,
will refer to the operational ethical committee of Inria (the COERLE)
before proceeding.



\section{Security}

The LOGIPEDIA project does not involve any activities or results
raising security issues nor contain any ``EU-classified information''
as background or results.

%%% Local Variables:
%%% mode: latex
%%% TeX-master: "propB"
%%% End:


\end{proposal}

\end{document}

%%% Local Variables: 
%%% mode: LaTeX
%%% TeX-master: t
%%% mode: flyspell
%%% ispell-local-dictionary: "english"
%%% End: 

% LocalWords:  efo efoRM baz bazRM miko acrolong ntelligent iting pn pnlong
% LocalWords:  textsc newpage compactht texttt euproposal.cls callname callid
% LocalWords:  challengeid objectiveid outcomeid tableofcontents
